\documentclass[12pt,a4paper]{article}

% Margins.
\setlength{\oddsidemargin}{0in}
\setlength{\evensidemargin}{0in}
\setlength{\headheight}{12pt}
\setlength{\headsep}{42pt}
\setlength{\topmargin}{-54pt}
\setlength{\textwidth}{6.5in}
\setlength{\textheight}{10in}

\usepackage{amsmath}
\usepackage{float}
\usepackage{graphicx}
\usepackage[hyphens]{url}
\usepackage{hyperref}	% Clickable links to figures, references and urls.
\usepackage{enumerate}
\usepackage{datetime}

% Drawing.
\usepackage{pgf}
\usepackage{tikz}

% Listings for formatting code.
\usepackage{listings}
\usepackage{textcomp}
% General options.
\lstset{breaklines=true, basicstyle=\small\ttfamily, tabsize=4, numbers=left, stepnumber=1, frame=single, showstringspaces=false, upquote=true}
% C++ specific high-lighting. Comments are 50/50 shades of green/black and strings coloured with 60/40 red/black mixture.
\lstset{language=[ISO]C++, commentstyle=\color{green!50!black}, keywordstyle=\color{blue}, stringstyle=\color{red!60!black}}

%opening
\title{\vspace{-2cm}Programming for Engineers II\\Class 22\\Implicit Object Return\\Explicit and Mutable Keywords}
\author{Attique Dawood}
\date{July 04, 2013\\[0.2cm] Last Modified: \today, \currenttime}
\begin{document}
\maketitle
\section{Announcements}
\begin{itemize}
\item Quiz today.
\end{itemize}
\section{Revision}
\begin{itemize}
\item Variables declared \verb|static| are only destroyed when the program ends.
\item Even if declared in a function, \verb|static| variables will continue to exist and take up memory until the end of program.
\item \verb|static| variable declaration/creation will only happen once.
\item There is no point making global variables \verb|static| because they already function as if they were \verb|static|... but nobody is stopping you from doing so.
\item One possible use of using static variables in a function is when you're passing a pointer or reference to a local variable. If the local variable isn't static it's pointer or reference is no longer useful outside the function.
\item A \verb|static| data member of class has only one instance. It belongs to the class rather than any object.
\item \verb|static| members are not associated with object but with the class.
\item Even there aren't any objects, \verb|static| data member will continue to exist.
\item One use of \verb|static| is to provide a common variable that can be accessed by objects. Communication?
\item \verb|static| functions can only access \verb|static| data members. Why?
\item \verb|static| functions can be called even if objects don't exist.
\item If objects don't exist, \verb|static| variables still exist!
\end{itemize}
\section{Implicitly Returning Objects}
\begin{itemize}
\item This is a cute way of returning objects when an appropriate constructor is already provided.
\item Assume complex class with two--argument constructor already defined. We can return an object by writing class name and passing constructor arguments.
\end{itemize}
\begin{lstlisting}[caption={Implicitly Returning Objects}]
#include <iostream>
using namespace std;
class Complex
{
	private:
	float real;
	float img;
	public:
	Complex(): real(0.f), img(0.f) {}
	Complex(float a, float b): real(a), img(b) {}
	Complex operator+ (Complex& rhs)
	{
		return Complex(real+rhs.real, img+rhs.img); // Implicit object return.
	}
	// Prefix.
	Complex operator++ ()
	{
		return Complex(++real, ++img); // Implicit object return.
	}
	// Postfix.
	Complex operator++ (int dummy)
	{
		return Complex(real++, img++); // Implicit object return.
	}
};
int main()
{
	Complex A(0.f,0.f);
	A++;
	++A;
	Complex C = A + Complex(2.4f, 22.4f); // Notice implicit rhs operand.
	
	return 0;
}
\end{lstlisting}
\section{\texttt{mutable}}
\begin{itemize}
\item A \texttt{mutable} variable can be changed by \texttt{const} functions.
\item As data member of a class, \verb|mutable| variables can be changed even if the object is constant.
\end{itemize}
\begin{lstlisting}[caption={\texttt{mutable} variables}]
#include <iostream>
using namespace std;

class A
{
	private:
	mutable int x; // Declaration
	public:
	void Setx(int t) const { x = t; }
	int Getx() const { return x; }
};

int main()
{
	A obj;	
	A.Setx(3);
	cout << "x = " << A.Getx() << endl;

	return 0;
}
\end{lstlisting}
\begin{lstlisting}[caption={\texttt{const} object with \texttt{mutable} data members}]
#include <iostream>
using namespace std;

class A
{
	private:
	mutable int x; // Declaration
	int y;
	public:
	void Setx(int t) const { x = t; }
	void Sety(int t) { y = t; }
	int Getx() const { return x; }
	int Gety() const { return y; }
};

int main()
{
	const A obj;	
	A.Setx(3); // Changing mutable data. This is OK.
	cout << "x = " << A.Getx() << endl;

	A.Sety(5); // This will not work, object is const.
	
	return 0;
}
\end{lstlisting}
\section{\texttt{explicit} Keyword}
\begin{itemize}
\item Sometimes we don't want any implicit conversions.
\item \verb|explicit| keyword can be used to specify if don't want a certain conversion to take place implicitly.
\item \verb|explicit| is used with one argument constructors.
\end{itemize}
\begin{lstlisting}[caption={\texttt{explicit} constructor with distance class}]
#include <iostream>
using namespace std;

class distance
{
	private:
	float m;
	public:
	distance(): m(0)
	{
	}
	explicit distance(float km)
	{
		m = 1000 * km;
	}
	void display()
	{
		cout << "Distance = " << m << " meters." << endl;
	}
};

void showdist(distance d)
{
	d.display();
}

int main()
{
	distance d1(2.4);
	showdist(d1);
	showdist(3.4); // This won't work with explicit constructor.

	return 0;
}
\end{lstlisting}
%\nocite{*}
\bibliographystyle{plain}
\bibliography{OOPref}
\end{document}
