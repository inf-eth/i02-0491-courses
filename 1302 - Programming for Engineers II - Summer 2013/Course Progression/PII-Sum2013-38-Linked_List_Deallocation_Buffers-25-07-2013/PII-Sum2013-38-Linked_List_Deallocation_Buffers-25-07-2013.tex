\documentclass[12pt,a4paper]{article}

% Margins.
\setlength{\oddsidemargin}{0in}
\setlength{\evensidemargin}{0in}
\setlength{\headheight}{12pt}
\setlength{\headsep}{42pt}
\setlength{\topmargin}{-54pt}
\setlength{\textwidth}{6.5in}
\setlength{\textheight}{10in}

\usepackage{amsmath}
\usepackage{float}
\usepackage{graphicx}
\usepackage[hyphens]{url}
\usepackage{hyperref}	% Clickable links to figures, references and urls.
\usepackage{enumerate}
\usepackage{datetime}

% Drawing.
\usepackage{pgf}
\usepackage{tikz}

% Listings for formatting code.
\usepackage{listings}
\usepackage{textcomp}
% General options.
\lstset{breaklines=true, basicstyle=\small\ttfamily, tabsize=4, numbers=left, stepnumber=1, frame=single, showstringspaces=false, upquote=true}
% C++ specific high-lighting. Comments are 50/50 shades of green/black and strings coloured with 60/40 red/black mixture.
\lstset{language=[ISO]C++, commentstyle=\color{green!50!black}, keywordstyle=\color{blue}, stringstyle=\color{red!60!black}}

%opening
\title{\vspace{-2cm}Programming for Engineers II\\Class 38\\Linked List De--allocation}
\author{Attique Dawood}
\date{July 25, 2013\\[0.2cm] Last Modified: \today, \currenttime}
\begin{document}
\maketitle
\section{Announcements}
\begin{itemize}
\item None
\end{itemize}
\section{Input/Output Buffers}
\begin{itemize}
\item Buffers are temporary storage that are associated with hardware input/output (I/O) devices.
\item The stream of data flow between devices may not be always constant. Data may flow in bursts.
\item In order to streamline data flow, buffers are used.
\item As an example, hard disk access is slow and if we're writing a byte onto the hard disk every second then that would be slower compared to writing a whole chunk of bytes at once. Buffers store the incoming stream of data temporarily and when they're full, data is written at once.
\item Writing the contents of buffer onto hard disk is known as \textit{synchronisation}.
\item Buffers are automatically synchronised when they're full.
\item Buffers can also be synchronised explicitly with \verb|endl|, \verb|flush()| and \verb|sync()| calls.
\end{itemize}
\begin{table}[H]
\begin{center}
\vspace{0.3cm}
	\begin{tabular}{cl}
	\hline \hline
		\textbf{Stream Object} \rule{0pt}{2.6ex} & \textbf{Description}\\
		\hline
		\texttt{cin}\rule{0pt}{2.6ex} &  Buffered input\\
		\texttt{cout}\rule{0pt}{2.6ex} &  Buffered output\\
		\texttt{cerr}\rule{0pt}{2.6ex} &  Unbuffered error stream\\
		\texttt{clog}\rule{0pt}{2.6ex} &  Same as \verb|cerr| but buffered\\
	\hline \hline
	\end{tabular}
\end{center}
\label{StreamBuffers}
\caption{Stream Buffers}
\end{table}
\section{Linked List De--allocation}
\begin{lstlisting}[caption={Linked List Destructor}]
~LinkedList()
{
	Node* temp1;
	Node* temp2;
	temp1 = temp2 = First;
	if (First == NULL)
		return;
	else
	{
		temp2 = temp2->Next;
		while(true)
		{
			cout << "Deleting " << temp1->Data << endl;
			delete temp1;
			temp1 = temp2;
			if (temp1 == NULL)
				break;
			temp2 = temp2->Next;
		}
	}
}
\end{lstlisting}
%\nocite{*}
\bibliographystyle{plain}
\bibliography{OOPref}
\end{document}
