\documentclass[12pt,a4paper]{article}

% Margins.
\setlength{\oddsidemargin}{0in}
\setlength{\evensidemargin}{0in}
\setlength{\headheight}{12pt}
\setlength{\headsep}{42pt}
\setlength{\topmargin}{-54pt}
\setlength{\textwidth}{6.5in}
\setlength{\textheight}{10in}

\usepackage{amsmath}
\usepackage{float}
\usepackage{graphicx}
\usepackage[hyphens]{url}
\usepackage{hyperref}	% Clickable links to figures, references and urls.
\usepackage{datetime}

% Drawing.
\usepackage{pgf}
\usepackage{tikz}

% Listings for formatting code.
\usepackage{listings}
\usepackage{textcomp}
% General options.
\lstset{breaklines=true, basicstyle=\small\ttfamily, tabsize=4, numbers=left, stepnumber=1, frame=single, showstringspaces=false, upquote=true}
% C++ specific high-lighting. Comments are 50/50 shades of green/black and strings coloured with 60/40 red/black mixture.
\lstset{language=[ISO]C++, commentstyle=\color{green!50!black}, keywordstyle=\color{blue}, stringstyle=\color{red!60!black}}

%opening
\title{\vspace{-2cm}Programming for Engineers II\\Class 20\\Data Conversion}
\author{Attique Dawood}
\date{July 02, 2013\\[0.2cm] Last Modified: \today, \currenttime}
\begin{document}
\maketitle
\section{Announcements}
\begin{itemize}
\item Today's lecture is from Chapter 8 of Lafore \cite{Lafore}.
\item Please read ``Conversions Between Objects of Different Classes'' from Chapter 8 of Lafore \cite{Lafore}.
\item Understand code for class time12 and class time24
\item Understand how conversion from time12 to time24 and vice versa had been handled.
\item Understand what does the following mean:
\begin{enumerate}
\item Routine in source object
\item Routine in destination object
\end{enumerate}
\end{itemize}
\section{Revision}
\begin{itemize}
\item Subscript operator \verb|[]|?
\item Normally subscript operator doesn't check bounds. Crashes happen at runtime if bad indices are provided.
\item We modified the subscript operator to handle crashes.
\end{itemize}
\section{Casting: Conversion Between Basic Data Types}
\begin{itemize}
\item \underline{Variable classification on the basis of `scope'.}
\item Global variables are accessible everywhere in a single file, in class, any function and main. Their scope is `global'.
\item Local variables are limited to the structure in which they are declared.
\item Global variables exist for the entirety of the program and will only be destroyed when program exits.
\item Local variables will only exist for the duration of the function in which they are declared. They will be destroyed when function ends.
\end{itemize}
\begin{lstlisting}[caption={Variable Casting}]
int main()
{
	int x = 3;
	float y = 4.3f;

	// Implicit conversion.
	x = y;
	y = x;

	// Explicit conversion.
	x = static_cast<int>(y);
	x = (int)y;
	y = (float)x;
	
	return 0;
}
\end{lstlisting}
\section{Conversion Between Objects}
\begin{itemize}
\item Source and destination objects.
\item Sometimes you cannot change one data type. Can you modify how int or float works?
\item We have facility to modify either source or destination code to perform a conversion.
\item If the destination code cannot be changed we need to use \verb|operator type()| function (type is destination type) to convert source variable into destination variable.
\item An example is conversion of complex number to a real number, which we mathematically define as taking real part and discarding imaginary part.
\end{itemize}
\begin{lstlisting}[caption={Conversion Between Objects}]
int intvar;
float floatvar
Complex complexobj;

intvar = floatvar;
floatvar = intvar;

complexobj = intvar; // complexobj.operator=(intvar)
complexobj = floatvar; // complexobj.operator=(floatvar)

intvar = complexobj; // overload int?
floatvar = complexobj; // overload float?

objA = objB; // objA.operator=(objB)
objB = objA; // objB.operator=(objA)
\end{lstlisting}
\begin{lstlisting}[caption={Object from Float}]
class Complex
{
	private:
	float real;
	float img;
	public:
	Complex(): real(0), img(0) {}
	Complex(float x, float y): real(x), img(y) {}
	void operator=(float pfloatvar)
	{
		real = pfloatvar;
		img = 0;
	}
};
int main()
{
	float resistance = 3.3;
	Complex impedance;
	impedance = resistance; // Complex from int conversion: impedance.operator=(resistance)
	// impedance is now 3.3+j0
	return 0;
}
\end{lstlisting}
\begin{lstlisting}[caption={Float from Object}]
class Complex
{
	private:
	float real;
	float img;
	public:
	Complex(): real(0), img(0) {}
	Complex(float x, float y): real(x), img(y) {}
	operator float()
	{
		return real;
	}
};
int main()
{
	float resistance;
	Complex impedance(3.3,4.7);
	resistance = impedance; // float from complex conversion: impedance.operator float()
	// resistance now contains 3.3.

	return 0;
}
\end{lstlisting}
\section{Object Conversion}
\begin{itemize}
\item What if we want to convert between different classes? Procedure remains the same.
\item We have two classes, A and B.
\item Ideally, to do \verb|A=B|, we would want to make \verb|void operator=(B)| in class A. Similarly, for \verb|B=A| we would want to make \verb|void operator=(A)| in class B.
\item What if we cannot change class B?
\item We can define \verb|operator B()| in class A that returns object of class B.
\end{itemize}
\begin{minipage}{7.5cm}
\begin{lstlisting}[caption={class A}]
class A
{
	private:
	float x;
	public:
	void operator=(B);
	operator B();
};
\end{lstlisting}
\end{minipage}
\hspace*{1cm}
\begin{minipage}{7.5cm}
\begin{lstlisting}[caption={class B}]
class B
{
	private:
	int y;
	public:
	// Cannot change code.
};
\end{lstlisting}
\end{minipage}
\begin{lstlisting}[caption={Object Conversion}]
int main()
{
	A Aobj;
	B Bobj;
	
	Aobj = Bobj; // conversion from B to A. Aobj.operator=(B) called.
	Bobj = Aobj; // conversion from A to B. Aobj.operator B() called.

	return 0;
}
\end{lstlisting}
%\nocite{*}
\bibliographystyle{plain}
\bibliography{OOPref}
\end{document}
