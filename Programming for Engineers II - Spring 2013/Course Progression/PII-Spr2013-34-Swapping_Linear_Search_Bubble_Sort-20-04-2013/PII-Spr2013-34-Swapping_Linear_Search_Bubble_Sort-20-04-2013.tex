\documentclass[12pt,a4paper]{article}

% Margins.
\setlength{\oddsidemargin}{0in}
\setlength{\evensidemargin}{0in}
\setlength{\headheight}{12pt}
\setlength{\headsep}{42pt}
\setlength{\topmargin}{-54pt}
\setlength{\textwidth}{6.5in}
\setlength{\textheight}{10in}

\usepackage{amsmath}
\usepackage{float}
\usepackage{graphicx}
\usepackage[hyphens]{url}
\usepackage{hyperref}	% Clickable links to figures, references and urls.
\usepackage{enumerate}
\usepackage{datetime}

% Drawing.
\usepackage{pgf}
\usepackage{tikz}

% Listings for formatting code.
\usepackage{listings}
\usepackage{textcomp}
% General options.
\lstset{breaklines=true, basicstyle=\small\ttfamily, tabsize=4, numbers=left, stepnumber=1, frame=single, showstringspaces=false, upquote=true}
% C++ specific high-lighting. Comments are 50/50 shades of green/black and strings coloured with 60/40 red/black mixture.
\lstset{language=[ISO]C++, commentstyle=\color{green!50!black}, keywordstyle=\color{blue}, stringstyle=\color{red!60!black}}

%opening
\title{\vspace{-2cm}Programming for Engineers II\\Class 34\\Swapping\\Linear Search\\Bubble Sort}
\author{Attique Dawood}
\date{April 20, 2013\\[0.2cm] Last Modified: \today, \currenttime}
\begin{document}
\maketitle
\section{Announcements}
\begin{itemize}
\item Sessional -- II papers. Exam average for section C = 25.45.
\end{itemize}
\section{Revision}
\begin{itemize}
\item Virtual destructors.
\item Resolving ambiguity in multiple inheritance using polymorphism.
\end{itemize}
\section{Swapping}
\begin{itemize}
\item To swap two variables, we need a temporary variables.
\item A swap function may use either pointers or reference to swap values.
\end{itemize}
\begin{lstlisting}[caption={Swapping Functions Using Pointer and Reference},escapechar=$]
#include <iostream>
using namespace std;

// Swapping function using pointers.
void Swap(int* x, int* y)
{
	int temp = *x;
	*x = *y;
	*y = temp;
}
// Swapping function using reference.
void Swap(int& x, int& y)
{
	int temp = x;
	x = y;
	y = temp;
}

int main()
{
	int A = 3;
	int B = 6;
	cout << "Initial values: A = " << A << ", B = " << B << endl;

	Swap(&A, &B);
	cout << "After swapping (using pointers): A = " << A << ", B = " << B << endl;

	Swap(A, B);
	cout << "After swapping again (using reference): A = " << A << ", B = " << B << endl;
	return 0;
}
\end{lstlisting}
\section{Linear Search: Sorted and Unsorted Data}
\begin{itemize}
\item Assuming data is sorted in ascending order, we can terminate the search as unsuccessful if match is not found and we get to values greater than our key.
\item If data is unsorted then we need to search the whole list for a match.
\item Conclusion: If data is sorted, it takes less effort to search.
\end{itemize}
\begin{lstlisting}[caption={Linear Search for sorted and unsorted data},escapechar=$]
#include <iostream>
#include <cstdlib>
using namespace std;

// Sorted (ascending)
bool SearchSorted(int* Data, int Size, int Key)
{
	for (int i=0; i<Size; i++)
	{
		// If match found return true.
		if (Data[i] == Key)
			return true;
		// If values are greater than key then no need to search further.
		if (Data[i] > Key)
			return false;
	}
	return false; // If we ever get here, search was unsuccessful.
}
// Unsorted.
bool SearchUnSorted(int* Data, int Size, int Key)
{
	for (int i=0; i<Size; i++)
	{
		// If match found return true.
		if (Data[i] == Key)
			return true;
	}
	return false; // If we ever get here, search was unsuccessful.
}
int main()
{
	const int Size = 10;
	int Data[Size];
	for (int i=0; i<10; i++)
	{
		Data[i] = rand()%10;
		cout << "Data[i] = " << Data[i] << endl;
	}

	int Key;
	cout << "Enter a value to search: ";
	cin >> Key;

	if (SearchUnsorted(Data, Size, Key) == true)
		cout << "Search successful!" << endl;

	return 0;
}
\end{lstlisting}
\section{Bubblesort}
\begin{itemize}
\item Bubblesort is a sorting algorithm.
\item Example: Sorting an array in ascending order.
\item Larger values are `bubbled' to the top.
\end{itemize}
\begin{lstlisting}[caption={Bubble Sort}]
#include <cstdlib>
#include <iostream>
using namespace std;

int main()
{
	const int Size = 5;
	int data[Size];

	// Storing random numbers in range 0-99 in data[]
	for (int i=0; i<Size; i++)
		data[i] = rand() % 100;
		
	// Displaying data before sorting.
	cout << "Data before sorting: ";
	for (int i=0; i<Size; i++)
		cout << data[i] << " ";
	cout << endl;

	// Bubblesort
	for (int c=0; c<Size; c++)
	{
		int currentsize = Size-c;
		cout << "===== Iteration no: " << c+1 << " ====" << endl;
		for (int i=0; i<currentsize-1; i++)
		{
			// Sorting condition.
			if (data[i] > data[i+1])
			{
				// Swap.
				int temp = data[i];
				data[i] = data[i+1];
				data[i+1] = temp;
			}
			cout << "Array after swapping no: " << i << endl;
			for (int z=0; z<Size; z++)
				cout << data[z] << " ";
			cout << endl;
		}
		cout << "===================================" << endl;
	}

	// Displaying data after sorting.
	cout << "Data after sorting: ";
	for (int i=0; i<Size; i++)
		cout << data[i] << " ";
	cout << endl;
	
	return 0;
}
\end{lstlisting}
%\nocite{*}
%\bibliographystyle{plain}
%\bibliography{OOPref}
\end{document}
