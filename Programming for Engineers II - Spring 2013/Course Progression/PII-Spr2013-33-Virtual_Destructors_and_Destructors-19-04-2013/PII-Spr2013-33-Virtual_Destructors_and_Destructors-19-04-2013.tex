\documentclass[12pt,a4paper]{article}

% Margins.
\setlength{\oddsidemargin}{0in}
\setlength{\evensidemargin}{0in}
\setlength{\headheight}{12pt}
\setlength{\headsep}{42pt}
\setlength{\topmargin}{-54pt}
\setlength{\textwidth}{6.5in}
\setlength{\textheight}{10in}

\usepackage{amsmath}
\usepackage{float}
\usepackage{graphicx}
\usepackage[hyphens]{url}
\usepackage{hyperref}	% Clickable links to figures, references and urls.
\usepackage{enumerate}
\usepackage{datetime}

% Drawing.
\usepackage{pgf}
\usepackage{tikz}

% Listings for formatting code.
\usepackage{listings}
\usepackage{textcomp}
% General options.
\lstset{breaklines=true, basicstyle=\small\ttfamily, tabsize=4, numbers=left, stepnumber=1, frame=single, showstringspaces=false, upquote=true}
% C++ specific high-lighting. Comments are 50/50 shades of green/black and strings coloured with 60/40 red/black mixture.
\lstset{language=[ISO]C++, commentstyle=\color{green!50!black}, keywordstyle=\color{blue}, stringstyle=\color{red!60!black}}

%opening
\title{\vspace{-2cm}Programming for Engineers II\\Class 33\\Polymorphism\\Virtual Destructors\\Virtual Inheritance to Resolve Ambiguities}
\author{Attique Dawood}
\date{April 19, 2013\\[0.2cm] Last Modified: \today, \currenttime}
\begin{document}
\maketitle
\section{Announcements}
\begin{itemize}
\item Reminder: Quiz on Saturday, 20-04-2013. Topics are linked list and polymorphism.
\item Next ABQ is based on linked list and polymorphism. Detailed implementation may be required.
\item You may be required to modify list functions, like, instead of adding a node at end, insert node at beginning.
\item Practice with linked lists.
\end{itemize}
\section{Revision}
\begin{itemize}
\item Polymorphism: Don't look at type of pointer, look at type of object it points.
\item Polymorphism is achieved through virtual functions.
\item Abstract classes. Making a virtual function, pure virtual results in an abstract class. Objects of abstract class cannot be instantiated.
\end{itemize}
\section{Virtual Destructors}
\begin{itemize}
\item Remember when an object is created, base class constructor is called first and then derived class constructor. \item When the object is destroyed destructors are called in reverse order, i.e., derived class first and then base class.
\item If we're using the base class pointer to de--allocate memory then our compiler will think the object is of base class and only call base class destructor. We don't want this. We want to call the destructor of derived class.
\item When using base class pointer, we need to make the destructors virtual so derived class destructors are called first.
\end{itemize}
\section{Virtual Inheritance: Resolving Ambiguities in Multiple Inheritance}
\begin{itemize}
\item Consider base class A with a data member named \verb|int data|.
\item Classes B and C are both derived individually from class A and both contain a copy of \verb|data|.
\item Class D is derived from both B and C containing \verb|A::data| and \verb|B::data|.
\item Accessing \verb|data| directly in D will create an ambiguity.
\item By specifying \verb|virtual| when deriving B and C from A, any classes further derived from B and C will only contain a single copy of \verb|data|.
\end{itemize}
\begin{lstlisting}[caption={Ambiguity in Multiple Inheritance}]
#include <iostream>
using namespace std;

class A
{
	public:
	int Data;
};
class B: public A
{
};
class C: public A
{
};
class D: public B, public C
{
	void Display()
	{
		cout << Data << endl; // Error: which data to display? B::data or C::data
	}
};
\end{lstlisting}
\begin{lstlisting}[caption={Resolving Ambiguity in Multiple Inheritance},escapechar=$]
#include <iostream>
using namespace std;

class A
{
	public:
	int Data;
};
class B: $\textcolor{red}{virtual}$ public A
{
};
class C: $\textcolor{red}{virtual}$ public A
{
};
class D: public B, public C
{
	void Display()
	{
		cout << Data << endl; // No ambiguity, there is only one data which is A::data.
	}
};
\end{lstlisting}
%\nocite{*}
%\bibliographystyle{plain}
%\bibliography{OOPref}
\end{document}
