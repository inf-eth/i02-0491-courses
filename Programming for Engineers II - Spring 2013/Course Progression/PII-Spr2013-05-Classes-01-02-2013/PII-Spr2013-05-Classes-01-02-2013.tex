\documentclass[12pt,a4paper]{article}

% Margins.
\setlength{\oddsidemargin}{0in}
\setlength{\evensidemargin}{0in}
\setlength{\headheight}{12pt}
\setlength{\headsep}{42pt}
\setlength{\topmargin}{-54pt}
\setlength{\textwidth}{6.5in}
\setlength{\textheight}{10in}

\usepackage{amsmath}
\usepackage{float}
\usepackage{graphicx}
\usepackage[hyphens]{url}
\usepackage{hyperref}	% Clickable links to figures, references and urls.

% Drawing.
\usepackage{pgf}
\usepackage{tikz}

% Listings for formatting code.
\usepackage{listings}
\usepackage{textcomp}
% General options.
\lstset{breaklines=true, basicstyle=\small\ttfamily, tabsize=4, numbers=left, stepnumber=1, frame=single, showstringspaces=false, upquote=true}
% C++ specific high-lighting. Comments are 50/50 shades of green/black and strings coloured with 60/40 red/black mixture.
\lstset{language=[ISO]C++, commentstyle=\color{green!50!black}, keywordstyle=\color{blue}, stringstyle=\color{red!60!black}}

%opening
\title{\vspace{-2cm}Programming for Engineers II\\Class 05\\Classes\\Complex Number Program}
\author{Attique Dawood}
\date{February 01, 2013\\[0.2cm] Last Modified: \today}
\begin{document}
\maketitle
\section{Announcements}
\begin{itemize}
\item Quiz?
\end{itemize}
\section{Revision}
\begin{itemize}
\item Complex number refinements from procedural to OOP.
\item Introduction to classes. Similar to struct but comes with more bells and whistles.
\item Classes and objects. A class--object relationship is similar to common noun and proper noun.
\item Objects are physical whereas class is abstract. Person is a class and a particular person like Einstein is object of that class.
\item Keep in mind, in OOP we are more interested in objects rather than anything else. We need to first identify objects in a particular problem.
\end{itemize}
\section{Data Encapsulation, Data Hiding and Access Restriction}
\begin{itemize}
\item Encapsulate means to group something and put it in a container. Structs and classes both provide data encapsulation capability.
\item Data hiding means to hide or restrict outside world from accessing sensitive data. In struct all data is visible everywhere. Classes provide option to make data \verb|private| to hide it from global access.
\item Classes can provide `controlled' access to its private data to other trusted classes or entities. This is known as access control. Member functions provide means to implement access restriction.
\end{itemize}
\section{Defining Member Functions of a Class}
\begin{itemize}
\item Variable \textbf{declaration} and \textbf{initialisation}.
\item Function \textbf{prototype/declaration} and function \textbf{definition}.
\item Class \textbf{definition} and \textbf{implementation}
\item Member functions of a class are its property. Member functions are the only way to communicate with object(s) of a class.
\item Member functions can either be defined inside or outside the class. Keep in mind, member functions of a class can freely access any data (public or private) of that class.
\item Scope resolution operator \verb|::| can be used to define class functions outside the class definition.
\end{itemize}
\section{Working With Complex Numbers and Matrices}
\begin{itemize}
\item Create a program that can add and subtract complex numbers.
\item Now create a program than can add and subtract matrices. Assume 1D matrix of size $1\times 3$.
\item Add the functionality to multiply with a scalar.
\item Add functionality to multiply and divide two complex numbers or matrices.
\item Create a matrix of complex numbers.
\item Compare how procedural, structured and OOP implementation differ.
\end{itemize}
\section{Procedural Approach}
\begin{lstlisting}[caption={Simple prodecural approach to complex number addition}]
#include <iostream>
using std::cin;
using std::cout;
using std::endl;
int main()
{
	// First complex number.
	float complex1real;
	float complex1img;
	// Second complex number.
	float complex2real;
	float complex2img;
	// Resultant complex number.
	float complex3real;
	float complex3img;
	
	// First complex input.
	cout << "Enter first complex: ";
	cin >> complex1real >> complex1img;
	// Second complex input.
	cout << "Enter second complex: ";
	cin >> complex2real >> complex2img;
	
	// Addition.
	complex3real = complex1real + complex2real;
	complex3img = complex1img + complex2img;
	
	// Result.
	cout << "Result is " << complex3real << " + j" << complex3img << endl;
	
	return 0;
}
\end{lstlisting}
\begin{lstlisting}[caption={Simple prodecural approach to matrix addition}]
#include <iostream>
using std::cin;
using std::cout;
using std::endl;
int main()
{
	// First matrix.
	float Mat1[4];
	// Second matrix.
	float Mat2[4];
	// Resultant complex number.
	float Mat3[4];
	
	// First matrix input.
	cout << "Enter first matrix: ";
	for (int i=0; i<4; i++)
		cin >> Mat1[i];
	// Second matrix input.
	cout << "Enter second matrix: ";
	for (int i=0; i<4; i++)
		cin >> Mat2[i];
	
	// Addition.
	for (int i=0; i<4; i++)
		Mat3[i] = Mat1[i] + Mat2[i];
	
	// Result.
	cout << "Result is: " << endl;
	for (int i=0; i<4; i++)
		cout << "Mat3[" << i << "] = " << Mat3[i] << endl;
	
	return 0;
}
\end{lstlisting}
\section{Procedural Approach Using Functions}
\begin{lstlisting}[caption={Procedural approach to complex number addition using functions}]
#include <iostream>
using std::cin;
using std::cout;
using std::endl;

// Input function prototype.
void InputComplex(float &, float &);
// Input function definition.
void InputComplex(float &real, float &img)
{
	cout << "Enter complex number: ";
	cin >> real >> img;
}

// Addition function prototype.
void AddComplex(float, float, float, float, float &, float &);
// Addition function definition.
void AddComplex(float c1r, float c1i, float c2r, float c2i, float &c3r, float &c3i)
{
	c3r = c1r + c2r;
	c3i = c1i + c2i;
}

// Display function prototype.
void DisplayComplex(float, float);
// Display function definition.
void DisplayComplex(float real, float img)
{
	if (img < 0.f)
		cout << real << " - j" << -1.f*img << endl;
	else
		cout << real << " + j" << img << endl;
}

int main()
{
	// First complex number.
	float complex1real;
	float complex1img;
	// Second complex number.
	float complex2real;
	float complex2img;
	// Resultant complex number.
	float complex3real;
	float complex3img;
	
	// First complex input.
	InputComplex(complex1real, complex1img);
	// Second complex input.
	InputComplex(complex2real, complex2img);
	
	// Addition.
	AddComplex(complex1real, complex1img, complex2real, complex2img, complex3real, complex3img);
	
	// Result.
	cout << "Result is: ";
	DisplayComplex(complex3real, complex3img);
	
	return 0;
}
\end{lstlisting}
\section{\texttt{struct}}
\begin{itemize}
\item Can create new data types from basic types.
\item Types can be anything int, float, double. Different types can be mixed in a \texttt{struct}.
\item Group similar data together.
\item Can also include other structs (matrix of complex numbers?).
\end{itemize}
\begin{lstlisting}[caption={\texttt{struct}}]
struct complex
{
	float real;
	float img;
};

int main()
{
	complex google;
	complex facebook;
	google.real = 1.f;
	google.img = 0.f;
	facebook.real = google.real + 34.f;
	facebook.img = 3.3f * google.img - 2.2f;
	
	return 0;
}
\end{lstlisting}
\section{Structured Approach}
\begin{lstlisting}[caption={Using \texttt{struct} to define complex number and add them using functions}]
#include <iostream>
using std::cin;
using std::cout;
using std::endl;

struct complex
{
	float real;
	float img;
};

// Input function prototype.
void InputComplex(complex &);
// Input function definition.
void InputComplex(complex &c)
{
	cout << "Enter complex number: ";
	cin >> c.real >> c.img;
}

// Addition function prototype.
void AddComplex(complex, complex, complex &);
// Addition function definition.
void AddComplex(complex c1, complex c2, complex &c3)
{
	c3.real = c1.real + c2.real;
	c3.img = c1.img + c2.img;
}

// Display function prototype.
void DisplayComplex(complex);
// Display function definition.
void DisplayComplex(complex c)
{
	if (c.img < 0.f)
		cout << c.real << " - j" << -1.f*c.img << endl;
	else
		cout << c.real << " + j" << c.img << endl;
}

int main()
{
	// First complex number.
	complex complex1;
	// Second complex number.
	complex complex2;
	// Resultant complex number.
	complex complex3;
	
	// First complex input.
	InputComplex(complex1);
	// Second complex input.
	InputComplex(complex2);
	
	// Addition.
	AddComplex(complex1, complex2, complex3);
	
	// Result.
	cout << "Result is: ";
	DisplayComplex(complex3);
	
	return 0;
}
\end{lstlisting}
\section{\texttt{class} Definition in C++}
\begin{lstlisting}[caption={\texttt{class} definition}]
class classname
{
	private:
	// Data and/or functions hidden from outside world.
	public:
	// Data and/or functions used to interact with or manipulate the object by outside world.	
};
\end{lstlisting}
\section{Classes and Objects}
\begin{itemize}
\item class is abstract, object is physical. Think common noun and proper noun.
\item Objects are physical manifestation of class.
\item Person is a class; a particular person like Stephen Hawking would be object of that class.
\item Similarly complex is a class whereas a particular complex number $2-j21$ is an object.
\item In C++, class is a datatype. Variables of that data type are called objects.
\end{itemize}
\section{Object--Oriented Approach}
\begin{lstlisting}[caption={Object--oriented approach to add complex numbers}]
#include <iostream>
using std::cin;
using std::cout;
using std::endl;

class complex
{
	private:
	float real;
	float img;
	
	public:
	void Input();
	void Display();
	complex Add(complex);
	complex DivideBy(complex);
};

// Class functions can be defined inside or outside the class.
// Here, functions are defined outside the class.
// Notice the `::' operator.
void complex::Input()
{
	cout << "Enter complex number: ";
	cin >> real >> img;
}
void complex::Display()
{
	if (img < 0.f)
		cout << real << " - j" << -1.f*img << endl;
	else
		cout << real << " + j" << img << endl;
}
complex complex::Add(complex c)
{
	complex result;
	result.real = real + c.real;
	result.img = img + c.img;
	
	return result;
}

int main()
{
	// First complex number.
	complex complex1;	// complex1 is an object of complex class.
	// Second complex number.
	complex complex2;
	// Resultant complex number.
	complex complex3;
	
	// First complex input.
	complex1.Input();
	// Second complex input.
	complex2.Input();
	
	// Addition.
	complex3 = complex1.Add(complex2);
	
	// Result.
	cout << "Result is: ";
	complex3.Display();
	
	return 0;
}
\end{lstlisting}
\begin{itemize}
\item What if we're dealing with matrices?
\item Function names won't change. Only functionality. User not concerned with how things are done. He/she only asks objects to do his or her bidding.
\end{itemize}
\nocite{*}
\bibliographystyle{plain}
\bibliography{OOPref}
\end{document}
