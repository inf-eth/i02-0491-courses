\documentclass[12pt,a4paper]{article}

% Margins.
\setlength{\oddsidemargin}{0in}
\setlength{\evensidemargin}{0in}
\setlength{\headheight}{12pt}
\setlength{\headsep}{42pt}
\setlength{\topmargin}{-54pt}
\setlength{\textwidth}{6.5in}
\setlength{\textheight}{10in}

\usepackage{amsmath}
\usepackage{float}
\usepackage{graphicx}
\usepackage[hyphens]{url}
\usepackage{hyperref}	% Clickable links to figures, references and urls.
\usepackage{enumerate}
\usepackage{datetime}

% Drawing.
\usepackage{pgf}
\usepackage{tikz}

% Listings for formatting code.
\usepackage{listings}
\usepackage{textcomp}
% General options.
\lstset{breaklines=true, basicstyle=\small\ttfamily, tabsize=4, numbers=left, stepnumber=1, frame=single, showstringspaces=false, upquote=true}
% C++ specific high-lighting. Comments are 50/50 shades of green/black and strings coloured with 60/40 red/black mixture.
\lstset{language=[ISO]C++, commentstyle=\color{green!50!black}, keywordstyle=\color{blue}, stringstyle=\color{red!60!black}}

%opening
\title{\vspace{-2cm}Programming for Engineers II\\Class 38\\File Handling\\Modes and Flags\\Reading/Writing Objects}
\author{Attique Dawood}
\date{April 26, 2013\\[0.2cm] Last Modified: \today, \currenttime}
\begin{document}
\maketitle
\section{Announcements}
\begin{itemize}
\item ABQ and an additional quiz on Friday, 4-05-2013.
\item ABQ is from linked list.
\item Quiz is from topics covered since last quiz like, searching, sorting, file handling and recursion.
\end{itemize}
\section{File Modes}
Modes are options that can be specified when opening a file. Options are specified as the second argument of \verb|open()| function. Options can be combined with the `OR' operator \verb\'|'\. A few examples are given below:
\begin{table}[htb]
\begin{center}
\vspace{0.3cm}
	\begin{tabular}{lp{9cm}}
	\hline \hline
		\textbf{Mode} \rule{0pt}{2.6ex} & \textbf{Description}\\
		\hline
		\verb|ios::in| \rule{0pt}{2.6ex} & Read mode\\
		\verb|ios::out| \rule{0pt}{2.6ex} & Write mode\\
		\verb|ios::binary| \rule{0pt}{2.6ex} & Binary mode operation\\
		\verb|ios::app| \rule{0pt}{2.6ex} & Append mode. Set the stream's position indicator to the end of the stream before each output operation. \cite{Ref:SOF-ios-ate-app}, \cite{Ref:CPP-ios-ate-app}\\
		\verb|ios::ate| \rule{0pt}{2.6ex} & At end: Sets the stream's position indicator to the end of the stream on opening.\\
		\verb|ios::create| \rule{0pt}{2.6ex} & Create file if it does not exist.\\
		\verb|ios::nocreate| \rule{0pt}{2.6ex} & If file does not exist, do not create new one. File.open() will fail.\\
		\verb|ios::trunc| \rule{0pt}{2.6ex} & Existing file contents are discarded. Destroys existing data in file. Use with caution!\\
		\verb|File.open("abc.txt", ios::in);| \rule{0pt}{2.6ex} & Open file for reading. File must exist.\\
		\verb|File.open("abc.txt", ios::out);| \rule{0pt}{2.6ex} & Open file for writing. File will be created if it doesn't exist. Existing file will be overwritten.\\
		\verb\..., ios::out|ios::app);\ \rule{0pt}{2.6ex} & Open file for writing. File will be created if it doesn't exist. Existing data will be preserved. Data will be written at the end of file.\\
		\verb\..., ios::out|ios::binary);\ \rule{0pt}{2.6ex} & Open file for writing in binary mode. File will be created if it doesn't exist. Existing file will be overwritten.\\
		\verb\ios::out|ios::binary|ios::app\ \rule{0pt}{2.6ex} & Open file for writing in binary mode. File will be created if it doesn't exist. Existing data will be preserved. Data will be written at the end of file.\\
	\hline \hline
	\end{tabular}
\end{center}
\label{FileModes}
\caption{File modes}
\end{table}
\newline
\noindent\textbf{Note: }An object of class \verb|fstream| will have \verb$ios::in|ios::out$ mode if no mode is specified. Similarly, ifstream and ofstream objects have \verb|ios::in| and \verb|ios::out| modes by default. For example, \verb|fstream file("abc.txt");|.
\section{File Handling Flags}
Flags are boolean variables that indicate the state of file and effect of different operations. The flags can be checked with functions given in table below~\cite{Ref:CPP-ios-flags}:
\begin{table}[H]
\begin{center}
\vspace{0.3cm}
	\begin{tabular}{lp{9cm}}
	\hline \hline
		\textbf{Function} \rule{0pt}{2.6ex} & \textbf{Description}\\
		\hline
		\verb|eof()| \rule{0pt}{2.6ex} & Returns \verb|true| if end of file was reached during last read operation.\\
		\verb|fail()| \rule{0pt}{2.6ex} & Returns \verb|true| if last read operation failed.\\
		\verb|bad()| \rule{0pt}{2.6ex} & Returns \verb|true| if an error is encountered due to failure of input/output operation.\\
		\verb|good()| \rule{0pt}{2.6ex} & Returns \verb|true| if none of the above errors have encountered and everything is good. Universal check.\\
		\verb|is_open()| \rule{0pt}{2.6ex} & Returns \verb|true| if file was successfully opened.\\
		\hline \hline
	\end{tabular}
\end{center}
\label{Flags}
\caption{Flags}
\end{table}
\begin{lstlisting}[caption={Reading a line from text file with error checking}]
#include <iostream>
#include <fstream>		// File handling library.
using namespace std;
int main()
{
	fstream InFile;	// File handling object.

	// Open "abc.txt" for reading (ios::in).
	InFile.open("abc.txt", ios::in);

	// Checking if file opened successfully.
	if (InFile.is_open() == true)
	{
		cout << "File opened successfully." << endl;

		InFile.getline(Line, 50); // Reading a line of text from file.

		// Checking if data was read successfully.
		if (InFile.fail() == true)
			cout << "Some error occured during read operation." << endl;
		else
		{
			cout << "Data read is..." << endl;
			cout << Line << endl;
		}

		// Checking end of file.
		if (InFile.eof() == true)
			cout << "End of file was encountered during last read operation. " << endl;

		InFile.close();	// Close file.
	}
	else
	{
		cout << "Error opening file." << endl;
		return -1;
	}

	return 0;
}
\end{lstlisting}
\begin{lstlisting}[caption={Reading a line from text file}]
#include <iostream>
#include <fstream>		// File handling library.
using namespace std;
int main()
{
	char Line[50];

	fstream InFile;	// File handling object.

	// Open "abc.txt" for reading (ios::in).
	InFile.open("abc.txt", ios::in);
	InFile.getline(Line, 50);
	InFile.close();	// Close file.
	
	cout << "Data read is..." << endl;
	cout << Line << endl;

	return 0;
}
\end{lstlisting}
\begin{lstlisting}[caption={Reading a whole text file line--by--line}]
#include <iostream>
#include <fstream>		// File handling library.
using namespace std;
int main()
{
	char Line[50];

	fstream InFile;	// File handling object.

	// Open "abc.txt" for reading (ios::in).
	InFile.open("abc.txt", ios::in);
	while(InFile)
	{
		InFile.getline(Line, 50);
		cout << "Read: " << Line << endl;
	}
	InFile.close();	// Close file.

	return 0;
}
\end{lstlisting}
\section{Writing to Binary File}
\begin{lstlisting}[caption={Writing to binary file}]
#include <fstream>		// File handling library.
using namespace std;

int main()
{
	fstream OutFile;	// File handling object.

	int x = 7;
	double d = -1.32489;
	char c = '$';

	// Open "abc.txt" for writing in binary mode (ios::out|ios::binary). This will overwrite existing file.
	OutFile.open("abc.txt", ios::out|ios::binary);
	OutFile.write((char*)&x, sizeof(int));
	OutFile.write((char*)&d, sizeof(double));
	OutFile.write((char*)&c, sizeof(char));
	OutFile.close();	// Close file.

	return 0;
}
\end{lstlisting}
\section{Reading from Binary File}
\begin{lstlisting}[caption={Reading from Binary file}]
#include <iostream>
#include <fstream>		// File handling library.
using namespace std;

int main()
{
	int x;
	double d;
	char c;

	fstream InFile;	// File handling object.

	// Open "abc.txt" for reading in binary mode (ios::in|ios::binary).
	InFile.open("abc.txt", ios::in|ios::binary);
	InFile.read((char*)&x, sizeof(int));
	InFile.read((char*)&d, sizeof(double));
	InFile.read((char*)&c, sizeof(char));
	InFile.close();	// Close file.

	cout << "x = " << x << endl;
	cout << "d = " << d << endl;
	cout << "c = " << c << endl;

	return 0;
}
\end{lstlisting}
\section{Writing Object to Binary File}
\begin{lstlisting}[caption={Writing an object to binary file}]
#include <iostream>
#include <fstream>		// File handling library.
using namespace std;

class StudentInfo
{
	public:
	char Name[20];
	int Age;
	float CGPA;
	char Gender;
};

int main()
{
	StudentInfo S;
	cout << "Enter student information: " << endl;
	cout << "Name: ";
	cin.getline(S.Name, 20);
	cout << "Age: ";
	cin >> S.Age;
	cout << "CGPA: ";
	cin >> S.CGPA;
	cout << "Gender (M/F): ";
	cin >> S.Gender;

	fstream OutFile;	// File handling object.

	// Open "abc.txt" for writing in binary mode (ios::out|ios::binary). This will overwrite existing file.
	OutFile.open("abc.txt", ios::out|ios::binary);
	OutFile.write((char*)&S, sizeof(StudentInfo));
	OutFile.close();	// Close file.

	return 0;
}
\end{lstlisting}
\newpage
\section{Reading Object from Binary File}
\begin{lstlisting}[caption={Reading an object from binary file}]
#include <iostream>
#include <fstream>		// File handling library.
using namespace std;
class StudentInfo
{
	public:
	char Name[20];
	int Age;
	float CGPA;
	char Gender;
};

int main()
{
	StudentInfo S;

	fstream InFile;	// File handling object.

	// Open "abc.txt" for reading in binary mode (ios::in|ios::binary).
	InFile.open("abc.txt", ios::in|ios::binary);
	InFile.read((char*)&S, sizeof(StudentInfo));
	InFile.close();	// Close file.

	switch (S.Gender)
	{
		case 'M':
			cout << "Name: Mr. " << S.Name << endl;
			break;
		case 'F':
			cout << "Name: Ms. " << S.Name << endl;
			break;
	}
	cout << "Age: " << S.Age << endl;
	cout << "CGPA: " << S.CGPA << endl;

	return 0;
}
\end{lstlisting}
%\nocite{*}
\bibliographystyle{plain}
\bibliography{OOPref}
\end{document}
