\documentclass[12pt,a4paper]{article}

% Margins.
\setlength{\oddsidemargin}{0in}
\setlength{\evensidemargin}{0in}
\setlength{\headheight}{12pt}
\setlength{\headsep}{42pt}
\setlength{\topmargin}{-54pt}
\setlength{\textwidth}{6.5in}
\setlength{\textheight}{10in}

\usepackage{amsmath}
\usepackage{float}
\usepackage{graphicx}
\usepackage[hyphens]{url}
\usepackage{hyperref}	% Clickable links to figures, references and urls.

% Drawing.
\usepackage{pgf}
\usepackage{tikz}

% Listings for formatting code.
\usepackage{listings}
\usepackage{textcomp}
% General options.
\lstset{breaklines=true, basicstyle=\small\ttfamily, tabsize=4, numbers=left, stepnumber=1, frame=single, showstringspaces=false, upquote=true}
% C++ specific high-lighting. Comments are 50/50 shades of green/black and strings coloured with 60/40 red/black mixture.
\lstset{language=[ISO]C++, commentstyle=\color{green!50!black}, keywordstyle=\color{blue}, stringstyle=\color{red!60!black}}

%opening
\title{\vspace{-2cm}Programming for Engineers II\\Class 13\\Copy Constructor and Assignment Operator\\Overloading Arithmetic Operators}
\author{Attique Dawood}

\begin{document}
\maketitle
\section{Announcements}
\begin{itemize}
\item Paper pattern is similar to ABQ. Class diagram, writing and understanding code, constructor/destructor and operator overloading.
\item Today's topics taken from \cite{Ref:CPP-Copy-Constructor-and-Assignment-Operator} and \cite{Ref:CPP-Operator-Overloading}.
\end{itemize}
\section{Revision}
\begin{itemize}
\item Memory leaks. Using NULL pointer for safety.
\item A destructor, very much like constructor, is a special function that is automatically called every time an object is destroyed.
\item Constructor and destructor calls when dynamically creating objects.
\end{itemize}
\section{Copy Constructor}
\begin{itemize}
\item Default constructor and user--defined constructor. If you do not define a default constructor then compiler provides one implicitly.
\item Copy constructor can make copies of object at creation. If user does not define a copy constructor then compiler provides one implicitly.
\item Copy constructor is used to create an object when another object is passed as constructor argument.
\item Copy constructor does a member--wise copy of object elements (data members).
\end{itemize}
\begin{lstlisting}[caption={Copy Constructor}]
// Notice, complex class doesn't define any constructors.
// 1. Default constructor will be provided by compiler.
// Complex(){}
// 2. Also, copy constructor will be provided by compiler.
// Prototypes for copy constructor.
// Complex(Complex &);
// Complex(const Complex &);

class Complex
{
	private:
	float real;
	float img;
};

int main()
{
	Complex x;
	Complex y(x);
	
	return 0;
}
\end{lstlisting}
\section{Assignment Operator =}
\begin{itemize}
\item This is another facility provided by compiler if not explicitly defined by user.
\item Assignment operator is just a member function.
\item An assignment operator essentially does a member--wise copy of elements.
\item Difference between copy constructor and assignment operator? Copy constructor can only be called at object creation because it's a constructor.
\end{itemize}
\begin{lstlisting}[caption={Calling Assignment Operator}]
class Complex
{
	private:
	float real;
	float img;
};

int main()
{
	Complex x, y;
	Complex z(x);  // Copy constructor called.
	Complex y = x; // Assignment operator called. y was already created using default constructor.
	
	return 0;
}
\end{lstlisting}
\begin{lstlisting}[caption={Defining Assignment Operator}]
class Complex
{
	private:
	float real;
	float img;
	public:
	Complex& operator=(const Complex& rhs)
	{
		real = rhs.real;
		img = rhs.img;
		return (*this);
	}	
};

int main()
{
	Complex x, y;
	Complex z(x);  // Copy constructor called.
	Complex y = x; // Assignment operator called.
	
	return 0;
}
\end{lstlisting}
\section{Putting it Together}
Compiler will generate following functions if not defined by user:
\begin{enumerate}
\item A default constructor with empty body if user does not provide any constructor.
\item A copy constructor that performs member--wise copy.
\item Assignment operator for member--wise copy.
\end{enumerate}
\begin{lstlisting}[caption={User and Compiler Code}]
// User code for class.
class Complex
{
	private:
	float real;
	float img;
};

// Compiler code.
class Complex
{
	private:
	float real;
	float img;
	public:
	// Default constructor.
	Complex()
	{
	}
	// Copy constructor. Notice, member initialiser list.
	Complex(Complex& other): real(other.real), img(other.img)
	{
	}
	// Assignment operator.
	Complex& operator=(const Complex& rhs)
	{
		real = rhs.real;
		img = rhs.img;
		return (*this);  // 'this' is the pointer to calling object.
	}
};
\end{lstlisting}
\section{Operator Overloading}
\begin{itemize}
\item Arithmetic operators +, -, *, / are not provided by compiler and must be defined by user.
\item For our \verb|Complex| class compiler doesn't know how to perform addition, subtraction or multiplication between two objects (variables) of \verb|Complex| type.
\item Defining these operators as member functions of the class is known as ``operator overloading'', a somewhat misleading term.
\end{itemize}
\begin{lstlisting}[caption={Overloading Arithmetic Operators}]
#include <iostream>
using std::cout;
using std::endl;

class Complex
{
	private:
	float real;
	float img;
	public:
	Complex();
	Complex(float, float);
	Complex operator+ (const Complex&); // Addition operator.
	Complex operator/ (const Complex&); // Division operator.
	void Display();
};

// Default constructor.
Complex::Complex(): real(0.f), img(0.f)
{
}
// User-defined constructor.
Complex::Complex(float preal, float pimg): real(preal), img(pimg)
{
}
// Addition operator.
Complex Complex::operator+ (const Complex& rhs)
{
	Complex result;
	result.real = real + rhs.real;
	result.img = img + rhs.img;
	return result;
}
// Division operator.
Complex Complex::operator/ (const Complex& rhs)
{
	Complex result;
	float a, b, c, d;
	a = real;
	b = img;
	c = rhs.real;
	d = rhs.img;

	float den = c*c + d*d; // denominator.
	result.real = (a*c+b*d)/den;
	result.img = (b*c-a*d)/den;

	return result;
}
// Display function.
void Complex::Display()
{
	if (img < 0.f)
		cout << real << " - j" << -1.f*img << endl;
	else
		cout << real << " + j" << img << endl;
}

int main()
{
	Complex x(1.1f, -2.7f);
	Complex y(2.1f, 3.3f);
	Complex z;

	z = x+y;
	cout << "x+y = ";
	z.Display();

	z = x/y;
	cout << "x/y = ";
	z.Display();

	z = y/x;
	cout << "y/x = ";
	z.Display();

	return 0;
}
\end{lstlisting}
%\nocite{*}
\bibliographystyle{plain}
\bibliography{OOPref}
\end{document}
