\documentclass[12pt,a4paper]{article}

% Margins.
\setlength{\oddsidemargin}{0in}
\setlength{\evensidemargin}{0in}
\setlength{\headheight}{12pt}
\setlength{\headsep}{42pt}
\setlength{\topmargin}{-54pt}
\setlength{\textwidth}{6.5in}
\setlength{\textheight}{10in}

\usepackage{amsmath}
\usepackage{float}
\usepackage{graphicx}
\usepackage[hyphens]{url}
\usepackage{hyperref}	% Clickable links to figures, references and urls.
\usepackage{enumerate}
\usepackage{datetime}

% Drawing.
\usepackage{pgf}
\usepackage{tikz}

% Listings for formatting code.
\usepackage{listings}
\usepackage{textcomp}
% General options.
\lstset{breaklines=true, basicstyle=\small\ttfamily, tabsize=4, numbers=left, stepnumber=1, frame=single, showstringspaces=false, upquote=true}
% C++ specific high-lighting. Comments are 50/50 shades of green/black and strings coloured with 60/40 red/black mixture.
\lstset{language=[ISO]C++, commentstyle=\color{green!50!black}, keywordstyle=\color{blue}, stringstyle=\color{red!60!black}}

%opening
\title{\vspace{-2cm}Programming for Engineers II\\Class 36\\Recursion\\Recursive Binary Search}
\author{Attique Dawood}
\date{April 22, 2013\\[0.2cm] Last Modified: \today, \currenttime}
\begin{document}
\maketitle
\section{Announcements}
\begin{itemize}
\item None.
\end{itemize}
\section{Revision}
\begin{itemize}
\item Computation order.
\item Insertion sort.
\item Binary search.
\end{itemize}
\section{Recursion}
\begin{itemize}
\item When a function is defined in terms of itself.
\item Recursion must have a base case or termination condition. Typical examples are factorial and Fibonacci series.\\
\noindent\textbf{Factorial:} $n! = n\times(n-1)!$ for $n\geq 1$, where base case is $0! = 1! = 1.$\\
\noindent\textbf{Fibnonacci Series:} $fib(n) = fib(n-1) + fib(n-2)$ for $n\geq 2$, where $fib(0) = 0$ and $fib(1) = 1.$
\end{itemize}
\section{Insertion Sort}
\begin{lstlisting}[caption={Recursive Factorial},escapechar=$]
#include <iostream>
using namespace std;

// Factorial function.
int Factorial(int n)
{
	if (n==0||n==1)
		return 1; // Base cases.
	else
		return n*Factorial(n-1);
}

int main()
{
	int x;
	cout << "Enter n: ";
	cin >> x;
	cout << "n! = " << Factorial(x) << endl;

	return 0;
}
\end{lstlisting}
\begin{lstlisting}[caption={Fibonacci Series},escapechar=$]
#include <iostream>
using namespace std;

// Calculate nth term of Fibonacci Series.
int fib(int n)
{
	// Base cases.
	if (n==0)
		return 0;
	if (n==1)
		return 1;
	// If not base case go recursive.
	else
		return fib(n-1)+fib(n-2);
}

int main()
{
	int x;
	cout << "Enter number of terms: ";
	cin >> x;
	cout << "Fibonacci Series upto " << x << " terms is: ";
	for (int i=0; i<x; i++)
	{
		cout << fib(i) << " ";
	}
	cout << endl;

	return 0;
}
\end{lstlisting}
\section{Recursive Binary Search}
\begin{lstlisting}[caption={Binary Search}]
#include <cstdlib>
#include <iostream>
using namespace std;

bool BinarySearch(int* Data, int Size, int Key)
{
	int left = 0;
	int right = Size-1;
	int mid = (left+right)/2;

	cout << "================================" << endl;
	cout << "Data = ";
	for (int i=0; i<Size; i++)
		cout << Data[i] << " ";
	cout << endl;
	cout << "Size = " << Size << ", Key = " << Key << endl;
	cout << "left = " << left << ", right = " << right << ", mid = " << mid << endl;

	if (Data[mid] == Key)
		return true;
	if (Size == 1)
		return false;

	// Left half.
	if (Key < Data[mid])
		right = mid-1;
	else
		left = mid+1;

	return BinarySearch(Data+left, right-left+1, Key);
}

void BubbleSort(int* Data, int Size)
{
	// Bubblesort
	for (int c=0; c<Size; c++)
	{
		int currentsize = Size-c;
		for (int i=0; i<currentsize-1; i++)
		{
			// Sorting condition.
			if (Data[i] > Data[i+1])
			{
				// Swap.
				int temp = Data[i];
				Data[i] = Data[i+1];
				Data[i+1] = temp;
			}
		}
	}
}
int main()
{
	int Size;
	cout << "Enter size of data: ";
	cin >> Size;
	int* data = new int[Size];

	// Storing random numbers in range 0-99 in data[]
	for (int i=0; i<Size; i++)
		data[i] = rand() % 100;
		
	// Displaying data before sorting.
	cout << "Data before sorting: ";
	for (int i=0; i<Size; i++)
		cout << data[i] << " ";
	cout << endl;

	BubbleSort(data, Size);

	// Displaying data after sorting.
	cout << "Data after sorting: ";
	for (int i=0; i<Size; i++)
		cout << data[i] << " ";
	cout << endl;

	int Key;
	cout << "Enter key to search: ";
	cin >> Key;
	if (BinarySearch(data, Size, Key) == true)
		cout << "Key: " << Key << " was found!" << endl;

	return 0;
}
\end{lstlisting}
%\nocite{*}
%\bibliographystyle{plain}
%\bibliography{OOPref}
\end{document}
