\documentclass[12pt,a4paper]{article}

% Margins.
\setlength{\oddsidemargin}{0in}
\setlength{\evensidemargin}{0in}
\setlength{\headheight}{12pt}
\setlength{\headsep}{42pt}
\setlength{\topmargin}{-54pt}
\setlength{\textwidth}{6.5in}
\setlength{\textheight}{10in}

\usepackage{amsmath}
\usepackage{float}
\usepackage{graphicx}
\usepackage[hyphens]{url}
\usepackage{hyperref}	% Clickable links to figures, references and urls.
\usepackage{enumerate}
\usepackage{datetime}

% Drawing.
\usepackage{pgf}
\usepackage{tikz}

% Listings for formatting code.
\usepackage{listings}
\usepackage{textcomp}
% General options.
\lstset{breaklines=true, basicstyle=\small\ttfamily, tabsize=4, numbers=left, stepnumber=1, frame=single, showstringspaces=false, upquote=true}
% C++ specific high-lighting. Comments are 50/50 shades of green/black and strings coloured with 60/40 red/black mixture.
\lstset{language=[ISO]C++, commentstyle=\color{green!50!black}, keywordstyle=\color{blue}, stringstyle=\color{red!60!black}}

%opening
\title{\vspace{-2cm}Programming for Engineers II\\Class 19\\Static Data Members\\Explicit and Mutable}
\author{Attique Dawood}
\date{March 08, 2013\\[0.2cm] Last Modified: \today, \currenttime}
\begin{document}
\maketitle
\section{Announcements}
\begin{itemize}
\item Quiz.
\end{itemize}
Project announcements blatantly copied from lecture \# 20 (Sections A/B).
\subsection{Project}
\begin{itemize}
\item \textbf{Project = 8 Abs marks}
\item Instead of a take-home project, an \textbf{On-The-Spot (OTS) programming} exercise will be done, \textbf{individually} (no groups).
	\begin{enumerate}
	\item \textbf{IDEA:} Project idea will be explained on the spot (class or Lab will be decided later. Preference is in Lab. Based on availability of Labs it can be a Saturday).
	\item \textbf{DESIGN:} You will be given some time to design (write layman description, decide on classes, their connection i.e., Class Diagrams, Flow Charts etc)
	\item \textbf{IMPLEMENT:} Write error free executable code within specified time that achieves the required functionality.
	\item \textbf{TEST:} You will then develop test cases and test the code and when found valid, you will submit it to the teacher.
	\end{enumerate}
\item There ‘may’ be some extra time with some penalty.
\item Tentatively, OTS project will be held at the end of 2nd last week of classes, this semester. The week from \textbf{29th April to 4th May 2013}
\end{itemize}
\newpage
\subsection{Bonus Project}
\begin{itemize}
\item \textbf{Bonus Project = 10 Abs BONUS marks}
\item These bonus marks will primarily be awarded on the Final Exam paper. In case more bonus marks are earned than we can award, they will be adjusted in midterm marks etc.
\item In the bonus project, you can implement any OOP project of your own choice provided that:
\begin{itemize}
\item the idea is approved by the respective instructor. The latest to do this is \textbf{23rd March 2013}.
\item submit the idea in written form on SLATE to your respective instructor.
\end{itemize}
\item Grading:
\begin{enumerate}
\item A \textbf{5--10 minute presentation} may be required by a panel of examiners (both instructors and/or Lab engineer and TAs).
\item The panel will establish whether it is your own code or not.
\item You may be required to do at the spot minor changes in the code as required by the panel to demonstrate ownership and understanding of the code.
\item These presentations MUST be scheduled before \textbf{4th May 2013}. The student is responsible to arrange it.
\item These will be \textbf{individual} projects. Groups are not allowed.
\end{enumerate}
\end{itemize}
\subsection{Interesting Observations}
\begin{itemize}
\item ONLY way to prepare for the project is to start writing code with a keyboard, today.
\item If you start working on bonus project, you'll be prepared for the `actual' project.
\end{itemize}
\newpage
\section{Revision}
\begin{itemize}
\item Data conversion.
\item To do \verb|A=B| we would ideally want to overload assignment operator for class A that takes object of class B as argument, \verb|void operator=(B)|.
\item To do \verb|B=A| we would ideally want to overload assignment operator for class B that takes object of class A as argument, \verb|void operator=(A)|.
\item If we cannot change class B code and still want to do \verb|B=A| we'll define \verb|operator B()| in class A.
\end{itemize}
\section{\texttt{static} Variables}
\begin{itemize}
\item \textbf{WARNING!} We have loosely used the term `static' for non-dynamic variables. In the context of C/C++ as a language \verb|static| actually means something else.
\item Variables declared \verb|static| are only destroyed when the program ends.
\item Even if declared in a function, \verb|static| variables will continue to exist and take up memory until the end of program.
\item \verb|static| variable declaration/creation will only happen once.
\item There is no point making global variables \verb|static| because they already function as if they were \verb|static|... but nobody is stopping you from doing so.
\item One possible use of using static variables in a function is when you're passing a pointer or reference to a local variable. If the local variable isn't static it's pointer or reference is no longer useful outside the function.
\end{itemize}
\begin{minipage}{7.5cm}
\begin{lstlisting}[caption={Local Variable in a Function}]
#include <iostream>
using namespace std;

int Simple()
{
	int x = 0;
	x++;
	return x;
}
int main()
{
	cout << x << endl;

	cout << Simple() << endl;
	cout << Simple() << endl;
	cout << Simple() << endl;
	
	return 0;
}
\end{lstlisting}
\end{minipage}
\hspace*{1cm}
\begin{minipage}{7.5cm}
\begin{lstlisting}[caption={\texttt{static} Local Variable in a Function}]
#include <iostream>
using namespace std;

int Simple()
{
	static int x = 0;
	x++;
	return x;
}
int main()
{
	cout << x << endl;

	cout << Simple() << endl;
	cout << Simple() << endl;
	cout << Simple() << endl;
	
	return 0;
}
\end{lstlisting}
\end{minipage}
\section{\texttt{static} Data Member}
\begin{itemize}
\item A \verb|static| data member of class has only one instance. It belongs to the class rather than any object.
\item \verb|static| members are not associated with object but with the class.
\item Even there aren't any objects, \verb|static| data member will continue to exist.
\item One use of \verb|static| is to provide a common variable that can be accessed by objects. Communication?
\end{itemize}
\begin{lstlisting}[caption={\texttt{static} Data Member}]
#include <iostream>
using namespace std;

class Complex
{
	private:
	static int count; // Declaration.
	public:
	Complex()
	{
		count++;
	}
	int GetCount()
	{
		return count;
	}
	~Complex()
	{
		count--;
	}
};
int Complex::count = 0; // Initialisation (think of definition).

int main()
{
	Complex A, B, C;
	cout << A.GetCount() << endl;
	Complex* Ap;
	Ap = new Complex;
	cout << A.GetCount() << endl;
	delete Ap;
	cout << A.GetCount() << endl;
	
	return 0;
}
\end{lstlisting}
\section{\texttt{static} Functions}
\begin{itemize}
\item \verb|static| functions can only access \verb|static| data members. Why?
\item \verb|static| functions can be called even if objects don't exist.
\item If objects don't exist, \verb|static| variables still exist!
\end{itemize}
\begin{lstlisting}[caption={\texttt{static} Member Function}]
#include <iostream>
using namespace std;

class Complex
{
	private:
	static int count; // Declaration.
	public:
	Complex()
	{
		count++;
	}
	static int GetCount()
	{
		return count;
	}
	~Complex()
	{
		count--;
	}
};
int Complex::count = 0; // Initialisation (think of definition).

int main()
{
	Complex A, B, C;
	cout << Complex::GetCount() << endl;
	Complex* Ap;
	Ap = new Complex;
	cout << Complex::GetCount() << endl;
	delete Ap;
	cout << Complex::GetCount() << endl;
	
	return 0;
}
\end{lstlisting}
\section{\texttt{Mutable}}
\begin{itemize}
\item A \texttt{mutable} variable can be changed by \texttt{const} functions.
\item As data member of a class, \verb|mutable| variables can be changed even if the object is constant.
\end{itemize}
\begin{lstlisting}[caption={\texttt{mutable} variables}]
#include <iostream>
using namespace std;

class A
{
	private:
	mutable int x; // Declaration
	public:
	void Setx(int t) const { x = t; }
	int Getx() const { return x; }
};

int main()
{
	A obj;	
	A.Setx(3);
	cout << "x = " << A.Getx() << endl;

	return 0;
}
\end{lstlisting}
\begin{lstlisting}[caption={\texttt{const} object with \texttt{mutable} data members}]
#include <iostream>
using namespace std;

class A
{
	private:
	mutable int x; // Declaration
	int y;
	public:
	void Setx(int t) const { x = t; }
	void Sety(int t) { y = t; }
	int Getx() const { return x; }
	int Gety() const { return y; }
};

int main()
{
	const A obj;	
	A.Setx(3); // Changing mutable data. This is OK.
	cout << "x = " << A.Getx() << endl;

	A.Sety(5); // This will not work, object is const.
	
	return 0;
}
\end{lstlisting}
\section{\texttt{explicit} Keyword}
\begin{itemize}
\item Sometimes we don't want any implicit conversions.
\item \verb|explicit| keyword can be used to specify if don't want a certain conversion to take place implicitly.
\item \verb|explicit| is used with one argument constructors.
\end{itemize}
\begin{lstlisting}[caption={\texttt{explicit} constructor with distance class}]
#include <iostream>
using namespace std;

class distance
{
	private:
	float m;
	public:
	distance(): m(0)
	{
	}
	explicit distance(float km)
	{
		m = 1000 * km;
	}
	void display()
	{
		cout << "Distance = " << m << " meters." << endl;
	}
};

void showdist(distance d)
{
	d.display();
}

int main()
{
	distance d1(2.4);
	showdist(d1);
	showdist(3.4); // This won't work with explicit constructor.

	return 0;
}
\end{lstlisting}
%\nocite{*}
\bibliographystyle{plain}
\bibliography{OOPref}
\end{document}
