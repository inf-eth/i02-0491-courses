\documentclass[12pt,a4paper]{article}

% Margins.
\setlength{\oddsidemargin}{0in}
\setlength{\evensidemargin}{0in}
\setlength{\headheight}{12pt}
\setlength{\headsep}{42pt}
\setlength{\topmargin}{-54pt}
\setlength{\textwidth}{6.5in}
\setlength{\textheight}{10in}

\usepackage{amsmath}
\usepackage{float}
\usepackage{graphicx}
\usepackage[hyphens]{url}
\usepackage{hyperref}	% Clickable links to figures, references and urls.

% Drawing.
\usepackage{pgf}
\usepackage{tikz}

% Listings for formatting code.
\usepackage{listings}
\usepackage{textcomp}
% General options.
\lstset{breaklines=true, basicstyle=\small\ttfamily, tabsize=4, numbers=left, stepnumber=1, frame=single, showstringspaces=false, upquote=true}
% C++ specific high-lighting. Comments are 50/50 shades of green/black and strings coloured with 60/40 red/black mixture.
\lstset{language=[ISO]C++, commentstyle=\color{green!50!black}, keywordstyle=\color{blue}, stringstyle=\color{red!60!black}}

%opening
\title{\vspace{-2cm}Programming for Engineers II\\Class 09\\Constructor}
\author{Attique Dawood}
\date{February 13, 2013\\[0.2cm] Last Modified: \today}
\begin{document}
\maketitle
\section{Announcements}
\begin{itemize}
\item None.
\end{itemize}
\section{Revision}
\begin{itemize}
\item Relationship between classes.
\item Inheritance: Base class and derived class.
\item Derived class inherits all attributes and functions of base class. A student has all the attributes of a person in addition to CGPA and roll number.
\item Inheritance is represented by an empty or hollow arrow--head from derived class to base class.
\item Ownership or `has a' relationship. A car \textbf{has a}n engine. A person \textbf{has a} brain.
\item Two types of ownership: composition and aggregation.
\item Composition is strong link. If life--cycle of container object ends, then contained objects are destroyed or cease to exist. A person \textbf{has a} brain. When that person dies, his brain will cease to function.
\item Aggregation is weak link. Life--cycle of contained objects does not end with that of container. For example, a person \textbf{has a} mobile phone. If that person dies, mobile will continue to function.
\item Association is a `using a' or `knows a' relationship. Represented by a single stealth arrow from `using' class to `used' class. Class being used may not be aware of class using it.
\item A person driving a car is an association between two objects. Person is `using' a car.
\end{itemize}
\section{\texttt{inline} Functions}
\begin{itemize}
\item Functions can be defined as \verb|inline|.
\item This is mainly a compiler level optimisation.
\end{itemize}
\section{Function Overloading}
\begin{itemize}
\item Functions with same name but different arguments.
\item Function being called depends on arguments being passed.
\item Sum function example. Can be defined for int, float or double types.
\end{itemize}
\section{Constructor}
\begin{itemize}
\item A constructor is a very special function in a class.
\item Name of constructor is the name of class.
\item It doesn't have a return type. Mentioning a return type is an error.
\item A constructor is called exactly once throughout the life of an object; only at its creation. Actually, just before the creation.
\item The purpose of a constructor is to initialise data members or variables etc.
\item Even if a constructor isn't defined, compiler will create a default constructor which doesn't do anything.
\item A constructor, default or otherwise, will always be called upon object creation.
\item Constructor can take arguments.
\item Different constructors can be defined. They can be differentiated through their argument list.
\end{itemize}
\begin{lstlisting}[caption={Default and User--Defined Constructor}]
#include <iostream>
using std::cin;
using std::cout;
using std::endl;

class complex
{
	private:
	float real;
	float img;
	
	public:
	complex(); // Default constructor.
	complex(float,float); // User-defined constructor.
	
	void Input();
	void Display();
	complex Add(complex);
	complex DivideBy(complex);
};

// Class functions can be defined inside or outside the class.
// Here, functions are defined outside the class.
// Notice the `::' operator.
complex::complex()
{
	real = 0.f;
	img = 0.f;
}
complex::complex(float pr, float pi)
{
	real = pr;
	img = pi;
}
void complex::Input()
{
	cout << "Enter complex number: ";
	cin >> real >> img;
}
void complex::Display()
{
	if (img < 0.f)
		cout << real << " - j" << -1.f*img << endl;
	else
		cout << real << " + j" << img << endl;
}
complex complex::Add(complex c)
{
	complex result;
	result.real = real + c.real;
	result.img = img + c.img;
	
	return result;
}

int main()
{
	complex complex1;	// Default constructor called. 0+j0.
	complex complex2(3.2, 2.5); // Complex initialised to 3.2+j2.5
	complex complex3;	// Default constructor.
	
	// Addition.
	complex3 = complex1.Add(complex2);
	
	// Result.
	cout << "Result is: ";
	complex3.Display();
	
	return 0;
}
\end{lstlisting}
\section{\texttt{const} Variables}
\begin{itemize}
\item A variable declared with \verb|const| directive.
\item Value of a constant variable cannot change throughout the execution of a program.
\item Can only be initialised at declaration.
\item Member initialiser list.
\end{itemize}
%\nocite{*}
%\bibliographystyle{plain}
%\bibliography{OOPref}
\end{document}
