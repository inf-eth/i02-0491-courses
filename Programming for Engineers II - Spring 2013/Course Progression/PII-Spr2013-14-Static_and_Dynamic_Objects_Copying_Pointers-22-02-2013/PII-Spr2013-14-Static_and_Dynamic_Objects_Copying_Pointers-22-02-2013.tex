\documentclass[12pt,a4paper]{article}

% Margins.
\setlength{\oddsidemargin}{0in}
\setlength{\evensidemargin}{0in}
\setlength{\headheight}{12pt}
\setlength{\headsep}{42pt}
\setlength{\topmargin}{-54pt}
\setlength{\textwidth}{6.5in}
\setlength{\textheight}{10in}

\usepackage{amsmath}
\usepackage{float}
\usepackage{graphicx}
\usepackage[hyphens]{url}
\usepackage{hyperref}	% Clickable links to figures, references and urls.

% Drawing.
\usepackage{pgf}
\usepackage{tikz}

% Listings for formatting code.
\usepackage{listings}
\usepackage{textcomp}
% General options.
\lstset{breaklines=true, basicstyle=\small\ttfamily, tabsize=4, numbers=left, stepnumber=1, frame=single, showstringspaces=false, upquote=true}
% C++ specific high-lighting. Comments are 50/50 shades of green/black and strings coloured with 60/40 red/black mixture.
\lstset{language=[ISO]C++, commentstyle=\color{green!50!black}, keywordstyle=\color{blue}, stringstyle=\color{red!60!black}}

%opening
\title{\vspace{-2cm}Programming for Engineers II\\Class 14\\Static and Dynamic Objects\\Copying Pointers}
\author{Attique Dawood}
\date{February 22, 2013\\[0.2cm] Last Modified: \today}
\begin{document}
\maketitle
\section{Announcements}
\begin{itemize}
\item Quiz.
\end{itemize}
\section{Revision}
\begin{itemize}
\item Memory leaks. Using NULL pointer for safety.
\item A destructor, very much like constructor, is a special function that is automatically called every time an object is destroyed.
\item Copy constructor and assignment operator are provided by compiler. They can also be re--defined by user.
\end{itemize}
\section{Static and Dynamic Objects}
\begin{itemize}
\item Statically created variables or objects are explicitly declared before the program is compiled or run.
\item Static variables/objects created in \verb|main()| are destroyed when \verb|main()| exits.
\item Destruction of statically created variables/objects is the responsibility of compiler.
\item Constructor is called when an object is dynamically created using \verb|new|.
\item Destructor is called when an object is destroyed with \verb|delete|.
\item Destruction (or deallocation) of dynamically created variables/objects is the responsibility of programmer.
\item Programmer must ensure that all dynamically allocated memory (using \verb|new|) is properly deallocated (with \verb|delete|) before the program exits.
\end{itemize}
\begin{lstlisting}[caption={Static and Dynamic Objects}]
#include <iostream>
using namespace std;

class Complex
{
	char ID;
	public:
	// Default constructor.
	Complex(): ID('z') { cout << ID << " constructor called." << endl; }
	// User-defined constructor.
	Complex(char pID): ID(pID) { cout << ID << " constructor called." << endl; }
	// Destructor.
	~Complex() { cout << ID << " destructor called." << endl; }
};
int main()
{
	Complex x('x');  // Static object x. Constructor called.
	
	Complex *y; // Pointer declaration. No constructor called.
	y = new Complex('y'); // Constructor called. New object created.
	delete y;        // Memory deallocation. Dynamic object destroyed. Destructor called.
	
	Complex z[2]; // Default constructor called for both z[0] and z[1]. Two static objects created with ID 'z'.
	
	Complex *t[2]; // Two pointers declared. No object creation.
	t[0] = new Complex('a');	// Object creation. Constructor called.
	t[1] = new Complex('b');    // Object creation. Constructor called.
	
	delete t[0]; // Destructor called for t[0] 'a';
	delete t[1]; // Destructor called for t[1] 'b';
	
	cout << "main() is exiting now..." << endl;
	
	return 0;
}
// All static object destructors are called after main ends.
\end{lstlisting}
\section{Copying Pointers}
\begin{itemize}
\item What happens if you copy pointers?
\item Only pointer address will be copied. Pointer values must be explicitly copied.
\item Copy constructor is used to create an object when another object is passed as constructor argument.
\item Copy constructor does a member--wise copy of object elements (data members).
\item Copy constructor performs a `dumb' copy. Simply equates member variables.
\item This isn't a problem with statically defined variables but copying pointers presents problems.
\end{itemize}
\begin{lstlisting}[caption={Copying Pointers}]
int main()
{
	int *x;
	x = new int;
	*x = 44;
	
	int *y;
	y = x;   // Cannot simply just copy pointers. Must allocate new memory for y.
	*y = 33; // Both *x and *y become 33.
	
	return 0;
}

\end{lstlisting}
\begin{lstlisting}[caption={Proper Way to Copy Pointers}]
int main()
{
	int *x;
	x = new int;
	*x = 44;
	
	int *y;
	y = new int; // pointer y is assigned its own memory.
	*y = *x;     // *x is now copied to *y. Both *x and *y contain 44.
	*y = 33;     // *x remains 44 and *y is now 33. Both pointers are independent.
	
	return 0;
}

\end{lstlisting}
\begin{lstlisting}[caption={Copying Pointers Using Copy Constructor}]
class Simple
{
	public:
	int *value;
};

int main()
{
	Simple x;
	x.value = new int;
	*(x.value) = 3;
	
	Simple z(x);  // Copy constructor called.
	z.value = 5;  // What happens? *(x.value) and *(z.value) both become 5 because both pointers are pointing to same memory location.
	
	return 0;
}
\end{lstlisting}
\noindent\textbf{Solution:} Must re--define copy constructor and allocate memory for pointers explicitly before copying pointer values.
%\nocite{*}
\bibliographystyle{plain}
\bibliography{OOPref}
\end{document}
