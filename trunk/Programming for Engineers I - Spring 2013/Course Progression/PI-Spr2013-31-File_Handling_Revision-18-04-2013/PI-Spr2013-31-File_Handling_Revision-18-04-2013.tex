\documentclass[12pt,a4paper]{article}

% Margins.
\setlength{\oddsidemargin}{0in}
\setlength{\evensidemargin}{0in}
\setlength{\headheight}{12pt}
\setlength{\headsep}{42pt}
\setlength{\topmargin}{-54pt}
\setlength{\textwidth}{6.5in}
\setlength{\textheight}{10in}

\usepackage{amsmath}
\usepackage{float}
\usepackage{graphicx}
\usepackage[hyphens]{url}
\usepackage{hyperref}	% Clickable links to figures, references and urls.
\usepackage{datetime}

% Drawing.
\usepackage{pgf}
\usepackage{tikz}

% Listings for formatting code.
\usepackage{listings}
\usepackage{textcomp}
% General options.
\lstset{breaklines=true, basicstyle=\small\ttfamily, tabsize=4, numbers=left, stepnumber=1, frame=single, showstringspaces=false, upquote=true}
% C++ specific high-lighting. Comments are 50/50 shades of green/black and strings coloured with 60/40 red/black mixture.
\lstset{language=[ISO]C++, commentstyle=\color{green!50!black}, keywordstyle=\color{blue}, stringstyle=\color{red!60!black}}

%opening
\title{\vspace{-2cm}Programming for Engineers I\\Class 31\\File Handling\\Revision}
\author{Attique Dawood}
\date{April 18, 2013\\[0.2cm] Last Modified: \today, \currenttime}
\begin{document}
\maketitle
\section{File Handling in C++}
\begin{itemize}
\item In C++ file handling library is \verb|<fstream>|.
\item You create `objects' of type (or class) \verb|fstream| and use them for reading to and writing from files.
\item Text files can be manipulated using \verb|<<| and \verb|>>| operators exactly like \verb|cout/cin|.
\end{itemize}
\section{Text and Binary File Formats}
\begin{itemize}
\item The number -3.7182193 can be either a double or float depending on data type being used.
\item For example \verb|float f = -3.7182193;| is float and takes 4 bytes for storage in memory. These 4 bytes contain the binary representation of float.
\item If stored as a string, it is 10 characters long including the minus sign and decimal point: \verb|char c[] = "-3.7182193";|. Note: \verb|NULL| is implicit and doesn't count towards length of string.
\item Take another example of an integer \verb|int x = 1;|. If stored as a string of characters, it will take only 1 byte. If stored as an integer it will take 4 bytes, regardless of the value.
\item In a text file, data is stored as a string of characters. It's a copy of what you see on your console. As can be expected, a text file is easily readable if opened in a text editor.
\item In a binary file, data is stored as it would in a memory. It's a copy of what is in the memory. A binary file, just like memory, isn't so obviously readable in raw form.
\end{itemize}
\section{Reading a Text File Character--By--Character}
\begin{lstlisting}[caption={Reading a whole text file character--by--character}]
#include <iostream>
#include <fstream>		// File handling library.
using namespace std;
int main()
{
	char c;

	fstream InFile;	// File handling object.

	// Open "abc.txt" for reading (ios::in).
	InFile.open("abc.txt", ios::in);
	while(InFile)
	{
		InFile >> c;
		if (InFile.eof())
			break;
		else
			cout << c;
	}
	cout << endl;
	InFile.close();	// Close file.

	return 0;
}
\end{lstlisting}
\end{document}
