\documentclass[12pt,a4paper]{article}

% Margins.
\setlength{\oddsidemargin}{0in}
\setlength{\evensidemargin}{0in}
\setlength{\headheight}{12pt}
\setlength{\headsep}{42pt}
\setlength{\topmargin}{-54pt}
\setlength{\textwidth}{6.5in}
\setlength{\textheight}{10in}

\usepackage{amsmath}
\usepackage{float}
\usepackage{graphicx}
\usepackage[hyphens]{url}
\usepackage{hyperref}	% Clickable links to figures, references and urls.

% Drawing.
\usepackage{pgf}
\usepackage{tikz}

% Listings for formatting code.
\usepackage{listings}
\usepackage{textcomp}
% General options.
\lstset{breaklines=true, basicstyle=\small\ttfamily, tabsize=4, numbers=left, stepnumber=1, frame=single, showstringspaces=false, upquote=true}
% C++ specific high-lighting. Comments are 50/50 shades of green/black and strings coloured with 60/40 red/black mixture.
\lstset{language=[ISO]C++, commentstyle=\color{green!50!black}, keywordstyle=\color{blue}, stringstyle=\color{red!60!black}}

%opening
\title{\vspace{-2cm}Programming for Engineers I\\Class 11\\Switch}
\author{Attique Dawood}

\begin{document}
\maketitle
\section{Announcements}
\begin{itemize}
\item Quiz.
\end{itemize}
\section{Revision}
\begin{itemize}
\item \verb|while|, \verb|do-while| and \verb|for| loops.
\item \verb|do-while| is normally used when at least execution of loop is required. Not very common.
\item \verb|while| is used normally when you don't know how many times loop will execute. Terminating condition is used.
\item \verb|for| loop is generally used when you know exactly how many times loop will execute.
\item There is no definitive rule that says a particular loop \emph{has to be used} for a certain situation. You should use whichever loop you're comfortable with.
\item \verb|break| and \verb|continue| statements.
\item You can exit the program at any time inside \verb|main()| with \verb|return 0;|.
\end{itemize}
\section{\texttt{switch} Structure}
\begin{itemize}
\item \verb|switch| is a convenient way of handling multiple conditions.
\item \underline{\texttt{switch} can only be used when you know exactly what the match is}; unlike the range specification in \verb|if| conditions.
\item Every condition inside switch structure must end with a break.
\item A default case can be defined for unexpected conditions.
\end{itemize}
\begin{lstlisting}[caption={Using switch to Display GPA}]
#include <iostream>
using namespace std;

int main()
{
	char Grade;
	cout << "Enter a grade (A-F): ";
	cin >> Grade;
	
	switch(Grade)	// Check Grade variable.
	{
		case 'A':
		{
			cout << "4.0 GPA" << endl;
			break;
		}
		case 'B':
		{
			cout << "3.0 GPA" << endl;
			break;
		}
		case 'C':
		{
			cout << "2.0 GPA" << endl;
			break;
		}
		case 'D':
		{
			cout << "1.0 GPA" << endl;
			break;
		}
		case 'F':
		{
			cout << "0.0 GPA" << endl;
			break;
		}
		default:
		{
			cout << "Invalid grade entered." << endl;
			break;
		}
	}	
	return 0;
}
\end{lstlisting}
\end{document}