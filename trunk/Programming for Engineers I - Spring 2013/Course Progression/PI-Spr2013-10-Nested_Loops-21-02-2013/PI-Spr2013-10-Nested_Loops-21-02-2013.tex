\documentclass[12pt,a4paper]{article}

% Margins.
\setlength{\oddsidemargin}{0in}
\setlength{\evensidemargin}{0in}
\setlength{\headheight}{12pt}
\setlength{\headsep}{42pt}
\setlength{\topmargin}{-54pt}
\setlength{\textwidth}{6.5in}
\setlength{\textheight}{10in}

\usepackage{amsmath}
\usepackage{float}
\usepackage{graphicx}
\usepackage[hyphens]{url}
\usepackage{hyperref}	% Clickable links to figures, references and urls.

% Drawing.
\usepackage{pgf}
\usepackage{tikz}

% Listings for formatting code.
\usepackage{listings}
\usepackage{textcomp}
% General options.
\lstset{breaklines=true, basicstyle=\small\ttfamily, tabsize=4, numbers=left, stepnumber=1, frame=single, showstringspaces=false, upquote=true}
% C++ specific high-lighting. Comments are 50/50 shades of green/black and strings coloured with 60/40 red/black mixture.
\lstset{language=[ISO]C++, commentstyle=\color{green!50!black}, keywordstyle=\color{blue}, stringstyle=\color{red!60!black}}

%opening
\title{\vspace{-2cm}Programming for Engineers I\\Class 10\\Nested Loops}
\author{Attique Dawood}

\begin{document}
\maketitle
\section{Announcements}
\begin{itemize}
\item Quiz tomorrow on loops.
\item First sessional exam syllabus upto loops.
\end{itemize}
\section{Revision}
\begin{itemize}
\item \verb|while|, \verb|do-while| and \verb|for| loops.
\item \verb|do-while| is normally used when at least execution of loop is required. Not very common.
\item \verb|while| is used normally when you don't know how many times loop will execute. Terminating condition is used.
\item \verb|for| loop is generally used when you know exactly how many times loop will execute.
\item There is no definitive rule that says a particular loop \emph{has to be used} for a certain situation. You should use whichever loop you're comfortable with.
\end{itemize}
\section{\texttt{break} and \texttt{continue}}
\begin{itemize}
\item A loop can be forcefully stopped by using the \verb|break| keyword.
\item \verb|break| keyword \emph{must} be used in a loop. It will immediately terminate the loop without executing anything else.
\item If a \verb|continue| keyword is encountered in a loop then all the remaining statements in loop will not execute. The loop itself will continue to execute.
\end{itemize}
\begin{lstlisting}[caption={Breaking a Loop}]
#include <iostream>
using namespace std;

int main()
{
	int x;
	while (true) // Infinite loop.
	{
		cout << "Enter a number: ";
		cin >> x;
		if (x < 0)
		{
			cout << "You've entered a negative number. Exiting..." << endl;
			break; // Loop will break if negative number is entered.
		}
		cout << "You've entered a positive number. Continuing..." << endl;
	}
	
	return 0;
}
\end{lstlisting}
\begin{lstlisting}[caption={Using continue in a Loop}]
#include <iostream>
using namespace std;

int main()
{
	for (int i=0; i<20; i++)
	{
		cout << "This statement will show and counter is: " << i << endl;
		continue;
		cout << "This statement will not show as it is skipped because of continue..." << endl;
		cout << "This statement will also be skipped..." << endl;
	}
	
	return 0;
}
\end{lstlisting}
\section{Nested Loops}
\begin{itemize}
\item Nested loops can be used to perform complex tasks.
\item A simple example is printing symmetric 2D shapes like square, rectangle, triangles and diamonds.
\end{itemize}
\begin{lstlisting}[caption={Using for Loop to Make Rectangle and Square}]
#include <iostream>
using namespace std;

int main()
{
	int rows = 5;
	int columns = 10;
	
	for (int i=0; i<rows; i++)
	{
		for (int j=0; j<columns; j++)
		{
			cout << "*";
		}
		cout << endl;			
	}
	
	return 0;
}
\end{lstlisting}
What would happen if you change \verb|j<columns| to \verb|j<i|?\\
\noindent \textbf{Exercise:} Print all the even numbers between 40 and 150\\
\noindent \textbf{Exercise:} Print all the numbers divisible by 11 between 0 and 1000\\
\noindent \textbf{Exercise:} Print all the numbers between 100 and 1000 whose sum of digits is divisible by 5\\
\end{document}