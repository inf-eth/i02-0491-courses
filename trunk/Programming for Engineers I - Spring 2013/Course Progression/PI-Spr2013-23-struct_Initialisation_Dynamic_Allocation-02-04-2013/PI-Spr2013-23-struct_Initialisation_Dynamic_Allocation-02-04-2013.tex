\documentclass[12pt,a4paper]{article}

% Margins.
\setlength{\oddsidemargin}{0in}
\setlength{\evensidemargin}{0in}
\setlength{\headheight}{12pt}
\setlength{\headsep}{42pt}
\setlength{\topmargin}{-54pt}
\setlength{\textwidth}{6.5in}
\setlength{\textheight}{10in}

\usepackage{amsmath}
\usepackage{float}
\usepackage{graphicx}
\usepackage[hyphens]{url}
\usepackage{hyperref}	% Clickable links to figures, references and urls.

% Drawing.
\usepackage{pgf}
\usepackage{tikz}

% Listings for formatting code.
\usepackage{listings}
\usepackage{textcomp}
% General options.
\lstset{breaklines=true, basicstyle=\small\ttfamily, tabsize=4, numbers=left, stepnumber=1, frame=single, showstringspaces=false, upquote=true}
% C++ specific high-lighting. Comments are 50/50 shades of green/black and strings coloured with 60/40 red/black mixture.
\lstset{language=[ISO]C++, commentstyle=\color{green!50!black}, keywordstyle=\color{blue}, stringstyle=\color{red!60!black}}

%opening
\title{\vspace{-2cm}Programming for Engineers I\\Class 23\\Initialisation of \texttt{struct} Variables\\Dynamic \texttt{struct}s}
\author{Attique Dawood}

\begin{document}
\maketitle
\section{Announcements}
\begin{itemize}
\item None.
\end{itemize}
\section{Revision}
\begin{itemize}
\item Creating new data types using \verb|struct|.
\item Defining \verb|struct|.
\item Data members of a \verb|struct|. The dot operator for accessing data members.
\end{itemize}
\section{\texttt{struct} Initialisation}
\begin{itemize}
\item Members of a \verb|struct| can be initialised, like array, by passing comma separated arguments enclosed in braces.
\item Initialisation is done at creation of variable.
\end{itemize}
\begin{lstlisting}[caption={\texttt{struct} Initialisation}]
#include <iostream>
#include <cstring>
using namespace std;

struct Student
{
	char Name[20];
	int Age;
	float CGPA;
	char Gender;
};

int main()
{
	Student S1 = {"Ahsan", 19, 3.97f, 'M'};
	Student S2 = {"Sahar", 19, 3.5f, 'F'};

	// Alternate way of setting data members.
	strcpy(S1.Name, "Ahsan");
	S1.Age = 19;
	S1.CGPA = 3.97f;
	S1.Gender = 'M';

	return 0;
}
\end{lstlisting}
\section{Copying \texttt{struct}s}
\begin{itemize}
\item \texttt{struct}s can be copied normally via assignment operator.
\end{itemize}
\begin{lstlisting}
Student S1 = {"Ahsan", 19, 3.97f, 'M'};
Student S2 = S1; // Copying S1 into S2.
\end{lstlisting}
\section{Passing \texttt{struct} Variables to Functions}
\begin{itemize}
\item \texttt{struct}s are passed just like normal variables.
\item It is better to pass as reference to avoid unnecessary memory and speed overheads associated with pass--by--value.
\end{itemize}
\section{Dynamic \texttt{struct} Variables and Arrays}
\begin{itemize}
\item Arrays of \texttt{struct}s can be created just like that of basic data types.
\item Dynamic \texttt{struct} variables are created just like basic types. Only difference is accessing members of \texttt{struct}.
\item The \verb|->| is used instead of dot operator to access members when using a pointer to some \texttt{struct} variable.
\end{itemize}
\begin{lstlisting}[caption={Dynamic \texttt{struct}}]
#include <iostream>
using namespace std;

struct Complex
{
	float real;
	float img;
};
int main()
{
	// Static struct access using pointer.
	Complex t;
	Complex* tp;
	tp = &t;
	tp->real = 1.0f;
	tp->img = 0.0f;

	// Dynamic allocation.
	Complex* c;
	c = new Complex;
	c->real = 3.2f; // same as (*c).real = 3.2f;
	c->img = 4.4f;  // same as (*c).img = 4.4f;

	cout << c->real << c->img << endl;
	delete c;

	return 0;
}
\end{lstlisting}
\begin{lstlisting}[caption={Array of \texttt{struct}}]
#include <iostream>
using namespace std;

struct Complex
{
	float real;
	float img;
};
int main()
{
	Complex* c;
	c = new Complex[5];

	for (int i=0; i<5; i++)
	{
		c[i].real = 3.2f;
		c[i].img = 4.4f;
	}

	cout << c[0].real << c[0].img << endl;

	delete[] c;

	return 0;
}
\end{lstlisting}

\end{document}