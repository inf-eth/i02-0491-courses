\documentclass[12pt,a4paper]{article}

% Margins.
\setlength{\oddsidemargin}{0in}
\setlength{\evensidemargin}{0in}
\setlength{\headheight}{12pt}
\setlength{\headsep}{42pt}
\setlength{\topmargin}{-54pt}
\setlength{\textwidth}{6.5in}
\setlength{\textheight}{10in}

\usepackage{amsmath}
\usepackage{float}
\usepackage{graphicx}
\usepackage[hyphens]{url}
\usepackage{hyperref}	% Clickable links to figures, references and urls.

% Drawing.
\usepackage{pgf}
\usepackage{tikz}

% Listings for formatting code.
\usepackage{listings}
\usepackage{textcomp}
% General options.
\lstset{breaklines=true, basicstyle=\small\ttfamily, tabsize=4, numbers=left, stepnumber=1, frame=single, showstringspaces=false, upquote=true}
% C++ specific high-lighting. Comments are 50/50 shades of green/black and strings coloured with 60/40 red/black mixture.
\lstset{language=[ISO]C++, commentstyle=\color{green!50!black}, keywordstyle=\color{blue}, stringstyle=\color{red!60!black}}

%opening
\title{\vspace{-2cm}Programming for Engineers I\\Class 24\\Nested Structures}
\author{Attique Dawood}

\begin{document}
\maketitle
\section{Announcements}
\begin{itemize}
\item None.
\end{itemize}
\section{Revision}
\begin{itemize}
\item Creating new data types using \verb|struct|.
\item Defining \verb|struct|.
\item Data members of a \verb|struct|. The dot operator for accessing data members.
\item Initialisation.
\item Copying \texttt{struct}.
\item Passing \texttt{struct} to functions.
\item Arrays of \texttt{struct}.
\end{itemize}
\section{Nested \texttt{struct}}
\begin{itemize}
\item Just like ordinary variables, variables of \texttt{struct} type can be used inside other structures.
\item We can define a point with x and y coordinates.
\item A line has two points.
\item A triangle has three points.
\end{itemize}
\begin{lstlisting}[caption={Nested \texttt{struct}}]
#include <iostream>
#include <cstring>
using namespace std;

struct Point
{
	int x;
	int y;
};

struct Line
{
	Point P1;
	Point P2;
};

struct Triangle
{
	Point P1;
	Point P2;
	Point P3;
};

int main()
{
	Triangle T;

	// Taking input.
	T.P1.x = 1;
	T.P1.y = 1;
	T.P2.x = 3;
	T.P2.y = 2;
	T.P3.x = 4;
	T.P3.y = 5;

	return 0;
}
\end{lstlisting}

\end{document}