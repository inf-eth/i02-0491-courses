\documentclass[12pt,a4paper]{article}

% Margins.
\setlength{\oddsidemargin}{0in}
\setlength{\evensidemargin}{0in}
\setlength{\headheight}{12pt}
\setlength{\headsep}{42pt}
\setlength{\topmargin}{-54pt}
\setlength{\textwidth}{6.5in}
\setlength{\textheight}{10in}

\usepackage{amsmath}
\usepackage{float}
\usepackage{graphicx}
\usepackage[hyphens]{url}
\usepackage{hyperref}	% Clickable links to figures, references and urls.

% Drawing.
\usepackage{pgf}
\usepackage{tikz}

% Listings for formatting code.
\usepackage{listings}
% General options.
\lstset{breaklines=true, basicstyle=\small\ttfamily, tabsize=4, numbers=left, stepnumber=1, frame=single, showstringspaces=false}
% C++ specific high-lighting. Comments are 50/50 shades of green/black and strings coloured with 60/40 red/black mixture.
\lstset{language=[ISO]C++, commentstyle=\color{green!50!black}, keywordstyle=\color{blue}, stringstyle=\color{red!60!black}}

%opening
\title{\vspace{-2cm}Programming for Engineers I\\Class 03\\Computing Concepts: Operators and Logic}
\author{Attique Dawood}

\begin{document}
\maketitle
\section{Revision}
\begin{itemize}
\item To become a good programmer take interest. Intelligence helps but practice can make up for that to a certain extent.
\item Programming languages: assembly, C, C++, Java, C\#, scripting languages like Visual Basic, python etc.
\item Programming cycle: Code, pre--process, compile, link, debug.
\item Basic working of a computer. I/O devices, CPU, RAM, ROM.
\end{itemize}
\section{Operators}
\begin{itemize}
\item There are mathematical operators and logic operators.
\item Math or arithmetic operators are +, -, *, /, \%. Operator precedence?
\item Basic logic operations are NOT, AND, OR, XOR.
\item In C/C++ there are two types of logic operations: bitwise and logical statements.
\item The bitwise logic operators are \verb|~| for NOT, \verb|&| for AND, \verb@|@ for OR and \verb|^| for XOR. These operate on individual bits of variable(s).
\item The logic operators for evaluating statements are \verb|!| for NOT, \verb|&&| for AND and \verb@||@ for OR. Logical operators can operate on logical statements.
\item The operators \verb|<<| and \verb|>>| have different uses depending on context.
\item As stream operator used with \verb|cin| and \verb|cout| for input/output.
\item As shift operators, \verb|<<| and \verb|>>| operate on individual bits of a variable. For example, \verb|x<<2| which means bits of variable \verb|x| left--shifted by 2.

\end{itemize}
\section{Exercise: Using \% to Separate Digits of a Number}
\begin{enumerate}
\item Take a number, let's say 432.
\item Take mod ($432 \% 10$) and you get 2.
\item Divide this by 10 and you get 43.
\item Take mod again and you get 3.
\item Divide again by 10 and you get 4.
\item Taking mod again will give 4.
\end{enumerate}
\section{Truth Table for Logic Gates}
\begin{itemize}
\item 4 bit binary numbers.
\item Logic operations between binary numbers.
\end{itemize}
\section{Shift Operations}
\begin{itemize}
\item Right shift is integer division by base.
\item Left shift is multiplication by base.
\end{itemize}
\end{document}