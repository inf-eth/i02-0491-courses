\documentclass[12pt,a4paper]{article}

% Margins.
\setlength{\oddsidemargin}{0in}
\setlength{\evensidemargin}{0in}
\setlength{\headheight}{12pt}
\setlength{\headsep}{42pt}
\setlength{\topmargin}{-54pt}
\setlength{\textwidth}{6.5in}
\setlength{\textheight}{10in}

\usepackage{amsmath}
\usepackage{float}
\usepackage{graphicx}
\usepackage[hyphens]{url}
\usepackage{hyperref}	% Clickable links to figures, references and urls.

% Drawing.
\usepackage{pgf}
\usepackage{tikz}
\usepackage{amssymb}  % Tick mark
\usepackage{textcomp} % Cross
\usepackage{datetime}

% Listings for formatting code.
\usepackage{listings}
\usepackage{textcomp}
% General options.
\lstset{breaklines=true, basicstyle=\small\ttfamily, tabsize=4, numbers=left, stepnumber=1, frame=single, showstringspaces=false, upquote=true}
% C++ specific high-lighting. Comments are 50/50 shades of green/black and strings coloured with 60/40 red/black mixture.
\lstset{language=[ISO]C++, commentstyle=\color{green!50!black}, keywordstyle=\color{blue}, stringstyle=\color{red!60!black}}

%opening
\title{Programming for Engineers I\\Lab 12\\Linked List}
\author{Attique Dawood}
\date{May 09, 2013\\[0.2cm] Last Modified: \today, \currenttime}
\begin{document}
\maketitle
\section{Linked List}
\begin{itemize}
\item Storage problems with fragmented memory or hard disk.
\item Array requires contiguous space in memory. Array cannot dynamically grow incrementally. Must de--allocate and re--allocate whole array to change size.
\item Linked list solves the above problems. Doesn't require contiguous memory. Linked list is a chain of pointer--connected nodes. Nodes store information just like array indices.
\item Linked list access is slower than array. Must traverse whole list to reach a particular node. But, storage and size is flexible.
\item Array is preferable when speed is desired and storage size is known. Static data. Example: Number of students in a class cannot exceed 50. Create an array of 50 size for students.
\item Linked list is preferable when speed is not a major concern and storage size is highly dynamic. Example: Daily record of cars using motorway. On weekends traffic is high but there aren't many cars on weekdays. Numbers of cars can be highly varied.
\end{itemize}
\begin{lstlisting}[caption={Basic Linked List}]
#include <iostream>
using namespace std;

struct Node
{
	int Data;
	Node* Next;
};

Node* First = NULL;

void AddAtHead(int pData)
{
	Node* NewNode;
	NewNode = new Node;
	NewNode->Data = pData;

	if (First == NULL)
	{
		First = NewNode;
		NewNode->Next = NULL;
	}
	else
	{
		NewNode->Next = First;
		First = NewNode;
	}
}

void AddAtTail(int pData)
{
	Node* NewNode;
	NewNode = new Node;
	NewNode->Data = pData;
	NewNode->Next = NULL;

	if (First == NULL)
		First = NewNode;
	else
	{
		Node* temp = First;
		while (temp->Next != NULL)
			temp = temp->Next;
		// After loop, temp now points to last node.
		temp->Next = NewNode;
	}
}

void Display()
{
	Node* temp;
	temp = First;
	if (temp == NULL)
		return;
	while (temp != NULL)
	{
		cout << temp->Data << " ";
		temp = temp->Next;
	}
	cout << endl;
}

int Search(int key)
{
	Node* temp;
	temp = First;
	while (temp != NULL)
	{
		if (temp->Data == key)
			return 0;
		temp = temp->Next;
	}
	return -1;
}

int Delete(int key)
{
	Node* temp1;
	Node* temp2;
	temp1 = First;
	temp2 = First;
	if (First == NULL)
		return -1;
	if (First->Data == key)
	{
		First = First->Next;
		delete temp1;
	}
	else
	{
		temp2 = temp2->Next;
		while (temp2 != NULL)
		{
			if (temp2->Data == key)
			{
				temp1->Next = temp2->Next;
				delete temp2;
				return 0;
			}
			temp1 = temp1->Next;
			temp2 = temp2->Next;
		}
	}
	return -1;
}

int main()
{
	char choice;
	int x;
	while (true)
	{
		cout << "1. Add at head." << endl
			 << "2. Add at tail." << endl
			 << "3. Search." << endl
			 << "4. Delete." << endl
			 << "5. Display list." << endl
			 << "6. Exit." << endl;
		cout << "Please enter a choice: ";
		cin >> choice;

		switch (choice)
		{
			case '1':
			{
				cout << "Enter a number to add at head: ";
				cin >> x;
				AddAtHead(x);
				break;
			}
			case '2':
			{
				cout << "Enter a number to add at tail: ";
				cin >> x;
				AddAtTail(x);
				break;
			}
			case '3':
			{
				cout << "Enter a number to search: ";
				cin >> x;
				if (Search(x) == 0)
					cout << x << " is in the list." << endl;
				else
					cout << x << " is not in the list." << endl;
				break;
			}
			case '4':
			{
				cout << "Enter a number to delete: ";
				cin >> x;
				if (Delete(x) == 0)
					cout << x << " was deleted from the list." << endl;
				else
					cout << x << " is not in the list." << endl;
				break;
			}
			case '5':
			{
				cout << "Linked List: ";
				Display();
				break;
			}
			case '6':
			{
				return 0;
			}
			default:
			{
				cout << "Invalid choice..." << endl;
				break;
			}
		} // switch end
	}// while end

	return 0;
}
\end{lstlisting}
\section{Exercise}
\noindent\textbf{Note: }You are only required to write \texttt{main()} that accomplishes following tasks.\\
\noindent\textbf{Question No. 1:} Create a linked list that contains first 11 whole numbers (0, 1, 2, 3, ..., 10).\\
\noindent\textbf{Question No. 2:} Add first 10 negative integers to the linked list in Q1 such that the list is in sorted order (-10, -9, -8, ..., 0, 1, 2, 3, ..., 10).\\
\noindent\textbf{Question No. 3:} From Q2, swap the first and last numbers in linked list, i.e., 10 and -10.\\
\end{document}
