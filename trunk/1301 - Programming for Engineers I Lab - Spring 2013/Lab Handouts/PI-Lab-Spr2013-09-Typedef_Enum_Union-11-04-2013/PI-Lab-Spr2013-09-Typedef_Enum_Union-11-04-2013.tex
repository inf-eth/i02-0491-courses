\documentclass[12pt,a4paper]{article}

% Margins.
\setlength{\oddsidemargin}{0in}
\setlength{\evensidemargin}{0in}
\setlength{\headheight}{12pt}
\setlength{\headsep}{0pt}
\setlength{\topmargin}{-60pt}
\setlength{\textwidth}{6.5in}
\setlength{\textheight}{10.75in}

\usepackage{amsmath}
\usepackage{float}
\usepackage{graphicx}
\usepackage[hyphens]{url}
\usepackage{hyperref}	% Clickable links to figures, references and urls.
\usepackage{datetime}
% Drawing.
\usepackage{pgf}
\usepackage{tikz}
\usepackage{amssymb}  % Tick mark
\usepackage{textcomp} % Cross

% Listings for formatting code.
\usepackage{listings}
\usepackage{textcomp}
% General options.
\lstset{breaklines=true, basicstyle=\small\ttfamily, tabsize=4, numbers=left, stepnumber=1, frame=single, showstringspaces=false, upquote=true}
% C++ specific high-lighting. Comments are 50/50 shades of green/black and strings coloured with 60/40 red/black mixture.
\lstset{language=[ISO]C++, commentstyle=\color{green!50!black}, keywordstyle=\color{blue}, stringstyle=\color{red!60!black}}

%opening
\title{\vspace{-1cm}Programming for Engineers I\\Lab 09\\Typedef, Enum and Union}
\author{Attique Dawood}
\date{April 11, 2013\\[0.2cm] Last Modified: \today}
\begin{document}
\maketitle
\section{\texttt{typedef}}
\verb|typedef| is used to define new names, aliases or nicknames for already existing data types.
\begin{lstlisting}[caption={\texttt{typedef}}]
#include <iostream>
using namespace std;

typedef int I;				// typedef for single int
typedef int IntArray[5];	// typedefining an int array of size 5.

int main()
{
	I x = 3;
	cin >> x;
	cout << x;
	
	IntArray data;
	for (int i=0; i<5; i++)
		data[i] = rand()%6;
	for (int i=0; i<5; i++)
		cout << "data[" << i << "] = " << data[i] << endl;
	
	return 0;
}
\end{lstlisting}
\section{enum}
\verb|enum| is used to create new datatypes with pre--defined values. It is like a set of values you can define and then pick/choose from.
\begin{lstlisting}[caption={\texttt{enum}}]
#include <iostream>
using namespace std;

enum Direction{north, south, east, west};

int main()
{

	Direction d;
	int temp;
	cout << "Enter a direction: " << endl
		 << "0 = north, 1 = south, 2 = east, 3 = west: ";
	cin >> temp;

	d = (Direction)temp; // casting from int to Direction.

	if (d == north)
		cout << "NORTH!" << endl;
	else if (d == south)
		cout << "SOUTH!" << endl;
	else if (d == east)
		cout << "EAST!" << endl;
	else if (d == west)
		cout << "WEST!" << endl;
	else
		cout << "Invalid direction!" << endl;

	return 0;
}
\end{lstlisting}
\section{\texttt{union}}
\verb|union| is similar to a \verb|struct| but all the data members in \verb|union| share same memory.
\begin{lstlisting}[caption={\texttt{union}}]
#include <iostream>
using namespace std;

union test
{
	int i;
	double d;
	char c;
};

int main()
{
	test u;
	u.i = 65;
	cout << "u.c = " << u.c << endl;
	
	u.c = 'E';
	cout << u.i << endl;
	
	return 0;
}
\end{lstlisting}
\newpage
\section{Exercise}
\textbf{Question No. 1:} Using \verb|typedef| create a character array. Input a string and then display it.\\
\textbf{Question No. 2:} Use enum to create a set/collection of items. Items should be related. Test your set.\\
\textbf{Question No. 3:} Create a union containing int, double and char. Test it by taking input and display the members.\\
\end{document}
