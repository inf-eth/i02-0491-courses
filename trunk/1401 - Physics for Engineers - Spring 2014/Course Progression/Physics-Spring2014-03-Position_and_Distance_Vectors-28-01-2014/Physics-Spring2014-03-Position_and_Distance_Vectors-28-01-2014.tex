\documentclass[12pt,a4paper]{article}

% Margins.
\setlength{\oddsidemargin}{0in}
\setlength{\evensidemargin}{0in}
\setlength{\headheight}{12pt}
\setlength{\headsep}{42pt}
\setlength{\topmargin}{-54pt}
\setlength{\textwidth}{6.5in}
\setlength{\textheight}{10in}

\usepackage{amsmath}
\usepackage{float}
\usepackage{graphicx}
\usepackage[hyphens]{url}
\usepackage{hyperref}	% Clickable links to figures, references and urls.
\usepackage{datetime}
\usepackage{longtable}

% Drawing.
\usepackage{pgf}
\usepackage{tikz}

% Listings for formatting code.
\usepackage{listings}
\usepackage{textcomp}
% General options.
\lstset{breaklines=true, basicstyle=\small\ttfamily, tabsize=4, numbers=left, stepnumber=1, frame=single, showstringspaces=false, upquote=true}
% C++ specific high-lighting. Comments are 50/50 shades of green/black and strings coloured with 60/40 red/black mixture.
\lstset{language=[ISO]C++, commentstyle=\color{green!50!black}, keywordstyle=\color{blue}, stringstyle=\color{red!60!black}}

%opening
\title{\vspace{-2cm}Physics for Engineers\\Class 03\\Position and Distance Vectors}
\author{Attique Dawood}
\date{January 28, 2014\\[0.2cm] Last Modified: \today, \currenttime}
\begin{document}
\maketitle
\section{Announcements}
\begin{itemize}
\item None.
\end{itemize}
\section{Revision}
\begin{itemize}
\item Scalars and vectors.
\item Component form of vector.
\item Unit vector.
\item Vector addition and subtraction.
\end{itemize}
\section{Usage of rec() and pol()}
\begin{itemize}
\item rec() and pol() are two functions available in all calculators.
\item rec() takes magnitude and angle as input and converts them into rectangular components.
\item pol() takes rectangular components and gives magnitude and angle.
\item In older calculators result is stored in memory locations `E' and `F'. To access a memory location use `Rcl' or `Alpha' keys.
\end{itemize}
\section{Function}
\noindent\textbf{Warning:} This is a very loose definition of function but one that is convenient and suitable for us. In calculus you might come across a stricter definition.
\begin{itemize}
\item A function is a `rule' (or relationship) between two quantities.
\item Two variables, independent and dependent are associated with a function.
\item All possible values that independent variable can take is called `domain' of function.
\item All possible values that dependent variable can take is called `range' of function.
\item In the function $x(t)=2t$, $x$ is dependent variable, $t$ is independent variable and the rule is `value of $x$ is twice that of $t$' or simply $2t$.
\end{itemize}
\section{Scalar and Vector Fields}
\subsection{Field}
A field is a function of space or the coordinates x, y and z. A field specifies or defines a scalar quantity in the whole space or region.
\subsection{Scalar Field}
A scalar field is a function that assigns to each point in space a scalar or number. For example, temperature in a room at each point is defined by a function $T(x, y, z)=2x+3y-z$.
\subsection{Vector Field}
A vector field is a function that assigns to each point in space a vector. For example, electric field in a region is given by $\bar{E}(x, y, z)=2y\hat{x}+3z\hat{y}+xy\hat{z}$.
\section{Position Vector (1.6S)}
A position vector gives the position of a point with respect to the origin. Mathematically we can express the position of a point $(x,y,z)$ as,
\begin{equation}
\textbf{r}=x\hat x+y \hat y+z\hat z
\end{equation}
We normally use a subscript with $\textbf{r}$ to represent position vectors to specific points. For example, $\textbf{r}_P$, $\textbf{r}_Q$ etc.
\section{Distance Vector (1.6S)}
Given two points P and Q, the distance vector from P to Q is the vector,
\begin{equation}
\textbf{r}_{PQ}=\textbf{r}_{Q}-\textbf{r}_{P}
\end{equation}
Distance vector is the displacement between P and Q. Magnitude of distance vector ($|\textbf{r}_{PQ}|$) gives the distance between P and Q.

\begin{itemize}
\item A point P(x, y, z) and a vector \textbf{A}($A_x$, $A_y$, $A_z$) may be written in a similar manner but they are not the same. Only the position of point P, $\textbf{r}_P$, is a vector. \cite[page 8]{Sadiku}
\item A vector field is said to be uniform and constant if it is the same at every point x, y and z in space. Example of a uniform field is $\textbf A=3\hat x+4\hat y-\hat z$.
\item A vector field may not necessarily be constant throughout the space. In fact, most vector fields are variable (direction, as well as magnitude). Example of such a field is $\textbf A=-y\hat x+x\hat y$.
\end{itemize}

\section{Exercises}
\noindent\textbf{Question 1:} Points $P$ and $Q$ are located at (0, 2, 4) and (-3, 1, 5). Calculate
\begin{enumerate}
\item[(1)] The position vector $\textbf{r}_P$.
\item[(2)] The distance vector, $\textbf{r}_{PQ}$, from $P$ to $Q$.
\item[(3)] The distance between $P$ and $Q$.
\item[(4)] A vector parallel to $PQ$ with magnitude of 10.
\end{enumerate}
\noindent\textbf{Question 2:} A vector field is given by $\textbf{A}=-y\hat x+x\hat y$.
\begin{enumerate}
\item[(1)] Draw the field vector at points (1, 0, 0), (0, 1, 0), (-1, 0, 0) and (0, -1, 0).
\item[(2)] What is the magnitude of the field at (2, 3, 0).
\item[(3)] Find at unit vector along \textbf{A} at (1, 0, 0) and (0, 1, 0).
\item[(4)] The given field \textbf{A} has a counter--clockwise rotation around the z-axis. Can you suggest a vector field with a clockwise rotation?
\end{enumerate}
%\nocite{*}
\bibliographystyle{plain}
\bibliography{PhysicsRef}
\end{document}
