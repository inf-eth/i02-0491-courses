\documentclass[12pt,a4paper]{article}

% Margins.
\setlength{\oddsidemargin}{0in}
\setlength{\evensidemargin}{0in}
\setlength{\headheight}{12pt}
\setlength{\headsep}{42pt}
\setlength{\topmargin}{-54pt}
\setlength{\textwidth}{6.5in}
\setlength{\textheight}{10in}

\usepackage{amsmath}
\usepackage{float}
\usepackage{graphicx}
\usepackage[hyphens]{url}
\usepackage{hyperref}	% Clickable links to figures, references and urls.
\usepackage{datetime}

% Drawing.
\usepackage{pgf}
\usepackage{tikz}

% Listings for formatting code.
\usepackage{listings}
\usepackage{textcomp}
% General options.
\lstset{breaklines=true, basicstyle=\small\ttfamily, tabsize=4, numbers=left, stepnumber=1, frame=single, showstringspaces=false, upquote=true}
% C++ specific high-lighting. Comments are 50/50 shades of green/black and strings coloured with 60/40 red/black mixture.
\lstset{language=[ISO]C++, commentstyle=\color{green!50!black}, keywordstyle=\color{blue}, stringstyle=\color{red!60!black}}

%opening
\title{\vspace{-2cm}Programming for Engineers II\\Class 07\\Relationship Between Classes\\Composition and Aggregation}
\author{Attique Dawood}
\date{June 17, 2013\\[0.2cm] Last Modified: \today, \currenttime}
\begin{document}
\maketitle
\section{Announcements}
\begin{itemize}
\item None.
\end{itemize}
\section{Revision}
\begin{itemize}
\item Relationship between classes.
\item Inheritance: Base class and derived class.
\item Derived class inherits all attributes and functions of base class. A student has all the attributes of a person in addition to CGPA and roll number.
\item UML class diagram.
\item Class is represented by name, data and attributes grouped in a box.
\item Private and public methods are preceded by `-' or `+' sign.
\item Inheritance is represented by an empty or hollow arrow--head from derived class to base class.
\end{itemize}
\section{Relationship Between Classes}
\subsection{``is a'' Relationship}
\begin{itemize}
\item Car is the base class.
\item Honda \textbf{is a} car.
\item Toyota \textbf{is a} car.
\end{itemize}
\begin{itemize}
\item Mobile phone is the base class.
\item Apple iPhone \textbf{is a} mobile phone.
\item Samsung Galaxy \textbf{is a} mobile phone.
\end{itemize}
\begin{itemize}
\item Person is the base class.
\item Teacher \textbf{is a} person.
\item Student \textbf{is a} person.
\end{itemize}
Derived classes have all the features (attributes and functionality) of base class in addition to some features specific to it. A person will have a name and age. A student in addition to name and age will also have a CGPA.
\subsection{``has a'' Relationship or Aggregation}
A class can be a member of another classes. Weak link, empty diamond.
\begin{itemize}
\item That teacher \textbf{has a}n Apple iPhone.
\item That student \textbf{has a} Samsung Galaxy.
\item A person has address. If a person dies, address is unaffected.
\end{itemize}
\subsection{``part of''/``composed of'' Relationship or Composition}
A class can be a part of another classes. Strong link, filled diamond. Some authors may refer to composition as a form of aggregation. If the container class is destroyed so is the contained class.
\begin{itemize}
\item Compressor is a \textbf{part of} AC. If AC is scrapped, compressor goes with it.
\item Engine is a \textbf{part of} car. 
\item Tyre is also a \textbf{part of} car. If a car is destroyed, this implies everything contained within is scrap metal and useless.
\item Organs of a person cannot sustain if the person dies.
\end{itemize}
\nocite{*}
\bibliographystyle{plain}
\bibliography{OOPref}
\end{document}
