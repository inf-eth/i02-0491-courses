\documentclass[12pt,a4paper]{article}

% Margins.
\setlength{\oddsidemargin}{0in}
\setlength{\evensidemargin}{0in}
\setlength{\headheight}{12pt}
\setlength{\headsep}{42pt}
\setlength{\topmargin}{-54pt}
\setlength{\textwidth}{6.5in}
\setlength{\textheight}{10in}

\usepackage{amsmath}
\usepackage{float}
\usepackage{graphicx}
\usepackage[hyphens]{url}
\usepackage{hyperref}	% Clickable links to figures, references and urls.
\usepackage{enumerate}

% Drawing.
\usepackage{pgf}
\usepackage{tikz}

% Listings for formatting code.
\usepackage{listings}
\usepackage{textcomp}
% General options.
\lstset{breaklines=true, basicstyle=\small\ttfamily, tabsize=4, numbers=left, stepnumber=1, frame=single, showstringspaces=false, upquote=true}
% C++ specific high-lighting. Comments are 50/50 shades of green/black and strings coloured with 60/40 red/black mixture.
\lstset{language=[ISO]C++, commentstyle=\color{green!50!black}, keywordstyle=\color{blue}, stringstyle=\color{red!60!black}}

%opening
\title{\vspace{-2cm}Programming for Engineers II\\Class 21\\Friend Functions\\Overloading Stream Operators}
\author{Attique Dawood}

\begin{document}
\maketitle
\section{Announcements}
\begin{itemize}
\item Assignment based quiz on Monday.
\end{itemize}
\section{Revision}
\begin{itemize}
\item When an appropriate constructor is already provided we can return an object by writing class name and passing constructor arguments.
\item A \texttt{mutable} variable can be changed by \texttt{const} functions.
\item As data member of a class, \verb|mutable| variables can be changed even if the object is constant.
\item Sometimes we don't want any implicit conversions.
\item \verb|explicit| is used with one argument constructors to prevent unwanted conversions.
\end{itemize}
\section{\texttt{friend} Functions}
\begin{itemize}
\item Sometimes we need to access private data of a class in certain trusted functions.
\item These functions can be declared \verb|friend| of the class.
\item \verb|friend| is specified with the prototype of the function inside the class declaration.
\item \verb|friend| function definition/implementation is outside the class.
\item \verb|friend| \underline{functions are not members of the class}. Remember, member functions can already access private members.
\end{itemize}
\begin{lstlisting}[caption={\texttt{friend} function to display Complex number}]
#include <iostream>
using namespace std;

class Complex
{
	private:
	float real;
	float img;
	public:
	Complex(): real(0.f), img(0.f) {}
	Complex(float a, float b): real(a), img(b) {}
	friend void Display(Complex&); // Friend function specified.
};

// Display function definition. This is an ordinary function.
void Display(Complex& c)
{
	cout << c.real << "+j" << c.img << endl;
}

int main()
{
	Complex A(2.f,3.3f);
	Display(A);

	return 0;
}
\end{lstlisting}
\section{Overloading the Insertion and Extraction Operators}
\begin{itemize}
\item Stream operators \verb|>>| and \verb|<<| operators are used for input and output.
\item \verb|cin| and \verb|cout| are objects of \verb|istream| and \verb|ostream| classes.
\item Stream operators must be overloaded as friends. Convenient way to display() and input().
\item Warning! Following code snippet will not work with VC6.
\end{itemize}
\begin{lstlisting}[caption={Overloading Stream Operators for Complex class}]
#include <iostream>
using namespace std;

class Complex
{
	private:
	float real;
	float img;
	public:
	Complex(): real(0.f), img(0.f) {}
	Complex(float a, float b): real(a), img(b) {}
	friend ostream& operator <<(ostream&, Complex&);
	friend istream& operator >>(istream&, Complex&);
};

ostream& operator <<(ostream& out, Complex& c)
{
	out << c.real << "+j" << c.img;
	return out;
}
istream& operator >>(istream& in, Complex& c)
{
	in >> c.real >> c.img;
	return in;
}

int main()
{
	Complex A;
	cout << "Enter a complex number: ";
	cin >> A;
	cout << "You entered: " << A << endl;

	return 0;
}
\end{lstlisting}
%\nocite{*}
\bibliographystyle{plain}
\bibliography{OOPref}
\end{document}
