\documentclass[12pt,a4paper]{article}

% Margins.
\setlength{\oddsidemargin}{0in}
\setlength{\evensidemargin}{0in}
\setlength{\headheight}{12pt}
\setlength{\headsep}{42pt}
\setlength{\topmargin}{-54pt}
\setlength{\textwidth}{6.5in}
\setlength{\textheight}{10in}

\usepackage{amsmath}
\usepackage{float}
\usepackage{graphicx}
\usepackage[hyphens]{url}
\usepackage{hyperref}	% Clickable links to figures, references and urls.

% Drawing.
\usepackage{pgf}
\usepackage{tikz}

% Listings for formatting code.
\usepackage{listings}
\usepackage{textcomp}
% General options.
\lstset{breaklines=true, basicstyle=\small\ttfamily, tabsize=4, numbers=left, stepnumber=1, frame=single, showstringspaces=false, upquote=true}
% C++ specific high-lighting. Comments are 50/50 shades of green/black and strings coloured with 60/40 red/black mixture.
\lstset{language=[ISO]C++, commentstyle=\color{green!50!black}, keywordstyle=\color{blue}, stringstyle=\color{red!60!black}}

%opening
\title{\vspace{-2cm}Programming for Engineers II\\Class 10\\Macros\\Dynamic Memory Allocation}
\author{Attique Dawood}

\begin{document}
\maketitle
\section{Announcements}
\begin{itemize}
\item Quiz will be held tomorrow.
\end{itemize}
\section{Revision}
\begin{itemize}
\item Function Overloading.
\item Inline Functions.
\item Constructors: same as class name, no return type, only called once at object creation, can be overloaded.
\end{itemize}
\section{Macros}
\begin{itemize}
\item Replaces indicated patterns when preprocessing.
\item Can define symbols.
\item Can also make macro-ed functions.
\item Can also make a certain portion of code compile.
\end{itemize}
\begin{lstlisting}[caption={Using macro to define size}]
#include <iostream>
using std::cout;
using std::endl;

#define SIZE 10

int main()
{
	int list[SIZE];
	for (int i=0; i<SIZE; i++)
		list[i] = i;
	for (int i=0; i<SIZE; i++)
		cout << list[i] << endl;
	
	return 0;
}
\end{lstlisting}
\begin{lstlisting}[caption={Using macro to specify data type}]
#include <iostream>
using std::cin;
using std::cout;
using std::endl;

// Changing TYPE would make this work for any type of data.
// Try changing TYPE to short int, float or double.
#define TYPE int

TYPE sum(TYPE a, TYPE b)
{
	TYPE result = a+b;
	return result;
}

int main()
{
	TYPE x, y;
	cout << "Enter two numbers to sum:" << endl;
	cin >> x >> y;
	cout << "Sum is: " << sum(x, y) << endl;
	
	return 0;
}
\end{lstlisting}
\begin{lstlisting}[caption={Using macro to execute only portions of code}]
#include <iostream>
using std::cin;
using std::cout;
using std::endl;

#define Q 1

#if Q == 1
int main()
{
	cout << "This is question number 1" << endl;
	
	return 0;
}
#endif

#if Q == 2
int main()
{
	cout << "This is question number 2" << endl;
	
	return 0;
}
#endif

#if Q == 3
int main()
{
	cout << "This is question number 3" << endl;
	
	return 0;
}
#endif
\end{lstlisting}
\section{New and Delete}
\begin{itemize}
\item \verb|new| and \verb|delete| are the C++ equivalent of \verb|malloc()| and \verb|free()|.
\item Allocating memory to a single variable.
\item Arrays.
\item Deallocating memory.
\item Every piece of memory allocated using \verb|new| MUST be deallocated using \verb|delete| before the program exits.
\end{itemize}
\begin{lstlisting}[caption={new and delete}]
#include <iostream>
using std::cin;
using std::cout;
using std::endl;

int main()
{
	int x = 3;
	int *xp;
	xp = &x; // Pointing xp -> x.
	
	cout << "Value of x = " << x << endl;
	cout << "Address of x (&x) = " << &x << endl;
	cout << "Value of pointer xp = " << xp << endl;
	cout << "Value of x using *xp = " << *xp << endl;
	
	// Using xp to dynamically allocate memory.
	xp = new int; // xp now points to dynamically allocated memory.
	*xp = 33;
	cout << "Value stored by *xp is = " << *xp << endl;
	delete xp; // Dellocates memory assigned to xp.
	
	// Creating and deleting a dynamic array.
	xp = new int[10];
	delete []xp;
	
	return 0;
}
\end{lstlisting}
%\nocite{*}
%\bibliographystyle{plain}
%\bibliography{OOPref}
\end{document}
