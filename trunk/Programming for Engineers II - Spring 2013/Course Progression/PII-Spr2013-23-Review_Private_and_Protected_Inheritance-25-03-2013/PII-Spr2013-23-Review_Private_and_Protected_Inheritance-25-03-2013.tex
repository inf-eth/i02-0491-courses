\documentclass[12pt,a4paper]{article}

% Margins.
\setlength{\oddsidemargin}{0in}
\setlength{\evensidemargin}{0in}
\setlength{\headheight}{12pt}
\setlength{\headsep}{42pt}
\setlength{\topmargin}{-54pt}
\setlength{\textwidth}{6.5in}
\setlength{\textheight}{10in}

\usepackage{amsmath}
\usepackage{float}
\usepackage{graphicx}
\usepackage[hyphens]{url}
\usepackage{hyperref}	% Clickable links to figures, references and urls.
\usepackage{enumerate}

% Drawing.
\usepackage{pgf}
\usepackage{tikz}

% Listings for formatting code.
\usepackage{listings}
\usepackage{textcomp}
% General options.
\lstset{breaklines=true, basicstyle=\small\ttfamily, tabsize=4, numbers=left, stepnumber=1, frame=single, showstringspaces=false, upquote=true}
% C++ specific high-lighting. Comments are 50/50 shades of green/black and strings coloured with 60/40 red/black mixture.
\lstset{language=[ISO]C++, commentstyle=\color{green!50!black}, keywordstyle=\color{blue}, stringstyle=\color{red!60!black}}

%opening
\title{\vspace{-2cm}Programming for Engineers II\\Class 23\\Private and Protected Inheritance\\Review}
\author{Attique Dawood}
\date{March 25, 2013\\[0.2cm] Last Modified: \today}
\begin{document}
\maketitle
\section{Announcements}
\begin{itemize}
\item Bonus project proposal deadline extended to Tuesday, 26th March.
\end{itemize}
\section{Revision}
\subsection{\texttt{friend}s}
\begin{itemize}
\item Sometimes an external function needs to access private members of a class. In this case that function must be made \verb|friend| of that class.
\item Input/output stream operators \verb|<<| and \verb|>>| must be made \verb|friend|s of the class for which they are being overloaded. This is because you cannot change the \verb|cout| class.
\item \verb|friend| \underline{functions are not members of the class}. Remember, member functions can already access private members.
\item Arithmetic operators can also be defined outside the class by declaring them friends.
\begin{lstlisting}[caption={operator+ prototype as member function and as \texttt{friend}}]
Complex Complex::operator+ (Complex&); // member function
Complex operator+ (Complex&, Complex&); // global friend function
\end{lstlisting}
\begin{lstlisting}[caption={operator+ implemented as \texttt{friend}}]
#include <iostream>
using namespace std;

class Complex
{
	private:
	float real;
	float img;
	public:
	Complex(): real(0.f), img(0.f) {}
	Complex(float a, float b): real(a), img(b) {}
	friend Complex operator+(Complex&, Complex&);
};

// operator+ function in this case is external to the class.
// It cannot access private members of the class unless declared a `friend' in the class.
Complex operator+(Complex& lhs, Complex& rhs)
{
	return Complex(lhs.real+rhs.real, lhs.img+rhs.img);
}

int main()
{
	Complex A(2.f,3.3f);
	Complex B(2.32f,-1.3f);
	Complex C;
	C = A + B;

	return 0;
}
\end{lstlisting}
\end{itemize}
\subsection{Inheritance}
\begin{itemize}
\item Inheritance provides a mechanism to make new classes from existing classes.
\item Existing class is called `base' class and inherited class is called `derived' class.
\item Derived class contains all the data members and functions of base class.
\item Base class does not need to re--implement the functions already present in base class, therefore, existing code can be re--used.
\item Derived class can contain new data members and functions. This extends the capabilities of the base class.
\item Derived class can also change or re--implement functions in the base class. This provides flexibility in what derived class wants to do.
\end{itemize}
\subsection{Over--Riding Base Class Functions}
\begin{itemize}
\item If derived class contains a function of the same name as in base class then derived class function will `over--ride' base class function.
\item Base class function is effectively `hidden'. Normal function call with dot operator will only call derived class function.
\item To access base class function scope resolution operator is used with dot operator and base class name.
\end{itemize}
\section{Multiple Files}
\begin{itemize}
\item Until now you have only used a single file for your code. When creating large programs, code is spread between different header and code files according to relevance.
\item Class definition is usually put in a file with .hpp extension. For Complex class this header should be named Complex.hpp. .hpp refers to C++ header whereas .h headers are normally used for C code. Headers without any extension are standard C++ headers like \verb|<iostream>|.
\item Class implementation is put in a file with class name and .cpp extension. For Complex class this would be Complex.cpp.
\item In main() header file is included as normal.
\end{itemize}
\begin{minipage}{7.5cm}
\begin{lstlisting}[caption={Complex.hpp}]
class Complex
{
	private:
	float real, img;
	public:

	void Input();
	void Display();
};
\end{lstlisting}
\end{minipage}
\hspace*{1cm}
\begin{minipage}{7.5cm}
\begin{lstlisting}[caption={Complex.cpp}]
#include <iostream>
#include <Complex.hpp>
using namespace std;

void Complex::Input()
{
	cin >> real >> img;
}
void Complex::Display()
{
	cout << real << img;
}
\end{lstlisting}
\end{minipage}
\begin{lstlisting}[caption={main.cpp}]
#include <Complex.hpp>

int main()
{
	Complex A;
	A.Input();
	A.Display();

	return 0;
}
\end{lstlisting}
\section{Ambiguities in Multiple Inheritance}
\begin{enumerate}
\item class A and class B both have functions called disp(). class C being derived from both class A and class B will not know which disp() to call if using dot operator. This can be resolved with either over--riding and making a new disp() function in class C OR using scope resolution operator.
\item class A contains disp(). class B and class C are derived from class A. If another class D is derived from both B and C then class D won't know which disp() to call.
\end{enumerate}
\section{Private and Protected Members}
\begin{itemize}
\item Data or functions in a class can be either private, protected or public.
\item In main() or outside the class, private and protected data/functions (members) cannot be accessed.
\item A derived class cannot access private members of its base class. It can access protected and public members.
\end{itemize}
This can be summarised as:
\begin{enumerate}
\item Keeping members private is a way of saying, ``I do not want this accessible anywhere, in derived class(es) or in main()''.
\item Keeping members protected is a way of saying, ``I want this to be accessible in derived class but not in main()''.
\item Keeping members public is a way of saying, ``I want this to be accessible everywhere''.
\end{enumerate}
\begin{lstlisting}[caption={private and protected members}]
class A
{
	private:
	int x;
	protected:
	int y;
	public:
	int z;
};
class B: public A
{
	public:
	void Test()
	{
		x = 3; // ERROR! cannot access private members of class A.
		y = 3; // OK
		z = 3; // OK
	}
};
int main()
{
	B Bobj;
	Bobj.x = 0; // ERROR!
	Bobj.y = 0; // ERROR!
	Bobj.z = 0; // OK
	
	return 0;
}
\end{lstlisting}
What happens if we use \verb|class B: private A|? All protected and public members in A will now become private in B. They cannot be accessed in main().
%\nocite{*}
\bibliographystyle{plain}
\bibliography{OOPref}
\end{document}
