\documentclass[12pt,a4paper]{article}

% Margins.
\setlength{\oddsidemargin}{0in}
\setlength{\evensidemargin}{0in}
\setlength{\headheight}{12pt}
\setlength{\headsep}{42pt}
\setlength{\topmargin}{-54pt}
\setlength{\textwidth}{6.5in}
\setlength{\textheight}{10in}

\usepackage{amsmath}
\usepackage{float}
\usepackage{graphicx}
\usepackage[hyphens]{url}
\usepackage{hyperref}	% Clickable links to figures, references and urls.
\usepackage{enumerate}

% Drawing.
\usepackage{pgf}
\usepackage{tikz}

% Listings for formatting code.
\usepackage{listings}
\usepackage{textcomp}
% General options.
\lstset{breaklines=true, basicstyle=\small\ttfamily, tabsize=4, numbers=left, stepnumber=1, frame=single, showstringspaces=false, upquote=true}
% C++ specific high-lighting. Comments are 50/50 shades of green/black and strings coloured with 60/40 red/black mixture.
\lstset{language=[ISO]C++, commentstyle=\color{green!50!black}, keywordstyle=\color{blue}, stringstyle=\color{red!60!black}}

%opening
\title{\vspace{-2cm}Programming for Engineers II\\Class 26\\Advanced Pointers\\Keyword \texttt{this}}
\author{Attique Dawood}

\begin{document}
\maketitle
\section{Announcements}
\begin{itemize}
\item None.
\end{itemize}
\section{Revision}
\begin{itemize}
\item Dynamic memory allocation using pointer. Object pointers and \verb|->| notation.
\item Pointers and arrays.
\item Pointer levels: double and triple pointers.
\end{itemize}
\section{\texttt{this} Pointer}
\begin{itemize}
\item Inside class functions, we have access to a `magic' pointer named \verb|this|.
\item This pointer points to calling object itself.
\end{itemize}
\begin{lstlisting}[caption={\texttt{this} pointer}]
#include <iostream>
using namespace std;

class Complex
{
	private:
	float real;
	float img;
	public:
	void Input()
	{
		cin >> this->real;
		cin >> this->img;
	}
	void Display()
	{
		cout << this->real << endl;
		cout << this->img << endl;
	}
};

int main()
{
	Complex c;
	c.Input();
	c.Display();

	return 0;
}
\end{lstlisting}
%\nocite{*}
\bibliographystyle{plain}
\bibliography{OOPref}
\end{document}
