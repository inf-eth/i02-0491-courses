\documentclass[12pt,a4paper]{article}

% Margins.
\setlength{\oddsidemargin}{0in}
\setlength{\evensidemargin}{0in}
\setlength{\headheight}{12pt}
\setlength{\headsep}{42pt}
\setlength{\topmargin}{-54pt}
\setlength{\textwidth}{6.5in}
\setlength{\textheight}{10in}

\usepackage{amsmath}
\usepackage{float}
\usepackage{graphicx}
\usepackage[hyphens]{url}
\usepackage{hyperref}	% Clickable links to figures, references and urls.
\usepackage{enumerate}

% Drawing.
\usepackage{pgf}
\usepackage{tikz}

% Listings for formatting code.
\usepackage{listings}
\usepackage{textcomp}
% General options.
\lstset{breaklines=true, basicstyle=\small\ttfamily, tabsize=4, numbers=left, stepnumber=1, frame=single, showstringspaces=false, upquote=true}
% C++ specific high-lighting. Comments are 50/50 shades of green/black and strings coloured with 60/40 red/black mixture.
\lstset{language=[ISO]C++, commentstyle=\color{green!50!black}, keywordstyle=\color{blue}, stringstyle=\color{red!60!black}}

%opening
\title{\vspace{-2cm}Programming for Engineers II\\Class 22\\Inheritance}
\author{Attique Dawood}
\date{March 15, 2013\\[0.2cm] Last Modified: \today}
\begin{document}
\maketitle
\section{Announcements}
\begin{itemize}
\item Assignment based quiz on Monday.
\end{itemize}
\section{Revision}
\begin{itemize}
\item Sometimes an external function needs to access private members of a class. In this case that function must be made \verb|friend| of that class.
\item Input/output stream operators \verb|<<| and \verb|>>| must be made \verb|friend|s of the class for which they are being overloaded.
\item \verb|friend| \underline{functions are not members of the class}. Remember, member functions can already access private members.
\item Arithmetic operators can also be defined outside the class by declaring them friends.
\begin{lstlisting}[caption={operator+ implemented as \texttt{friend}}]
#include <iostream>
using namespace std;

class Complex
{
	private:
	float real;
	float img;
	public:
	Complex(): real(0.f), img(0.f) {}
	Complex(float a, float b): real(a), img(b) {}
	friend Complex operator+(Complex&, Complex&);
};

// operator+ function in this case is external to the class.
// It cannot access private members of the class unless declared a `friend' in the class.
Complex operator+(Complex& lhs, Complex& rhs)
{
	return Complex(lhs.real+rhs.real, lhs.img+rhs.img);
}

int main()
{
	Complex A(2.f,3.3f);
	Complex B(2.32f,-1.3f);
	Complex C;
	C = A + B;

	return 0;
}
\end{lstlisting}
\end{itemize}
\section{Inheritance}
\begin{itemize}
\item Inheritance provides a mechanism to make new classes from existing classes.
\item Existing class is called `base' class and inherited class is called `derived' class.
\item Derived class contains all the data members and functions of base class.
\item Base class does not need to re--implement the functions already present in base class, therefore, existing code can be re--used.
\item Derived class can contain new data members and functions. This extends the capabilities of the base class.
\item Derived class can also change or re--implement functions in the base class. This provides flexibility in what derived class wants to do.
\end{itemize}
\section{Constructor and Destructor Calls}
\begin{itemize}
\item Constructor of base class is called first and then constructor(s) of derived classes are called.
\item Destructor calls happen in reverse order.
\item If base class contains multiple constructors, derived class can specify the constructor it wants to call in the member initialiser list.
\end{itemize}
\begin{lstlisting}[caption={Example of Inheritance}]
#include <iostream>
using namespace std;

// AC = A constructor.
// BC = B constructor.
// AD = A destructor.
// BD = B destructor.
class A
{
	private:
	int x;
	public:
	A(): x(0)
	{
		cout << "AC" << endl;
	}
	A(int px): x(px)
	{
		cout << "AC" << endl;
	}
	~A()
	{
		cout << "AD" << endl;
	}
};
// class B is derived from class A.
// B will contain both x and y
class B: public A
{
	private:
	int y;
	public:
	B(): y(0) // This constructor will implicitly call A()
	{
		cout << "BC" << endl;
	}
	B(int py): A(), y(py) // This will explicitly call A()
	{
		cout << "BC" << endl;
	}
	B(int px, int py): A(py), y(py) // This will explicitly call A(int)
	{
		cout << "BC" << endl;
	}
	~B()
	{
		cout << "BD" << endl;
	}
};
int main()
{
	// Constructor calls in creation of B object:
	// 1. Constructor of A is called implicitly and x is set to 0.
	// 2. Constructor of B is called and y is set to 0.
	B Bobj;
	
	return 0;	
}
// Destructors are called in reverse order.
// First destructor of B is called and then destructor of A.
// Output of the program is:
// AC
// BC
// BD
// AD
\end{lstlisting}
\section{Over--Riding Base Class Functions}
\begin{itemize}
\item If derived class contains a function of the same name as in base class then derived class function will `over--ride' base class function.
\item Base class function is effectively `hidden'. Normal function call with dot operator will only call derived class function.
\item To access base class function scope resolution operator is used with dot operator and base class name.
\end{itemize}
\begin{lstlisting}[caption={Over--Riding Base Class Function}]
#include <iostream>
#include <string> // For string class.
using namespace std;

class Person
{
	private:
	string Name;
	int Age;
	public:
	Person(): Name("No name"), Age(-1)
	{
	}
	Person(string pName, int pAge): Name(pName), Age(pAge)
	{
	}
	void Input()
	{
		cin >> Name >> Age;
	}
	void Display()
	{
		cout << "Name: " << Name << endl
			 << "Age:  " << Age << endl;
	}
};

class Student: public Person
{
	private:
	float CGPA;
	public:
	Student(): Person(), CGPA(-1.f)
	{
	}
	Student(string pName, int pAge, float pCGPA): Person(pName, pAge), CGPA(pCGPA)
	{
	}
	// Input/display functions over-ride the ones in base class but call them explicitly.
	void Input()
	{
		Person::Input();
		cin >> CGPA;
	}
	void Display()
	{
		Person::Display();
		cout << "CGPA: " << CGPA << endl;
	}
};

int main()
{
	// Student object using default constructors.
	Student S1;
	
	// Student object by passing values.
	Student S2("Ahsan", 19, 3.97f);
	
	S1.Input();
	S1.Display(); // Default display: Displays name, age and cgpa.
	S1.Person::Display(); // This will only display name and age.
	
	return 0;
}
\end{lstlisting}
%\nocite{*}
\bibliographystyle{plain}
\bibliography{OOPref}
\end{document}
