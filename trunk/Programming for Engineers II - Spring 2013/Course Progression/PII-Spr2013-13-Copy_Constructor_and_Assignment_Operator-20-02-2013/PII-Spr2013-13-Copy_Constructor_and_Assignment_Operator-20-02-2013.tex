\documentclass[12pt,a4paper]{article}

% Margins.
\setlength{\oddsidemargin}{0in}
\setlength{\evensidemargin}{0in}
\setlength{\headheight}{12pt}
\setlength{\headsep}{42pt}
\setlength{\topmargin}{-54pt}
\setlength{\textwidth}{6.5in}
\setlength{\textheight}{10in}

\usepackage{amsmath}
\usepackage{float}
\usepackage{graphicx}
\usepackage[hyphens]{url}
\usepackage{hyperref}	% Clickable links to figures, references and urls.

% Drawing.
\usepackage{pgf}
\usepackage{tikz}

% Listings for formatting code.
\usepackage{listings}
\usepackage{textcomp}
% General options.
\lstset{breaklines=true, basicstyle=\small\ttfamily, tabsize=4, numbers=left, stepnumber=1, frame=single, showstringspaces=false, upquote=true}
% C++ specific high-lighting. Comments are 50/50 shades of green/black and strings coloured with 60/40 red/black mixture.
\lstset{language=[ISO]C++, commentstyle=\color{green!50!black}, keywordstyle=\color{blue}, stringstyle=\color{red!60!black}}

%opening
\title{\vspace{-2cm}Programming for Engineers II\\Class 13\\Copy Constructor and Assignment Operator}
\author{Attique Dawood}

\begin{document}
\maketitle
\section{Announcements}
\begin{itemize}
\item Paper pattern is similar to ABQ. Class diagram, writing and understanding code, constructor/destructor and operators.
\item Today's topics taken from \cite{Ref:CPP-Copy-Constructor-and-Assignment-Operator}
\end{itemize}
\section{Revision}
\begin{itemize}
\item Memory leaks. Using NULL pointer for safety.
\item A destructor, very much like constructor, is a special function that is automatically called every time an object is destroyed.
\item Constructor and destructor calls when dynamically creating objects.
\end{itemize}
\section{Copy Constructor}
\begin{itemize}
\item Default constructor and user--defined constructor. If you do not define a default constructor then compiler provides one implicitly.
\item Copy constructor can make copies of object at creation. If user does not define a copy constructor then compiler provides one implicitly.
\item Copy constructor is used to create an object when another object is passed as constructor argument.
\item Copy constructor does a member--wise copy of object elements (data members).
\end{itemize}
\begin{lstlisting}[caption={Copy Constructor}]
// Notice, complex class doesn't define any constructors.
// 1. Default constructor will be provided by compiler.
// Complex(){}
// 2. Also, copy constructor will be provided by compiler.
// Prototypes for copy constructor.
// Complex(Complex &);
// Complex(const Complex &);

class Complex
{
	private:
	float real;
	float img;
};

int main()
{
	Complex x;
	Complex y(x);
	
	return 0;
}
\end{lstlisting}
\section{Assignment Operator =}
\begin{itemize}
\item This is another facility provided by compiler if not explicitly defined by user.
\item Assignment operator is just a member function.
\item An assignment operator essentially does a member--wise copy of elements.
\item Difference between copy constructor and assignment operator? Copy constructor can only be called at object creation because it's a constructor.
\end{itemize}
\begin{lstlisting}[caption={Calling Assignment Operator}]
class Complex
{
	private:
	float real;
	float img;
};

int main()
{
	Complex x, y;
	Complex z(x);  // Copy constructor called.
	Complex y = x; // Assignment operator called. y was already created using default constructor.
	
	return 0;
}
\end{lstlisting}
\begin{lstlisting}[caption={Defining Assignment Operator}]
class Complex
{
	private:
	float real;
	float img;
	public:
	Complex& operator=(const Complex& rhs)
	{
		real = rhs.real;
		img = rhs.img;
		return (*this);
	}	
};

int main()
{
	Complex x, y;
	Complex z(x);  // Copy constructor called.
	Complex y = x; // Assignment operator called.
	
	return 0;
}
\end{lstlisting}
\section{Putting it Together}
\begin{lstlisting}[caption={User and Compiler Code}]
// User code for class.
class Complex
{
	private:
	float real;
	float img;
};

// Compiler code.
class Complex
{
	private:
	float real;
	float img;
	public:
	// Default constructor.
	Complex()
	{
	}
	// Copy constructor.
	Complex(Complex& other): real(other.real), img(other.img)
	{
	}
	// Assignment operator.
	Complex& operator=(const Complex& rhs)
	{
		real = rhs.real;
		img = rhs.img;
		return (*this);
	}
};
\end{lstlisting}
%\nocite{*}
\bibliographystyle{plain}
\bibliography{OOPref}
\end{document}
