\documentclass[12pt,a4paper]{article}

% Margins.
\setlength{\oddsidemargin}{0in}
\setlength{\evensidemargin}{0in}
\setlength{\headheight}{12pt}
\setlength{\headsep}{42pt}
\setlength{\topmargin}{-54pt}
\setlength{\textwidth}{6.5in}
\setlength{\textheight}{9in}

\usepackage{amsmath}
\usepackage{float}
\usepackage{graphicx}
\usepackage[hyphens]{url}
\usepackage{hyperref}	% Clickable links to figures, references and urls.

% Drawing.
\usepackage{pgf}
\usepackage{tikz}

% Listings for formatting code.
\usepackage{listings}
% General options.
\lstset{breaklines=true, basicstyle=\small\ttfamily, tabsize=4, numbers=left, stepnumber=1, frame=single, showstringspaces=false}
% C++ specific high-lighting. Comments are 50/50 shades of green/black and strings coloured with 60/40 red/black mixture.
\lstset{language=[ISO]C++, commentstyle=\color{green!50!black}, keywordstyle=\color{blue}, stringstyle=\color{red!60!black}}

%opening
\title{\vspace{-2cm}Programming for Engineers II\\Class 01\\Introduction}
\author{Attique Dawood}

\begin{document}
\maketitle
\section{About the Course}
\textbf{Course Title:} Programming for Engineers II\\
\textbf{Pre-requisite:} Programming for Engineers I\\
\textbf{Instructor:} Attique Dawood\\
\textbf{Contact:} attique DOT dawood AT nu DOT edu DOT pk\\
\textbf{Text Book:} C++: How to Program by Deitel \& Deitel\\
\textbf{Reference Book:} Object Oriented Programming in C++ by Robert Lafore\\
\section{Guidelines}
\begin{itemize}
\item \textbf{\underline{Your goal is to do all assignments (including lab work) by yourself.}}
\item Do read the book. It is very helpful.
\item You will only learn with practice.
\end{itemize}
\section{Course Progression}
\begin{itemize}
\item Two sessionals and a final.
\item Tri--weekly quizzes based on assignments carry significant weight and will be marked as `assignments'.
\item Weekly quizzes will taken at the end of week.
\item All quizzes will be announced.
\item Students are required to do a project. Creative ideas are encouraged.
\end{itemize}
\begin{table}[H]
\begin{center}
\vspace{0.3cm}
	\begin{tabular}{llc}
	\hline \hline
		\rule{0pt}{2.6ex} & \textbf{Type of Assessment} & \textbf{Marks}\\
		\hline
		1 \rule{0pt}{2.6ex} & Weekly assessment (Quiz) & 5\\
		2 & Quiz based on assigned homework& 12\\
		3 & Sessional I & 10\\
		4 & Sessional II & 10\\
		5 & Class participation & 5\\
		6 & Project & 8\\
		7 & Final Exam & 50\\
	\hline \hline
	\end{tabular}
\end{center}
\label{Marks Distribution}
\caption{Marks Distribution}
\end{table}
\section{C and C++}
\begin{itemize}
\item This course is where you learn \textbf{programming methodology}.
\item You will have to un--learn some concepts from previous course.
\item We will be following ANSI standards.
\item C is a procedural language while C++ is an object--oriented language.
\item C is a subset of C++. C++ contains C and much more. It enhances (extends) and provides better alternatives to most C procedures.
\item All the C is C++ since C is a subset of C++.
\item Practical programs are generally quite large and must be broken down into different modules.
\item Program maintenance is an important issue (patches?).
\end{itemize}
\section{A Simple C++ Program}
\begin{lstlisting}[caption={A Basic C++ Program}]
#include <iostream> // Input/output stream library.
int main()
{
	std::cout << "Hello world" << std::endl;
	
	return 0;
}
\end{lstlisting}
\begin{itemize}
\item Standard C++ header files do not have .h extensions.
\item Format specifiers can be used in \verb|cout| directives (\verb|\r, \n, \t| etc.).
\end{itemize}
\section{Namespaces}
\begin{itemize}
\item The functions (or objects) in C++ headers are enclosed in namespaces for modularity.
\item \texttt{std} is a standard namespace.
\item The namespace functions or entire namespace can be `included' for easier access.
\end{itemize}
\begin{lstlisting}[caption={The `using' Directive}]
#include <iostream> // Input/output stream library.
using std::cout;
using std::endl;
int main()
{
	cout << "Hello world" << endl;
	
	return 0;
}
\end{lstlisting}
\begin{lstlisting}[caption={Including a Namespace}]
#include <iostream> // Input/output stream library.
using namespace std;
int main()
{
	cout << "Hello world" << endl;
	
	return 0;
}
\end{lstlisting}
\section{How Namespaces Work?}
A very good tutorial is available at \url{http://www.cplusplus.com/doc/tutorial/namespaces/}
\end{document}
