\documentclass[12pt,a4paper]{article}

% Margins.
\setlength{\oddsidemargin}{0in}
\setlength{\evensidemargin}{0in}
\setlength{\headheight}{12pt}
\setlength{\headsep}{42pt}
\setlength{\topmargin}{-54pt}
\setlength{\textwidth}{6.5in}
\setlength{\textheight}{10in}

\usepackage{amsmath}
\usepackage{float}
\usepackage{graphicx}
\usepackage[hyphens]{url}
\usepackage{hyperref}	% Clickable links to figures, references and urls.

% Drawing.
\usepackage{pgf}
\usepackage{tikz}

% Listings for formatting code.
\usepackage{listings}
\usepackage{textcomp}
% General options.
\lstset{breaklines=true, basicstyle=\small\ttfamily, tabsize=4, numbers=left, stepnumber=1, frame=single, showstringspaces=false, upquote=true}
% C++ specific high-lighting. Comments are 50/50 shades of green/black and strings coloured with 60/40 red/black mixture.
\lstset{language=[ISO]C++, commentstyle=\color{green!50!black}, keywordstyle=\color{blue}, stringstyle=\color{red!60!black}}

%opening
\title{\vspace{-2cm}Programming for Engineers II\\Class 12\\Destructor}
\author{Attique Dawood}
\date{February 18, 2013\\[0.2cm] Last Modified: \today}
\begin{document}
\maketitle
\section{Announcements}
\begin{itemize}
\item None.
\end{itemize}
\section{Revision}
\begin{itemize}
\item Constructors: same as class name, no return type, only called once at object creation, can be overloaded.
\item Constant variables. Can only be initialised at declaration.
\item If defined as class member, MUST be initialised using ``member initialiser list''.
\item Member initialiser list is a comma separated list of variables that are initialised before actual body of the constructor.
\item Only constructors are allowed to have a member initialiser list.
\item Non-constant members can also  be initialised using member initialiser list and it is generally the recommended way of initialising variables. But, \textbf{member initialiser list is the only way to initialise constant members}.
\end{itemize}
\section{Memory Leaks}
\begin{itemize}
\item What happens if you allocate memory without deallocating memory?
\item Previously allocated memory remains allocated. But you're not using it and you cannot possibly use it again. Why? Lost pointer address.
\item This is called a memory leak.
\item Memory leak is the greatest pitfall when using dynamic memory allocation. Even good programs and computer games suffer from memory leaks.
\item Make sure you deallocate memory before allocating a new piece of memory.
\item Memory must also be deallocated before the program exits.
\end{itemize}
\begin{lstlisting}[caption={Typical Memory Leak}]
int main()
{
	int *x;
	x = new int;
	*x = 4;

	x = new int;  // Lost address of previously allocated memory!
	*x = 6;
	
	return 0;
}
\end{lstlisting}
\section{Using NULL Pointers}
\begin{itemize}
\item If a pointer isn't allocated memory or isn't pointing at anything, assign it NULL.
\item This is a good programming practice. It not only helps prevent memory leaks but also is a check if pointer is actually pointing at anything.
\item Trying to access value of a pointer if memory isn't allocated will result in a program crash.
\item Trying to deallocate memory if no memory was allocated will also result in a crash.
\end{itemize}
\begin{lstlisting}[caption={Two Typical Program Crashes}]
int main()
{
	int *x;
	*x = 4;	 // CRASH1! x contains garbage address.
	delete x;// CRASH2! can't delete a random garbage address.
	
	return 0;
}
\end{lstlisting}
\begin{lstlisting}[caption={Using NULL to Properly Handle Pointers}]
int main()
{
	int *x = NULL;	// Pointer initialisation with NULL.
	
	// Allocation.
	if (x==NULL)
		x = new int;
	else
	{
		delete x;
		x = new int;
	}
		
	// Use pointer.
	*x = 4;
	
	// Explicit deallocation. Notice pointer is set to NULL.
	if (x!=NULL)
	{
		delete x;		// Deallocating memory.
		x = NULL;		// Pointer assigned NULL.
	}

	// New memory allocation.
	if (x==NULL)
		x = new int;
	else
	{
		delete x;
		x = new int;
	}

	// Pointer usage.
	*x = 6;
	
	// Deallocation before exiting.
	if (x!=NULL)
	{
		delete x;		// Deallocating memory.
		x = NULL;		// Pointer assigned NULL.
	}
	
	return 0;
}
\end{lstlisting}

\section{$\sim$ Destructor}
\begin{itemize}
\item A destructor, very much like constructor, is a special function that is automatically called every time an object is destroyed.
\item Destructor has the same name as class but with a preceding $\sim$. Also, it doesn't return anything.
\item Typical function of destructor is to deallocate any dynamically allocated memory when object life ends or it is destroyed.
\end{itemize}
\begin{lstlisting}[caption={Constructor and Destructor}]
#include <iostream>
using std::cout;
using std::endl;

class Simple
{
	private:
	const int x;
	int y;
	public:
	Simple();  // Default constructor.
	~Simple(); // Destructor.
};
// Notice member initialiser list after ':'
Simple::Simple(): x(0), y(0)
{
	cout << "Constructor called." << endl;
}
Simple::~Simple()
{
	cout << "Destructor called." << endl;
}

int main()
{
	Simple test;
	cout << "Object created in main(). Now exiting..." << endl;
	return 0;
}
\end{lstlisting}
\begin{lstlisting}[caption={Constructor and Destructor With Dynamic Allocation}]
#include <iostream>
using std::cout;
using std::endl;

class Simple
{
	private:
	const int x;
	int y;
	public:
	Simple();  // Default constructor.
	~Simple(); // Destructor.
};
// Notice member initialiser list after ':'
Simple::Simple(): x(0), y(0)
{
	cout << "Constructor called." << endl;
}
Simple::~Simple()
{
	cout << "Destructor called." << endl;
}

int main()
{
	Simple *test;
	
	cout << "Creating object..." << endl;
	test = new Simple;
	
	cout << "Deallocating object memory..." << endl;
	delete test;
	
	cout << "Exiting main()..." << endl;
	return 0;
}
\end{lstlisting}
\section{Puting it All Together: The Dynamic Matrix Class}
\begin{lstlisting}[caption={Dynamic Matrix Class}]
#include <iostream>
using std::cin;
using std::cout;
using std::endl;

class matrix
{
private:
	float *Element;	// 1D matrix of size 1xSize.
	int Size;
public:
	matrix();
	matrix(int);
	~matrix();
	void SetSize(int);
	void Input();
	void Display();
	void Add(matrix&);
};

// Default constructor.
matrix::matrix(): Element(NULL), Size(0)
{
}
// User-defined constructor that takes in size.
matrix::matrix(int pSize)
{
	Element = new float[pSize];
	Size=pSize;

	// Initialising matrix.
	for (int i=0; i<Size; i++)
		Element[i] = 0.f;
}
// Destructor.
matrix::~matrix()
{
	if (Element != NULL)
		delete []Element;
}
// Resize matrix.
void matrix::SetSize(int pSize)
{
	if (Element == NULL)
		delete []Element;

	Element = new float[pSize];
	Size=pSize;
}
// Input matrix.
void matrix::Input()
{
	cout << "Enter matrix elements: ";
	for (int i=0; i<Size; i++)
		cin >> Element[i];
}
// Display matrix.
void matrix::Display()
{
	for (int i=0; i<Size; i++)
		cout << Element[i] << " ";

	cout << endl;
}
// Adds passed matrix into calling matrix.
void matrix::Add(matrix& m)
{
	for (int i=0; i<Size; i++)
		Element[i] = Element[i] + m.Element[i];
}

// Main.
int main()
{
	matrix matrix1(4);

	matrix matrix2;
	matrix2.SetSize(4);

	matrix1.Input();
	matrix2.Input();

	matrix1.Add(matrix2);

	cout << "Result is: ";
	matrix1.Display();

	// Resizing matrices.
	matrix1.SetSize(2);
	matrix2.SetSize(2);

	matrix1.Input();
	matrix2.Input();

	matrix1.Add(matrix2);
	cout << "Result is: ";
	matrix1.Display();

	return 0;
}
\end{lstlisting}
%\nocite{*}
%\bibliographystyle{plain}
%\bibliography{OOPref}
\end{document}
