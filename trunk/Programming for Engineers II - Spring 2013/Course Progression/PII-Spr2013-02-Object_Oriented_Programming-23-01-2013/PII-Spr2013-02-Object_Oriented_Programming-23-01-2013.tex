\documentclass[12pt,a4paper]{article}

% Margins.
\setlength{\oddsidemargin}{0in}
\setlength{\evensidemargin}{0in}
\setlength{\headheight}{12pt}
\setlength{\headsep}{42pt}
\setlength{\topmargin}{-54pt}
\setlength{\textwidth}{6.5in}
\setlength{\textheight}{10in}

\usepackage{amsmath}
\usepackage{float}
\usepackage{graphicx}
\usepackage[hyphens]{url}
\usepackage{hyperref}	% Clickable links to figures, references and urls.

% Drawing.
\usepackage{pgf}
\usepackage{tikz}

% Listings for formatting code.
\usepackage{listings}
% General options.
\lstset{breaklines=true, basicstyle=\small\ttfamily, tabsize=4, numbers=left, stepnumber=1, frame=single, showstringspaces=false}
% C++ specific high-lighting. Comments are 50/50 shades of green/black and strings coloured with 60/40 red/black mixture.
\lstset{language=[ISO]C++, commentstyle=\color{green!50!black}, keywordstyle=\color{blue}, stringstyle=\color{red!60!black}}

%opening
\title{\vspace{-2cm}Programming for Engineers II\\Class 02\\Introduction to Object--Oriented Programming}
\author{Attique Dawood}

\begin{document}
\maketitle
\section{Announcements}
\begin{itemize}
\item There is no weekly quiz this week. Quiz will be conducted on next Friday and will include topics covered in first two weeks.
\item \textbf{Homework:}\\
Chapter 1, sections 1.1--1.9 from Deitel.\\
Chapter 1, \textit{The Big Picture} from Lafore.\\
Chapter 2, \textit{Introduction to C++ Programming}. Sections 2.1--2.7 from Deitel.
\end{itemize}
\section{Weekly Schedule}
\begin{table}[H]
\begin{center}
\vspace{-0.3cm}
	\begin{tabular}{lccrc}
	\hline \hline
		\rule{0pt}{2.6ex} \textbf{Week} & \textbf{Date} & \textbf{Quiz} & \textbf{Evaluation} & \textbf{Weight}\\
		\hline
		1 \rule{0pt}{2.6ex} & 21 Jan & & & \\
		2 & 28 Jan & Yes & & 0.5 \\
		3 & 4 Feb & & ABQ 1 & 4 \\
		4 & 11 Feb & Yes & & 0.5 \\
		5 & 18 Feb & Yes & & 0.5 \\
		6 & 25 Feb & & Mid--I & 10 \\
		7 & 4 Mar & Yes & & 0.5 \\
		8 & 11 Mar & Yes & & 0.5 \\
		9 & 18 Mar & & ABQ 2& 4 \\
		10 & 25 Mar & Yes & & 0.5 \\
		11 & 1 Apr & Yes & & 0.5 \\
		12 & 8 Apr & & Mid--II& 10 \\
		13 & 15 Apr & Yes & & 0.5 \\
		14 & 22 Apr & Yes & & 0.5 \\
		15 & 29 Apr & & ABQ 3& 4 \\
		16 & 6 May & Yes & & 0.5 \\
		17 & 13 May & & & \\
		18 & 20 May & & Final & 50 \\
	\hline \hline
	\end{tabular}
\end{center}
\label{Tentative-Weekly-Schedule}
\caption{Tentative weekly schedule}
\end{table}
\section{Why Do We Need Object--Oriented Programming?\cite{Lafore}}
\begin{itemize}
\item Limitations in earlier programming approaches.
\item Need to understand limitations.
\end{itemize}
\subsection{Procedural Languages}
\begin{itemize}
\item C, Pascal, FORTRAN.
\item Procedural approach is to do a single task in a serial manner by executing a set of instructions.
\item This will work for smaller programs. Organisation (or paradigm) isn't needed.
\end{itemize}
\subsection{Division into Functions}
\begin{itemize}
\item For relatively large programs a list of instructions may be tedious whenever there is some repetition of code.
\item Functions can make the code readable and reduce repetition.
\item Function can be called a subprogram, subroutine or procedure.
\item Each function has a defined task or purpose in procedural programming in addition to an interface (arguments list and return value).
\end{itemize}
\subsection{Structured Programming}
\begin{itemize}
\item Similar functions can be grouped together to form a \textit{module}.
\item Program division into functions and then modules led to something called \textit{structured programming} before object--oriented approach was developed.
\end{itemize}
\subsection{Problems With Structured Programming}
\begin{itemize}
\item \textbf{Unrestricted access:} A lot of functions access same global data. If data is changed functions must be re--written.
\item \textbf{Poor real--world modelling:} Real--world objects (or entities) cannot be described by just data or functionality. They have both \textit{attributes} and \textit{behaviour}.
\end{itemize}
\subsection{Real--World Modelling}
\subsubsection{Attributes (Data)}
Attributes of a car can be colour, horsepower, model. A person can have a job description like teacher or student, eye colour, height, age etc.
\subsubsection{Behaviour (Function)}
A car can accelerate or decelerate. A person can talk, eat and sleep.
\section{The Object--Oriented Approach}
\begin{itemize}
\item Create a single unit that encloses \textit{data} and the \textit{functions that operate on that data}. Also called \textit{data encapsulation}.
\item This is called an \textit{object}.
\item An object's functions are the only way to access its data. You have to ask a person his name and he/she can choose to tell or not to tell you his/her name.
\item The data is \textit{hidden} to prevent accidental alteration.
\item \textit{Data hiding} and \textit{data encapsulation} are key terms in object--oriented languages.
\item Since you know exactly which functions will operate on data this simplifies writing, debugging and maintaining the program.
\item In C++, objects communicate with each other through their member functions.
\subsection{Example}
Different departments in an organisation do their specific tasks. They have their own records to maintain and only the concerned people from a particular department can modify records.
\end{itemize}
\section{UML}
UML stands for Unified Modelling Language. This is a standardised graphical language meant for graphical representation of a problem.
\section{C and C++}
\begin{itemize}
\item C++ extends the functionality of C to implement OOP.
\item C++ also adds some features on top of C.
\item Since the emphasis in C++ is that of OOP, some C features are seldom if ever used. There are several C++ alternatives to C functions in order to follow the OOP approach.
\end{itemize}
\section{A Simple CGPA Calclator}
\begin{lstlisting}[caption={CGPA Calculator}]
#include <iostream> // Input/output stream library.
using namespace std;
int main()
{
	float C1, C2, C3, C4, C5, CGPA;		// Variable declaration.
	cout << "Enter course grades: ";	// Message to input grades.
	cin >> C1 >> C2 >> C3 >> C4 >> C5;	// Input.
	CGPA = (C1+C2+C3+C4+C5)/5;			// CGPA calculation.
	cout << "CGA is " << CGPA << endl;	// Displaying CGPA.
	
	return 0;
}
\end{lstlisting}
\nocite{*}
\bibliographystyle{plain}
\bibliography{OOPref}
\end{document}
