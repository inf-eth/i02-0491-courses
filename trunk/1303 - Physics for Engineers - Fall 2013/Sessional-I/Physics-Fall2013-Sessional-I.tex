\documentclass[12pt,a4paper]{article}

% Margins.
\setlength{\oddsidemargin}{0in}
\setlength{\evensidemargin}{0in}
\setlength{\headheight}{12pt}
\setlength{\headsep}{42pt}
\setlength{\topmargin}{-54pt}
\setlength{\textwidth}{6.5in}
\setlength{\textheight}{10in}
\pagestyle{plain}

\usepackage{amsmath}
\usepackage{float}
\usepackage{graphicx}
\usepackage[hyphens]{url}
\usepackage[hidelinks]{hyperref}	% Clickable links to figures, references and urls.
\usepackage{lastpage}

% Drawing.
\usepackage{pgf}
\usepackage{tikz}

% Listings for formatting code.
\usepackage{listings}
\usepackage{textcomp}
% General options.
\lstset{breaklines=true, basicstyle=\footnotesize\ttfamily, tabsize=4, numbers=none, stepnumber=1, frame=single, showstringspaces=false, upquote=true}
% C++ specific high-lighting. Comments are 50/50 shades of green/black and strings coloured with 60/40 red/black mixture.
\lstset{language=[ISO]C++, commentstyle=\color{green!50!black}, keywordstyle=\color{blue}, stringstyle=\color{red!60!black}}

% Marks of each question.
\def\QOne{10}
\def\Qtwo{10}
\def\Qthree{10}
\def\Qfour{10}
\def\Qfive{10}
\def\TotalMarks{50}

\begin{document}
\begin{minipage}{0.55\textwidth}
{\LARGE \textbf{Physics for Engineers}}\\[0.15cm]
{\normalsize \textbf{Fall 2013 Semester}}\\
{\Large \textbf{$1^{st}$ Sessional Exam}}\\
{\normalsize \textbf{Saturday, September 21, 2013}}\\[0.30cm]
{\Large \textbf{Total Time: 60 minutes}}\\[0.15cm]
{\Large \textbf{Total Marks: 50}}\\
\textbf{Course Instructors:}\\
Arshad Hassan\\
Attique Dawood\\
\end{minipage}
\begin{minipage}{0.4\textwidth}
\textbf{Serial} \hrulefill \\[0.25cm]
\textbf{Name} \hrulefill\\[0.25cm]
\textbf{Section} \rule{1cm}{0.2mm} \textbf{Roll No:} \hrulefill\\[0.25cm]
\textbf{Signature:} \hrulefill\\[0.25cm]
\rule{6.6cm}{0.2mm}\\
\textbf{Signature of Invigilator}\\[0.25cm]
\end{minipage}
\begin{table}[H]
\begin{center}
\vspace{0.3cm}
	{\large \begin{tabular}{|l|c|c|c|c|c|c|}
	\hline
		\rule{0pt}{2.6ex} Question & \textbf{1} & \textbf{2} & \textbf{3} & \textbf{4} & \textbf{5} & \textbf{Total}\\
		\hline
		Total Marks \rule{0pt}{2.6ex} & \QOne & \Qtwo & \Qthree & \Qfour & \Qfive & \TotalMarks\\
		\hline
		Marks Obtained \rule{0pt}{2.6ex} & & & & & &\\
	\hline
	\end{tabular}}
\end{center}
\end{table}
\noindent \textbf{You are advised to READ these notes:}
\begin{enumerate}
\item \textbf{Attempt on the Question Paper. \underline{NO EXTRA SHEET} will be provided/accepted. No
additional sheet will be provided for rough work. Use the back of the page where
provided space is not sufficient.}
\item After asked to commence the exam, please verify that you have \textbf{\pageref{LastPage} different
printed pages} including this title page.
\item There are 5 questions. Attempt all of them. It is advisable to go through the paper once
before starting with the first question.
\item Exam is closed books, closed notes. Please see that the area in your threshold is clean.
You will be charged for any material which can be classified as \textbf{`helping in the paper'}
found near you.
\item \textbf{Calculator sharing is strictly prohibited.}
\item Students who attempt the paper with lead pencils lose the right to get them rechecked.
\item \textbf{The invigilator present is not supposed to answer any questions. No one may come
to your room for corrections and you are not supposed to request to call anyone.
Make assumptions wherever required and clearly mark them.}
\end{enumerate}
\newpage
\noindent\textbf{Question 1: Vector Fields\hfill \QOne~marks}\\
Given a vector field
\begin{equation*}
\textbf{A}=y\hat x+(y^2x-z)\hat y+z^2\hat z
\end{equation*}
Find and sketch the field vector at following points in the $xy$--plane.
\begin{enumerate}
\item[a.] (2, 1, 0)
\item[b.] (1, 1.5, 0)
\item[c.] (0, 1, 0)
\item[d.] (-2, 1, 0)
\end{enumerate}
\begin{figure}[H]
\begin{tikzpicture}
	\draw[thick] (0cm,0cm) rectangle (\textwidth, 18.5cm);
	\draw[thick, <->, >=stealth] (4cm, 2cm) -- (\textwidth-4cm, 2cm);
	\draw[thick, ->, >=stealth] (\textwidth/2, 2cm) -- (\textwidth/2, 7cm);
	% Drawing vertical grid lines.
	\foreach \x in {4.5cm,5.5cm,6.5cm,7.5cm,8.5cm,9.5cm,10.5cm,11.5cm,12.5cm}
		\draw[dashed] (\x-0.255cm,2cm) -- (\x-0.255cm,6.25cm-0.25cm); % Solid lines at +1 intervals.
	% Drawing horizontal grid lines.
	\foreach \y in {2cm,3cm,4cm,5cm,6cm}
		\draw[dashed] (4.5cm-0.255cm,\y) -- (\textwidth-4cm-0.25cm,\y); % Solid lines at +1 intervals.
\end{tikzpicture}
\end{figure}
\noindent \textbf{Question 2: Vectors Projections\hfill \Qtwo~marks}\\
A line passes through points P(3, -2, 4) and Q(-1, 0, 2). Find the shortest distance between this line and the point R(4, 1, -1).\\
\noindent \textbf{Hint:} Find vectors $\overrightarrow{PQ}$ and $\overrightarrow{PR}$. The perpendicular component of $\overrightarrow{PR}$ on $\overrightarrow{PQ}$ is the required distance.
\begin{figure}[H]
\begin{tikzpicture}
	\draw[thick] (0cm,0cm) rectangle (\textwidth, 22cm);
\end{tikzpicture}
\end{figure}
\noindent\textbf{Question 3: Flux of Vector Field \hfill \Qthree~marks}\\
A vector field is given by $\textbf{A}=xz\hat x+(2xy+z)\hat y+3y^2z\hat z$ in a region. Evaluate the flux, $\int\limits_{S}\textbf{A}\cdot d\textbf{S}$, through the surface given by $y=-1$, $0<x<2$ and $0<z<1$.
\begin{figure}[H]
\begin{tikzpicture}
	\draw[thick] (0cm,0cm) rectangle (\textwidth, 23cm);
\end{tikzpicture}
\end{figure}

\noindent\textbf{Question 4: Line Integrals \hfill \Qfour~marks}\\
Given a vector field $\textbf{A}=(3y^2+2x)\hat x+6xy\hat y$, find the circulation of \textbf{A} ($\oint\limits_{L} \textbf{A}\cdot d\textbf{\textit{l}}$) over closed path ABCA.
\begin{figure}[H]
\flushright
\begin{tikzpicture}[xscale=1.2,yscale=1.2,font=\small]
	\def\XD{0cm}
	\def\YD{0cm}

	% Drawing vertical grid lines.
	\foreach \x in {-2cm,-1cm,0cm,1cm,2cm}
		\draw[dashed] (\x,-2cm) -- (\x,2cm); % Solid lines at +1 intervals.
	\foreach \x/\xlabel in {-2cm/$-2$,-1cm/$-1$,1cm/$1$,2cm/$2$}
		\coordinate[label=left:\xlabel] (XLabel) at (\x,0.2cm);
	% Drawing horizontal grid lines.
	\foreach \y in {-2cm,-1cm,0cm,1cm,2cm}
		\draw[dashed] (-2cm,\y) -- (2cm,\y); % Solid lines at +1 intervals.
	\foreach \y/\ylabel in {-2cm/$-2$,-1cm/$-1$,1cm/$1$,2cm/$2$}
		\coordinate[label=left:\ylabel] (YLabel) at (0cm,\y+0.2cm);
	
	\draw[thick, ->, >=stealth] (2cm, -2cm) -- (-1cm, 1cm);
	\draw[thick, ->, >=stealth] (-1cm, 1cm) -- (2cm, 1cm);
	\draw[thick, ->, >=stealth] (2cm, 1cm) -- (2cm, -2cm);

	\coordinate[label=below:A] (A) at (2.2cm,-2cm);
	\coordinate[label=above:B] (D) at (-1.2cm,1cm);
	\coordinate[label=above:C] (C) at (2.2cm,1cm);
	
	\draw[thick, <->, >=stealth] (0cm, -3cm) -- (0cm, 3cm);
	\coordinate[label=above:$y$] (y) at (0cm,3cm);
	\draw[thick, <->, >=stealth] (-3cm, 0cm) -- (3cm, 0cm);
	\coordinate[label=right:$x$] (x) at (3cm,0cm);

\end{tikzpicture}
\end{figure}
\begin{figure}[H]
\vspace{-8.8cm}
\begin{tikzpicture}
	\draw[thick] (0cm,0cm) rectangle (\textwidth, 23.25cm);
\end{tikzpicture}
\end{figure}
\noindent\textbf{Question 5: Coulomb's Law \hfill \Qfive~marks}\\
Point charges $q_1$ and $q_2$ are located at (0, 1, 1) and (1, 0, 1), respectively. A third point charge $q_3$ is placed at origin. Find the force $\textbf{F}_3$ experienced by $q_3$ if $q_1=-3~\mu$C, $q_2=2~\mu$C and $q_3=-1~\mu$C.

\begin{figure}[H]
\begin{tikzpicture}
	\draw[thick] (0cm,0cm) rectangle (\textwidth, 22.5cm);
\end{tikzpicture}
\end{figure}

\end{document}