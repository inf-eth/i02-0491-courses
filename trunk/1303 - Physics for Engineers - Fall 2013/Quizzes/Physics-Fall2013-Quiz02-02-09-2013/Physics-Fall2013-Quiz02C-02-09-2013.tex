\documentclass[12pt,a4paper]{article}

% Margins.
\setlength{\oddsidemargin}{0in}
\setlength{\evensidemargin}{0in}
\setlength{\headheight}{12pt}
\setlength{\headsep}{42pt}
\setlength{\topmargin}{-54pt}
\setlength{\textwidth}{6.5in}
\setlength{\textheight}{10in}

\usepackage{amsmath}
\usepackage{float}
\usepackage{graphicx}
\usepackage[hyphens]{url}
\usepackage{hyperref}	% Clickable links to figures, references and urls.
\usepackage{datetime}

% Drawing.
\usepackage{pgf}
\usepackage{tikz}

% Listings for formatting code.
\usepackage{listings}
\usepackage{textcomp}
% General options.
\lstset{breaklines=true, basicstyle=\small\ttfamily, tabsize=4, numbers=left, stepnumber=1, frame=single, showstringspaces=false, upquote=true}
% C++ specific high-lighting. Comments are 50/50 shades of green/black and strings coloured with 60/40 red/black mixture.
\lstset{language=[ISO]C++, commentstyle=\color{green!50!black}, keywordstyle=\color{blue}, stringstyle=\color{red!60!black}}

%opening
\title{\vspace{-2.5cm}Physics for Engineers - Fall 2013\\Quiz \#02C}
\date{\vspace{-1.5cm}Date: 02--09--2013}
\begin{document}
\maketitle
\vspace{-0.5cm}
\noindent \textbf{Time: 10 minutes\hfill Total Marks: 10}\\[0.3cm]
\noindent \textbf{Name:\rule{8cm}{1pt}\hfill Roll Number:\rule{3cm}{1pt}}\\[0.5cm]
\noindent \textbf{Question:} Given two vector fields
\begin{equation*}
\begin{split}
&\textbf{A}=(x+2y)\hat x+(x-z)\hat y+x\hat z\mathrm{~and}\\
&\textbf{B}=(1-z)\hat x+(x+2y)\hat y+2xz\hat z
\end{split}
\end{equation*}
Find the angle between \textbf{A} and \textbf{B} at (-1, 2, 3).
\end{document}
