\documentclass[12pt,a4paper]{article}

% Margins.
\setlength{\oddsidemargin}{0in}
\setlength{\evensidemargin}{0in}
\setlength{\headheight}{12pt}
\setlength{\headsep}{42pt}
\setlength{\topmargin}{-54pt}
\setlength{\textwidth}{6.5in}
\setlength{\textheight}{10in}

\usepackage{amsmath}
\usepackage{float}
\usepackage{graphicx}
\usepackage[hyphens]{url}
\usepackage{hyperref}	% Clickable links to figures, references and urls.
\usepackage{datetime}
\usepackage{subfigure}

% Links direct to top of figures.
\usepackage[all]{hypcap}

% Drawing.
\usepackage{pgf}
\usepackage{tikz}

% Listings for formatting code.
\usepackage{listings}
\usepackage{textcomp}
% General options.
\lstset{breaklines=true, basicstyle=\small\ttfamily, tabsize=4, numbers=left, stepnumber=1, frame=single, showstringspaces=false, upquote=true}
% C++ specific high-lighting. Comments are 50/50 shades of green/black and strings coloured with 60/40 red/black mixture.
\lstset{language=[ISO]C++, commentstyle=\color{green!50!black}, keywordstyle=\color{blue}, stringstyle=\color{red!60!black}}

%opening
\title{\vspace{-3cm}Physics for Engineers\\Assignment 06\\Gauss's Law}
\author{Arshad Hassan\and Attique Dawood}
\date{October 21, 2013\\Due: October 28, 2013\\[0.2cm] Last Modified: \today, \currenttime}
\begin{document}
\maketitle
\section{Charge in Free Space}
Use Gauss's Law to derive electric field \textbf{everywhere} for given charge distributions.
\subsection{Infinite Line Charge}
An infinite line charge is placed along $z$--axis with given charge density.
\begin{itemize}
\item[a.] Uniform charge density.
\item[b.] $\rho_L=z$ C/m.
\item[c.] $\rho_L=\dfrac{1}{z}$ C/m.
\end{itemize}
\subsection{Spherical Charge Distribution}
\begin{itemize}
\item[a.] Uniform charge density $\rho_v$ in a sphere of radius $R$.
\item[b.] Uniform charge density $\rho_v$ in the region $r=a$ to $r=b$.
\item[c.] Charge density $\rho_v=\dfrac{1}{r}$ in a sphere of radius $R$.
\item[d.] Charge density $\rho_v=\dfrac{1}{r^2}$ in a sphere of radius $R$.
\end{itemize}
\section{Conductors}
\begin{itemize}
\item[a.] A spherical conductor of radius $R$ has total charge $q$. Use Gauss's Law to find electric field everywhere.
\item[b.] A conducting shell exists at $r=a$ to $r=b$. Space inside the shell is free space. If a charge $q$ is placed at the centre of the shell find electric field everywhere. Also describe how charge will be distributed on the surface of shell and calculate surface charge densities on inner and outer surface.
\item[c.] A conducting shell exists at $r=a$ to $r=b$ and has a net charge $2q$. Space inside the shell is free space. If a charge $-q$ is placed at the centre of the shell find electric field everywhere. Also describe how charge will be distributed on the surface of shell and calculate surface charge densities on inner and outer surface.
\end{itemize}
%\bibliographystyle{plain}
%\bibliography{PhysicsRef}
\end{document}
