\documentclass[12pt,a4paper]{article}

% Margins.
\setlength{\oddsidemargin}{0in}
\setlength{\evensidemargin}{0in}
\setlength{\headheight}{12pt}
\setlength{\headsep}{42pt}
\setlength{\topmargin}{-54pt}
\setlength{\textwidth}{6.5in}
\setlength{\textheight}{10in}

\usepackage{amsmath}
\usepackage{float}
\usepackage{graphicx}
\usepackage[hyphens]{url}
\usepackage{hyperref}	% Clickable links to figures, references and urls.

% Drawing.
\usepackage{pgf}
\usepackage{tikz}

% Listings for formatting code.
\usepackage{listings}
\usepackage{textcomp}
% General options.
\lstset{breaklines=true, basicstyle=\small\ttfamily, tabsize=4, numbers=left, stepnumber=1, frame=single, showstringspaces=false, upquote=true}
% C++ specific high-lighting. Comments are 50/50 shades of green/black and strings coloured with 60/40 red/black mixture.
\lstset{language=[ISO]C++, commentstyle=\color{green!50!black}, keywordstyle=\color{blue}, stringstyle=\color{red!60!black}}

%opening
\title{\vspace{-2cm}Programming for Engineers I\\Class 13\\Arrays}
\author{Attique Dawood}

\begin{document}
\maketitle
\section{Announcements}
\begin{itemize}
\item None.
\end{itemize}
\section{Revision}
\begin{itemize}
\item \verb|switch| statement.
\end{itemize}
\section{Arrays}
\begin{itemize}
\item Arrays can be used to create multiple variables of the same type.
\item To store marks of 50 students it would be tedious to declare 50 integers. However, an array makes it easier.
\item Arrays are normally used to group and store similar data.
\item Elements of an array can be accessed with subscript operator \verb|[]|.
\item A \verb|for| loop is naturally suited for array manipulation.
\end{itemize}
\begin{lstlisting}[caption={A Simple Array}]
#include <iostream>
using namespace std;

int main()
{
	int Grades[5];
	// Input all grades in a loop.
	for (int i=0; i<5; i++)
		cin >> Grades[i];
	// Output grades on console.
	cout << "Grades entered: ";
	for (int i=0; i<5; i++)
		cout << Grades[i] << " ";

	cout << endl;

	return 0;
}
\end{lstlisting}
\section{Initialization}
Array can either be initialised at declaration or at a later stage using loop.
\subsection{Initialisation at Declaration}
\begin{lstlisting}[caption={Array initialisation at declaration}]
int myArray1[10] = {0}; // Initialises whole array with zeroes.
int myArray2[10] = {-1,11,-12,10,15,2,4,-7,4,21}; // Initialise individual elements.
\end{lstlisting}
\subsection{Using \texttt{for} loop}
\begin{lstlisting}[caption={Array initialisation using for loop}]
int myArray1[10]; // Declaration.
for(int i=0; i<10; i++)
{
	myArray[i] = 0; // Initialise all elements to zero.
}
\end{lstlisting}
\section{Sample Program}
\begin{lstlisting}[caption={Sample program}]
#include <iostream>
using namespace std;

int main()
{
	int Age[4]; // Declaration.
	Age[0] = 20;
	Age[1] = 19;
	Age[2] = 22;
	Age[3] = 21;

	cout << "First element  = Age[0]" << endl;
	cout << "Second element  = Age[1]" << endl;
	cout << "Third element  = Age[2]" << endl;
	cout << "Fourth element  = Age[3]" << endl;

	return 0;
}
\end{lstlisting}
\section{Copying Arrays}
Arrays cannot be copied using assignment operator, so individual elements must be copied using a loop. Similarly, other operations like addition, subtraction etc. must performed on individual elements rather than whole arrays.
\begin{lstlisting}[caption={Copying an array}]
#include <iostream>
using namespace std;

int main()
{
	float Array1[5] = {-0.33f,-0.45f,0.22f,0.532f,0.101f};
	float Array2[5];

	// Copying...
	for (int i=0; i<5; i++)
	{
		Array2[i] = Array1[i];
	}
	// Display.
	for (int i=0; i<5; i++)
	{
		cout << "Element " << i+1 << " = " << Array2[i] << endl;
	}

	return 0;
}
\end{lstlisting}
\end{document}