\documentclass[12pt,a4paper]{article}

% Margins.
\setlength{\oddsidemargin}{0in}
\setlength{\evensidemargin}{0in}
\setlength{\headheight}{12pt}
\setlength{\headsep}{42pt}
\setlength{\topmargin}{-54pt}
\setlength{\textwidth}{6.5in}
\setlength{\textheight}{10in}

\usepackage{amsmath}
\usepackage{float}
\usepackage{graphicx}
\usepackage[hyphens]{url}
\usepackage{hyperref}	% Clickable links to figures, references and urls.

% Drawing.
\usepackage{pgf}
\usepackage{tikz}

% Listings for formatting code.
\usepackage{listings}
\usepackage{textcomp}
% General options.
\lstset{breaklines=true, basicstyle=\small\ttfamily, tabsize=4, numbers=left, stepnumber=1, frame=single, showstringspaces=false, upquote=true}
% C++ specific high-lighting. Comments are 50/50 shades of green/black and strings coloured with 60/40 red/black mixture.
\lstset{language=[ISO]C++, commentstyle=\color{green!50!black}, keywordstyle=\color{blue}, stringstyle=\color{red!60!black}}

%opening
\title{\vspace{-2cm}Programming for Engineers I\\Class 02\\Compilation Steps}
\author{Attique Dawood}

\begin{document}
\maketitle
\section{How Computers Work?}
\begin{itemize}
\item Different components; CPU, Memory (RAM), hard disk, I/O devices.
\item CPU is the brain of the computer.
\item The purpose of using a computer is to make things easier for us.
\item Need to know how to make a computer work efficiently.
\item CPU only understands binary. Example of CPU doing an addition.
\item CPU instruction is a binary stream with opcode and operands.
\end{itemize}
\section{Programming Languages}
\begin{itemize}
\item The simplest way to make CPU do something is to directly give it binary input.
\item In older systems this was the case but it was tedious.
\item Assembly language was invented that could translate symbolic directives into binary.
\item C/C++ is a higher level language than assembly and there are still higher level languages like Visual Basic.
\end{itemize}
\section{Steps in Compilation of a C/C++ Program}
\begin{enumerate}
\item Write/modify code. (.cpp file)
\item Pre--process.
\item Compile. (.obj file)
\item Link. (.exe file)
\item Execute.
\item Errors? (Go to step 1).
\end{enumerate}
\section{Who is a Good Programmer?}
\begin{itemize}
\item Someone with a creative mind.
\item Can come up with different ways to do the same thing.
\item Can think of improvements.
\item \textit{``Can perform a heart by--pass surgery with house--hold items!''}
\end{itemize}
\section{How to Become a Good Programmer?}
\begin{itemize}
\item Take an interest. Can't be good at something you hate doing.
\item Practice.
\item Get to know the ins and outs.
\end{itemize}
\section{A Simple CGPA Calclator}
\begin{lstlisting}[caption={CGPA Calculator}]
#include <iostream> // Input/output stream library.
using namespace std;
int main()
{
	float C1, C2, C3, C4, C5, CGPA;		// Variable declaration.
	cout << "Enter course grades: ";	// Message to input grades.
	cin >> C1 >> C2 >> C3 >> C4 >> C5;	// Input.
	CGPA = (C1+C2+C3+C4+C5)/5;			// CGPA calculation.
	cout << "CGA is " << CGPA << endl;	// Displaying CGPA.
	
	return 0;
}
\end{lstlisting}
\end{document}