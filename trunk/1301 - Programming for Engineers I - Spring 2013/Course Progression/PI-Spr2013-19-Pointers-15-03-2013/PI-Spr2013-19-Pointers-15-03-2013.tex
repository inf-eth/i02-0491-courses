\documentclass[12pt,a4paper]{article}

% Margins.
\setlength{\oddsidemargin}{0in}
\setlength{\evensidemargin}{0in}
\setlength{\headheight}{12pt}
\setlength{\headsep}{42pt}
\setlength{\topmargin}{-54pt}
\setlength{\textwidth}{6.5in}
\setlength{\textheight}{10in}

\usepackage{amsmath}
\usepackage{float}
\usepackage{graphicx}
\usepackage[hyphens]{url}
\usepackage{hyperref}	% Clickable links to figures, references and urls.

% Drawing.
\usepackage{pgf}
\usepackage{tikz}

% Listings for formatting code.
\usepackage{listings}
\usepackage{textcomp}
% General options.
\lstset{breaklines=true, basicstyle=\small\ttfamily, tabsize=4, numbers=left, stepnumber=1, frame=single, showstringspaces=false, upquote=true}
% C++ specific high-lighting. Comments are 50/50 shades of green/black and strings coloured with 60/40 red/black mixture.
\lstset{language=[ISO]C++, commentstyle=\color{green!50!black}, keywordstyle=\color{blue}, stringstyle=\color{red!60!black}}

%opening
\title{\vspace{-2cm}Programming for Engineers I\\Class 19\\Pointers}
\author{Attique Dawood}

\begin{document}
\maketitle
\section{Announcements}
\begin{itemize}
\item Quiz 4 on 2D arrays.
\end{itemize}
\section{Revision}
\begin{itemize}
\item C--strings.
\end{itemize}
\section{Pointers}
\begin{itemize}
\item Pointers are very special variables that let you manipulate other `normal' variables (int, float, double etc.).
\item A pointer stores the address of the variable it wants to access (manipulate).
\item Pointer declaration requires declaration of the type of variable that pointer will `point' to. \verb|int* xp| means a pointer named `xp' that can point to an integer variable.
\item Accessing the value of `pointed' variable is called `dereferencing' the pointer. This is done by using a \verb|*| with pointer. Example: \verb|Display: cout << *xp;|, \verb|Input: cin >> *xp;|
\item \textbf{Remember, like other variables, a pointer will contain garbage if it is not initialised with the address of a valid variable}.
\item \textbf{Any attempt to access an invalid address stored in pointer will likely result in runtime crash}.
\item Pointers allow dynamic memory access and is probably the most powerful feature in C. Mastering the use of pointers can result in efficient and faster programs.
\item Basic pointer usage is essential for understanding dynamic memory allocation/deallocation.
\end{itemize}
\begin{lstlisting}[caption={Pointer Basics}]
#include <iostream>
using namespace std;

int main()
{
	int x = 34;
	int* xp; // Pointer declaration.
	xp = &x; // Pointer 'xp' initialised with address of x.
	
	cout << "Value of x: " << x << endl; // Value stored in x.
	cout << "Address of x: " << &x << endl; // Address of x in memory.
	cout << "Value of xp: " << xp << endl; // xp stores address of x.
	cout << "Value of x using xp: " << *xp << endl; // Display value stored in x using pointer xp.
	*xp = -1; // Change value stored in x using pointer xp.
	cout << "New value of x: " << endl
		 << "Using x: " << x << endl
		 << "Using xp: " << *xp << endl; 
	
	return 0;
}
\end{lstlisting}
\end{document}