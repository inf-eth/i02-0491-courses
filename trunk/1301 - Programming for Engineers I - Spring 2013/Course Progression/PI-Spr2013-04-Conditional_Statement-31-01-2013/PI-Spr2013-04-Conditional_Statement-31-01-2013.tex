\documentclass[12pt,a4paper]{article}

% Margins.
\setlength{\oddsidemargin}{0in}
\setlength{\evensidemargin}{0in}
\setlength{\headheight}{12pt}
\setlength{\headsep}{42pt}
\setlength{\topmargin}{-54pt}
\setlength{\textwidth}{6.5in}
\setlength{\textheight}{10in}

\usepackage{amsmath}
\usepackage{float}
\usepackage{graphicx}
\usepackage[hyphens]{url}
\usepackage{hyperref}	% Clickable links to figures, references and urls.

% Drawing.
\usepackage{pgf}
\usepackage{tikz}

% Listings for formatting code.
\usepackage{listings}
\usepackage{textcomp}
% General options.
\lstset{breaklines=true, basicstyle=\small\ttfamily, tabsize=4, numbers=left, stepnumber=1, frame=single, showstringspaces=false, upquote=true}
% C++ specific high-lighting. Comments are 50/50 shades of green/black and strings coloured with 60/40 red/black mixture.
\lstset{language=[ISO]C++, commentstyle=\color{green!50!black}, keywordstyle=\color{blue}, stringstyle=\color{red!60!black}}

%opening
\title{\vspace{-2cm}Programming for Engineers I\\Class 04\\Conditional Statements}
\author{Attique Dawood}

\begin{document}
\maketitle
\section{Revision}
\begin{itemize}
\item There are mathematical operators and logic operators.
\item Math or arithmetic operators are +, -, *, /, \%. Operator precedence?
\item Basic logic operations are NOT, AND, OR, XOR.
\item In C/C++ there are two types of logic operations: bitwise and logical statements.
\item The bitwise logic operators are \verb|~| for NOT, \verb|&| for AND, \verb@|@ for OR and \verb|^| for XOR. These operate on individual bits of variable(s).
\item The logic operators for evaluating statements are \verb|!| for NOT, \verb|&&| for AND and \verb@||@ for OR. Logical operators can operate on logical statements.
\item The operators \verb|<<| and \verb|>>| have different uses depending on context.
\item As stream operator used with \verb|cin| and \verb|cout| for input/output.
\item As shift operators, \verb|<<| and \verb|>>| operate on individual bits of a variable. For example, \verb|x<<2| which means bits of variable \verb|x| left--shifted by 2.
\end{itemize}
\section{Bitwise Operation}
\begin{itemize}
\item Masking can be used to set a particular value for a range of bits.
\item ANDing with an all 1's mask will preserve the original bit pattern.
\item ORing with an all 1's mask will result in an all 1's.
\item ANDing with an all 0's mask will set all bits to 0.
\item ORing with an all 0's mask will preserve bit pattern.
\item We can also make a mask with certain bit pattern. Let's say we want to deal with 4 bit numbers.
\end{itemize}
\begin{lstlisting}[caption={Example: Masking}]
#include <iostream> 
using namespace std; 
int main()
{
	// All 0's mask.
	unsigned int MaskAll0s = 0;
	// All 1's mask.
	unsigned int MaskAll1s = ~MaskAll0s;
	
	int x = 5;
	int z;
	
	z = x & MaskAll0s;
	cout << "z = x & MaskAll0s = " << z << endl;
	z = x | MaskAll0s;
	cout << "z = x | MaskAll0s = " << z << endl;
	z = x & MaskAll1s;
	cout << "z = x & MaskAll1s = " << z << endl;
	z = x | MaskAll1s;
	cout << "z = x | MaskAll1s = " << z << endl;

	return 0;
}
\end{lstlisting}
\begin{lstlisting}[caption={Example: Masking 4 bits}]
#include <iostream> 
using namespace std; 
int main()
{
	// 4 bit mask.
	unsigned int Mask4bit = 15; // 0xF
	
	int x = 5;
	int z = (~x) & Mask4bit;
	cout << "Inverting bits of x=5, we get z = " << z << endl;
	
	z = (x<<2) & Mask4bit;
	cout << "x<<2 = " << z << endl;
	
	return 0;
}
\end{lstlisting}
\section{Comparison Operators}
The comparison operators can be used to compare two values. The values can also be variables.  For example, \verb|x == y|. This should be read as \textit{Is x equal to y?} Most commonly used operators are:
\begin{itemize}
\item \verb|==| (equals?)
\item \verb|!=| (not equals?)
\item \verb|>| (greater than?)
\item \verb|<| (less than?)
\item \verb|>=| (equals or greater than?)
\item \verb|<=| (equals or less than?)
\end{itemize}

\section{Conditional Statements}
The syntax for conditional statement is,
\begin{lstlisting}
if (condition...)
{
    do this if condition is true...
}
else
{
    do this if condition is false...
}
\end{lstlisting}
If the condition is evaluated to be 0 then it is false. If condition is non--zero it is true.
Operators in previous section are useful for testing a certain condition. For example,
\begin{lstlisting}
int x = 2;
int y = 4;
if (x == y)
{
    cout << "x is equal to y" << endl;
}
else
{
   cout << "x is not equal to y" endl;
}
\end{lstlisting}
In this case the output will be \verb|x is not equal to y|. The \verb|else| part is not mandatory and an \verb|if| statement can be used standalone. For example,
\begin{lstlisting}
int x;
cin >> x;
if (x > 0)
{
    cout << "You have entered a positive number" << endl;
}
\end{lstlisting}
\end{document}