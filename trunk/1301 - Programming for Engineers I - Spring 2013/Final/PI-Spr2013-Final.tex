\documentclass[12pt,a4paper]{article}

% Margins.
\setlength{\oddsidemargin}{0in}
\setlength{\evensidemargin}{0in}
\setlength{\headheight}{12pt}
\setlength{\headsep}{42pt}
\setlength{\topmargin}{-54pt}
\setlength{\textwidth}{6.5in}
\setlength{\textheight}{10in}
\pagestyle{plain}

\usepackage{amsmath}
\usepackage{float}
\usepackage{graphicx}
\usepackage[hyphens]{url}
\usepackage[hidelinks]{hyperref}	% Clickable links to figures, references and urls.
\usepackage{lastpage}

% Drawing.
\usepackage{pgf}
\usepackage{tikz}

% Listings for formatting code.
\usepackage{listings}
\usepackage{textcomp}
% General options.
\lstset{breaklines=true, basicstyle=\footnotesize\ttfamily, tabsize=4, numbers=none, stepnumber=1, frame=single, showstringspaces=false, upquote=true}
% C++ specific high-lighting. Comments are 50/50 shades of green/black and strings coloured with 60/40 red/black mixture.
\lstset{language=[ISO]C++, commentstyle=\color{green!50!black}, keywordstyle=\color{blue}, stringstyle=\color{red!60!black}}

% Marks of each question.
\def\QOne{5}
\def\Qtwo{10}
\def\Qthree{5}
\def\Qfour{5}
\def\Qfive{10}
\def\Qsix{5}
\def\Qseven{10}
\def\Qeight{10}
\def\Qnine{10}
\def\Qten{10}
\def\Qeleven{5}
\def\Qtwelve{15}
\def\TotalMarks{100}

\begin{document}
\begin{minipage}{0.55\textwidth}
{\LARGE \textbf{Programming for\\ Engineers I}}\\[0.15cm]
{\normalsize \textbf{Spring 2013 Semester}}\\
{\Large \textbf{Final Exam}}\\
{\normalsize \textbf{Friday, May 17, 2013}}\\[0.30cm]
{\Large \textbf{Total Time: 180 minutes}}\\[0.15cm]
{\Large \textbf{Total Marks: 100}}\\
\textbf{Course Instructor:}\\
Atif Mughees\\
Attique Dawood\\
\end{minipage}
\begin{minipage}{0.4\textwidth}
\textbf{Serial} \hrulefill \\[0.25cm]
\textbf{Name} \hrulefill\\[0.25cm]
\textbf{Section} \rule{1cm}{0.2mm} \textbf{Roll No:} \hrulefill\\[0.25cm]
\textbf{Signature:} \hrulefill\\[0.25cm]
\rule{6.6cm}{0.2mm}\\
\textbf{Signature of Invigilator}\\[0.25cm]
\end{minipage}
\begin{table}[H]
\begin{center}
\vspace{0.3cm}
	{\large \begin{tabular}{|l|c|c|c|c|c|c|c|c|c|c|c|c|c|}
	\hline
		\rule{0pt}{2.6ex} Question & \textbf{1} & \textbf{2} & \textbf{3} & \textbf{4} & \textbf{5} & \textbf{6} & \textbf{7} & \textbf{8} & \textbf{9} & \textbf{10} & \textbf{11} & \textbf{12} & \textbf{Total}\\
		\hline
		Total Marks \rule{0pt}{2.6ex} & \QOne & \Qtwo & \Qthree & \Qfour & \Qfive & \Qsix & \Qseven & \Qeight & \Qnine & \Qten & \Qeleven & \Qtwelve & \TotalMarks\\
		\hline
		Marks Obtained \rule{0pt}{2.6ex} & & & & & & & & & & & & &\\
	\hline
	\end{tabular}}
\end{center}
\end{table}
\noindent \textbf{You are advised to READ these notes:}
\begin{enumerate}
\item \textbf{Attempt on the Question Paper. \underline{NO EXTRA SHEET} will be provided/accepted. No
additional sheet will be provided for rough work. Use the back of the page where
provided space is not sufficient.}
\item After asked to commence the exam, please verify that you have \textbf{\pageref{LastPage} different
printed pages} including this title page.
\item There are 12 questions. Attempt all of them. It is advisable to go through the paper once
before starting with the first question.
\item Exam is closed books, closed notes. Please see that the area in your threshold is clean.
You will be charged for any material which can be classified as \textbf{`helping in the paper'}
found near you.
\item \textbf{Calculator sharing is strictly prohibited.}
\item Students who attempt the paper with lead pencils lose the right to get them rechecked.
\item \textbf{The invigilator present is not supposed to answer any questions. No one may come
to your room for corrections and you are not supposed to request to call anyone.
Make assumptions wherever required and clearly mark them.}
\end{enumerate}
\newpage
\noindent\textbf{Question 1: Code correction\hfill \QOne~marks}\\
Find errors in the given code and suggest corrections. Errors may be logical or in syntax.
\begin{lstlisting}
#include <iostream>
using namespace std;

struct Complex
{
	float real = 0;
	float img = 0;
};

int main()
{
	const Complex C1;
	C1.real = 7.1;
	C1.img = 2.3;

	Complex* C2ptr;
	C2ptr = new int;
	
	C2ptr.real = 1.8;
	C2ptr.img = 3.3
	
	cout << C2ptr << endl;

	return 0;
}
\end{lstlisting}
\begin{figure}[H]
\begin{tikzpicture}
	\draw[thick] (0cm,0cm) rectangle (\textwidth, 12.5cm);
\end{tikzpicture}
\end{figure}

\noindent\textbf{Question 2: Struct Arrays/Dynamic Allocation \hfill \Qtwo~marks}\\
Given definition of Book structure, you are required to dynamically create an array of books.
\begin{enumerate}
\item Take size of array as input from user.
\item Write code to dynamically create an array of books.
\item In a loop, user will enter attributes of all books using input function.
\item Display the attributes of all books using display function.
\item De--allocate memory before exiting main.
\end{enumerate}
\begin{lstlisting}
#include <iostream>
using namespace std;
struct Book
{
	char Title[50];
	int PublishedYear;
};
void Input(Book& B)
{




}
void Display(Book& B)
{




}
int main()
{
	Book* ArrayOfBooks;
	int Size;
	// 1. Take input size from user.
		
		
	
	// 2. Allocation.
		
	
	// 3. Input attributes from user.
		
	
	
	// 4. Display all books.
		
	
		
	// 5. De-allocation.
		
	
	return 0;
}
\end{lstlisting}

\noindent\textbf{Question 3: Bitwise and Shift Operators \hfill $1\times 5=$\Qthree~marks}\\
Assume 4 bit unsigned binary numbers. Evaluate the following:
\begin{enumerate}
\item[a.] \verb|  4 & 5|
\item[b.] \verb$  4 | 7$
\item[c.] \verb|  3 ^ 9|
\item[d.] \verb|  7 >> 2|
\item[e.] \verb|  8 << 1|
\end{enumerate}
\begin{figure}[H]
\begin{tikzpicture}
	\draw[thick] (0cm,0cm) rectangle (\textwidth, 6cm);
\end{tikzpicture}
\end{figure}

\noindent\textbf{Question 4: Signed Integers \hfill \Qfour~marks}\\
\textbf{Assuming 16 bit signed integer}, what is the bit representation of -1? Show how this number is stored in computer memory.
\begin{figure}[H]
\begin{tikzpicture}
	\draw[thick] (0cm,0cm) rectangle (\textwidth, 10cm);
\end{tikzpicture}
\end{figure}

\noindent\textbf{Question 5: Float Representation \hfill $5\times 2=$\Qfive~marks}\\
a. Convert 11.390625 to 32 bit floating point representation. Also show how this number is stored in computer memory.
\begin{figure}[H]
\begin{tikzpicture}
	\draw[thick] (0cm,0cm) rectangle (\textwidth, 11cm);
\end{tikzpicture}
\end{figure}
\noindent b. Convert the following 32 bit floating point number into decimal.\\
\verb|1 10000101 10110101000000000000000|
\begin{figure}[H]
\begin{tikzpicture}
	\draw[thick] (0cm,0cm) rectangle (\textwidth, 10cm);
\end{tikzpicture}
\end{figure}

\noindent\textbf{Question 6: Sorting \hfill \Qsix~marks}\\
Given an integer array of size 10 filled with random numbers, write code to sort this array in ascending order. You can use bubble sort OR insertion sort.\\
\verb|int Data[10] = {-1, 8, 2, 11, 7, 3, 5, 17, 0, 9};|
\begin{figure}[H]
\begin{tikzpicture}
	\draw[thick] (0cm,0cm) rectangle (\textwidth, 22.5cm);
\end{tikzpicture}
\end{figure}

\noindent\textbf{Question 7: C--Strings \hfill $5\times 2=$\Qseven~marks}\\
Implement the following library functions:
\begin{itemize}
\item[a.] \verb|strlen|: Calculates and returns length of string.
\item[b.] \verb|strcmp|: Compares two strings. Return value is \verb|true| if strings match, otherwise, \verb|false| is returned for mismatch.
\end{itemize}
\noindent\textbf{Note:} Standard \verb|strcmp| function actually returns an int and also compares alphabetical order. You are required to implement a simplified version that only checks mismatch.
\begin{lstlisting}
int strlen(char* str)
{












}
bool strcmp(char* str1, char* str2)
{































}
\end{lstlisting}

\noindent\textbf{Question 8: File Handling: Basics \hfill $2.5\times 4=$\Qeight~marks}\\
Write the purpose of following options:
\begin{enumerate}
\item[a.] \verb|ios::in|
\begin{figure}[H]
\begin{tikzpicture}
	\draw[thick] (0cm,0cm) rectangle (\textwidth, 4cm);
\end{tikzpicture}
\end{figure}
\item[b.] \verb|ios::out|
\begin{figure}[H]
\begin{tikzpicture}
	\draw[thick] (0cm,0cm) rectangle (\textwidth, 4cm);
\end{tikzpicture}
\end{figure}
\item[c.] \verb|ios::binary|
\begin{figure}[H]
\begin{tikzpicture}
	\draw[thick] (0cm,0cm) rectangle (\textwidth, 4cm);
\end{tikzpicture}
\end{figure}
\item[d.] \verb|ios::app|
\begin{figure}[H]
\begin{tikzpicture}
	\draw[thick] (0cm,0cm) rectangle (\textwidth, 4cm);
\end{tikzpicture}
\end{figure}
\end{enumerate}

\newpage
\noindent\textbf{Question 9: File Handling: Text Files \hfill $5\times 2=$\Qnine~marks}\\
\begin{enumerate}
\item[a.] Write code to store 100 random numbers in a text file named \verb|"random.txt"|. Numbers should be separated by spaces.
\begin{figure}[H]
\begin{tikzpicture}
	\draw[thick] (0cm,0cm) rectangle (\textwidth, 10cm);
\end{tikzpicture}
\end{figure}
\item[b.] Write code to read in \verb|"random.txt"| and display all the numbers. Assume file contains 100 numbers.
\begin{figure}[H]
\begin{tikzpicture}
	\draw[thick] (0cm,0cm) rectangle (\textwidth, 10cm);
\end{tikzpicture}
\end{figure}
\end{enumerate}
\noindent\textbf{Question 10: File Handling: Binary Files \hfill $5\times 2=$\Qten~marks}\\
\begin{enumerate}
\item[a.] Input attributes of a single student and store in a binary file named \verb|"records.rec"| in append mode.
\begin{lstlisting}
#include <fstream>
#include <iostream>
using namespace std;
struct StudentInfo
{
	char Name[20];
	int Age;
	float CGPA;
};
int main()
{
	StudentInfo S;
	fstream OutFile;
	// 1. Input attributes.
	
	
	
	
	
	// 2. Open file in append mode.
	
	
	
	// 3. Write S to file.
	
	
	
	// 4. Close file.
	
		
	return 0;
}
\end{lstlisting}
\item[b.] Read in \verb|"records.txt"| and display all records stored in the file.
\begin{figure}[H]
\begin{tikzpicture}
	\draw[thick] (0cm,0cm) rectangle (\textwidth, 8cm);
\end{tikzpicture}
\end{figure}
\end{enumerate}
\noindent\textbf{Question 11: Linked List: Basics \hfill $2.5\times 2=$\Qeleven~marks}\\
\begin{enumerate}
\item[a.] Write one advantage of linked list over array.
\begin{figure}[H]
\begin{tikzpicture}
	\draw[thick] (0cm,0cm) rectangle (\textwidth, 4cm);
\end{tikzpicture}
\end{figure}
\item[b.] Write on dis--advantage of linked list.
\begin{figure}[H]
\begin{tikzpicture}
	\draw[thick] (0cm,0cm) rectangle (\textwidth, 4cm);
\end{tikzpicture}
\end{figure}
\end{enumerate}

\noindent\textbf{Question 12: Linked List: Construction \hfill $5\times 3=$\Qtwelve~marks}\\
Write code for the specified linked list functions. Node structure is given.
\begin{lstlisting}
struct Node
{
	int Data;
	Node* Next;
};
Node* First = NULL;
\end{lstlisting}
\begin{enumerate}
\item[a.] Display linked list.
\begin{lstlisting}
void DisplayList()
{












}
\end{lstlisting}
\item[b.] Add node at head.
\begin{lstlisting}
void AddNodeAtHead(int pData)
{





































}
\end{lstlisting}
\item[c.] Search linked list. If key is found return \verb|true|, otherwise return \verb|false|.
\begin{lstlisting}
bool SearchList(int Key)
{












}
\end{lstlisting}
\end{enumerate}

%\begin{figure}[H]
%\begin{tikzpicture}
%	\draw[thick] (0cm,0cm) rectangle (\textwidth, 22.5cm);
%\end{tikzpicture}
%\end{figure}
\end{document}