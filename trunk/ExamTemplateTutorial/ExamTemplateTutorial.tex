\documentclass[12pt,a4paper]{article}

% Margins.
\setlength{\oddsidemargin}{0in}
\setlength{\evensidemargin}{0in}
\setlength{\headheight}{0pt}
\setlength{\headsep}{20pt}
\setlength{\topmargin}{-54pt}
\setlength{\textwidth}{6.5in}
\setlength{\textheight}{10.5in}

\usepackage{amsmath}
\usepackage{float}
\usepackage{graphicx}
\usepackage[hyphens]{url}
\usepackage{hyperref}	% Clickable links to figures, references and urls.
\usepackage{datetime}
\usepackage{longtable}
\usepackage{array}

% Links direct to top of figures.
\usepackage[all]{hypcap}

% Drawing.
\usepackage{pgf}
\usepackage{tikz}

% Figures folder.
\graphicspath{{Figures/}}

% Listings for formatting code.
\usepackage{listings}
\usepackage{textcomp}

% Table cell alignment directives.
\newcolumntype{L}[1]{>{\raggedright\let\newline\\\arraybackslash\hspace{0pt}}m{#1}}
\newcolumntype{C}[1]{>{\centering\let\newline\\\arraybackslash\hspace{0pt}}m{#1}}
\newcolumntype{R}[1]{>{\raggedleft\let\newline\\\arraybackslash\hspace{0pt}}m{#1}}

% General options.
\lstset{breaklines=true, basicstyle=\footnotesize\ttfamily, tabsize=4, numbers=none, stepnumber=1, frame=single, showstringspaces=false, upquote=true}
% C++ specific high-lighting. Comments are 50/50 shades of green/black and strings coloured with 60/40 red/black mixture.
\lstset{language=[ISO]C++, commentstyle=\color{green!50!black}, keywordstyle=\color{blue}, stringstyle=\color{red!60!black}}

\newcommand\TotalMarks{60}
% Marks of each question
\def\Qone{5}
\def\Qtwo{15}
\def\Qthree{10}
\def\Qfour{20}
\def\Qfive{10}

%opening
\title{\vspace{-2cm}Using Exam Template: Tutorial}
\author{}
\date{May 24, 2014\\[0.2cm] Last Modified: \today, \currenttime}
\begin{document}
\maketitle
\section{Introduction}
This is a tutorial that will get you started with the provided exam template. The template is intended to provide a convenient way of typesetting exam without having to deal with font sizes, spacing, header/footers etc. Exam--specific information, like department, subject, exam date etc. can be filled in like a form. There are examples demonstrating how to write different types of questions.
\section{Obtaining \LaTeX}
You can skip this section if you already have a working installation of \LaTeX. Before you can start using \LaTeX you need to install two things:
\begin{enumerate}
\item The \LaTeX compiler and packages suite: Output document type depends on compiler used. for example \verb|latex| produces \verb|.dvi| document and \verb|pdflatex| produces \verb|.pdf| document. Chances are you will have all these installed if using a popular distribution like \verb|Texlive| or \verb|MiKTeX|.
\item An editor: The most popular editor that works on both windows and linux is \verb|TeXstudio|. It was originally called \verb|TexMakerX|.
\end{enumerate}
\subsection{Installation on Windows}
\begin{enumerate}
\item Obtain the \verb|ProTeXt| bundle (a self--extracting \verb|exe| file) from \url{http://ctan.imsc.res.in/systems/windows/protext/} [fig. \ref{downloading-protext}]. The version available at the time of writing this tutorial is called \verb|ProTeXt-3.1.3-060313|.
\item Extract the \verb|ProTeXt| package [fig. \ref{protext-exe}] and run \verb|setup.exe| [fig. \ref{protext-setup}].
\item Select \verb|MiKTeX| [fig. \ref{miktex-install}] and choose `Complete' installation [fig. \ref{miktex-installation-type}] when prompted.
\item Install \verb|TexMakerX| or \verb|TeXstudio|, whichever came with your \verb|ProTeXt| package.
\item Start up \verb|TexMakerX|. Go to \verb|Options > Configure > Quick Build| and select\\ \verb|PdfLaTeX + View PDF| [fig. \ref{quickbuild-setup}]. Now you can compile and view your documents with the \verb|F1| key. This is not needed if using \verb|TeXstudio|.
\item A fresh installation of \verb|TeXstudio| may not detect correct path to \verb|MikTeX| installation. This can happen if pressing the \verb|F1| key or clicking on the double green arrow results in an error. To fix this, go to \verb|Options > Configure > Commands| and press all the buttons with yellow arrows to revert default paths.
\end{enumerate}
\subsection{Installation on Linux}
\begin{enumerate}
\item On terminal type \verb|yum install texlive| or \verb|apt-get install texlive| depending on your distribution package manager.
\item Install editor with \verb|yum install texmakerx| or \verb|yum install texstudio|. On debian based distributions use \verb|apt-get install texmakerx| or \verb|apt-get install texstudio| [fig. \ref{texstudio-paths}].
\end{enumerate}
\begin{figure}[H]
\centering
\includegraphics[scale=0.75]{Figures/DownloadProTeXt.png}
\caption{Downloading \texttt{ProTeXt}}
\label{downloading-protext}
\end{figure}
\begin{figure}[H]
\centering
\includegraphics[scale=0.75]{ProTeXtexe.png}
\caption{\texttt{ProTeXt} executible}
\label{protext-exe}
\end{figure}
\begin{figure}[H]
\centering
\includegraphics[scale=1.0]{ProTeXtSetup.png}
\caption{\texttt{ProTeXt} setup}
\label{protext-setup}
\end{figure}
\begin{figure}[H]
\centering
\includegraphics[scale=1.0]{MikTeXInstall.png}
\caption{\texttt{MikTeX} installation}
\label{miktex-install}
\end{figure}
\begin{figure}[H]
\centering
\includegraphics[scale=1.0]{MiKTeXComplete.png}
\caption{Selecting \texttt{MikTeX} installation type}
\label{miktex-installation-type}
\end{figure}
\begin{figure}[H]
\centering
\includegraphics[scale=0.8]{QuickBuildSetup.png}
\caption{Quick build setup}
\label{quickbuild-setup}
\end{figure}
\begin{figure}[H]
\centering
\includegraphics[scale=0.65]{TeXstudioPath.png}
\caption{Setting TeXstudio paths}
\label{texstudio-paths}
\end{figure}
\section{Using the Exam Template}
\subsection{Basics}
\LaTeX~uses filetypes with \verb|.tex| extension as input. These files contain `code' to typeset the document. A \verb|.tex| file needs to be compiled into an output \verb|.pdf| file. Open the \verb|ExamTemplate.tex| file in editor by double--clicking on it. Once opened, simply click the green arrow or press \verb|F1| to compile it [fig. \ref{build-and-view}]. Some temporary files will be created along with an \verb|ExamTemplate.pdf| file. The editor also has a built--in viewer [fig. \ref{compiled}].

If the file is already open in Adobe Reader then compiling the \verb|.tex| file will generate an error saying that it cannot write on file.
\begin{figure}[H]
\centering
\includegraphics[scale=0.675]{BuildandView.png}
\caption{Building and viewing a document}
\label{build-and-view}
\end{figure}
\begin{figure}[H]
\centering
\includegraphics[scale=0.6]{Compiled.png}
\caption{Successful compilation}
\label{compiled}
\end{figure}
\subsection{Essential Fields}
The very first section is called the `Fill--ables' section [fig. \ref{fill-ables}]. Here you can set following:
\begin{itemize}
\item University name.
\item Department.
\item Campus.
\item Semester.
\item Exam.
\item Subject with course code.
\item Date of exam.
\item Instructor names. Names of three instructors can be provided. Leave a \verb|\null| or empty brackets to keep an instructor field empty.
\item Exam time.
\item Total marks. There is a way of automatically calculating marks of all questions. Simply uncomment line starting with \verb|\pgfmathtruncatemacro| in fourth section.
\item Total questions.
\item Individual marks of each question can also be set. By default 10 questions have been defined here. If your exam contains less than 10 questions you can simply set marks of additional questions as 0 or delete/comment them altogether [fig. \ref{exam-with-5-qs}]. If there are more than 10 questions you can add additional question definitions.
\end{itemize}
\begin{figure}[H]
\centering
\includegraphics[scale=0.75]{Fill-ables.png}
\caption{Fill--ables section}
\label{fill-ables}
\end{figure}
\begin{figure}[H]
\centering
\includegraphics[scale=1.0]{ExamWith5Qs.png}
\caption{An exam with 5 questions}
\label{exam-with-5-qs}
\end{figure}
\subsection{Instructions Section}
Exam instructions can vary depending on subject. As an example most exams don't have an objective part. So this instruction can be removed by modifying the enumerated/bulleted list and deleting the last item [fig. \ref{exam-instructions}].
\begin{figure}[H]
\centering
\includegraphics[scale=0.85]{ExamInstructions.png}
\caption{Deleting last instruction}
\label{exam-instructions}
\end{figure}
\newpage
\subsection{Working With Tables}
A table is created with the \verb|\begin{tabular}| directive. Additionally, number of columns in the table and text alignment must be specified with the \verb|\begin{tabular}| directive. A few examples are given below:
\lstset{language=TeX}
\begin{lstlisting}
% 1. A table with 3 columns and text center aligned. No separator between columns.
\begin{tabular}{c c c}

% 2. A table with 3 columns. Left-most column has left-aligned text. No separator.
\begin{tabular}{l c c}

% 3. A table with 4 columns and centre-aligned text. Columns are separated with a single vertical line.
\begin{tabular}{|c|c|c|c|}

% 4. A table with 4 columns and centre-aligned text. Columns are separated with a single vertical line. Width of left-most column is fixed at 1cm. Notice capital 'C'.
\begin{tabular}{|C{1.0cm}|c|c|c|}
\end{lstlisting}
Here's the code for a simple table. Width of left-most and right-most columns are fixed at 1.5 cm and 3 cm respectively. Width of middle two columns is automatic and they are centre-aligned. Left-most column is left-aligned and right-most column is right-aligned. An \verb|&| ends a column. The last column is ended with newline directive \verb|\\|.
\begin{lstlisting}
\begin{tabular}{|L{1.5cm}|c|c|R{3cm}|}
\hline
	R1C1 & R1C2 & R1C3 & R1C4 \\ % Row 1
\hline
	R2C1 & R2C2 & R2C3 & R2C4 \\ % Row 2
\hline\hline
\end{tabular}
\end{lstlisting}
This produces:
\begin{tabular}{|L{1.5cm}|c|c|R{3cm}|}
\hline
	R1C1 & R1C2 & R1C3 & R1C4 \\
\hline
	R2C1 & R2C2 & R2C3 & R2C4 \\
\hline\hline
\end{tabular}
The \verb|\hline| draws a horizontal line between two rows. There are two lines after last row. If we want to remove third column from the table the modified code would be:
\begin{lstlisting}
\begin{tabular}{|L{1.5cm}|c|R{3cm}|}
\hline
	R1C1 & R1C2 & R1C4 \\ % Row 1
\hline
	R2C1 & R2C2 & R2C4 \\ % Row 2
\hline\hline
\end{tabular}
\end{lstlisting}
And the code produces:
\begin{tabular}{|L{1.5cm}|c|R{3cm}|}
\hline
	R1C1 & R1C2 & R1C4 \\
\hline
	R2C1 & R2C2 & R2C4 \\
\hline\hline
\end{tabular}
\subsection{Marks Table}
The marks table has centre-aligned columns with fixed width. By default the table lists 10 questions. If the exam contains less questions, say, 5 then extra columns need to be removed. The modified code would look like:\newpage
\begin{lstlisting}
\begin{tabular}{|C{1.8cm}|C{0.75cm}|C{0.75cm}|C{0.75cm}|C{0.75cm}|C{0.75cm}|c|}
	\hline
		\rule{0pt}{2.6ex} & Q-1 & Q-2 & Q-3 & Q-4 & Q-5 & \textbf{Total}\\
		\hline
		\textbf{Marks Obtained}\rule{0pt}{2.6ex}& & & & & & \\
		\hline
		\textbf{Total Marks}\rule{0pt}{2.6ex}& \Qone & \Qtwo & \Qthree & \Qfour & \Qfive & \TotalMarks\\
	\hline
\end{tabular}
\end{lstlisting}
The out put is:
\begin{table}[H]
	{\footnotesize \begin{tabular}{|C{1.8cm}|C{0.75cm}|C{0.75cm}|C{0.75cm}|C{0.75cm}|C{0.75cm}|c|}
	\hline
		\rule{0pt}{2.6ex} & Q-1 & Q-2 & Q-3 & Q-4 & Q-5 & \textbf{Total}\\
		\hline
		\textbf{Marks Obtained}\rule{0pt}{2.6ex}& & & & & & \\
		\hline
		\textbf{Total Marks}\rule{0pt}{2.6ex}& \Qone & \Qtwo & \Qthree & \Qfour & \Qfive & \TotalMarks\\
	\hline
\end{tabular}}
\end{table}
\underline{The number of \&'s in a single row must exactly be equal to one less than total columns.}
\subsection{Questions}
A simple question can be typeset as:
\begin{lstlisting}
\noindent\textbf{Question 1: Constructors/Destructors \hfill \Qone~marks}\\
Write the output of following program. Mention object name with constructor/destructor calls.
\end{lstlisting}
\begin{itemize}
\itemsep0em
\item A new paragraph begins with a little indentation or empty space. The \verb|\noindent| directive will remove this space.
\item Any text can be made bold with \verb|\textbf{bold text}|. This produces: \textbf{bold text}. The short cuts \verb|Ctrl+b| and \verb|Ctrl+i| can also be used to make the text bold or italic.
\item \verb|\hfill| will fill all horizontal space so that any following text is right-aligned. For example, \verb|I am left\hfill I am right| produces:\\I am left\hfill I am right
\item \verb|\\| starts a new line.
\item Equations are written inside a special math environment. The math environment is invoked with \verb|$|. The code \verb|$E=mc^2$| produces: $E=mc^2$.
\item Fractions can be written as \verb|\dfrac{numerator}{denominator}|.\\The code \verb|$\dfrac{1}{4}+\dfrac{1}{2}=\dfrac{3}{4}$| produces: $\dfrac{1}{4}+\dfrac{1}{2}=\dfrac{3}{4}$.
\end{itemize}
\section{Learning \LaTeX}
There are several \LaTeX~tutorials available online.
\begin{enumerate}
%\item A good place to start learning basics of \LaTeX~is: \url{http://en.wikibooks.org/wiki/LaTeX/Basics}
\item A comprehensive guide is available at: \url{https://www.tug.org/twg/mactex/tutorials/ltxprimer-1.0.pdf}.
\item Another reference book: \url{http://tobi.oetiker.ch/lshort/lshort.pdf}
\end{enumerate}
%\nocite{*}
%\bibliographystyle{plain}
%\bibliography{PhysicsRef}
\end{document}
