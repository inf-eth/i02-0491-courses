\documentclass[12pt,a4paper]{article}

% Margins.
\setlength{\oddsidemargin}{0in}
\setlength{\evensidemargin}{0in}
\setlength{\headheight}{12pt}
\setlength{\headsep}{0pt}
\setlength{\topmargin}{-60pt}
\setlength{\textwidth}{6.5in}
\setlength{\textheight}{10.75in}

\usepackage{amsmath}
\usepackage{float}
\usepackage{graphicx}
\usepackage[hyphens]{url}
\usepackage{hyperref}	% Clickable links to figures, references and urls.
\usepackage{datetime}
\usepackage{longtable}

% Drawing.
\usepackage{pgf}
\usepackage{tikz}

% Listings for formatting code.
\usepackage{listings}
\usepackage{textcomp}
% General options.
\lstset{breaklines=true, basicstyle=\small\ttfamily, tabsize=4, numbers=left, stepnumber=1, frame=single, showstringspaces=false, upquote=true}
% C++ specific high-lighting. Comments are 50/50 shades of green/black and strings coloured with 60/40 red/black mixture.
\lstset{language=[ISO]C++, commentstyle=\color{green!50!black}, keywordstyle=\color{blue}, stringstyle=\color{red!60!black}}

%opening
\title{Data Structures and Algorithms\\Class 11\\Introduction to Trees}
\author{Attique Dawood}
\date{March 03, 2015\\[0.2cm] Last Modified: \today, \currenttime}
\begin{document}
\maketitle
\section{Announcements}
\begin{itemize}
\item Assignment 02 deadline is now Thursday.
\end{itemize}
\section{Revision}
\begin{itemize}
\item Recursion.
\end{itemize}
\section{Trees: Basic Terminology~\cite[Sec. 3.1]{Aho}}
Tree is a collection of elements called $nodes$. One of the nodes is the $root$ which is parent of all the nodes. A node can store anything, a number, a string or a symbol. A $tree$ can be defined recursively as
\begin{itemize}
\item[1.] A single node is by itself a tree. This is also the root of the tree.
\item[2.] Suppose $n$ is a node. We have trees $T_1, T_2,..., T_k$ with roots $n_1, n_2,..., n_k$. We can make a new tree by making $n$ the parent of nodes $n_1, n_2,..., n_k$. In this tree, $n$ is the root and $T_1, T_2,..., T_k$ are the \textit{sub--trees} of the root. Nodes $n_1, n_2,..., n_k$ are called the $children$ of node $n$.
\subsection{Example: Table of Contents}
Table of contents in a book can be considered as a tree.
\subsection{Path}
If $n_1, n_2,..., n_k$ are sequence of nodes in a tree such that $n_i$ is the parent of $n_{i+1}$, then for $1\leq i<k$, this sequence is a $path$ from node $n_1$ to node $n_k$. There is a path of length zero from each node to itself. The length of a path is one less than the number of nodes in the path.
\subsection{Ancestor and Descendant}
If there is a path from node $a$ to node $b$ then $a$ is an $ancestor$ of $b$ and $b$ is a $descendent$ of $a$. Ancestor or descendent of a node, other than the node itself, is called a $proper$ ancestor or $descendant$. In a tree root is the only node with no proper ancestor. A node with no proper descendant is called a $leaf$.
\subsection{Height and Depth}
$Height$ of a node in a tree is the longest distance from that node to a leaf. $Height of a tree$ is the height of the root. $Depth$ of a node is the length of the unique path from root to that node.
\subsection{Ordering}
Left--to--right ordering of $siblings$ (children of the same node) can be extended to compare any two nodes that are not related by ancestor--descendent relationship. If $a$ and $b$ are siblings and $a$ is to the left of $b$ then all descendants of $a$ are to the left of all the descendants of $b$. A path can be drawn from a root a node $n$. All the nodes not in the path are either to the left or right of node $n$.
\subsection{Preorder, Inorder and Postorder Traversals}
\end{itemize}
\begin{lstlisting}
int fact(int n)
{
    if (n==0||n==1)
        return 1;
    else
        return n*fact(n-1);
}
\end{lstlisting}
%\section{Exercises}
%\noindent\textbf{Question:} ?\\[0.2cm]
%\nocite{*}
\bibliographystyle{plain}
\bibliography{DSRef}
\end{document}
