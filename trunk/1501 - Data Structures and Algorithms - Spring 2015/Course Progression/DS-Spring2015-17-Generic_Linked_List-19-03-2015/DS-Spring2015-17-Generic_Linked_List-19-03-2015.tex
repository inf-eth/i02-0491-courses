\documentclass[12pt,a4paper]{article}

% Margins.
\setlength{\oddsidemargin}{0in}
\setlength{\evensidemargin}{0in}
\setlength{\headheight}{12pt}
\setlength{\headsep}{0pt}
\setlength{\topmargin}{-60pt}
\setlength{\textwidth}{6.5in}
\setlength{\textheight}{10.75in}

\usepackage{amsmath}
\usepackage{float}
\usepackage{graphicx}
\usepackage[hyphens]{url}
\usepackage{hyperref}	% Clickable links to figures, references and urls.
\usepackage{datetime}
\usepackage{longtable}

% Drawing.
\usepackage{pgf}
\usepackage{tikz}

% Listings for formatting code.
\usepackage{listings}
\usepackage{textcomp}
% General options.
\lstset{breaklines=true, basicstyle=\small\ttfamily, tabsize=4, numbers=left, stepnumber=1, frame=single, showstringspaces=false, upquote=true}
% C++ specific high-lighting. Comments are 50/50 shades of green/black and strings coloured with 60/40 red/black mixture.
\lstset{language=[ISO]C++, commentstyle=\color{green!50!black}, keywordstyle=\color{blue}, stringstyle=\color{red!60!black}}

%opening
\title{Data Structures and Algorithms\\Class 17\\Generic Linked List}
\author{Attique Dawood}
\date{March 19, 2015\\[0.2cm] Last Modified: \today, \currenttime}
\begin{document}
\maketitle
\section{Announcements}
\begin{itemize}
\item None.
\end{itemize}
\section{Revision}
\begin{itemize}
\item Sessional--I solution.
\end{itemize}
\section{\texttt{void} Pointers}
\begin{itemize}
\item What if you don't know the type of data you're going to dynamically allocate?
\item Solution: Make a \verb|void*|. It can point to any datatype!
\item \underline{Must explicitly cast the pointer when accessing value.}
\end{itemize}
\begin{lstlisting}[caption={\texttt{void*} usage}]
#include <iostream>
using namespace std;

int main()
{
	void* vp;
	vp = new int[5];
	delete[] vp;

	vp = new float;
	*((float*)vp) = 3.3f; // Notice, explicit casting from void* to float* before dereferencing.
	cout << "*vp: " << *((float*)vp) << endl;
	delete vp;

	return 0;
}
\end{lstlisting}
\section{Templates in C++}
\begin{itemize}
\item Templates offer a mechanism to write generic code.
\item If we want to make a sum function to add two data types then data types can be \verb|int|, \verb|float|, \verb|double| or even \verb|complex|.
\item We would have to make sum function for all these types.
\item By making a template, we let compiler decide which instance of sum function to call.
\item Template calling can be implicit or explicit.
\item If no type is mentioned with function call, compiler will decide what is best instantiation of template.
\item If a type is mentioned with function call (e.g., \verb|sum<int> (3, 7)|) then that particular instance of template is instantiated by compiler.
\item Templates can also be defined for multiple types.
\item Templates can be defined for functions as well as classes.
\end{itemize}
\begin{lstlisting}[caption={Traditional way of writing sum function}]
int sum(int a, int b)
{
	return (a+b);
}
float sum(float a, float b)
{
	return (a+b);
}
...
complex sum(complex a, complex b)
{
	return (a+b);
}
...
\end{lstlisting}
\begin{lstlisting}[caption={Sum function template}]
// Sum function template.
// Depending on type of argument passed in main, that will be substituted in place of 'MyType'.
// For example, if called as sum(2.3, 7.7), MyType would be double.
template <class MyType> MyType sum(MyType a, MyType b)
{
	return (a+b);
}
int main()
{
	sum(3.2, 7.8); // implicit instantiation with double type.
	sum(3, 8); // implicit instantiation with int type.
	sum<double> (3.2, 7.8); // explicit double instantiation.
	
	return 0;
}
\end{lstlisting}
\begin{lstlisting}[caption={Sum function template for mixed types}]
template <class MyType1, class MyType2> MyType1 sum(MyType1 a, MyType2 b)
{
	return (a+b);
}
int main()
{
	sum(3.2, 7.8); // implicit instantiation with both double types.
	sum(3, 8.9); // implicit instantiation for int and double types.
	sum<int, double> (3, 7.8); // explicit instantiation.
	
	return 0;
}
\end{lstlisting}
\begin{lstlisting}[caption={Class template}]
#include <iostream>
using namespace std;
template <class MyType> class GenericArray
{
	private:
	MyType Data[5];
	public:
	void Input()
	{
		for (int i=0; i<5; i++)
			cin >> Data[i];
	}
	void Display()
	{
		for (int i=0; i<5; i++)
			cout << Data[i] << " ";
		cout << endl;
	}
};
int main()
{
	GenericArray<int> IntArray;
	GenericArray<float> FloatArray;
	IntArray.Input();
	IntArray.Display();
	FloatArray.Input();
	FloatArray.Display();
	
	return 0;
}
\end{lstlisting}
\newpage
\section{Generic Linked List}
Generic data structures can be created using void pointers or templates. Here is an example of a templatised linked list that can store any type of data.
\begin{lstlisting}[caption={Linked List Template},escapechar=$]
#include <iostream>
using namespace std;

// Node
template <class T> class Node
{
public:
	T Data;
	Node<T>* Next;
};

// Linked list class.
template <class Y> class LinkedList
{
private:
	Node$\textcolor{red}{<Y>}$* First;
	int Count;
public:
	// Constructor.
	LinkedList(): First(NULL), Count(0)
	{
	}
	// Adds a node at the end of linked list.
	void AddNode(Y pData)
	{
		// If creating first node we need to change first pointer.
		if (First == NULL)
		{
			Node$\textcolor{red}{<Y>}$* NewNode;
			NewNode = new Node$\textcolor{red}{<Y>}$;
			NewNode->Data = pData;
			NewNode->Next = NULL;
			First = NewNode;
			Count++;
		}
		// We need to traverse the whole list to get to last node.
		else
		{
			// Find last node.
			Node$\textcolor{red}{<Y>}$* temp;
			temp = First;
			while (temp->Next != NULL)
				temp = temp->Next;
			// temp now points to last node.
			Node$\textcolor{red}{<Y>}$* NewNode;
			NewNode = new Node$\textcolor{red}{<Y>}$;
			temp->Next = NewNode;
			NewNode->Data = pData;
			NewNode->Next = NULL;
			Count++;
		}
	}
	// Displays all entries in linked list.
	void Display()
	{
		Node$\textcolor{red}{<Y>}$* temp;
		temp = First;
		while(temp != NULL)
		{
			cout << "Data = " << temp->Data << endl;
			temp = temp->Next;
		}
	}
	// Search a data value.
	int Search(Y Key)
	{
		Node$\textcolor{red}{<Y>}$* temp;
		temp = First;
		while(temp != NULL)
		{
			if (temp->Data == Key)
				return 0;

			temp = temp->Next;
		}
		return -1; // If we ever get here then search was unsuccessful.
	}
	// Delete a Node by data value.
	int Delete(Y Key)
	{
		// If List is empty then nothing to do.
		if (First == NULL)
			return -1;

		Node$\textcolor{red}{<Y>}$* previous;
		Node$\textcolor{red}{<Y>}$* current;
		previous = current = First;
		// Checking first node.
		if (current->Data == Key)
		{
			// Delete First node. First will now point to next node.
			First = First->Next;
			delete current;
			Count--;
			return 0;
		}
		else
		{
			current = current->Next; // Move current pointer to next node.
			while(current != NULL)
			{
				// If entry is matched then link previous node to next node and delete current node.
				if (current->Data == Key)
				{
					previous->Next = current->Next;
					delete current;
					Count--;
					return 0;
				}
				// Move pointers forward.
				previous = current;
				current = current->Next;
			}
		}
		return -1; // If we ever get here then search was unsuccessful.
	}
	// Get size of linked list.
	int GetCount()
	{
		return Count;
	}
	// Subscript operator.
	Y& operator[](int Index)
	{
		// Checking index bounds.
		if (Index < 0 || Index >= Count)
		{
			cout << "Index out of bounds. Exiting..." << endl;
			exit(-1);
		}
		else
		{
			// Move temp pointer to point to node at desired index.
			Node$\textcolor{red}{<Y>}$* temp = First;
			for (int i=0; i<Index; i++)
				temp = temp->Next;

			return temp->Data;
		}
	}
};

int main()
{
	LinkedList<int> TestList;

	TestList.AddNode(-1);
	TestList.AddNode(2);
	TestList.AddNode(0);
	TestList.AddNode(7);

	cout << "Displaying Linked List: " << endl;
	TestList.Display();

	int SearchKey = 2;
	if (TestList.Search(SearchKey) == 0)
		cout << SearchKey << " is in the list." << endl;

	if (TestList.Delete(SearchKey) == 0)
		cout << SearchKey << " was deleted." << endl;

	cout << "Displaying Linked List: " << endl;
	TestList.Display();
	
	TestList[0] = -122;
	cout << "Displaying Linked List: " << endl;
	for (int i=0; i<TestList.GetCount(); i++)
		cout << TestList[i] << endl;

	return 0;
}
\end{lstlisting}
%\section{Exercises}
%\noindent\textbf{Question:} ?\\[0.2cm]
%\nocite{*}
\bibliographystyle{plain}
\bibliography{DSRef}
\end{document}
