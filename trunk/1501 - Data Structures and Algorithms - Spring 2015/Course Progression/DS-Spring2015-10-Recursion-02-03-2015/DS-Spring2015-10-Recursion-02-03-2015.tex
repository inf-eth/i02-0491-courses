\documentclass[12pt,a4paper]{article}

% Margins.
\setlength{\oddsidemargin}{0in}
\setlength{\evensidemargin}{0in}
\setlength{\headheight}{12pt}
\setlength{\headsep}{0pt}
\setlength{\topmargin}{-60pt}
\setlength{\textwidth}{6.5in}
\setlength{\textheight}{10.75in}

\usepackage{amsmath}
\usepackage{float}
\usepackage{graphicx}
\usepackage[hyphens]{url}
\usepackage{hyperref}	% Clickable links to figures, references and urls.
\usepackage{datetime}
\usepackage{longtable}

% Drawing.
\usepackage{pgf}
\usepackage{tikz}

% Listings for formatting code.
\usepackage{listings}
\usepackage{textcomp}
% General options.
\lstset{breaklines=true, basicstyle=\small\ttfamily, tabsize=4, numbers=left, stepnumber=1, frame=single, showstringspaces=false, upquote=true}
% C++ specific high-lighting. Comments are 50/50 shades of green/black and strings coloured with 60/40 red/black mixture.
\lstset{language=[ISO]C++, commentstyle=\color{green!50!black}, keywordstyle=\color{blue}, stringstyle=\color{red!60!black}}

%opening
\title{Data Structures and Algorithms\\Class 10\\Recursion}
\author{Attique Dawood}
\date{March 02, 2015\\[0.2cm] Last Modified: \today, \currenttime}
\begin{document}
\maketitle
\section{Announcements}
\begin{itemize}
\item None.
\end{itemize}
\section{Revision}
\begin{itemize}
\item Queues and mappings.
\end{itemize}
\section{Recursion}
Recursion is useful for solving recursively--defined problems like factorial and Fibonacci series. Recursive and sequential implementation of factorial and Fibonacci series are given.
\begin{lstlisting}[caption={Recursive Factorial}]
int fact(int n)
{
    if (n==0||n==1)
        return 1;
    else
        return n*fact(n-1);
}
\end{lstlisting}
\begin{lstlisting}[caption={Loop Based Factorial}]
int fact(int n)
{
	int factorial=1;
	for (int i=n; i>1; i--)
		factorial=factorial*i;
	return factorial;
}
\end{lstlisting}
\begin{lstlisting}[caption={Recursive Fibonacci Series}]
int Fib(int n)
{
	if (n==1||n==2)
		return 1;
	else return Fib(n-1)+Fib(n-2);
}
\end{lstlisting}
\begin{lstlisting}[caption={Loop Based Fibonacci Series}]
int Fib(int n)
{
	int Fib1 = 1;
	int Fib2 = 1;
	int Fib3 = 1;
	for (int i=2; i<n; i++)
	{
		Fib3 = Fib1+Fib2;
		Fib1 = Fib2;
		Fib2 = Fib3;
	}
	return Fib3;
}
\end{lstlisting}
\section{Exercises}
\noindent\textbf{Question:} What are the space and time complexities of given code listings?\\[0.2cm]
\nocite{*}
\bibliographystyle{plain}
\bibliography{DSRef}
\end{document}
