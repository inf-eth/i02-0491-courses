\documentclass[12pt,a4paper]{article}

% Margins.
\setlength{\oddsidemargin}{0in}
\setlength{\evensidemargin}{0in}
\setlength{\headheight}{12pt}
\setlength{\headsep}{0pt}
\setlength{\topmargin}{-60pt}
\setlength{\textwidth}{6.5in}
\setlength{\textheight}{10.75in}

\usepackage{amsmath}
\usepackage{float}
\usepackage{graphicx}
\usepackage[hyphens]{url}
\usepackage{hyperref}	% Clickable links to figures, references and urls.
\usepackage{datetime}
\usepackage{longtable}

% Drawing.
\usepackage{pgf}
\usepackage{tikz}

% Listings for formatting code.
\usepackage{listings}
\usepackage{textcomp}
% General options.
\lstset{breaklines=true, basicstyle=\small\ttfamily, tabsize=4, numbers=left, stepnumber=1, frame=single, showstringspaces=false, upquote=true}
% C++ specific high-lighting. Comments are 50/50 shades of green/black and strings coloured with 60/40 red/black mixture.
\lstset{language=[ISO]C++, commentstyle=\color{green!50!black}, keywordstyle=\color{blue}, stringstyle=\color{red!60!black}}

%opening
\title{Data Structures and Algorithms\\Class 09\\Queues and Mappings}
\author{Attique Dawood}
\date{February 26, 2015\\[0.2cm] Last Modified: \today, \currenttime}
\begin{document}
\maketitle
\section{Announcements}
\begin{itemize}
\item None.
\end{itemize}
\section{Revision}
\begin{itemize}
\item Stack.
\end{itemize}
\section{Queues~\cite[Sec. 2.4]{Aho}}
Queue is an ADT where all insertions take place at one end called $rear$ and deletions take place at the other end called $Front$. Following operations are defined for a queue ADT:
\begin{itemize}
\item \verb|EnQueue(x)|: Insert $x$ at rear.
\item \verb|Dequeue()|: Delete $front$.
\item \verb|Front()|: Returns the element at the front of queue..
\item \verb|MakeNULL()|: Empties queue by dequeuing all elements.
\item \verb|IsEmpty()|: Checks if queue is empty or not. Returns true or false.
\end{itemize}
Queues are useful for scheduling. One such example is multi--tasking by CPU. A special type of queue is called $priority queue$. It keeps track of priority of elements. These elements are then placed in the queue according to their $priority$. VIPs given preference in a queue is one such example. Tasks scheduled on CPU can also be given priority.
\section{Mappings~\cite[Sec. 2.5]{Aho}}
A $mapping$ is an association between two sets of values. For example each alphabet can be assigned a number from 1 to 26. Encryption/decryption is an example of a mapping where inverse mapping exists. If no inverse exists then the mapping is known as a $hash$ function.
%\section{Exercises}
%\noindent\textbf{Question:} ?\\[0.2cm]
%\nocite{*}
\bibliographystyle{plain}
\bibliography{DSRef}
\end{document}
