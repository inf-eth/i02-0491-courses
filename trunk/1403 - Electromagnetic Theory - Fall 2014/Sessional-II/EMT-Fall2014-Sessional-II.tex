\documentclass[12pt,a4paper]{article}

% To use this template make changes to following:
% 1. Fill-ables section.
% 2. Instructions.
% 3. Marks table.
% 4. Actual questions.

% ================================ 1. Fill-ables ================================
\newcommand\University{National University of Computer and Emerging Sciences}
\newcommand\Department{School of Engineering}
\newcommand\Campus{Islamabad Campus}
\newcommand\Semester{Fall 2014}
\newcommand\Exam{Sessional II}
\newcommand\Subject{EE305--Electromagnetic Theory}
\newcommand\ExamDate{Tuesday, November 11, 2014}
\newcommand\InstructorOne{Attique Dawood}
\newcommand\InstructorTwo{\null~}
\newcommand\InstructorThree{\null}
\newcommand\TotalTime{01 Hour}
\newcommand\TotalMarks{50}
\newcommand\TotalQuestions{5}
\newcommand\TotalPages{\pageref{LastPage}} % Automatic: No need to change this.
% Marks of each question
\def\Qone{10}
\def\Qtwo{10}
\def\Qthree{10}
\def\Qfour{10}
\def\Qfive{10}
\def\Qsix{0}
\def\Qseven{0}
\def\Qeight{0}
\def\Qnine{0}
\def\Qten{0}
% ============================================================================

% ============== 2. Packages ==============
\usepackage{amsmath}
\usepackage{float}
\usepackage{graphicx}
\usepackage[hyphens]{url}
\usepackage[hidelinks]{hyperref}	% Clickable links to figures, references and urls.
\usepackage{lastpage}
\usepackage{array}
\usepackage{fancyhdr}
\usepackage{afterpage}
% Drawing packages.
\usepackage{pgf}
\usepackage{tikz}
% Listings for formatting code.
\usepackage{listings}
\usepackage{textcomp}

% General listings options.
\lstset{breaklines=true, basicstyle=\footnotesize\ttfamily, tabsize=4, numbers=left, stepnumber=1, frame=none, showstringspaces=false, upquote=true}
% C++ specific high-lighting. Comments are 50/50 shades of green/black and strings coloured with 60/40 red/black mixture.
\lstset{language=[ISO]C++, commentstyle=\color{green!50!black}, keywordstyle=\color{blue}, stringstyle=\color{red!60!black}}

% Table cell alignment directives.
\newcolumntype{L}[1]{>{\raggedright\let\newline\\\arraybackslash\hspace{0pt}}m{#1}}
\newcolumntype{C}[1]{>{\centering\let\newline\\\arraybackslash\hspace{0pt}}m{#1}}
\newcolumntype{R}[1]{>{\raggedleft\let\newline\\\arraybackslash\hspace{0pt}}m{#1}}

% Line spacing.
\def\SingleSpacing{\def\baselinestretch{1}\large\normalsize}
\def\DoubleSpacing{\def\baselinestretch{1.5}\large\normalsize}

% Margins.
\setlength{\oddsidemargin}{0in}
\setlength{\evensidemargin}{0in}
\setlength{\headheight}{28pt}
\setlength{\headsep}{2.5pt}
\setlength{\topmargin}{-60pt}
\setlength{\textwidth}{6.5in}
\setlength{\textheight}{10.75in} % Actual: 10.75in

% ============================= 3. Header and Footer ============================
\pagestyle{empty}
% Header
\chead
{
	{\large\textbf{\University}}\\
	\begin{minipage}{0.45\textwidth}
	\begin{center}
	{\small\textbf{\Department}}
	\end{center}
	\end{minipage}
	\begin{minipage}{0.45\textwidth}
	\begin{center}
	{\small\textbf{\Campus}}
	\end{center}
	\end{minipage}
}
% Footer
\lfoot{{\small\Exam}}
\cfoot{{\small\Semester}}
\rfoot{{\small Page \textbf{\thepage}~of \textbf{\TotalPages}}}
\renewcommand{\headrulewidth}{0.4pt}
\renewcommand{\footrulewidth}{0.4pt}
% ================================= 4. Front Page ===============================
\begin{document}
% A cute macro to add up marks of all individual questions. Uncomment if you want to use this.
\pgfmathtruncatemacro\TotalMarks{\Qone+\Qtwo+\Qthree+\Qfour+\Qfive+\Qsix+\Qseven+\Qeight+\Qnine+\Qten}
% Use this macro if marks are in decimal points
%\newcommand\TotalMarks{\pgfmathsetmacro\TotalMarks{\Qone+\Qtwo+\Qthree+\Qfour+\Qfive+\Qsix+\Qseven+\Qeight+\Qnine+\Qten}}
\begin{minipage}[t]{0.6\textwidth}
\begin{flushleft}
\DoubleSpacing
{\Large\textbf{\Subject}}\\
{\normalsize\ExamDate}\\
{\large\textbf{Course Instructor}}\\
{\normalsize\InstructorOne}\\
{\normalsize\InstructorTwo}\\
{\normalsize\InstructorThree}
\end{flushleft}
\end{minipage}
\begin{minipage}[t]{0.01\textwidth}
~
\end{minipage}
\begin{minipage}[t]{0.325\textwidth}
\DoubleSpacing
{\normalsize Serial No:}\\
{\Large\textbf{\Exam}}\\
{\large\textbf{Total Time: \TotalTime}}\\
{\large\textbf{Total Marks: \TotalMarks}}\\[1cm]
\rule{5cm}{0.2mm}\\[-0.25cm]
{\small Signature of Invigilator}
\end{minipage}
\SingleSpacing
~\\[1.5cm] % Extra space.
\rule{7cm}{0.2mm}~\rule{2.5cm}{0.2mm}~\rule{2cm}{0.2mm}~\rule{4.5cm}{0.2mm}\\
{\small Student Name\hspace{4.75cm}Roll No\hspace{1.35cm}Section\hspace{0.95cm}Signature}\\[1cm]
% ============================ 5. Instructions ==================================
\textbf{DO NOT OPEN THE QUESTION BOOK OR START UNTIL INSTRUCTED.}\\
\textbf{Instructions:}
\begin{enumerate}
\itemsep0em
\item Verify at the start of the exam that you have a total of \TotalQuestions~questions printed on \TotalPages~pages including this title page.
\item Attempt all questions on the question-book and in the given order.
\item The exam is closed books, closed notes. Please see that the area in your threshold is free of any material classified as `useful in the paper' or else there may be a charge of cheating.
\item Read the questions carefully for clarity of context and understanding of meaning and make assumptions wherever required, for neither the invigilator will address your queries, nor the teacher/examiner will come to the examination hall for any assistance.
\item Fit in all your answers in the provided space. You may use extra space on the last page if required. If you do so, clearly mark question/part number on that page to avoid confusion. 
\item Use only your own stationery and calculator. If you do not have your own calculator, use manual calculations. 
\item Use only permanent ink-pens. Only the questions attempted with permanent ink-pens will be considered. Any part of paper done in lead pencil cannot be claimed for checking/rechecking.
\item \textbf{All distances and dimensions are in meters}.
\end{enumerate}
% =============================== 6. Marks Table ================================
\begin{table}[H]
\begin{center}
\vspace{0.3cm}
	{\footnotesize \begin{tabular}{|C{1.8cm}|C{0.75cm}|C{0.75cm}|C{0.75cm}|C{0.75cm}|C{0.75cm}|c|}
	\hline
		\rule{0pt}{4.6ex} & Q-1 & Q-2 & Q-3 & Q-4 & Q-5 & \textbf{Total}\\[-0.5ex]
		\hline
		\rule{0pt}{2.5ex}\textbf{Total Marks}& \Qone & \Qtwo & \Qthree & \Qfour & \Qfive & \TotalMarks\\
		\hline
		\rule{0pt}{2.5ex}\textbf{Marks Obtained}& & & & & & \\
	\hline
	\end{tabular}}
\end{center}
\end{table}
{\small \textbf{Vetted By: \rule{6cm}{0.2mm} Vetter Signature: \rule{4.5cm}{0.2mm}}}
\setlength{\textheight}{10.45in}
\newpage
\pagestyle{fancy}
% ================================== 7. Questions ===============================
\noindent\textbf{Question 1: Electric Field \hfill \Qone~marks}\\
The planes $z=-1$ and $z=3$ are uniformly sprinkled with some charge. The surface charge densities on these planes were found to be $-2\epsilon_0$ and $4\epsilon_0$ C/m$^2$, respectively. Find electric field everywhere.
\newpage
\noindent\textbf{Question 2: Gauss' Law \hfill \Qtwo~marks}\\
A volume charge with density $\rho_v=\dfrac{1}{2\pi r^2}$ C/m$^3$ exists in the region $r<1$. This charge is enclosed in a conducting shell at $2<r<3$. The conductor is  given a net charge 1 C. Sketch problem geometry and find electric field everywhere.
\newpage
\noindent\textbf{Question 3: Electric Potential\hfill \Qthree~marks}\\
A volume charge with density $\rho_v=r$ C/m$^3$ exists in the region $r<2$. Find electric potential at $r=2$.
\newpage
\noindent\textbf{Question 4: Current\hfill \Qfour~marks}\\
Electric potential in a dielectric region ($\epsilon=4\epsilon_0$) is $V=xy^2z$ V.\\[0.2cm]
a. Find the volume charge density $\rho_v$.\\
b. If this charge is flowing with a velocity $\mathrm{\textbf u}=10^9y\hat y$ m/s, calculate the current crossing the surface $0<x<1$, $0<z<1$, $y=2$.\\[0.2cm]
\noindent\textbf{Note:} $\mathrm{\textbf J}=\rho_v\mathrm{\textbf u}$.\\[0.2cm]
\newpage
\noindent\textbf{Question 5: Polarisation\hfill \Qfive~marks}\\
Electric field in a region of free space is $\mathrm{\textbf E}=5x\hat x$ V/m. An infinite dielectric slab ($\epsilon=2.5\epsilon_0$) is placed in the region $1<x<3$. Find\\[0.2cm]
a. \textbf{E}, \textbf{D} and \textbf{P} in slab.\\
b. $\rho_{pv}$ and $\rho_{ps}$.
\end{document}