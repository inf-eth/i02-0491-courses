\documentclass[12pt,a4paper]{article}

% Margins.
\setlength{\oddsidemargin}{0in}
\setlength{\evensidemargin}{0in}
\setlength{\headheight}{12pt}
\setlength{\headsep}{0pt}
\setlength{\topmargin}{-60pt}
\setlength{\textwidth}{6.5in}
\setlength{\textheight}{10.75in}

\usepackage{amsmath}
\usepackage{float}
\usepackage{graphicx}
\usepackage[hyphens]{url}
\usepackage{hyperref}	% Clickable links to figures, references and urls.
\usepackage{datetime}

% Drawing.
\usepackage{pgf}
\usepackage{tikz}

% Listings for formatting code.
\usepackage{listings}
\usepackage{textcomp}
% General options.
\lstset{breaklines=true, basicstyle=\small\ttfamily, tabsize=4, numbers=left, stepnumber=1, frame=single, showstringspaces=false, upquote=true}
% C++ specific high-lighting. Comments are 50/50 shades of green/black and strings coloured with 60/40 red/black mixture.
\lstset{language=[ISO]C++, commentstyle=\color{green!50!black}, keywordstyle=\color{blue}, stringstyle=\color{red!60!black}}

%opening
\title{\vspace{-1.5cm}Electromagnetic Theory - Fall 2014\\Quiz \#03A}
\date{\vspace{-1.5cm}Date: 15--10--2014}
\begin{document}
\maketitle
\vspace{-0.5cm}
\noindent \textbf{Time: 15 minutes\hfill Total Marks: 10}\\[0.3cm]
\noindent \textbf{Name:\rule{8cm}{1pt}\hfill Roll Number:\rule{3cm}{1pt}}\\[0.5cm]
\noindent \textbf{Question:} Find the flux of \textbf{A}$=xy^2\hat x+zx\hat y+xz\hat z$ over the surface of unit cube $0<x,y,z<1$.
\noindent \textbf{Solution:} We can use divergence theorem to find the flux. From divergence theorem $\oint\limits_S\mathrm{\textbf A}\cdot d\mathrm{\textbf S}=\int\limits_v(\nabla\cdot\mathrm{\textbf A})dv$.\\
$\nabla\cdot\mathrm{\textbf A}=y^2+x$\\
$\int\limits_v(\nabla\cdot\mathrm{\textbf A})dv=\int\limits_0^1\int\limits_0^1\int\limits_0^1(y^2+x)dxdydz=\dfrac{1}{3}+\dfrac{1}{2}=\dfrac{5}{6}$
\end{document}
