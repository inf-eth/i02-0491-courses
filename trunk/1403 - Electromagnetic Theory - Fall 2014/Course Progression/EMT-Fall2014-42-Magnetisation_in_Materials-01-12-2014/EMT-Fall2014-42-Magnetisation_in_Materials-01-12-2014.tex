\documentclass[12pt,a4paper]{article}

% Margins.
\setlength{\oddsidemargin}{0in}
\setlength{\evensidemargin}{0in}
\setlength{\headheight}{12pt}
\setlength{\headsep}{0pt}
\setlength{\topmargin}{-60pt}
\setlength{\textwidth}{6.5in}
\setlength{\textheight}{10.75in}

\usepackage{amsmath}
\usepackage{float}
\usepackage{graphicx}
\usepackage[hyphens]{url}
\usepackage{hyperref}	% Clickable links to figures, references and urls.
\usepackage{datetime}
\usepackage{longtable}
\usepackage{subfigure}

% Links direct to top of figures.
\usepackage[all]{hypcap}

% Drawing.
\usepackage{pgf}
\usepackage{tikz}

% Listings for formatting code.
\usepackage{listings}
\usepackage{textcomp}
% General options.
\lstset{breaklines=true, basicstyle=\small\ttfamily, tabsize=4, numbers=left, stepnumber=1, frame=single, showstringspaces=false, upquote=true}
% C++ specific high-lighting. Comments are 50/50 shades of green/black and strings coloured with 60/40 red/black mixture.
\lstset{language=[ISO]C++, commentstyle=\color{green!50!black}, keywordstyle=\color{blue}, stringstyle=\color{red!60!black}}

%opening
\title{Electromagnetic Theory\\Class 42\\Magnetisation in Materials}
\author{Attique Dawood}
\date{December 01, 2014\\[0.2cm] Last Modified: \today, \currenttime}
\begin{document}
\maketitle
\section{Revision}
\begin{itemize}
\item Vector magnetic potential.
\end{itemize}
\section{Magnetic Force on a Current Carrying Wire}
A current carrying wire placed in a magnetic field experiences a force given by
\begin{equation}
\mathrm{\textbf F}=\int\limits_L Id\textbf{l}\times\mathrm{\textbf B}.
\end{equation}
\section{Magnetic Dipole Moment}
A magnetic dipole is formed from a current carrying loop. The magnetic moment is defined as
\begin{equation}
\mathrm{\textbf m}=IS\hat n.
\end{equation}
Where $I$ is the current in the loop, $S$ is the surface area of loop and $\hat n$ is a unit vector normal to the loop.
\section{Magnetisation in Materials}
When a material is placed in magnetic field, magnetic dipoles are formed in the material. This phenomenon is called `magnetisation'. Magnetisation is defined as the number of magnetic dipoles per unit volume. Mathematically
\begin{equation}
\mathrm{\textbf M}=\lim_{\Delta v \to 0}\dfrac{\Sigma\mathrm{\textbf m}_n}{\Delta v}~\mathrm{A/m}.
\end{equation}
\textbf{M} and \textbf{B} in a medium are related by
\begin{equation}
\mathrm{\textbf B}=\mu_0\left(\mathrm{\textbf H}+\mathrm{\textbf M}\right)
\end{equation}
and
\begin{equation}
\mathrm{\textbf M}=\chi_m\mathrm{\textbf H}.
\end{equation}
$\chi_m$ is magnetic susceptibility of a medium and given by
\begin{equation}
\chi_m=\mu_r-1.
\end{equation}
Magnetic susceptibility shows how much a material will magnetise when subject to magnetic field. Susceptibility of free space is understandably 0. Also note that in the relationship
\begin{equation}
\mathrm{\textbf B}=\mu\mathrm{\textbf H}=\mu_r\mu_0\mathrm{\textbf H}
\end{equation}
magnetisation is implicit where $\mu_r=\chi_me+1$. Another way of writing this is
\begin{equation}
\mathrm{\textbf B}=\mu\mathrm{\textbf H}=\mu_0\left(\mathrm{\textbf H}+\mathrm{\textbf M}\right)=\mu_0\mathrm{\textbf H}+\chi_m\mu_0\mathrm{\textbf H}.
\end{equation}
\section{Surface and Volume Bound Current Densities}
Due to magnetisation currents are set up on the surface and inside the magnetised material. The volume bound current density is given by
\begin{equation}
\mathrm{\textbf J}_b=\nabla\times\mathrm{\textbf M}~\mathrm{A/m}^2
\end{equation}
and surface bound current density on any surface with normal vector $\hat n$ is given by
\begin{equation}
\mathrm{\textbf K}_b=\mathrm{\textbf M}\times\hat n~\mathrm{A/m}.
\end{equation}
\section{Exercises}
\noindent\textbf{Question 1:} Magnetic flux density in a region is $\mathrm{\textbf B}=4\hat x$ Wb/m$^2$. A wire segment part of a circuit carrying 2 A current is along $z$--axis at $0<z<5$. Find the magnetic force on this wire segment.\\[0.2cm]
\noindent\textbf{Question 2:} In a ferromagnetic material ($\mu=4.5\mu_0$) placed at $0<y<2$, $\mathrm{\textbf B}=4y\hat z$ mWb/m$^2$. Calculate
\begin{itemize}
\item[a.] $\chi_m$
\item[b.] \textbf{H}
\item[c.] \textbf{M}
\item[d.] \textbf{J}$_b$
\item[e.] \textbf{K}$_b$
\end{itemize}
%%\nocite{*}
\bibliographystyle{plain}
\bibliography{EMTRef}
\end{document}
