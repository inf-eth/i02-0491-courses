\documentclass[12pt,a4paper]{article}

% Margins.
\setlength{\oddsidemargin}{0in}
\setlength{\evensidemargin}{0in}
\setlength{\headheight}{12pt}
\setlength{\headsep}{0pt}
\setlength{\topmargin}{-60pt}
\setlength{\textwidth}{6.5in}
\setlength{\textheight}{10.75in}

\usepackage{amsmath}
\usepackage{float}
\usepackage{graphicx}
\usepackage[hyphens]{url}
\usepackage{hyperref}	% Clickable links to figures, references and urls.
\usepackage{datetime}
\usepackage{longtable}
\usepackage{subfigure}

% Links direct to top of figures.
\usepackage[all]{hypcap}

% Drawing.
\usepackage{pgf}
\usepackage{tikz}

% Listings for formatting code.
\usepackage{listings}
\usepackage{textcomp}
% General options.
\lstset{breaklines=true, basicstyle=\small\ttfamily, tabsize=4, numbers=left, stepnumber=1, frame=single, showstringspaces=false, upquote=true}
% C++ specific high-lighting. Comments are 50/50 shades of green/black and strings coloured with 60/40 red/black mixture.
\lstset{language=[ISO]C++, commentstyle=\color{green!50!black}, keywordstyle=\color{blue}, stringstyle=\color{red!60!black}}

%opening
\title{Electromagnetic Theory\\Class 41\\Vector Magnetic Potential}
\author{Attique Dawood}
\date{November 28, 2014\\[0.2cm] Last Modified: \today, \currenttime}
\begin{document}
\maketitle
\section{Revision}
\begin{itemize}
\item Magnetic field of a coaxial wire.
\end{itemize}
\section{Magnetic Flux and Maxwell's Equation}
The fourth and final Maxwell's equation is
\begin{equation}
\nabla\times\mathrm{\textbf B}=0.
\end{equation}
Magnetic flux is defined as
\begin{equation}
\psi=\int\limits_S\mathrm{\textbf B}\cdot d\mathrm{\textbf S}.
\end{equation}
\section{Scalar Magnetic Potential}
If $\mathrm{\textbf J}=0$ then we can define a scalar magnetic potential for magnetic field
\begin{equation}
\mathrm{\textbf H}=-\nabla V_m.
\end{equation}
\section{Vector Magnetic Potential}
Vector magnetic potential \textbf{A} is defined as
\begin{equation}
\mathrm{\textbf B}=\nabla\times\mathrm{\textbf A}.
\end{equation}
Just as electric potential can be obtained from charge, magnetic vector potential can also be obtained from current as
\begin{equation}
\mathrm{\textbf A}=\int\limits_L\dfrac{\mu_0Id\textbf{l}}{4\pi\left|\mathrm{\textbf r}-\mathrm{\textbf r}'\right|}=\int\limits_S\dfrac{\mathrm{\mu_0\textbf K}dS}{4\pi\left|\mathrm{\textbf r}-\mathrm{\textbf r}'\right|}=\int\limits_v\dfrac{\mu_0\mathrm{\textbf J}dv}{4\pi\left|\mathrm{\textbf r}-\mathrm{\textbf r}'\right|}.
\end{equation}
Magnetic flux can also be obtained from vector magnetic potential as
\begin{equation}
\psi=\int\limits_S\mathrm{\textbf B}\cdot d\mathrm{\textbf S}=\int\limits_S\left(\nabla\times\mathrm{\textbf A}\right)\cdot d\mathrm{\textbf S}=\oint\limits_L\mathrm{\textbf A}\cdot d\textbf l.
\end{equation}
\section{Exercises}
\noindent\textbf{Question} Vector magnetic potential in a region is $\mathrm{\textbf A}=x^2y\hat x+y^2x\hat y-4xyz\hat z$ Wb/m. Calculate
\begin{itemize}
\item[a.] \textbf{B} at (-1, 2, 5)
\item[b.] Magnetic flux through the surface defined by $z=1$, $0<x<1$, $-1<y<4$.
\end{itemize}
\noindent\textbf{Answer: } a. $20\hat x+40\hat y+3\hat z$ Wb/m$^2$, b. 20 Wb.
%%\nocite{*}
\bibliographystyle{plain}
\bibliography{EMTRef}
\end{document}
