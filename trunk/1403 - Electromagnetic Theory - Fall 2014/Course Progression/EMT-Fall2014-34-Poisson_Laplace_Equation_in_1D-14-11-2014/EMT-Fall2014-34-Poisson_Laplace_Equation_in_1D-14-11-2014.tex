\documentclass[12pt,a4paper]{article}

% Margins.
\setlength{\oddsidemargin}{0in}
\setlength{\evensidemargin}{0in}
\setlength{\headheight}{12pt}
\setlength{\headsep}{0pt}
\setlength{\topmargin}{-60pt}
\setlength{\textwidth}{6.5in}
\setlength{\textheight}{10.75in}

\usepackage{amsmath}
\usepackage{float}
\usepackage{graphicx}
\usepackage[hyphens]{url}
\usepackage{hyperref}	% Clickable links to figures, references and urls.
\usepackage{datetime}
\usepackage{longtable}
\usepackage{subfigure}

% Links direct to top of figures.
\usepackage[all]{hypcap}

% Drawing.
\usepackage{pgf}
\usepackage{tikz}

% Listings for formatting code.
\usepackage{listings}
\usepackage{textcomp}
% General options.
\lstset{breaklines=true, basicstyle=\small\ttfamily, tabsize=4, numbers=left, stepnumber=1, frame=single, showstringspaces=false, upquote=true}
% C++ specific high-lighting. Comments are 50/50 shades of green/black and strings coloured with 60/40 red/black mixture.
\lstset{language=[ISO]C++, commentstyle=\color{green!50!black}, keywordstyle=\color{blue}, stringstyle=\color{red!60!black}}

%opening
\title{Electromagnetic Theory\\Class 34\\Electrostatic Boundary Value Problems\\Poisson's and Laplace's Equations in 1D}
\author{Attique Dawood}
\date{November 14, 2014\\[0.2cm] Last Modified: \today, \currenttime}
\begin{document}
\maketitle
\section{Revision}
\begin{itemize}
\item Electric boundary conditions.
\end{itemize}
\section{Poisson's and Laplace's Equations for Electrostatic Fields}
According to Gauss' law
\begin{equation}
\nabla\cdot\mathrm{\textbf E}=\dfrac{\rho_v}{\epsilon}.
\end{equation}
Substituting $\mathrm{\textbf E}=-\nabla V$ in above equation gives
\begin{equation}
\nabla^2V=\dfrac{\rho_v}{\epsilon}.
\end{equation}
This is known as Poisson's equation. For charge free zone $\rho_v=0$ and this equation becomes
\begin{equation}
\nabla^2V=0.
\end{equation}
This is called Laplace's equation.
\section{Exercises}
\noindent\textbf{Question 1:} An EHD pumping electrode in the form of a cylinder of cross--sectional area S is placed along z--axis. The electrode at $z=0$ is maintained at a constant potential $V=V_0$ and the electrode at $z=d$ is maintaned at $V=0$. The space between the electrodes contain a uniform charge density $\rho_0$. Find the potential, electric field and net force on charge between the electrodes. If $\rho_0=25$ mC/m$^3$ and $V_0=22$ kV find the pressure (force per unit area) of the pump. (Ans: 550 N/m$^2$)\\[0.2cm]
\noindent\textbf{Question 2:} A capacitor is placed along x--axis filled with a dielectric of permittivity $\epsilon=1.5\epsilon_0$. If $V(x=0)=0$ and $V(x=2\mathrm{~mm})=50$ V, find V(x), \textbf{E} and \textbf{D} in the dielectric medium between the plates. Also find surface charge density $\rho_s$ on plates.
%%\nocite{*}
%\bibliographystyle{plain}
%\bibliography{EMTRef}
\end{document}
