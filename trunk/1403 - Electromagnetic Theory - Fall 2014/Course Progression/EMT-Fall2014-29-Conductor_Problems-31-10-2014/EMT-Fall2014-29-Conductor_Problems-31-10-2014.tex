\documentclass[12pt,a4paper]{article}

% Margins.
\setlength{\oddsidemargin}{0in}
\setlength{\evensidemargin}{0in}
\setlength{\headheight}{12pt}
\setlength{\headsep}{0pt}
\setlength{\topmargin}{-60pt}
\setlength{\textwidth}{6.5in}
\setlength{\textheight}{10.75in}

\usepackage{amsmath}
\usepackage{float}
\usepackage{graphicx}
\usepackage[hyphens]{url}
\usepackage{hyperref}	% Clickable links to figures, references and urls.
\usepackage{datetime}
\usepackage{longtable}
\usepackage{subfigure}

% Links direct to top of figures.
\usepackage[all]{hypcap}

% Drawing.
\usepackage{pgf}
\usepackage{tikz}

% Listings for formatting code.
\usepackage{listings}
\usepackage{textcomp}
% General options.
\lstset{breaklines=true, basicstyle=\small\ttfamily, tabsize=4, numbers=left, stepnumber=1, frame=single, showstringspaces=false, upquote=true}
% C++ specific high-lighting. Comments are 50/50 shades of green/black and strings coloured with 60/40 red/black mixture.
\lstset{language=[ISO]C++, commentstyle=\color{green!50!black}, keywordstyle=\color{blue}, stringstyle=\color{red!60!black}}

%opening
\title{Electromagnetic Theory\\Class 29\\Conductor Problems}
\author{Attique Dawood}
\date{October 31, 2014\\[0.2cm] Last Modified: \today, \currenttime}
\begin{document}
\maketitle
\section{Revision}
\begin{itemize}
\item Free charges on a conductor will always move to the surface of the conductor so as to establish a state of equilibrium.
\item Electric field inside a conductor is zero at all times.
\end{itemize}
\section{Using Spherical Symmetry to Simplify Gauss's Law}
For the Gaussian integral $\oint\limits_{S} \textbf{E}\cdot d\textbf{S}=\dfrac{q_{enc}}{\epsilon_0}$, if $q_{enc}$ is zero then does this imply that electric field is also zero? To rephrase this question, can you think of an electric field for which flux comes out to be zero on a certain closed surface? If there is a uniform electric field in a charge--free zone and you take a spherical Gaussian surface (actually any closed surface) then total electric flux through that surface will be zero. Number of field lines entering will be equal to field lines leaving the closed surface. Generally speaking, \textbf{if electric flux or charge enclosed in Gaussian surface is zero then this does not imply that electric field is also zero in that region}.

There is, however, a certain scenario where electric field must be zero if charge enclosed in Gaussian surface is zero. There are two conditions for this:
\begin{itemize}
\item Electric field must be constant or uniform on the Gaussian surface.
\item Electric field must be normal on the Gaussian surface.
\end{itemize}
The Gaussian integral is simplified to
\begin{equation}
\begin{split}
&E\oint_{S}dS=\dfrac{q_{enc}}{\epsilon_0}\\
&EA=\dfrac{q_{enc}}{\epsilon_0}\\
&E=\dfrac{q_{enc}}{\epsilon_0A}.
\end{split}
\end{equation}
For problems with spherical symmetry and spherical Gaussian surface of radius $r$
\begin{equation}
\begin{split}
&EA=\dfrac{q_{enc}}{\epsilon_0}\\
&E\times 4\pi r^2=\dfrac{q_{enc}}{\epsilon_0}\\
&E=\dfrac{q_{enc}}{ 4\pi\epsilon_0 r^2}
\end{split}
\end{equation}
\section{Exercises}
\noindent\textbf{Question 1:} A spherical conductor of radius 2 m has total charge 1 C. Use Gauss's Law to find electric field everywhere and surface charge density.\\[0.2cm]
\noindent\textbf{Question 2:} A conducting shell exists at $r=1$ to $r=2$. Space inside the shell is free space. If the shell is given a -1 C charge find electric field everywhere and surface charge densities.\\[0.2cm]
\noindent\textbf{Question 3:} A conducting shell exists at $r=1$ to $r=3$ and has a net charge 1 C. Space inside the shell is free space. If a 2 C point charge is placed at the centre of the shell find electric field everywhere and surface charge densities.\\[0.2cm]
\noindent\textbf{Question 4:} A neutral conductor of arbitrary shape has two spherical cavities of radii $r_a$ and $r_b$ cut out. Point charges $q_a$ and $q_b$ are placed at the centre of cavities, respectively.
\begin{itemize}
\item[a.] Find Electric field in cavities.
\item[b.] Calculate total charge and charge density on cavity walls.
\item[c.] Force on $q_a$ and $q_b$.
\item[d.] Charge on surface of conductor.
\end{itemize}
%\nocite{*}
%\bibliographystyle{plain}
%\bibliography{EMTRef}
\end{document}
