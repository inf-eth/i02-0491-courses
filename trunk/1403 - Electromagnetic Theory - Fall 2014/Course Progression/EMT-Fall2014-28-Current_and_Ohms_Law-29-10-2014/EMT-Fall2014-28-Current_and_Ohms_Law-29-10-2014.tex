\documentclass[12pt,a4paper]{article}

% Margins.
\setlength{\oddsidemargin}{0in}
\setlength{\evensidemargin}{0in}
\setlength{\headheight}{12pt}
\setlength{\headsep}{0pt}
\setlength{\topmargin}{-60pt}
\setlength{\textwidth}{6.5in}
\setlength{\textheight}{10.75in}

\usepackage{amsmath}
\usepackage{float}
\usepackage{graphicx}
\usepackage[hyphens]{url}
\usepackage{hyperref}	% Clickable links to figures, references and urls.
\usepackage{datetime}
\usepackage{longtable}
\usepackage{subfigure}

% Links direct to top of figures.
\usepackage[all]{hypcap}

% Drawing.
\usepackage{pgf}
\usepackage{tikz}

% Listings for formatting code.
\usepackage{listings}
\usepackage{textcomp}
% General options.
\lstset{breaklines=true, basicstyle=\small\ttfamily, tabsize=4, numbers=left, stepnumber=1, frame=single, showstringspaces=false, upquote=true}
% C++ specific high-lighting. Comments are 50/50 shades of green/black and strings coloured with 60/40 red/black mixture.
\lstset{language=[ISO]C++, commentstyle=\color{green!50!black}, keywordstyle=\color{blue}, stringstyle=\color{red!60!black}}

%opening
\title{Electromagnetic Theory\\Class 28\\Convection and Conduction Current\\Ohm's Law}
\author{Attique Dawood}
\date{October 29, 2014\\[0.2cm] Last Modified: \today, \currenttime}
\begin{document}
\maketitle
\section{Revision}
\begin{itemize}
\item Free charges on a conductor will always move to the surface of the conductor so as to establish a state of equilibrium.
\item Electric field inside a conductor is zero at all times.
\end{itemize}
\section{Current Density}
Current is defined as charge flowing (through a cross--sectional area) per unit time. Mathematically
\begin{equation}
I=\dfrac{\Delta q}{\Delta t}=\dfrac{dq}{dt}.
\end{equation}
Current density is amount of current flowing through a cross--sectional area. Current density is a vector quantity and direction is that of current flow. It is given by
\begin{equation}
J=\dfrac{\Delta I}{\Delta S}.
\end{equation}
Current can be obtained from current density by integrating over area.
\begin{equation}
I=\int\limits_S\mathrm{\textbf J}\cdot d\mathrm{\textbf S}.
\end{equation}
Here `I' is the amount of current flowing through surface `S'. If \textbf{J} is uniform and normal over `S' then
\begin{equation}
I=J\times S.
\end{equation}
\section{Convection Current}
Convection current is normally defined as the current due to flow of charges in free space or vacuum or any medium that is not a conductor.
\begin{equation}
\Delta I=\dfrac{\Delta Q}{\Delta t}.
\end{equation}
Now
\begin{equation}
\Delta Q=\rho_v\times\Delta v=\rho_v\times\Delta S\Delta l.
\end{equation}
And
\begin{equation}
\Delta I=\rho_v\times\Delta S\dfrac{\Delta l}{\Delta t}=\rho_v\times\Delta S u.
\end{equation}
Current density can be written as
\begin{equation}
J=\dfrac{\Delta I}{\Delta S}=\rho_v u.
\end{equation}
Or
\begin{equation}
\mathrm{\textbf J}=\rho_v\mathrm{\textbf u}.
\end{equation}
\section{Conduction Current}
In conductors, current is due to flow of electrons and electrons continually collide with each other. These collisions offer a resistance to the flow of charges. Force on an electron moving under the influence of electric field \textbf{E} is
\begin{equation}
\mathrm{\textbf F}=-e\mathrm{\textbf E}.
\end{equation}
This force is equal to change in momentum when a collision occurs.
\begin{equation}
\dfrac{m_e\mathrm{\textbf u}}{\tau}=-e\mathrm{\textbf E}.
\end{equation}
Or
\begin{equation}
\mathrm{\textbf u}=-\dfrac{e\tau}{m_e}\mathrm{\textbf E}.
\end{equation}
If number of electrons in 1 m$^3$ is $n$ then volume charge density is
\begin{equation}
\rho_v=-ne.
\end{equation}
Since \textbf{J}$=\rho_v$\textbf{u},
\begin{equation}
\mathrm{\textbf J}=\dfrac{ne^2\tau}{m_e}\mathrm{\textbf E}.
\end{equation}
Since conductivity is $\sigma=\dfrac{ne^2\tau}{m_e}$, current density can written as
\begin{equation}
\mathrm{\textbf J}=\sigma\mathrm{\textbf E}.
\label{Ohms-Law}
\end{equation}
\section{Ohm's Law for Conductors}
Let's suppose a conductor of cross--sectional area `S' and length `$l$' is connected with a voltage source `V'. Electric field across the conductor is given by
\begin{equation}
E=\dfrac{V}{l}
\end{equation}
and current density is
\begin{equation}
J=\dfrac{I}{S}.
\end{equation}
From \ref{Ohms-Law}
\begin{equation}
\sigma E=\sigma\dfrac{V}{l}=\dfrac{I}{S}
\end{equation}
and
\begin{equation}
\dfrac{V}{I}=\dfrac{l}{\sigma S}=R.
\end{equation}
%\nocite{*}
%\bibliographystyle{plain}
%\bibliography{EMTRef}
\end{document}
