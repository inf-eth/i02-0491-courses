\documentclass[12pt,a4paper]{article}

% Margins.
\setlength{\oddsidemargin}{0in}
\setlength{\evensidemargin}{0in}
\setlength{\headheight}{12pt}
\setlength{\headsep}{0pt}
\setlength{\topmargin}{-60pt}
\setlength{\textwidth}{6.5in}
\setlength{\textheight}{10.75in}

\usepackage{amsmath}
\usepackage{float}
\usepackage{graphicx}
\usepackage[hyphens]{url}
\usepackage{hyperref}	% Clickable links to figures, references and urls.
\usepackage{datetime}
\usepackage{longtable}
\usepackage{subfigure}

% Links direct to top of figures.
\usepackage[all]{hypcap}

% Drawing.
\usepackage{pgf}
\usepackage{tikz}

% Listings for formatting code.
\usepackage{listings}
\usepackage{textcomp}
% General options.
\lstset{breaklines=true, basicstyle=\small\ttfamily, tabsize=4, numbers=left, stepnumber=1, frame=single, showstringspaces=false, upquote=true}
% C++ specific high-lighting. Comments are 50/50 shades of green/black and strings coloured with 60/40 red/black mixture.
\lstset{language=[ISO]C++, commentstyle=\color{green!50!black}, keywordstyle=\color{blue}, stringstyle=\color{red!60!black}}

%opening
\title{Electromagnetic Theory\\Class 20\\Electric Field of Ring and Disk}
\author{Attique Dawood}
\date{October 13, 2014\\[0.2cm] Last Modified: \today, \currenttime}
\begin{document}
\maketitle
\section{Announcements}
\begin{itemize}
\item None.
\end{itemize}
\section{Continuous Charge Distributions}
In real life charges mostly appear bunched in different formations resembling the shape of the object on which they reside. It is more practical to think of continuous charge distributions rather than individual point charges.
\begin{equation}
dq=\rho_Ldl=\rho_SdS=\rho_Vdv
\end{equation}
Integration of differential charge expression over line, surface or volume gives the total charge.
Note: Serway and Jewet \cite{Serway} refer to these charge densities as $\lambda$, $\sigma$ and $\rho$.
\section{Position Vector}
The position vector in cartesian, cylindrical and spherical coordinates is,
\begin{equation}
\textbf{r}=r\hat r=\rho\hat{\rho}+z\hat z=x\hat x+y\hat y+z\hat z
\end{equation}
where
\begin{equation}
\rho\hat{\rho}=x\hat x+y\hat y
\end{equation}
\section{Exercises}
\noindent\textbf{Question 1 \cite[Example 4.4, page 117]{Sadiku}\cite[Example 23.8, page 721]{Serway}:} A circular ring of radius $a$ carries a uniform charge $\rho_L$ C/m and is placed on the $xy$--plane with axis the same as the $z$--axis. Find the electric field at a point P(0, 0, $h$) on $z$--axis.\\[0.2cm]
\noindent\textbf{Question 2 \cite[Practice Exercise 4.4, page 119]{Sadiku}\cite[Example 23.9, page 722]{Serway}:} A circular disk of radius $a$ carries a uniform charge $\rho_S$ or $\sigma$ C/m$^2$ and is placed on the $xy$--plane with axis the same as the $z$--axis. Find the electric field at a point P(0, 0, $h$) on $z$--axis.
%\nocite{*}
\bibliographystyle{plain}
\bibliography{EMTRef}
\end{document}
