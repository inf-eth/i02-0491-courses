\documentclass[12pt,a4paper]{article}

% Margins.
\setlength{\oddsidemargin}{0in}
\setlength{\evensidemargin}{0in}
\setlength{\headheight}{12pt}
\setlength{\headsep}{0pt}
\setlength{\topmargin}{-60pt}
\setlength{\textwidth}{6.5in}
\setlength{\textheight}{10.75in}

\usepackage{amsmath}
\usepackage{float}
\usepackage{graphicx}
\usepackage[hyphens]{url}
\usepackage{hyperref}	% Clickable links to figures, references and urls.
\usepackage{datetime}
\usepackage{longtable}

% Links direct to top of figures.
\usepackage[all]{hypcap}

% Drawing.
\usepackage{pgf}
\usepackage{tikz}

% Listings for formatting code.
\usepackage{listings}
\usepackage{textcomp}
% General options.
\lstset{breaklines=true, basicstyle=\small\ttfamily, tabsize=4, numbers=left, stepnumber=1, frame=single, showstringspaces=false, upquote=true}
% C++ specific high-lighting. Comments are 50/50 shades of green/black and strings coloured with 60/40 red/black mixture.
\lstset{language=[ISO]C++, commentstyle=\color{green!50!black}, keywordstyle=\color{blue}, stringstyle=\color{red!60!black}}

%opening
\title{Electromagnetic Theory\\Class 10\\Line, Surface and Volume Integrals -- II\\Flux and Density Integrals}
\author{Attique Dawood}
\date{September 15, 2014\\[0.2cm] Last Modified: \today, \currenttime}
\begin{document}
\maketitle
\section{Announcements}
\begin{itemize}
\item Quiz \#02 today.
\end{itemize}
\section{Revision}
\begin{itemize}
\item Differential length, surface and volume.
\item Area and volume integrals.
\end{itemize}
\section{Surface and Volume Integrals}
In electromagnetics we need to solve surface and volume integrals. Some common integrals are given below.
\begin{itemize}
\item $\int\limits_{S} d\mathrm{S}$~: Gives the total area of the surface `S'. This is generalised expression actually solved as a double integral.
\item $\int\limits_{v} dv$~: Calculate the total volume defined by limits of `v'. This is also a generalised expression actually solved as a triple integral.
\item $\int\limits_{S}\textbf{A}\cdot d\textbf{S}$~: Gives the total flux of the vector field \textbf{A} through the surface `S'.
\item $\int\limits_{v}\rho_v dv$~: If $\rho_v$ is volumetric charge density in $C/m^3$ then this integrals gives the total charge in volume `v'.
\item $\int\limits_{v}\rho_m dv$~: If $\rho_m$ is volumetric mass density in $kg/m^3$ then this integrals gives the total mass in volume `v'.
\end{itemize}
\section{Exercises}
\noindent\textbf{Question 1:} A vector field is given by $\textbf{A}=xyz\hat y$ in a region. Evaluate the flux, $\int\limits_{S}\textbf{A}\cdot d\textbf{S}$, through the surface defined by $y=2$, $0<x<2$ and $2<z<5$.\\[0.2cm]
\noindent\textbf{Question 2:} Electric field of a point charge is $\textbf{E}=\dfrac{1}{r^2}\hat r$ in a region. Evaluate the flux, $\int\limits_{S}\textbf{A}\cdot d\textbf{S}$, through the spherical surface $r=2$.\\[0.2cm]
\noindent\textbf{Question 3:} The average density of Earth is $\rho_m=5540$ kg/m$^3$. Find the mass of Earth if Earth's radius is $r=6400$ km.\\[0.2cm]
%\noindent\textbf{Question 4:} The density of Earth increases with depth and can be approximated as a linear function given by $\rho_m(r)=14000-0.0005r$ kg/m$^3$. Find the mass of Earth.
%\nocite{*}
\bibliographystyle{plain}
\bibliography{EMTRef}
\end{document}
