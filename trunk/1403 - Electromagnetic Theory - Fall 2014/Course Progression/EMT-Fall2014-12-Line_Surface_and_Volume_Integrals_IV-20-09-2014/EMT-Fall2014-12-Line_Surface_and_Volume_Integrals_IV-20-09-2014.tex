\documentclass[12pt,a4paper]{article}

% Margins.
\setlength{\oddsidemargin}{0in}
\setlength{\evensidemargin}{0in}
\setlength{\headheight}{12pt}
\setlength{\headsep}{0pt}
\setlength{\topmargin}{-60pt}
\setlength{\textwidth}{6.5in}
\setlength{\textheight}{10.75in}

\usepackage{amsmath}
\usepackage{float}
\usepackage{graphicx}
\usepackage[hyphens]{url}
\usepackage{hyperref}	% Clickable links to figures, references and urls.
\usepackage{datetime}
\usepackage{longtable}
\usepackage{subfigure}

% Links direct to top of figures.
\usepackage[all]{hypcap}

% Drawing.
\usepackage{pgf}
\usepackage{tikz}

% Listings for formatting code.
\usepackage{listings}
\usepackage{textcomp}
% General options.
\lstset{breaklines=true, basicstyle=\small\ttfamily, tabsize=4, numbers=left, stepnumber=1, frame=single, showstringspaces=false, upquote=true}
% C++ specific high-lighting. Comments are 50/50 shades of green/black and strings coloured with 60/40 red/black mixture.
\lstset{language=[ISO]C++, commentstyle=\color{green!50!black}, keywordstyle=\color{blue}, stringstyle=\color{red!60!black}}

%opening
\title{Electromagnetic Theory\\Class 12\\Line, Surface and Volume Integrals -- IV\\Solving Line Integrals}
\author{Attique Dawood}
\date{September 20, 2014\\[0.2cm] Last Modified: \today, \currenttime}
\begin{document}
\maketitle
\section{Announcements}
\begin{itemize}
\item None.
\end{itemize}
\section{Revision}
\begin{itemize}
\item Line integrals.
\end{itemize}
\section{Open and Closed Paths}
If the path of integration in a line integral starts and ends at the same point then it is called a closed path (curve or loop). \textit{A closed path always bounds an open surface or area}. Integration over a closed path is written with a $\oint$.

An important closed line integral in electromagnetics is
\begin{equation}
\oint\limits_L \textbf{E}\cdot d\textbf{\textit{l}}=0.
\end{equation}
Although not obvious but if this integral is carried out on an open path then it gives the potential difference between initial and final points. In circuit theory this integral is the familiar Kirchoff's voltage law (KVL).
\section{Open and Closed Surfaces}
\textit{A closed surface bounds some volume}. For example, surface of a sphere, surface of a closed box, surface of a cylinder including top and bottom faces, etc.

Two important closed surface integral in electromagnetics are
\begin{equation}
\oint\limits_S \textbf{E}\cdot d\textbf{\textit{S}}=\dfrac{q}{\epsilon_0}
\end{equation}
and
\begin{equation}
\oint\limits_S \textbf{B}\cdot d\textbf{\textit{S}}=0.
\end{equation}
These are the Gauss's law for electric and magnetic fields.
\section{Exercises}
Electric field in a region is $\textbf{E}=\dfrac{x\hat x+y\hat y}{(x^2+y^2)^\frac{3}{2}}$.\\[0.2cm]
\noindent\textbf{Question 1:} Find $\oint\limits_L \textbf{E}\cdot d\textbf{\textit{l}}$ over the closed path ABCDA.\\[0.2cm]
\noindent\textbf{Question 2:} Find $\oint\limits_L \textbf{E}\cdot d\textbf{\textit{l}}$ over the closed path ABCA.\\[0.2cm]
\noindent\textbf{Question 3:} Find the potential difference between
\begin{itemize}
\item[(1)] A and B.
\item[(3)] B and D.
\item[(4)] A and C.
\end{itemize}
\noindent It is recommended you use a calculator to solve the integrals.\\[0.2cm]
\noindent\textbf{Question 4:} Find the circulation of $\textbf{A}=\rho \cos\phi\hat \rho+\sin\phi\hat\phi$ over the path shown in figure.\\[0.2cm]
\begin{figure}[H]
\centering
\mbox{
\subfigure[Path of integration for electric field]{
\begin{tikzpicture}[xscale=1.2,yscale=1.2,font=\small]
	\def\XD{0cm}
	\def\YD{0cm}

	\draw[thick, ->, >=stealth] (1cm, 2cm) -- (2cm, 2cm);
	\draw[thick, ->, >=stealth] (2cm, 2cm) -- (2cm, 4cm);
	\draw[thick, ->, >=stealth] (2cm, 4cm) -- (1cm, 4cm);
	\draw[thick, ->, >=stealth] (1cm, 4cm) -- (1cm, 2cm);
	\draw[thick, ->, >=stealth] (2cm, 4cm) -- (1cm, 2cm);

	\draw[dashed] (0cm, 0cm) -- (1cm, 2cm);
	\draw[dashed] (0cm, 2cm) -- (1cm, 2cm);
	\draw[dashed] (0cm, 4cm) -- (1cm, 4cm);
	\draw[dashed] (1cm, 0cm) -- (1cm, 1.6cm);
	\draw[dashed] (2cm, 0cm) -- (2cm, 1.6cm);
	\draw[white,fill=white] (0.5cm,0.5cm) rectangle (1.5cm, 0.9cm);
	\node at (1cm, 0.7cm){$y=2x$};

	\coordinate[label=above:$C$] (C) at (2cm,4cm);
	\coordinate[label=above:$D$] (D) at (1cm,4cm);
	\coordinate[label=below:$A$] (A) at (1cm,2cm);
	\coordinate[label=below:$B$] (B) at (2cm,2cm);
	
	\coordinate[label=left:$2$] (y1) at (0cm,2cm);
	\coordinate[label=left:$4$] (y2) at (0cm,4cm);
	\coordinate[label=below:$1$] (x1) at (1cm,0cm);
	\coordinate[label=below:$2$] (x2) at (2cm,0cm);
	
	\draw[thick, ->, >=stealth] (0cm, 0cm) -- (0cm, 5cm);
	\coordinate[label=above:$y$] (y) at (0cm,5cm);
	\draw[thick, ->, >=stealth] (0cm, 0cm) -- (5cm, 0cm);
	\coordinate[label=right:$x$] (x) at (5cm,0cm);

\end{tikzpicture}
}
\quad\subfigure[Circulation of \textbf{A} for question 4.]{
\begin{tikzpicture}[xscale=1.2,yscale=1.2,font=\small]
	\def\XD{0cm}
	\def\YD{0cm}

	\draw[thick, ->, >=stealth] (0cm, 0cm) -- (4cm, 0cm);
	\draw[thick, ->, >=stealth] (0:4)  arc (0:60:4) -- (0cm, 0cm);
	\draw[thick, ->, >=stealth] (0:4)  arc (0:30:4);
	\draw[thick, ->, >=stealth] (0:0.6) arc (0:60:0.6);
	\node[above] at (0.75cm, 0.2cm){$60^0$};
	
	\coordinate[label=below:$2$] (x2) at (4cm,0cm);
	
	\draw[thick, ->, >=stealth] (0cm, 0cm) -- (0cm, 5cm);
	\coordinate[label=above:$y$] (y) at (0cm,5cm);
	\draw[thick, ->, >=stealth] (0cm, 0cm) -- (5cm, 0cm);
	\coordinate[label=right:$x$] (x) at (5cm,0cm);

\end{tikzpicture}
}}
\caption{Circulation of vector fields.}
\label{Circulation}
\end{figure}
%\nocite{*}
\bibliographystyle{plain}
\bibliography{EMTRef}
\end{document}
