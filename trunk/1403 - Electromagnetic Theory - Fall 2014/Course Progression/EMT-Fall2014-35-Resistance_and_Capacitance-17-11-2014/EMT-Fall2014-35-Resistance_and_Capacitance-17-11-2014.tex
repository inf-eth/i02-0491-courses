\documentclass[12pt,a4paper]{article}

% Margins.
\setlength{\oddsidemargin}{0in}
\setlength{\evensidemargin}{0in}
\setlength{\headheight}{12pt}
\setlength{\headsep}{0pt}
\setlength{\topmargin}{-60pt}
\setlength{\textwidth}{6.5in}
\setlength{\textheight}{10.75in}

\usepackage{amsmath}
\usepackage{float}
\usepackage{graphicx}
\usepackage[hyphens]{url}
\usepackage{hyperref}	% Clickable links to figures, references and urls.
\usepackage{datetime}
\usepackage{longtable}
\usepackage{subfigure}

% Links direct to top of figures.
\usepackage[all]{hypcap}

% Drawing.
\usepackage{pgf}
\usepackage{tikz}

% Listings for formatting code.
\usepackage{listings}
\usepackage{textcomp}
% General options.
\lstset{breaklines=true, basicstyle=\small\ttfamily, tabsize=4, numbers=left, stepnumber=1, frame=single, showstringspaces=false, upquote=true}
% C++ specific high-lighting. Comments are 50/50 shades of green/black and strings coloured with 60/40 red/black mixture.
\lstset{language=[ISO]C++, commentstyle=\color{green!50!black}, keywordstyle=\color{blue}, stringstyle=\color{red!60!black}}

%opening
\title{Electromagnetic Theory\\Class 35\\Resistance and Capacitance}
\author{Attique Dawood}
\date{November 17, 2014\\[0.2cm] Last Modified: \today, \currenttime}
\begin{document}
\maketitle
\section{Revision}
\begin{itemize}
\item Poisson's and Laplace's equation.
\end{itemize}
\section{Resistance and Capacitance}
Resistance and capacitance of structures can be found using Laplace's equation. Resistance is given by
\begin{equation}
R=\dfrac{V}{I}.
\end{equation}
Capacitance is given by
\begin{equation}
C=\dfrac{Q}{V}.
\end{equation}
\section{Exercises}
\noindent\textbf{Question 1:} Find the capacitance of a parallel--plate capacitor.\\[0.2cm]
\noindent\textbf{Question 2:} Find the capacitance of a cylindrical and spherical capacitor.
%%\nocite{*}
%\bibliographystyle{plain}
%\bibliography{EMTRef}
\end{document}
