\documentclass[12pt,a4paper]{article}

% Margins.
\setlength{\oddsidemargin}{0in}
\setlength{\evensidemargin}{0in}
\setlength{\headheight}{12pt}
\setlength{\headsep}{0pt}
\setlength{\topmargin}{-60pt}
\setlength{\textwidth}{6.5in}
\setlength{\textheight}{10.75in}

\usepackage{amsmath}
\usepackage{float}
\usepackage{graphicx}
\usepackage[hyphens]{url}
\usepackage{hyperref}	% Clickable links to figures, references and urls.
\usepackage{datetime}
\usepackage{longtable}

% Drawing.
\usepackage{pgf}
\usepackage{tikz}

% Listings for formatting code.
\usepackage{listings}
\usepackage{textcomp}
% General options.
\lstset{breaklines=true, basicstyle=\small\ttfamily, tabsize=4, numbers=left, stepnumber=1, frame=single, showstringspaces=false, upquote=true}
% C++ specific high-lighting. Comments are 50/50 shades of green/black and strings coloured with 60/40 red/black mixture.
\lstset{language=[ISO]C++, commentstyle=\color{green!50!black}, keywordstyle=\color{blue}, stringstyle=\color{red!60!black}}

%opening
\title{Introduction to Computing\\Lab 01\\Programming Basics}
\author{Moomal Bukhari\and Attique Dawood}
\date{February 17, 2015\\[0.2cm] Last Modified: \today, \currenttime}
\begin{document}
\maketitle
\section{A Hello World Program}
C++ is a sequential language. This means you write statements in a sequence and these will be executed one--by--one until the program ends. Just like a calculator, C++ has certain \textit{functions} to perform different tasks. The functions are included in a \textit{library} file. Our very first program uses the function \verb|cout| from the \verb|iostream| library to output a message on computer screen. The sample program is given bellow.
\begin{lstlisting}[caption={Hello World Program}]
#include <iostream> 
using namespace std; 

int main() 
{ 
	cout << "Hello, world!" << endl; 
	return 0; 
}
\end{lstlisting}
The very first line is directive to tell our compiler that we are going to include the \verb|iostream| library. The \verb|namespace std| is a sub--library in \verb|iostream| containing the \verb|cout| function. All C++ programs have exactly one \verb|main()| directive which is the starting point of program. The \verb|cout| function can be thought of as a connection to the monitor and \verb|<<| indicate that some message is being passed to \verb|cout| which in turn passes it onto the monitor. The \verb|endl| indicates end line. It moves the cursor at the start of next line. \verb|return 0| is the exit statement. After this the program exits.
\section{Integer Data Type}
Integer is the most common data type used in C++. An \verb|int| stores a number. The following code demonstrates addition of two numbers.
\begin{lstlisting}[caption={Addition of Two Numbers}]
#include <iostream> 
using namespace std; 

int main() 
{ 
	int x = 5;
	int y = -2;
	int sum = x+y;
	cout << "Result is" << sum << endl;
	return 0; 
}
\end{lstlisting}
\section{Exercises}
\noindent\textbf{Question No. 1:} Using multiple \verb|cout| statements write your roll number on screen. Your output on screen should be large enough that it can be read from a distance of 3 meters.\\[0.2cm]
\noindent\textbf{Question No. 2:} Write a program to evaluate
\begin{enumerate}
\item[a.] $3423+495332-342428\div219\times201$
\item[b.] $(3423+495332-342428)\div219\times201$
\item[c.] $3423+(495332-342428)\div219\times201$
\end{enumerate}
\noindent\textbf{Question No. 3:} Write a program (or two) that demonstrates your understanding of everything you have learned in this lab. There are extra marks for creativity.
\end{document}