\documentclass[12pt,a4paper]{article}

% Margins.
\setlength{\oddsidemargin}{0in}
\setlength{\evensidemargin}{0in}
\setlength{\headheight}{12pt}
\setlength{\headsep}{0pt}
\setlength{\topmargin}{-60pt}
\setlength{\textwidth}{6.5in}
\setlength{\textheight}{10.75in}

\usepackage{amsmath}
\usepackage{float}
\usepackage{graphicx}
\usepackage[hyphens]{url}
\usepackage{hyperref}	% Clickable links to figures, references and urls.
\usepackage{datetime}
\usepackage{longtable}

% Drawing.
\usepackage{pgf}
\usepackage{tikz}

% Listings for formatting code.
\usepackage{listings}
\usepackage{textcomp}
% General options.
\lstset{breaklines=true, basicstyle=\small\ttfamily, tabsize=4, numbers=left, stepnumber=1, frame=single, showstringspaces=false, upquote=true}
% C++ specific high-lighting. Comments are 50/50 shades of green/black and strings coloured with 60/40 red/black mixture.
\lstset{language=[ISO]C++, commentstyle=\color{green!50!black}, keywordstyle=\color{blue}, stringstyle=\color{red!60!black}}

%opening
\title{Introduction to Computing\\Lab 04\\Conditional Statements}
\author{Moomal Bukhari\and Attique Dawood}
\date{March 10, 2015\\[0.2cm] Last Modified: \today, \currenttime}
\begin{document}
\maketitle
\section{Comparison Operators}
The comparison operators can be used to compare two values. The values can also be variables.  For example, \verb|x == y|. This should be read as \textit{Is x equal to y?} Most commonly used operators are:
\begin{itemize}
\item \verb|==| (equals?)
\item \verb|!=| (not equals?)
\item \verb|>| (greater than?)
\item \verb|<| (less than?)
\item \verb|>=| (equals or greater than?)
\item \verb|<=| (equals or less than?)
\end{itemize}
\section{Conditional Statements}
The syntax for conditional statement is,
\begin{lstlisting}
if (condition...)
{
    do this if condition is true...
}
else
{
    do this if condition is false...
}
\end{lstlisting}
If the condition is evaluated to be 0 then it is false. If condition is non--zero it is true.
Operators in previous section are useful for testing a certain condition. For example,
\begin{lstlisting}
int x = 2;
int y = 4;
if (x == y)
{
    cout << "x is equal to y" << endl;
}
else
{
   cout << "x is not equal to y" endl;
}
\end{lstlisting}
In this case the output will be \verb|x is not equal to y|. The \verb|else| part is not mandatory and an \verb|if| statement can be used standalone. For example,
\begin{lstlisting}
int x;
cin >> x;
if (x > 0)
{
    cout << "You have entered a positive number" << endl;
}
\end{lstlisting}
Multiple logical conditions can be combined together using AND (\verb|&&|) and OR (\verb$||$) operators. For example
\begin{lstlisting}
int x, y, z;
cin >> x >> y >> z;
if (x > y && y > z)
	cout << "x is largest number" << endl;
if (x == y || y == z || x == z)
	cout << "Two numbers are equal" << endl;
if (x != y && y != z && x != z)
	cout << "All numbers are dissimilar" << endl;
\end{lstlisting}
\newpage
\section{Exercises/Lab Tasks}
\subsection{Task 1: SGPA Calculator}
Ask the user to input GPAs of five courses. Calculate semester GPA from GPAs of individual courses. Assign a letter grade to the student based on SGPA calculated. If student fails a course or his/her SGPA is less than 2.0 then he/she needs to repeat at least two courses in next semester. Check this condition and output an appropriate message.
\subsection{Task 2: ATM Simulation}
Simulate an ATM machine using the following algorithm. Any added functionality can earn you bonus marks.
\begin{lstlisting}
VARIABLE PIN     {4 digit roll number}
VARIABLE Balance 15000
VARIABLE Input 0
START
	OUTPUT "Enter 4 DIGIT PIN":
	INPUT Input
	IF Input is not equal to PIN
		EXIT
	ELSE
		WHILE Input is not equal to 3
			OUTPUT "Enter your choice"
			OUTPUT "1. Check balance"
			OUTPUT "2. Withdraw money"
			OUTPUT "3. Exit"
			INPUT Input
			IF Input is equal to 3
				EXIT
			ESLEIF INPUT is equal to 2
				OUTPUT Balance
			ELSEIF INPUT is equal to 1
				OUTPUT "How much money do you want to withdraw?"
				INPUT Input
				IF Input is greater than balance
					OUTPUT "ERROR: Insufficient funds"
				ELSE
					SUBTRACT Input from Balance
					OUTPUT Remaining balance
			ELSE
				OUTPUT "Incorrect input"
		ENDWHILE
	EXIT
END
\end{lstlisting}
\end{document}