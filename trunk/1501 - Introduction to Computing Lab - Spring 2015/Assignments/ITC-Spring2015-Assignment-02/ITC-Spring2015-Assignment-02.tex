\documentclass[12pt,a4paper]{article}

% Margins.
\setlength{\oddsidemargin}{0in}
\setlength{\evensidemargin}{0in}
\setlength{\headheight}{12pt}
\setlength{\headsep}{0pt}
\setlength{\topmargin}{-60pt}
\setlength{\textwidth}{6.5in}
\setlength{\textheight}{10.75in}

\usepackage{amsmath}
\usepackage{float}
\usepackage{array}
\usepackage{graphicx}
\usepackage[hyphens]{url}
\usepackage{hyperref}	% Clickable links to figures, references and urls.
\usepackage{datetime}

% Drawing.
\usepackage{pgf}
\usepackage{tikz}

% Listings for formatting code.
\usepackage{listings}
\usepackage{textcomp}
% General options.
\lstset{breaklines=true, basicstyle=\small\ttfamily, tabsize=4, numbers=left, stepnumber=1, frame=single, showstringspaces=false, upquote=true}
% C++ specific high-lighting. Comments are 50/50 shades of green/black and strings coloured with 60/40 red/black mixture.
\lstset{language=[ISO]C++, commentstyle=\color{green!50!black}, keywordstyle=\color{blue}, stringstyle=\color{red!60!black}}

% Table cell alignment directives.
\newcolumntype{L}[1]{>{\raggedright\let\newline\\\arraybackslash\hspace{0pt}}m{#1}}
\newcolumntype{C}[1]{>{\centering\let\newline\\\arraybackslash\hspace{0pt}}m{#1}}
\newcolumntype{R}[1]{>{\raggedleft\let\newline\\\arraybackslash\hspace{0pt}}m{#1}}

%opening
\title{Introduction to Computing Lab\\Assignment 02\\More Loops}
\author{Moomal Bukhari\and Attique Dawood}
\date{March 24, 2015\\Due: March 30, 2015\\[0.2cm] Last Modified: \today, \currenttime}
\begin{document}
\maketitle
\noindent \textbf{Exercise 1:} Print all the even numbers between 40 and 150\\[0.2cm]
\noindent \textbf{Exercise 2:} Print all the numbers divisible by 11 between 0 and 1000\\[0.2cm]
\noindent \textbf{Exercise 3:} Print all the numbers between 100 and 1000 whose sum of digits is divisible by 5\\[0.2cm]
\noindent \textbf{Exercise 4:} Series calculation: Sum up $1/2 + 1/4 + 1/8 + 1/16 + ...$ ?\\[0.2cm]
\noindent \textbf{Exercise 5:} Write a C++ code to print out all Armstrong numbers from 100 to 999. A number is called Armstrong number if sum of cubes of each digit of the number is equal to the number itself. For example $153=(1^3+5^3+3^3)$.\\[0.2cm]
\noindent \textbf{Exercise 6:} Write a program to print the following shapes using nested \verb|for| loops (Take Height from the user). 
\begin{verbatim}
*           	*****               *       	 *****
**          	****               **       	  ****
***         	***               ***       	   ***
****        	**               ****       	    **
*****       	*               *****       	     *
\end{verbatim}
\end{document}
