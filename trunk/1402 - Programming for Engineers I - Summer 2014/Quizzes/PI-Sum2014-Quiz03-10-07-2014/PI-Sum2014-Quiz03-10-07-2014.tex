\documentclass[12pt,a4paper]{article}

% Margins.
\setlength{\oddsidemargin}{0in}
\setlength{\evensidemargin}{0in}
\setlength{\headheight}{12pt}
\setlength{\headsep}{0pt}
\setlength{\topmargin}{-60pt}
\setlength{\textwidth}{6.5in}
\setlength{\textheight}{10.75in}

\usepackage{amsmath}
\usepackage{float}
\usepackage{graphicx}
\usepackage[hyphens]{url}
\usepackage{hyperref}	% Clickable links to figures, references and urls.
\usepackage{datetime}

% Drawing.
\usepackage{pgf}
\usepackage{tikz}

% Listings for formatting code.
\usepackage{listings}
\usepackage{textcomp}
% General options.
\lstset{breaklines=true, basicstyle=\small\ttfamily, tabsize=4, numbers=none, stepnumber=1, frame=none, showstringspaces=false, upquote=true}
% C++ specific high-lighting. Comments are 50/50 shades of green/black and strings coloured with 60/40 red/black mixture.
\lstset{language=[ISO]C++, commentstyle=\color{green!50!black}, keywordstyle=\color{blue}, stringstyle=\color{red!60!black}}

%opening
\title{\vspace{-1.5cm}Programming for Engineers I - Summer 2014\\Quiz \#03}
\date{\vspace{-1.5cm}Date: 10--07--2014}
\begin{document}
\maketitle
\vspace{-0.5cm}
\noindent\textbf{Time: 10 minutes\hfill Total Marks: 10}\\[0.3cm]
\noindent\textbf{Name:\rule{8cm}{1pt}\hfill Roll Number:\rule{3cm}{1pt}}\\[0.5cm]
\noindent\textbf{Question:} Write C++ code to find the transpose of a matrix.
\begin{lstlisting}
/* ROWS and COLS are const with some pre-defined values */
...
float Matrix[ROWS][COLS] = {-1.6, 4.9 ... /* initialised with random values */};
float Transpose[COLS][ROWS];

for (int i=0; i<ROWS; i++)
{
	for (int j=0; j<COLS; j++)
	{
		Transpose[j][i] = Matrix[i][j];
	}
}
\end{lstlisting}
Another possible solution is
\begin{lstlisting}
for (int i=0; i<COLS; i++)
{
	for (int j=0; j<ROWS; j++)
	{
		Transpose[i][j] = Matrix[j][i];
	}
}
\end{lstlisting}
\end{document}
