\documentclass[12pt,a4paper]{article}

% Margins.
\setlength{\oddsidemargin}{0in}
\setlength{\evensidemargin}{0in}
\setlength{\headheight}{12pt}
\setlength{\headsep}{42pt}
\setlength{\topmargin}{-54pt}
\setlength{\textwidth}{6.5in}
\setlength{\textheight}{10in}

\usepackage{amsmath}
\usepackage{float}
\usepackage{graphicx}
\usepackage[hyphens]{url}
\usepackage{hyperref}	% Clickable links to figures, references and urls.
\usepackage{datetime}

% Drawing.
\usepackage{pgf}
\usepackage{tikz}

% Listings for formatting code.
\usepackage{listings}
\usepackage{textcomp}
% General options.
\lstset{breaklines=true, basicstyle=\small\ttfamily, tabsize=4, numbers=left, stepnumber=1, frame=single, showstringspaces=false, upquote=true}
% C++ specific high-lighting. Comments are 50/50 shades of green/black and strings coloured with 60/40 red/black mixture.
\lstset{language=[ISO]C++, commentstyle=\color{green!50!black}, keywordstyle=\color{blue}, stringstyle=\color{red!60!black}}

%opening
\title{\vspace{-2cm}Programming for Engineers I\\Class 11\\File Handling\\Stream Pointers and Random Access}
\author{Attique Dawood}
\date{July 22, 2014\\[0.2cm] Last Modified: \today, \currenttime}
\begin{document}
\maketitle
\section{Writing to Binary File}
\begin{lstlisting}[caption={Writing to binary file}]
#include <fstream>		// File handling library.
using namespace std;

int main()
{
	fstream OutFile;	// File handling object.

	int x = 7;
	double d = -1.32489;
	char c = '$';

	// Open "abc.txt" for writing in binary mode (ios::out|ios::binary). This will overwrite existing file.
	OutFile.open("abc.txt", ios::out|ios::binary);
	OutFile.write((char*)&x, sizeof(int));
	OutFile.write((char*)&d, sizeof(double));
	OutFile.write((char*)&c, sizeof(char));
	OutFile.close();	// Close file.

	return 0;
}
\end{lstlisting}
\section{Reading from Binary File}
\begin{lstlisting}[caption={Reading from Binary file}]
#include <iostream>
#include <fstream>		// File handling library.
using namespace std;

int main()
{
	int x;
	double d;
	char c;

	fstream InFile;	// File handling object.

	// Open "abc.txt" for reading in binary mode (ios::in|ios::binary).
	InFile.open("abc.txt", ios::in|ios::binary);
	InFile.read((char*)&x, sizeof(int));
	InFile.read((char*)&d, sizeof(double));
	InFile.read((char*)&c, sizeof(char));
	InFile.close();	// Close file.

	cout << "x = " << x << endl;
	cout << "d = " << d << endl;
	cout << "c = " << c << endl;

	return 0;
}
\end{lstlisting}
\section{Writing \texttt{struct} to Binary File}
\begin{lstlisting}[caption={Writing a \texttt{struct} to binary file}]
#include <iostream>
#include <fstream>		// File handling library.
using namespace std;

struct StudentInfo
{
	char Name[20];
	int Age;
	float CGPA;
	char Gender;
};

int main()
{
	StudentInfo S;
	cout << "Enter student information: " << endl;
	cout << "Name: ";
	cin.getline(S.Name, 20);
	cout << "Age: ";
	cin >> S.Age;
	cout << "CGPA: ";
	cin >> S.CGPA;
	cout << "Gender (M/F): ";
	cin >> S.Gender;

	fstream OutFile;	// File handling object.

	// Open "abc.txt" for writing in binary mode (ios::out|ios::binary). This will overwrite existing file.
	OutFile.open("abc.txt", ios::out|ios::binary);
	OutFile.write((char*)&S, sizeof(StudentInfo));
	OutFile.close();	// Close file.

	return 0;
}
\end{lstlisting}
\section{Reading \texttt{struct} from Binary File}
\begin{lstlisting}[caption={Reading a \texttt{struct} from Binary file}]
#include <iostream>
#include <fstream>		// File handling library.
using namespace std;
struct StudentInfo
{
	char Name[20];
	int Age;
	float CGPA;
	char Gender;
};

int main()
{
	StudentInfo S;

	fstream InFile;	// File handling object.

	// Open "abc.txt" for reading in binary mode (ios::in|ios::binary).
	InFile.open("abc.txt", ios::in|ios::binary);
	InFile.read((char*)&S, sizeof(StudentInfo));
	InFile.close();	// Close file.

	switch (S.Gender)
	{
		case 'M':
			cout << "Name: Mr. " << S.Name << endl;
			break;
		case 'F':
			cout << "Name: Ms. " << S.Name << endl;
			break;
	}
	cout << "Age: " << S.Age << endl;
	cout << "CGPA: " << S.CGPA << endl;

	return 0;
}
\end{lstlisting}
\section{Put and Get Pointers}
\begin{itemize}
\item When you open a file a special read pointer is set to point at the start of file.
\item The read pointer (called a `\textbf{get}' pointer in C++) is moved forward as you read from file.
\item If you read 1 byte, get pointer is moved 1 byte forward.
\item Essentially the get pointer points to the location in the file where next read is expected.
\item Similar to the get pointer, a `\textbf{put}' is used when writing to a file.
\item The put pointer keeps track where last write was performed. Normally it points to the end of file if data is being appended.
\item The get and put pointers can be manipulated to write/read data from any random location in file.
\end{itemize}
\begin{table}[H]
\begin{center}
\vspace{0.3cm}
	\begin{tabular}{lcc}
	\hline \hline
		\textbf{Type} \rule{0pt}{2.6ex} & \textbf{Get Position} & \textbf{Set Position}\\
		\hline
		Read (get)\rule{0pt}{2.6ex} &  \verb|tellg()| & \verb|seekg()|\\
		Write (put)\rule{0pt}{2.6ex} & \verb|tellp()| & \verb|seekp()|\\
	\hline \hline
	\end{tabular}
\end{center}
\label{GetandPutPointers}
\caption{Get and Put Pointers}
\end{table}
\begin{table}[H]
\begin{center}
\vspace{0.3cm}
	\begin{tabular}{lp{9cm}}
	\hline \hline
		\textbf{Syntax} \rule{0pt}{2.6ex} & \textbf{Description}\\
		\hline
		\verb|cout << (int) tellp();| \rule{0pt}{2.6ex} & Display the current location of put pointer\\
		\verb|seekg(0, ios::beg);| \rule{0pt}{2.6ex} & Set the location of get pointer at start of file\\
		\verb|seekp(0, ios::end);| \rule{0pt}{2.6ex} & Set the location of put pointer at end of file\\
		\verb|seekp(10, ios::end);| \rule{0pt}{2.6ex} & Set the location of put pointer 10 bytes from end of file\\
		\verb|seekp(10);| \rule{0pt}{2.6ex} & Set the location of put pointer to 10$^{th}$ byte in file\\
	\hline \hline
	\end{tabular}
\end{center}
\label{ManipulatingGetandPutPointers}
\caption{Manipulating Get and Put Pointers}
\section{Example: Random Access of Student Record from Binary File}
\noindent\textbf{Note:} For the following code to work, the size of object \textit{must} be constant. You cannot use\\ \texttt{string} class to store student name.\\
\end{table}
\begin{lstlisting}[caption={Random access in binary file}]
#include <iostream>
#include <fstream>		// File handling library.
using namespace std;

class StudentInfo
{
	public:
	char Name[20];
	int Age;
	float CGPA;
	char Gender;
};

void Input(StudentInfo& S)
{
	cout << "Enter student information: " << endl;
	cout << "Name: ";
	if (cin.peek() == '\n') cin.ignore();
	cin.getline(S.Name, 20);
	cout << "Age: ";
	cin >> S.Age;
	cout << "CGPA: ";
	cin >> S.CGPA;
	cout << "Gender (M/F): ";
	cin >> S.Gender;
}

void Display(StudentInfo& S)
{
	switch (S.Gender)
	{
		case 'M':
			cout << "Name: Mr. " << S.Name << endl;
			break;
		case 'F':
			cout << "Name: Ms. " << S.Name << endl;
			break;
	}
	cout << "Age: " << S.Age << endl;
	cout << "CGPA: " << S.CGPA << endl;
}

int main()
{
	char Option;
	cout << "=========== Menu ==========" << endl;
	cout << "1. Add a new record." << endl
		 << "2. Display all records." << endl
		 << "3. Search for a record." << endl
		 << "4. Update a record." << endl;
	cout << "===========================" << endl;
	cout << "Select an option: ";
	cin >> Option;

	switch(Option)
	{
		// =============================== Add New Record =============================
		case '1':
		{
			StudentInfo S;
			// Input record.
			Input(S);
			fstream OutFile;	// File handling object.

			// Open "records.rec" for writing in binary mode (ios::out|ios::binary|ios::app). This will append data to file.
			OutFile.open("records.rec", ios::out|ios::binary|ios::app);
			OutFile.write((char*)&S, sizeof(StudentInfo));
			OutFile.close();	// Close file.

			break;
		}
		// =============================== Display All Records =============================
		case '2':
		{
			StudentInfo S;
			fstream InFile;	// File handling object.

			// Open "records.rec" for reading in binary mode (ios::in|ios::binary).
			InFile.open("records.rec", ios::in|ios::binary);

			// Find number of records in file.
			InFile.seekg(0, ios::end); // Set get pointer at end of file.
			int NoOfRecords = (int)InFile.tellg() / sizeof(StudentInfo); // Total records is total bytes in file / size of one record.
			cout << "Total records in file = " << NoOfRecords << endl;
			InFile.seekg(ios::beg); // Set pointer to start of file before reading.

			// Traverse the whole file.
			for (int i=0; i<NoOfRecords; i++)
			
			{
				// Read a record.
				InFile.read((char*)&S, sizeof(StudentInfo));

				// Display record.
				cout << "-------- Record " << i+1 << " --------" << endl;
				Display(S);
			}
			InFile.close();	// Close file.

			break;
		}
		// =============================== Search for a Record =============================
		case '3':
		{
			StudentInfo S;
			fstream InFile;	// File handling object.
			char SearchName[20];
			bool Found = false; // Was the record found?
			cout << "Enter a name to search: ";
			if (cin.peek() == '\n') cin.ignore();
			cin.getline(SearchName, 20);

			// Open "records.rec" for reading in binary mode (ios::in|ios::binary).
			InFile.open("records.rec", ios::in|ios::binary);

			// Find number of records in file.
			InFile.seekg(0, ios::end); // Set get pointer at end of file.
			int NoOfRecords = (int)InFile.tellg() / sizeof(StudentInfo); // Total records is total bytes in file / size of one record.
			cout << "Total Records in file = " << NoOfRecords << endl;
			InFile.seekg(ios::beg); // Set pointer to start of file before reading.

			// Traverse the whole file.
			for (int i=0; i<NoOfRecords; i++)
			{
				// Read a record.
				InFile.read((char*)&S, sizeof(StudentInfo));

				if (strcmp(S.Name, SearchName) == 0)
				{
					cout << "Record was found!" << endl;
					Found = true;

					// Display record.
					cout << "-------- Record " << i+1 << " --------" << endl;
					Display(S);
				}
			}
			InFile.close();	// Close file.

			if (Found == false)
				cout << "Record was not found!" << endl;

			break;
		}
		// =============================== Update Record =============================
		case '4':
		{
			StudentInfo S;
			fstream File;	// File handling object.
			char SearchName[20];
			bool Found = false; // Was the record found?
			cout << "Enter a name to search: ";
			if (cin.peek() == '\n') cin.ignore();
			cin.getline(SearchName, 20);

			// Open "records.rec" for reading and writing in binary mode (ios::in|ios::binary).
			File.open("records.rec", ios::in|ios::out|ios::binary);

			// Find number of records in file.
			File.seekg(0, ios::end); // Set get pointer at end of file.
			int NoOfRecords = (int)File.tellg() / sizeof(StudentInfo); // Total records is total bytes in file / size of one record.
			cout << "Total Records in file = " << NoOfRecords << endl;
			File.seekg(ios::beg); // Set pointer to start of file before reading.

			// Traverse the whole file.
			for (int i=0; i<NoOfRecords; i++)
			{
				// Read a record.
				File.read((char*)&S, sizeof(StudentInfo));

				if (strcmp(S.Name, SearchName) == 0)
				{
					cout << "Record was found!" << endl;
					Found = true;

					// Display record.
					cout << "-------- Record " << i+1 << " --------" << endl;
					Display(S);

					cout << "Please update student information: " << endl;
					Input(S);
					File.seekp(i*sizeof(StudentInfo)); // Set put/write pointer at location of record.
					File.write((char*)&S, sizeof(StudentInfo));
					cout << "Record was updated!" << endl;
				}
			}
			File.close();	// Close file.

			if (Found == false)
				cout << "Record was not found!" << endl;

			break;
		}
		// =============================== Default Case =============================
		default:
		{
			cout << "Invalid choice." << endl;
			break;
		}
	}
	return 0;
}
\end{lstlisting}
\end{document}
