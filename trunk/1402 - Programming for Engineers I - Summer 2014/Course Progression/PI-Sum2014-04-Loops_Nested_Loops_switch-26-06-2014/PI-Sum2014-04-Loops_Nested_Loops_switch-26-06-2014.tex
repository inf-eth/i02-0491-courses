\documentclass[12pt,a4paper]{article}

% Margins.
\setlength{\oddsidemargin}{0in}
\setlength{\evensidemargin}{0in}
\setlength{\headheight}{12pt}
\setlength{\headsep}{0pt}
\setlength{\topmargin}{-60pt}
\setlength{\textwidth}{6.5in}
\setlength{\textheight}{10.75in}

\usepackage{amsmath}
\usepackage{float}
\usepackage{graphicx}
\usepackage[hyphens]{url}
\usepackage{hyperref}	% Clickable links to figures, references and urls.
\usepackage{datetime}
% Drawing.
\usepackage{pgf}
\usepackage{tikz}

% Listings for formatting code.
\usepackage{listings}
\usepackage{textcomp}
% General options.
\lstset{breaklines=true, basicstyle=\small\ttfamily, tabsize=4, numbers=left, stepnumber=1, frame=single, showstringspaces=false, upquote=true}
% C++ specific high-lighting. Comments are 50/50 shades of green/black and strings coloured with 60/40 red/black mixture.
\lstset{language=[ISO]C++, commentstyle=\color{green!50!black}, keywordstyle=\color{blue}, stringstyle=\color{red!60!black}}

%opening
\title{Programming for Engineers I\\Class 04\\Loops\\Nested Loops\\\texttt{switch}}
\author{Attique Dawood}
\date{June 26, 2014\\[0.2cm] Last Modified: \today, \currenttime}
\begin{document}
\maketitle
\section{Revision}
\begin{itemize}
\item Float representation.
\item Float conversion.
\end{itemize}
\section{While Loop}
\begin{itemize}
\item \verb|while| loop runs until a condition is met.
\item The loop should have a terminating condition or it will run indefinitely.
\item \verb|while| loop checks condition first and then executes loop structure.
\end{itemize}
\begin{lstlisting}[caption={while Loop}]
while (condition is true)
{
	// Do this
}
\end{lstlisting}
\begin{lstlisting}[caption={Using while Loop to Sum}]
#include <iostream>
using namespace std;

int main()
{
	int Sum=0;
	while (Sum<20)
	{
		cout << "Sum is now: " << i << endl;
		Sum = Sum + 6;		
	}
	
	return 0;
}
\end{lstlisting}
\begin{lstlisting}[caption={Using Counter With while Loop}]
#include <iostream>
using namespace std;

int main()
{
	int i=0;
	while (i<10)
	{
		cout << "Counter is now: " << i << endl;
		i=i+1;		
	}
	
	return 0;
}
\end{lstlisting}
\section{do--while Loop}
\begin{itemize}
\item Executes loop structure first and then checks for condition.
\item Guaranteed to execute at least once.
\item Generally used when at least one execution of loop is required.
\end{itemize}
\begin{lstlisting}[caption={Using Counter With do-while Loop}]
#include <iostream>
using namespace std;

int main()
{
	int i=0;
	do
	{
		cout << "Counter is now: " << i << endl;
		i=i+1;		
	}
	while (i<10); // Notice ';' after while.
	
	return 0;
}
\end{lstlisting}
\section{for Loop}
\begin{itemize}
\item A somewhat better while loop.
\item Provides counter in addition to condition check. Don't need to explicitly make our own counter.
\item A \verb|for| is generally used instead of \verb|while| when it is known beforehand how many times the loop will run. Although not a strict rule but \verb|for| and \verb|while| can be used interchangeably.
\item A \verb|for| loop effectively becomes \verb|while| loop if counter isn't used.
\end{itemize}
\begin{lstlisting}[caption={for Loop}]
for (initialise counter; condition; increment counter)
{
	// Do this
}
\end{lstlisting}
\begin{lstlisting}[caption={for Loop With Counter}]
#include <iostream>
using namespace std;

int main()
{
	for (int i=0; i<10; i++)
	{
		cout << "Counter is now: " << i << endl;
	}
	
	return 0;
}
\end{lstlisting}
\section{Applications of Loops}
\begin{itemize}
\item Series calculation: Sum up $1/2 + 1/4 + 1/8 + 1/16 + ...$ ?
\item Print a box or fixed width.
\item Take student grades as input in a loop and calculate CGPA.
\end{itemize}
\section{\texttt{break} and \texttt{continue}}
\begin{itemize}
\item A loop can be forcefully stopped by using the \verb|break| keyword.
\item \verb|break| keyword \emph{must} be used in a loop. It will immediately terminate the loop without executing anything else.
\item If a \verb|continue| keyword is encountered in a loop then all the remaining statements in loop will not execute. The loop itself will continue to execute.
\end{itemize}
\begin{lstlisting}[caption={Breaking a Loop}]
#include <iostream>
using namespace std;
int main()
{
	int x;
	while (true) // Infinite loop.
	{
		cout << "Enter a number: ";
		cin >> x;
		if (x < 0)
		{
			cout << "You've entered a negative number. Exiting..." << endl;
			break; // Loop will break if negative number is entered.
		}
		cout << "You've entered a positive number. Continuing..." << endl;
	}
	return 0;
}
\end{lstlisting}
\begin{lstlisting}[caption={Using continue in a Loop}]
#include <iostream>
using namespace std;
int main()
{
	for (int i=0; i<20; i++)
	{
		cout << "This statement will show and counter is: " << i << endl;
		continue;
		cout << "This statement will not show as it is skipped because of continue..." << endl;
		cout << "This statement will also be skipped..." << endl;
	}
	return 0;
}
\end{lstlisting}
\section{Nested Loops}
\begin{itemize}
\item Nested loops can be used to perform complex tasks.
\item A simple example is printing symmetric 2D shapes like square, rectangle, triangles and diamonds.
\end{itemize}
\begin{lstlisting}[caption={Using for Loop to Make Rectangle and Square}]
#include <iostream>
using namespace std;
int main()
{
	int rows = 5;
	int columns = 10;
	for (int i=0; i<rows; i++)
	{
		for (int j=0; j<columns; j++)
		{
			cout << "*";
		}
		cout << endl;			
	}
	return 0;
}
\end{lstlisting}
What would happen if you change \verb|j<columns| to \verb|j<i|?\\
\noindent \textbf{Exercise:} Print all the even numbers between 40 and 150\\
\noindent \textbf{Exercise:} Print all the numbers divisible by 11 between 0 and 1000\\
\noindent \textbf{Exercise:} Print all the numbers between 100 and 1000 whose sum of digits is divisible by 5
\section{\texttt{switch} Structure}
\begin{itemize}
\item \verb|switch| is a convenient way of handling multiple conditions.
\item \underline{\texttt{switch} can only be used when you know exactly what the match is}; unlike the range specification in \verb|if| conditions.
\item Every condition inside switch structure must end with a break.
\item A default case can be defined for unexpected conditions.
\end{itemize}
\begin{lstlisting}[caption={Using switch to Display GPA}]
#include <iostream>
using namespace std;
int main()
{
	char Grade;
	cout << "Enter a grade (A-F): ";
	cin >> Grade;
	
	switch(Grade)	// Check Grade variable.
	{
		case 'A':
		{
			cout << "4.0 GPA" << endl;
			break;
		}
		case 'B':
		{
			cout << "3.0 GPA" << endl;
			break;
		}
		case 'C':
		{
			cout << "2.0 GPA" << endl;
			break;
		}
		case 'D':
		{
			cout << "1.0 GPA" << endl;
			break;
		}
		case 'F':
		{
			cout << "0.0 GPA" << endl;
			break;
		}
		default:
		{
			cout << "Invalid grade entered." << endl;
			break;
		}
	}	
	return 0;
}
\end{lstlisting}
\end{document}