\documentclass[12pt,a4paper]{article}

% Margins.
\setlength{\oddsidemargin}{0in}
\setlength{\evensidemargin}{0in}
\setlength{\headheight}{12pt}
\setlength{\headsep}{0pt}
\setlength{\topmargin}{-60pt}
\setlength{\textwidth}{6.5in}
\setlength{\textheight}{10.75in}

\usepackage{amsmath}
\usepackage{float}
\usepackage{graphicx}
\usepackage[hyphens]{url}
\usepackage{hyperref}	% Clickable links to figures, references and urls.
\usepackage{datetime}

% Drawing.
\usepackage{pgf}
\usepackage{tikz}

% Listings for formatting code.
\usepackage{listings}
\usepackage{textcomp}
% General options.
\lstset{breaklines=true, basicstyle=\small\ttfamily, tabsize=4, numbers=left, stepnumber=1, frame=single, showstringspaces=false, upquote=true}
% C++ specific high-lighting. Comments are 50/50 shades of green/black and strings coloured with 60/40 red/black mixture.
\lstset{language=[ISO]C++, commentstyle=\color{green!50!black}, keywordstyle=\color{blue}, stringstyle=\color{red!60!black}}

%opening
\title{Programming for Engineers I\\Class 08\\Dynamic Memory Allocation}
\author{Attique Dawood}
\date{July 10, 2014\\[0.2cm] Last Modified: \today, \currenttime}
\begin{document}
\maketitle
\section{Announcements}
\begin{itemize}
\item Quiz \# 3 today.
\end{itemize}
\section{Revision}
\begin{itemize}
\item Pointers.
\item Passing arguments to functions.
\end{itemize}
\section{Pointer Arithmetic}
\begin{itemize}
\item Pointers can be incremented or decremented.
\item If a pointer is incremented by one then the value of address it stores is incremented by the size of data type.
\item Example: An int pointer contains 10. If it is incremented then the new value is 14 because size of int is 4.
\item This makes sense when viewed from the perspective of array access. In an int array each location occupies 4 bytes. The address of each location is separated by 4 bytes. To access 3rd element \verb|*(array+2)| is used. If first address is at 0 then second would be at 4 and third at 8.
\end{itemize} 
\begin{lstlisting}[caption={Pointer Arithmetic}]
#include <iostream>
using namespace std;

int main()
{
	int Array[5] = {3,7,2,1,5};
	int* pA;
	pA = Array; // pA stores starting address of Array.
	
	for (int i=0; i<5; i++)
	{
		cout << *pA << endl;
		pA++; // Increment pointer.
	}

	return 0;
}
\end{lstlisting}
\section{Dynamic Memory Allocation in C++}
\begin{itemize}
\item In C++ \verb|new| is used for dynamic memory allocation and \verb|delete| to de--allocate memory. Both are keywords.
\item Dynamic memory can only be allocated and accessed with a pointer.
\item Dynamically allocated memory must be de--allocated with \verb|delete| or \verb|delete[]| before program exits.
\end{itemize}
\begin{lstlisting}[caption={Dynamic memory allocation with pointer}]
int main()
{
	int* p;
	
	// Single integer.
	p = new int; // Allocation.
	*p = 3;
	delete p;    // De-allocation.
	
	// Integer array.
	int size = 5;
	p = new int[size]; // Array allocation.
	for (int i=0; i<size; i++)
	{
		p[i] = i;
	}		
	delete[] p;     // Array de-allocation.
	
	return 0;
}
\end{lstlisting}
\end{document}