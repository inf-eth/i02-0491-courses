\documentclass[12pt,a4paper]{article}

% Margins.
\setlength{\oddsidemargin}{0in}
\setlength{\evensidemargin}{0in}
\setlength{\headheight}{12pt}
\setlength{\headsep}{0pt}
\setlength{\topmargin}{-60pt}
\setlength{\textwidth}{6.5in}
\setlength{\textheight}{10.75in}

\usepackage{amsmath}
\usepackage{float}
\usepackage{graphicx}
\usepackage[hyphens]{url}
\usepackage{hyperref}	% Clickable links to figures, references and urls.
\usepackage{datetime}

% Drawing.
\usepackage{pgf}
\usepackage{tikz}

% Listings for formatting code.
\usepackage{listings}
\usepackage{textcomp}
% General options.
\lstset{breaklines=true, basicstyle=\small\ttfamily, tabsize=4, numbers=left, stepnumber=1, frame=single, showstringspaces=false, upquote=true}
% C++ specific high-lighting. Comments are 50/50 shades of green/black and strings coloured with 60/40 red/black mixture.
\lstset{language=[ISO]C++, commentstyle=\color{green!50!black}, keywordstyle=\color{blue}, stringstyle=\color{red!60!black}}

%opening
\title{Programming for Engineers I\\Assignment 02\\Float Representation\\Conditional Statements and Loops}
\author{Attique Dawood}
\date{June 26, 2014\\Due: July 01, 2014\\[0.2cm] Last Modified: \today, \currenttime}
\begin{document}
\maketitle
\noindent\textbf{Question 1:} Convert the following floating point values into their equivalent 32 bit binary representation. Also show how they are stored in RAM.
\begin{enumerate}
\item[a.] \verb|float x1 = 31.78125f;|
\item[b.] \verb|float x2 = 131.3828125;|
\item[c.] \verb|float x3 = 231.521484375f;|
\item[d.] \verb|float x4 = 11.125f;|
\item[e.] \verb|float x5 = 0.16015625f;|
\end{enumerate}
\noindent\textbf{Question 2:} Convert the following 32 bit binary and hexadecimal floating point values into their equivalent decimal form.
\begin{enumerate}
\item[a.] \verb|0100 0001  0000 1110  0000 0000  0000 0000|
\item[b.] \verb|1100 0001  1010 1100  0100 0000  0000 0000|
\item[c.] \verb|0100 0011  1110 1000  0000 0000  0000 0000|
\item[d.] \verb|b6 72 00 00|
\item[e.] \verb|40 03 10 00|
\end{enumerate}
\noindent\textbf{Question 3:} Take an integer as input. Determine if the number is positive or negative.\\
\noindent\textbf{Question 4:} Take letter grade as \verb|char| input and display corresponding GPA. Input can be A, B, C, D or F. Any other input should result in an error.\\
\noindent\textbf{Question 5:} Take GPA as a \verb|float| input. Display the corresponding letter grade.\\
\noindent\textbf{Question 6:} Take a three digit integer as input. Calculate the sum of digits. Determine from the sum of digits if the number is divisible by 3, 6 and 9.\\
\noindent\textbf{Question 7:} Take an integer and determine if it is prime.\\
\noindent\textbf{Question 8:} Using nested \verb|for| loops display following patterns on screen. Make the patterns separately.
\begin{lstlisting}
*      *****      *     *****
**      ****     ***    *****
***      ***    *****   *****
****      **   *******  *****
*****      *  ********* *****
\end{lstlisting}
\end{document}
