\documentclass[12pt,a4paper]{article}

% Margins.
\setlength{\oddsidemargin}{0in}
\setlength{\evensidemargin}{0in}
\setlength{\headheight}{12pt}
\setlength{\headsep}{0pt}
\setlength{\topmargin}{-60pt}
\setlength{\textwidth}{6.5in}
\setlength{\textheight}{10.75in}

\usepackage{amsmath}
\usepackage{float}
\usepackage{graphicx}
\usepackage[hyphens]{url}
\usepackage{hyperref}	% Clickable links to figures, references and urls.
\usepackage{datetime}

% Drawing.
\usepackage{pgf}
\usepackage{tikz}

% Listings for formatting code.
\usepackage{listings}
\usepackage{textcomp}
% General options.
\lstset{breaklines=true, basicstyle=\small\ttfamily, tabsize=4, numbers=left, stepnumber=1, frame=single, showstringspaces=false, upquote=true}
% C++ specific high-lighting. Comments are 50/50 shades of green/black and strings coloured with 60/40 red/black mixture.
\lstset{language=[ISO]C++, commentstyle=\color{green!50!black}, keywordstyle=\color{blue}, stringstyle=\color{red!60!black}}

%opening
\title{Programming for Engineers I\\Assignment 03\\Arrays and Sorting}
\author{Attique Dawood}
\date{July 03, 2014\\Due: July 10, 2014\\[0.2cm] Last Modified: \today, \currenttime}
\begin{document}
\maketitle
\noindent\textbf{Objectives}
\begin{itemize}
\item[1.] Learn how C++ arrays work.
\item[2.] Generation of random range of values.
\item[3.] Size of C++ arrays cannot be changed at runtime. Learn how to declare an appropriately large array of some maximum size and work within a certain size. The user effectively gets the impression that he/she is working with variable sized arrays.
\item[4.] Learn how to work with 2D arrays.
\end{itemize}
\noindent\textbf{Question 1:} Make a program that lets user enter an array size and range of data to be generated. Provide options to display and sort the data. The actual array should be allocated a fixed maximum size \verb|MAXSIZE| which should be \verb|const|. The size entered by user cannot exceed \verb|MAXSIZE|. Program menu with different options is given below.
\begin{verbatim}
1. Set array size (MAXSIZE=100)
2. Generate data
   - Input lower and upper range
3. Sort data
   1. Ascending order
   2. Descending order
4. Display data
5. Exit program
\end{verbatim}
\noindent\textbf{Error Checks:} Before the menu is displayed for the first time, initialise the array with all zeros and set a default size (e.g. 10). If the user inputs invalid value for size (negative or exceeding MAXSIZE) then display an appropriate error message.\\[0.2cm]
\noindent\textbf{Question 2:} 2D arrays function just like matrices. In this program provide functionality to add, subtract and multiply matrices of any size. First define \verb|MAXSIZE| for both row and column. Matrix dimensions cannot exceed \verb|MAXSIZE|. Declare three matrices A, B and C of dimensions \verb|MAXSIZE|$\times$\verb|MAXSIZE|. User should be provided a menu to set the dimensions of A and B, input matrix manually or generate random matrix data and perform matrix operations. The program menu with sub--menu options is given.
\begin{verbatim}
1. Set matrix size (Maximum dimensions: 10*10)
   1. Matrix A
      1. Input rows of A
      2. Input columns of A
   2. Matrix B
      1. Input rows of B
      2. Input columns of B
2. Generate data
   1. Manual input using cin
      1. Matrix A
      2. Matrix B
   2. Random generation
      1. Matrix A
         - Input lower and upper range
      2. Matrix B
         - Input lower and upper range
3. Matrix operations
   1. Add (A-B)
   2. Subtract (A-B)
   3. Multiply (A*B)
   4. Scalar multiplication (cA)
      - Input scalar c
   5. Inverse of A (A should be 2*2 matrix)
4. Display matrix
   1. Matrix A
   2. Matrix B
   3. Matrix C (resultant)
   4. Display all matrices
5. Exit program.
\end{verbatim}
\noindent\textbf{Error Checks:}
\begin{itemize}
\item[-] Initialise all matrices before the menu is displayed for the first time. Set default dimensions for A and B (e.g. $2\times2$).
\item[-] The rows or columns of any matrix cannot be less than 1 and greater than MAXSIZE defined already.
\item[-] If A and B have different dimensions then they cannot be added or subtracted. Display an error if user tries to add/subtract matrices whose dimensions are not equal. 
\item[-] A and B can only multiplied if number of columns of A are equal to number of rows of B. Display an appropriate error if user tries to multiply matrices that are not conformable for multiplication.
\item[-] Inverse can only be computed if A is $2\times2$ matrix. Also display a message if determinant of A comes out to be zero.
\item[-] Dimensions of C depends on matrix operation. In case of addition/subtraction, dimensions of C is set equal to that of A (or B). In case of multiplication, dimensions of C is set equal to $rowsA\times colsB$.
\end{itemize}
\end{document}
