\documentclass[12pt,a4paper]{article}

% Margins.
\setlength{\oddsidemargin}{0in}
\setlength{\evensidemargin}{0in}
\setlength{\headheight}{12pt}
\setlength{\headsep}{42pt}
\setlength{\topmargin}{-54pt}
\setlength{\textwidth}{6.5in}
\setlength{\textheight}{10in}

\usepackage{amsmath}
\usepackage{float}
\usepackage{graphicx}
\usepackage[hyphens]{url}
\usepackage{hyperref}	% Clickable links to figures, references and urls.
\usepackage{enumerate}
\usepackage{datetime}

% Drawing.
\usepackage{pgf}
\usepackage{tikz}

% Listings for formatting code.
\usepackage{listings}
\usepackage{textcomp}
% General options.
\lstset{breaklines=true, basicstyle=\small\ttfamily, tabsize=4, numbers=left, stepnumber=1, frame=single, showstringspaces=false, upquote=true}
% C++ specific high-lighting. Comments are 50/50 shades of green/black and strings coloured with 60/40 red/black mixture.
\lstset{language=[ISO]C++, commentstyle=\color{green!50!black}, keywordstyle=\color{blue}, stringstyle=\color{red!60!black}}

%opening
\title{\vspace{-2cm}Programming for Engineers II\\Class 42\\\texttt{iomanip}\\Function and Class Templates}
\author{Attique Dawood}
\date{May 08, 2013\\[0.2cm] Last Modified: \today, \currenttime}
\begin{document}
\maketitle
\section{Announcements}
\begin{itemize}
\item None
\end{itemize}
\section{Revision}
\begin{itemize}
\item Buffers and synchronisation.
\item Linked list de--allocation.
\end{itemize}
\section{\texttt{iomanip} Library}
\begin{itemize}
\item \verb|<iomanip>| libaray is used for manipulating out put.
\item \verb|setw(x)| will fix the output to x chars.
\item \verb|setprecision(x)| will set the desired floating point precision.
\item \verb|setbase(x)| and \verb|dec|, \verb|oct|, \verb|hex| will format the output for indicated base. Valid bases are 10, 8 and 16.
\end{itemize}
\section{Templates in C++}
\begin{itemize}
\item Templates offer a mechanism to write generic code.
\item If we want to make a sum function to add two data types then data types can be \verb|int|, \verb|float|, \verb|double| or even \verb|complex|.
\item We would have to make sum function for all these types.
\item By making a template, we let compiler decide which instance of sum function to call.
\item Template calling can be implicit or explicit.
\item If no type is mentioned with function call, compiler will decide what is best instantiation of template.
\item If a type is mentioned with function call (e.g., \verb|sum<int> (3, 7)|) then that particular instance of template is instantiated by compiler.
\item Templates can also be defined for multiple types.
\item Templates can be defined for functions as well as classes.
\end{itemize}
\begin{lstlisting}[caption={Traditional way of writing sum function}]
int sum(int a, int b)
{
	return (a+b);
}
float sum(float a, float b)
{
	return (a+b);
}
...
complex sum(complex a, complex b)
{
	return (a+b);
}
...
\end{lstlisting}
\begin{lstlisting}[caption={Sum function template}]
// Sum function template.
// Depending on type of argument passed in main, that will be substituted in place of 'MyType'.
// For example, if called as sum(2.3, 7.7), MyType would be double.
template <class MyType> MyType sum(MyType a, MyType b)
{
	return (a+b);
}
int main()
{
	sum(3.2, 7.8); // implicit instantiation with double type.
	sum(3, 8); // implicit instantiation with int type.
	sum<double> (3.2, 7.8); // explicit double instantiation.
	
	return 0;
}
\end{lstlisting}
\begin{lstlisting}[caption={Sum function template for mixed types}]
template <class MyType1, class MyType2> MyType1 sum(MyType1 a, MyType2 b)
{
	return (a+b);
}
int main()
{
	sum(3.2, 7.8); // implicit instantiation with both double types.
	sum(3, 8.9); // implicit instantiation for int and double types.
	sum<int, double> (3, 7.8); // explicit instantiation.
	
	return 0;
}
\end{lstlisting}
\begin{lstlisting}[caption={Class template}]
#include <iostream>
using namespace std;
template <class MyType> class GenericArray
{
	private:
	MyType Data[5];
	public:
	void Input()
	{
		for (int i=0; i<5; i++)
			cin >> Data[i];
	}
	void Display()
	{
		for (int i=0; i<5; i++)
			cout << Data[i] << " ";
		cout << endl;
	}
};
int main()
{
	GenericArray<int> IntArray;
	IntArray.Input();
	IntArray.Display();
	
	return 0;
}
\end{lstlisting}
%\nocite{*}
\bibliographystyle{plain}
\bibliography{OOPref}
\end{document}
