\documentclass[12pt,a4paper]{article}

% Margins.
\setlength{\oddsidemargin}{0in}
\setlength{\evensidemargin}{0in}
\setlength{\headheight}{12pt}
\setlength{\headsep}{42pt}
\setlength{\topmargin}{-54pt}
\setlength{\textwidth}{6.5in}
\setlength{\textheight}{10in}

\usepackage{amsmath}
\usepackage{float}
\usepackage{graphicx}
\usepackage[hyphens]{url}
\usepackage{hyperref}	% Clickable links to figures, references and urls.
\usepackage{enumerate}
\usepackage{datetime}

% Drawing.
\usepackage{pgf}
\usepackage{tikz}

% Listings for formatting code.
\usepackage{listings}
\usepackage{textcomp}
% General options.
\lstset{breaklines=true, basicstyle=\small\ttfamily, tabsize=4, numbers=left, stepnumber=1, frame=single, showstringspaces=false, upquote=true}
% C++ specific high-lighting. Comments are 50/50 shades of green/black and strings coloured with 60/40 red/black mixture.
\lstset{language=[ISO]C++, commentstyle=\color{green!50!black}, keywordstyle=\color{blue}, stringstyle=\color{red!60!black}}

%opening
\title{\vspace{-2cm}Programming for Engineers II\\Class 30\\Linked List\\Display, Search and Delete}
\author{Attique Dawood}
\date{April 12, 2013\\[0.2cm] Last Modified: \today, \currenttime}
\begin{document}
\maketitle
\section{Announcements}
\begin{itemize}
\item None.
\end{itemize}
\section{Linked List}
\begin{itemize}
\item Storage problems with fragmented memory or hard disk.
\item Array requires contiguous space in memory. Array cannot dynamically grow incrementally. Must de--allocate and re--allocate whole array to change size.
\item Linked list solves the above problems. Doesn't require contiguous memory. Linked list is a chain of pointer--connected nodes. Nodes store information just like array indices.
\item Linked list access is slower than array. Must traverse whole list to reach a particular node. But, storage and size is flexible.
\item Array is preferable when speed is desired and storage size is known. Static data. Example: Number of students in a class cannot exceed 50. Create an array of 50 size for students.
\item Linked list is preferable when speed is not a major concern and storage size is highly dynamic. Example: Daily record of cars using motorway. On weekends traffic is high but there aren't many cars on weekdays. Numbers of cars can be highly varied.
\end{itemize}
\begin{lstlisting}[caption={Basic Linked List With Add and Display functions}]
#include <iostream>
using namespace std;

class Node
{
	public:
	int Data;
	Node* Next;
};

class LinkedList
{
	private:
	Node* First;
	public:
	LinkedList(): First(NULL)
	{
	}
	// Adds a node at the end of linked list.
	void AddNode(int pData)
	{
		// If creating first node we need to change first pointer.
		if (First == NULL)
		{
			Node* NewNode;
			NewNode = new Node;
			NewNode->Data = pData;
			NewNode->Next = NULL;
			First = NewNode;
		}
		// We need to traverse the whole list to get to last node.
		else
		{
			// Find last node.
			Node* temp;
			temp = First;
			while (temp->Next != NULL)
				temp = temp->Next;
			// temp now points to last node.
			Node* NewNode;
			NewNode = new Node;
			temp->Next = NewNode;
			NewNode->Data = pData;
			NewNode->Next = NULL;
		}
	}
	// Displays all entries in linked list.
	void Display()
	{
		Node* temp;
		temp = First;
		while(temp != NULL)
		{
			cout << "Data = " << temp->Data << endl;
			temp = temp->Next;
		}
	}			
};

int main()
{
	LinkedList TestList;

	TestList.AddNode(-1);
	TestList.AddNode(2);
	TestList.AddNode(0);
	TestList.AddNode(7);
	
	TestList.Display();
	
	return 0;
}
\end{lstlisting}
\section{Linked List Search}
\begin{itemize}
\item This implementation assumes that entries in linked list are unique.
\item As soon as we find our entry we return 0.
\item If entry isn't found, -1 is returned.
\end{itemize}
\begin{lstlisting}[caption={Linked List Search}]
class LinkedList
{
...
	int Search(int Key)
	{
		Node* temp;
		temp = First;
		while(temp != NULL)
		{
			if (temp->Data == Key)
				return 0;
	
			temp = temp->Next;
		}
		return -1; // If we ever get here then search was unsuccessful.
	}
...
};

int main()
{
	LinkedList TestList;

	TestList.AddNode(-1);
	TestList.AddNode(2);
	TestList.AddNode(0);
	TestList.AddNode(7);
	
	TestList.Display();
	
	int SearchKey = 0;
	if (TestList.Search(SearchKey) == 0)
		cout << SearchKey << " is in the list." << endl;
	
	return 0;
}
\end{lstlisting}
\section{Deleting a Node}
\begin{itemize}
\item We can delete a node by its data value.
\item We need to keep track of previous node pointer for deletion.
\item We'll use two temporary pointers to traverse the array and search for our value.
\item Deletion of first node is a special case.
\end{itemize}
\begin{lstlisting}[caption={Linked List Delete Node}]
class LinkedList
{
...
	int Delete(int Key)
	{
		// If List is empty then nothing to do.
		if (First == NULL)
			return -1;

		Node* previous;
		Node* current;
		previous = current = First;
		// Checking first node.
		if (current->Data == Key)
		{
			// Delete First node. First will now point to next node.
			First = First->Next;
			delete current;
			return 0;
		}
		else
		{
			current = current->Next; // Move current pointer to next node.
			while(current != NULL)
			{
				// If entry is matched then link previous node to next node and delete current node.
				if (current->Data == Key)
				{
					previous->Next = current->Next;
					delete current;
					return 0;
				}
				// Move pointers forward.
				previous = current;
				current = current->Next;
			}
		}
		return -1; // If we ever get here then search was unsuccessful.
	}
...
};

int main()
{
	LinkedList TestList;

	TestList.AddNode(-1);
	TestList.AddNode(2);
	TestList.AddNode(0);
	TestList.AddNode(7);
	
	TestList.Display();
	
	int SearchKey = 0;
	if (TestList.Search(SearchKey) == 0)
		cout << SearchKey << " is in the list." << endl;
	
	if (TestList.Delete(SearchKey) == 0)
			cout << SearchKey << " was deleted." << endl;
	
	TestList.Display();
	
	return 0;
}
\end{lstlisting}
\textbf{Question:} Instead of always adding a node at the end we can choose to add a new node at the beginning! In this case first pointer will point to the newly added node. Create a function \verb|AddAtStart()| in the linked list code provided in Listing 1 to provide this functionality. User may choose to use either function (at end or beginning) to add nodes.\\
\textbf{Note:} This question is worth 1 absolute point. Only first correct submission will be accepted.
%\nocite{*}
%\bibliographystyle{plain}
%\bibliography{OOPref}
\end{document}
