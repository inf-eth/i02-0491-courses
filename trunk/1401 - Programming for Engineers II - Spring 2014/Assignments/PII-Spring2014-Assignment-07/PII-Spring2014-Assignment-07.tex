\documentclass[12pt,a4paper]{article}

% Margins.
\setlength{\oddsidemargin}{0in}
\setlength{\evensidemargin}{0in}
\setlength{\headheight}{12pt}
\setlength{\headsep}{42pt}
\setlength{\topmargin}{-54pt}
\setlength{\textwidth}{6.5in}
\setlength{\textheight}{10in}

\usepackage{amsmath}
\usepackage{float}
\usepackage{graphicx}
\usepackage[hyphens]{url}
\usepackage{hyperref}	% Clickable links to figures, references and urls.
\usepackage{datetime}

% Drawing.
\usepackage{pgf}
\usepackage{tikz}

% Listings for formatting code.
\usepackage{listings}
\usepackage{textcomp}
% General options.
\lstset{breaklines=true, basicstyle=\small\ttfamily, tabsize=4, numbers=left, stepnumber=1, frame=single, showstringspaces=false, upquote=true}
% C++ specific high-lighting. Comments are 50/50 shades of green/black and strings coloured with 60/40 red/black mixture.
\lstset{language=[ISO]C++, commentstyle=\color{green!50!black}, keywordstyle=\color{blue}, stringstyle=\color{red!60!black}}

%opening
\title{\vspace{-2cm}Programming for Engineers II\\Assignment 07\\File Handling}
\author{Attique Dawood}
\date{April 22, 2014\\Due: April 28, 2014\\[0.2cm] Last Modified: \today, \currenttime}
\begin{document}
\maketitle
Create a \verb|StudentDatabase| class to handle student records. All student data must be stored in a binary file. All operations should be performed on file. Provide a menu in \verb|main()| to perform the file operations.
\begin{lstlisting}
// Note: Student object size in memory is 32 bytes if using 8 byte structure boundaries (default).
class Student
{
	private:
	char Name[20];
	int Age;
	float CGPA;
	char Gender; // 'M' or 'F'.
	public:
	// Can overload stream operators for input and output.
	void Input();
	void Display();
};
class StudentDatabase
{
	public:
	void DisplayAllRecords(); // Reads the whole file and displays all records.
	void AddRecord(); // Input Student attributes and stores record on file.
	void SearchRecord(); // Search record from file and display it if found.
	void DeleteRecord(); // Deletes a record from file.
	// To delete a record from file: Write all the records in the file, except for the record to be deleted, in a temporary file. Overwrite the original file with temporary file. Original filesize should decrease by 32 bytes.
};
\end{lstlisting}
\end{document}
