\documentclass[12pt,a4paper]{article}

% Margins.
\setlength{\oddsidemargin}{0in}
\setlength{\evensidemargin}{0in}
\setlength{\headheight}{12pt}
\setlength{\headsep}{42pt}
\setlength{\topmargin}{-54pt}
\setlength{\textwidth}{6.5in}
\setlength{\textheight}{10in}

\usepackage{amsmath}
\usepackage{float}
\usepackage{graphicx}
\usepackage[hyphens]{url}
\usepackage{hyperref}	% Clickable links to figures, references and urls.
\usepackage{datetime}

% Drawing.
\usepackage{pgf}
\usepackage{tikz}

% Listings for formatting code.
\usepackage{listings}
\usepackage{textcomp}
% General options.
\lstset{breaklines=true, basicstyle=\small\ttfamily, tabsize=4, numbers=left, stepnumber=1, frame=single, showstringspaces=false, upquote=true}
% C++ specific high-lighting. Comments are 50/50 shades of green/black and strings coloured with 60/40 red/black mixture.
\lstset{language=[ISO]C++, commentstyle=\color{green!50!black}, keywordstyle=\color{blue}, stringstyle=\color{red!60!black}}

%opening
\title{\vspace{-2cm}Programming for Engineers II\\Assignment 06\\Doubly--Linked List}
\author{Attique Dawood}
\date{April 15, 2014\\Due: April 22, 2014\\[0.2cm] Last Modified: \today, \currenttime}
\begin{document}
\maketitle
Create a doubly--linked list. Linked list class template is given. 
\begin{lstlisting}
class Node
{
	public:
	int Data;
	Node* Next;
	Node* Previous;
};
class LinkedList
{
	private:
	Node* First;
	Node* Last;
	int Count;
	public:
	LinkedList();  // Constructor.
	~LinkedList(); // Destructor.
	
	// Node addition functions.
	void AddAtStart(int); // Adds node at beginning of list.
	void AddAtEnd(int);   // Adds node at end of list (default).
	void AddSorted(int);  // Adds a node in sorted order assuming list is already sorted in ascending order.
	
	void Display(); // Display linked list.
	int Search(int); // If search is successful, 0 is returned otherwise -1 is returned.
	int Delete(int); // If search is successful and node is deleted, 0 is returned otherwise -1 is returned.
	int& operator[](int); // Array subscript operator.
	void BubbleSort(); // Applies bubble-sort on linked list and sorts in ascending order.
	
};	
\end{lstlisting}
\newpage
\textbf{{\Large Bubblesort Code}}
\begin{lstlisting}[caption={Bubble Sort}]
#include <cstdlib>
#include <iostream>
using namespace std;

int main()
{
	const int Size = 5;
	int data[Size];

	// Storing random numbers in range 0-99 in data[]
	for (int i=0; i<Size; i++)
		data[i] = rand() % 100;
		
	// Displaying data before sorting.
	cout << "Data before sorting: ";
	for (int i=0; i<Size; i++)
		cout << data[i] << " ";
	cout << endl;

	// Bubblesort
	for (int c=0; c<Size; c++)
	{
		int currentsize = Size-c;
		cout << "===== Iteration no: " << c+1 << " ====" << endl;
		for (int i=0; i<currentsize-1; i++)
		{
			// Sorting condition.
			if (data[i] > data[i+1])
			{
				// Swap.
				int temp = data[i];
				data[i] = data[i+1];
				data[i+1] = temp;
			}
			cout << "Array after swapping no: " << i << endl;
			for (int z=0; z<Size; z++)
				cout << data[z] << " ";
			cout << endl;
		}
		cout << "===================================" << endl;
	}

	// Displaying data after sorting.
	cout << "Data after sorting: ";
	for (int i=0; i<Size; i++)
		cout << data[i] << " ";
	cout << endl;
	
	return 0;
}
\end{lstlisting}
\end{document}
