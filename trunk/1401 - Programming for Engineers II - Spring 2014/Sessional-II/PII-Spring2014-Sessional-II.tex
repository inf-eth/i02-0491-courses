\documentclass[12pt,a4paper]{article}

% Margins.
\setlength{\oddsidemargin}{0in}
\setlength{\evensidemargin}{0in}
\setlength{\headheight}{12pt}
\setlength{\headsep}{42pt}
\setlength{\topmargin}{-70pt}
\setlength{\textwidth}{6.5in}
\setlength{\textheight}{10in}
\pagestyle{empty}

\usepackage{amsmath}
\usepackage{float}
\usepackage{graphicx}
\usepackage[hyphens]{url}
\usepackage[hidelinks]{hyperref}	% Clickable links to figures, references and urls.
\usepackage{lastpage}

% Drawing.
\usepackage{pgf}
\usepackage{tikz}

% Listings for formatting code.
\usepackage{listings}
\usepackage{textcomp}
% General options.
\lstset{breaklines=true, basicstyle=\footnotesize\ttfamily, tabsize=4, numbers=none, stepnumber=1, frame=single, showstringspaces=false, upquote=true}
% C++ specific high-lighting. Comments are 50/50 shades of green/black and strings coloured with 60/40 red/black mixture.
\lstset{language=[ISO]C++, commentstyle=\color{green!50!black}, keywordstyle=\color{blue}, stringstyle=\color{red!60!black}}

% Marks of each question.
\def\QOne{10}
\def\Qtwo{10}
\def\Qthree{10}
\def\Qfour{10}
\def\Qfive{10}
\def\TotalMarks{50}

\begin{document}
\begin{minipage}{0.55\textwidth}
{\LARGE \textbf{Programming for Engineers}}\\[0.15cm]
{\normalsize \textbf{Spring 2014 Semester}}\\
{\Large \textbf{$2^{nd}$ Sessional Exam}}\\
{\normalsize \textbf{Saturday, April 05, 2014}}\\[0.30cm]
{\Large \textbf{Total Time: 60 minutes}}\\[0.15cm]
{\Large \textbf{Total Marks: 50}}\\
\textbf{Course Instructors:}\\
Attique Dawood\\
\end{minipage}
\begin{minipage}{0.4\textwidth}
\textbf{Serial} \hrulefill \\[0.25cm]
\textbf{Name} \hrulefill\\[0.25cm]
\textbf{Section} \rule{1cm}{0.2mm} \textbf{Roll No:} \hrulefill\\[0.25cm]
\textbf{Signature:} \hrulefill\\[0.25cm]
\rule{6.6cm}{0.2mm}\\
\textbf{Signature of Invigilator}\\[0.25cm]
\end{minipage}
\begin{table}[H]
\begin{center}
\vspace{0.3cm}
	{\large \begin{tabular}{|l|c|c|c|c|c|c|}
	\hline
		\rule{0pt}{2.6ex} Question & \textbf{1} & \textbf{2} & \textbf{3} & \textbf{4} & \textbf{5} & \textbf{Total}\\
		\hline
		Marks Obtained \rule{0pt}{2.6ex} & & & & & &\\
		\hline
		Total Marks \rule{0pt}{2.6ex} & \QOne & \Qtwo & \Qthree & \Qfour & \Qfive & \TotalMarks\\
	\hline
	\end{tabular}}
\end{center}
\end{table}
\noindent \textbf{You are advised to READ these notes:}
\begin{enumerate}
\item \textbf{Attempt on the Question Paper. \underline{NO EXTRA SHEET} will be provided/accepted. No
additional sheet will be provided for rough work. Use the back of the page where
provided space is not sufficient.}
\item After asked to commence the exam, please verify that you have \textbf{\pageref{LastPage} different
printed pages} including this title page.
\item There are 5 questions. Attempt all of them. It is advisable to go through the paper once
before starting with the first question.
\item Exam is closed books, closed notes. Please see that the area in your threshold is clean.
You will be charged for any material which can be classified as \textbf{`helping in the paper'}
found near you.
\item \textbf{Calculator sharing is strictly prohibited.}
\item Students who attempt the paper with lead pencils lose the right to get them rechecked.
\item \textbf{The invigilator present is not supposed to answer any questions. No one may come
to your room for corrections and you are not supposed to request to call anyone.
Make assumptions wherever required and clearly mark them.}
\end{enumerate}
\newpage

\noindent\textbf{Question 1: Stream Operators \hfill \QOne~marks}\\
Overload stream operators for the \verb|Complex| class.
\begin{lstlisting}[escapechar=!]
#include <iostream>
using namespace std;
class Complex
{
	private:
	int real;
	int img;
	public:









































int main()
{
	Complex A;
	cin >> A;
	cout << A;
	
	return 0;
}
\end{lstlisting}
\newpage
\noindent\textbf{Question 2: Data Conversion\hfill \Qtwo~marks}\\
Write appropriate functions in \verb|Complex| class so that code in \verb|main()| runs without any errors.
\begin{lstlisting}[escapechar=!]
#include <iostream>
using namespace std;
class Complex
{
	private:
	int real;
	int img;
	public:






































int main()
{
	Complex A(3, 5); // A.real is set 3 and A.img is set 5.
	int x = -2;
	A = x; // A.real is set -2 and A.img is set 0.

	Complex B(-9, 7);
	int y = B; // B.real is copied into y.
	
	return 0;
}
\end{lstlisting}
\newpage
\noindent\textbf{Question 3: 2D Dynamic Array \hfill \Qfive~marks}\\
Write code to allocate and de--allocate a 2D array using a double pointer.
\begin{lstlisting}[escapechar=!]
int main()
{
	int** Array2D;
	
	
	
	
	
	
	
	
	
	
	
	
	
	
	
	
	
	
	
	
	
	
	
	
	
	
	
	
	
	
	
	
	
	
	
	
	
	
	
	
	
	
	
	
	
	
	
	
	
	
	

	return 0;
}
\end{lstlisting}
\newpage
\noindent\textbf{Question 4: Pointers \hfill \Qfour~marks}\\
For the following code draw a memory map showing the contents of memory before \verb|return 0;| statement. It is easier to divide the memory into chunks of 4 bytes.
\begin{lstlisting}[escapechar=!]
int main()
{
	int** ipp;

	int x1 = 5;
	int x2 = 6;
	int x3 = 7;

	int* ip1;
	int* ip2;

	ip1 = &x1;
	ip2 = &x2;
	ipp = &ip1;

	*ipp = ip2;
	*ipp = &x3;
	**ipp = -1;

	return 0;
}
\end{lstlisting}
\newpage
\noindent\textbf{Question 5: Access Control in Inheritance \hfill \Qfive~marks}\\
Fill the following tables with ticks or crosses indicating accessibility. Assume \verb|public| filter for derived class.
\begin{table}[H]
\begin{center}
\vspace{0.3cm}
	{\large \begin{tabular}{|l|c|c|c|}
	\hline
		\rule{0pt}{2.6ex} & \textbf{Base} & \textbf{Derived} & \textbf{Main}\\
		\hline
		Private \rule{0pt}{2.6ex} & & & \\
		\hline
		Protected \rule{0pt}{2.6ex} & & & \\
		\hline
		Public \rule{0pt}{2.6ex} & & & \\
	\hline
	\end{tabular}}
\end{center}
\end{table}
\end{document}