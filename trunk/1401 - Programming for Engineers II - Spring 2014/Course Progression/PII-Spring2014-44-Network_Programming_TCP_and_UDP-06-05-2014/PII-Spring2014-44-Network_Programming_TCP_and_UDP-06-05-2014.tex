\documentclass[12pt,a4paper]{article}

% Margins.
\setlength{\oddsidemargin}{0in}
\setlength{\evensidemargin}{0in}
\setlength{\headheight}{12pt}
\setlength{\headsep}{42pt}
\setlength{\topmargin}{-54pt}
\setlength{\textwidth}{6.5in}
\setlength{\textheight}{10in}

\usepackage{amsmath}
\usepackage{float}
\usepackage{graphicx}
\usepackage[hyphens]{url}
\usepackage{hyperref}	% Clickable links to figures, references and urls.
\usepackage{enumerate}
\usepackage{datetime}

% Drawing.
\usepackage{pgf}
\usepackage{tikz}

% Listings for formatting code.
\usepackage{listings}
\usepackage{textcomp}
% General options.
\lstset{breaklines=true, basicstyle=\small\ttfamily, tabsize=4, numbers=left, stepnumber=1, frame=single, showstringspaces=false, upquote=true}
% C++ specific high-lighting. Comments are 50/50 shades of green/black and strings coloured with 60/40 red/black mixture.
\lstset{language=[ISO]C++, commentstyle=\color{green!50!black}, keywordstyle=\color{blue}, stringstyle=\color{red!60!black}}

%opening
\title{\vspace{-2cm}Programming for Engineers II\\Class 44\\Network Programming: TCP and UDP}
\author{Attique Dawood}
\date{May 06, 2014\\[0.2cm] Last Modified: \today, \currenttime}
\begin{document}
\maketitle
\section{Announcements}
\begin{itemize}
\item None.
\end{itemize}
\section{TCP and UDP}
TCP and UDP are two modes of communication in IP.
\begin{itemize}
\item TCP uses \verb|send()| and \verb|receive()| while UDP uses \verb|sendto()| and \verb|receivefrom()|.
\item UDP is like posting a letter. Any packet can be sent to or received from any IP address. 
\item TCP is like a phone call. A connection is established between a Server and Client. Client is the one calling and Server is the one accepting calls.
\item Although Client and Server don't exist in UDP technically, the two computers communicating using UDP are commonly distinguished by calling one Server and the other Client. Normally Server is the one that starts in waiting (\verb|receivefrom()|) state.
\end{itemize}
\section{TCP Model}
\subsection{Server}
\begin{itemize}
\item 2 sockets, 1 for accepting connections and another for communication with  Client.
\item Maintains at least two address structures, one for itself and one to store Client information. 
\item There can be multiple Clients connected to Server.
\item Sequence of Operation:
	\begin{enumerate}
    \item Create Server \verb|socket()|.
    \item \verb|bind()| Server socket with its address and port.
    \item Set Server socket options, \verb|setsockopt()| (optional).
    \item Configure Server socket to \verb|listen()| to any incoming connection requests.
    \item Set Server socket state to \verb|accept()| any connection request. This is blocking. When a connection is accepted it will return Client socket and Client address is stored.
    \item Use Client socket to \verb|send()| and \verb|recv()| data to that particular Client.
    \item \verb|close()| sockets before exiting.
	\end{enumerate}
\end{itemize}
\subsection{Client}
\begin{itemize}
\item 1 socket to connect and \verb|send()| and \verb|recv()| data.
\item Maintains two address structures, for Client and Server.
\item Sequence of Operation:
	\begin{enumerate}
    \item Create Client \verb|socket()|.
    \item \verb|bind()| Client socket with its adress and port.
    \item \verb|connect()| to a Server using Server Address and Server Port.
    \item \verb|send()| and \verb|recv()| using Client socket.
    \item \verb|close()| Client socket before exiting.
	\end{enumerate}
\end{itemize}
\section{UDP Model}
\begin{itemize}
\item Server does not need to \verb|listen()| or \verb|accept()| connections.
\item Client does not need to \verb|connect()|.
\item Server and Client are mirrors. Only logical difference is sequence of \verb|sendto()| and \verb|recvfrom()| calls.
\item Sequence of Operation:
	\begin{enumerate}
    \item Create \verb|socket()|.
    \item \verb|bind()| socket to own address and port.
    \item \verb|sendto()| `their' address or \verb|recvfrom()| `random' address on own socket.
    \item \verb|close()| socket.
    \end{enumerate}
\end{itemize}
%\nocite{*}
\bibliographystyle{plain}
\bibliography{OOPref}
\end{document}
