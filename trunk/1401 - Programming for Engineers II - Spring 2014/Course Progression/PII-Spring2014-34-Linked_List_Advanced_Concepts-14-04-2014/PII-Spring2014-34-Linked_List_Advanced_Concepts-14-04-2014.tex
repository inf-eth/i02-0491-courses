\documentclass[12pt,a4paper]{article}

% Margins.
\setlength{\oddsidemargin}{0in}
\setlength{\evensidemargin}{0in}
\setlength{\headheight}{12pt}
\setlength{\headsep}{42pt}
\setlength{\topmargin}{-54pt}
\setlength{\textwidth}{6.5in}
\setlength{\textheight}{10in}

\usepackage{amsmath}
\usepackage{float}
\usepackage{graphicx}
\usepackage[hyphens]{url}
\usepackage{hyperref}	% Clickable links to figures, references and urls.
\usepackage{enumerate}
\usepackage{datetime}

% Drawing.
\usepackage{pgf}
\usepackage{tikz}

% Listings for formatting code.
\usepackage{listings}
\usepackage{textcomp}
% General options.
\lstset{breaklines=true, basicstyle=\small\ttfamily, tabsize=4, numbers=left, stepnumber=1, frame=single, showstringspaces=false, upquote=true}
% C++ specific high-lighting. Comments are 50/50 shades of green/black and strings coloured with 60/40 red/black mixture.
\lstset{language=[ISO]C++, commentstyle=\color{green!50!black}, keywordstyle=\color{blue}, stringstyle=\color{red!60!black}}

%opening
\title{\vspace{-2cm}Programming for Engineers II\\Class 34\\Linked List\\ Advanced Concepts}
\author{Attique Dawood}
\date{April 14, 2014\\[0.2cm] Last Modified: \today, \currenttime}
\begin{document}
\maketitle
\section{Announcements}
\begin{itemize}
\item None.
\end{itemize}
\section{Advanced Linked List Concepts}
\begin{itemize}
\item \textbf{Doubly--linked list:} Can traverse in both directions. Electronic quiz; can skip questions and then come back again later.
\item \textbf{Circular--linked list:} First and last nodes are joined together.
\item \textbf{Indexing:} Index in a book? Helps you remember where you left off. Just a pointer to some node just like first and last. Can move forward or backward. Get the value at current index. Can reset index at 
\end{itemize}
\section{Adding a node at Beginning, Middle or End: Sorted Insertion?}
\begin{itemize}
\item Why do we need to insert at some random location? Insertion sort.
\item Creates a sorted list.
\item Easier to search a sorted list. Don't have to go through all the list.
\end{itemize}
\textbf{Question 1:} Modify the list in listing 1 and add a \verb|SortedInsertion()| function. This function will assume the list is sorted and add the data at appropriate position in the list.\\
\textbf{Question 2:} Create a doubly--linked list. The node contains a previous pointer in addition to next pointer. Previous pointer will point to previous node. Make functions provide add, display, search and delete functionality.\\
\textbf{Question 3:} Make the above list circular, i.e., link first and last nodes. Keep an index pointer in the linked list. Index can point to any node. Make functions to move index forward or backward and a \verb|GetIndex()| function to return the value at current location of index.\\
\textbf{Note:} Each of the above questions is worth 1 absolute point. Only first correct submission will be accepted. Students with lesser grades will be preferred.
%\nocite{*}
%\bibliographystyle{plain}
%\bibliography{OOPref}
\end{document}
