\documentclass[12pt,a4paper]{article}

% Margins.
\setlength{\oddsidemargin}{0in}
\setlength{\evensidemargin}{0in}
\setlength{\headheight}{12pt}
\setlength{\headsep}{42pt}
\setlength{\topmargin}{-54pt}
\setlength{\textwidth}{6.5in}
\setlength{\textheight}{10in}

\usepackage{amsmath}
\usepackage{float}
\usepackage{graphicx}
\usepackage[hyphens]{url}
\usepackage{hyperref}	% Clickable links to figures, references and urls.
\usepackage{datetime}

% Drawing.
\usepackage{pgf}
\usepackage{tikz}

% Listings for formatting code.
\usepackage{listings}
\usepackage{textcomp}
% General options.
\lstset{breaklines=true, basicstyle=\small\ttfamily, tabsize=4, numbers=left, stepnumber=1, frame=single, showstringspaces=false, upquote=true}
% C++ specific high-lighting. Comments are 50/50 shades of green/black and strings coloured with 60/40 red/black mixture.
\lstset{language=[ISO]C++, commentstyle=\color{green!50!black}, keywordstyle=\color{blue}, stringstyle=\color{red!60!black}}

%opening
\title{\vspace{-3cm}Physics for Engineers - Spring 2014\\Quiz \#05}
\date{\vspace{-1.5cm}Date: 01--04--2014}
\begin{document}
\maketitle
\vspace{-0.5cm}
\noindent \textbf{Time: 10 minutes\hfill Total Marks: 10}\\[0.3cm]
\noindent \textbf{Name:\rule{8cm}{1pt}\hfill Roll Number:\rule{3cm}{1pt}}\\[0.5cm]
\noindent \textbf{Question:} A conducting shell exists at $r=a$ to $r=b$ and has a net charge $2$ C. Space inside the shell is free space. If a charge $-1$ C is placed at the centre of the shell find electric field everywhere. Also describe how charge will be distributed on the surface of shell and calculate surface charge densities on inner and outer surface.
\end{document}
