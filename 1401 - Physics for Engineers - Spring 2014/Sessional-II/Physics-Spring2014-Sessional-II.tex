\documentclass[12pt,a4paper]{article}

% Margins.
\setlength{\oddsidemargin}{0in}
\setlength{\evensidemargin}{0in}
\setlength{\headheight}{12pt}
\setlength{\headsep}{42pt}
\setlength{\topmargin}{-70pt}
\setlength{\textwidth}{6.5in}
\setlength{\textheight}{10in}
\pagestyle{empty}

\usepackage{amsmath}
\usepackage{float}
\usepackage{graphicx}
\usepackage[hyphens]{url}
\usepackage[hidelinks]{hyperref}	% Clickable links to figures, references and urls.
\usepackage{lastpage}

% Drawing.
\usepackage{pgf}
\usepackage{tikz}

% Listings for formatting code.
\usepackage{listings}
\usepackage{textcomp}
% General options.
\lstset{breaklines=true, basicstyle=\footnotesize\ttfamily, tabsize=4, numbers=none, stepnumber=1, frame=single, showstringspaces=false, upquote=true}
% C++ specific high-lighting. Comments are 50/50 shades of green/black and strings coloured with 60/40 red/black mixture.
\lstset{language=[ISO]C++, commentstyle=\color{green!50!black}, keywordstyle=\color{blue}, stringstyle=\color{red!60!black}}

% Marks of each question.
\def\QOne{10}
\def\Qtwo{10}
\def\Qthree{10}
\def\Qfour{10}
\def\Qfive{10}
\def\TotalMarks{50}

\begin{document}
\begin{minipage}{0.55\textwidth}
{\LARGE \textbf{Physics for Engineers}}\\[0.15cm]
{\normalsize \textbf{Spring 2014 Semester}}\\
{\Large \textbf{$2^{nd}$ Sessional Exam}}\\
{\normalsize \textbf{Saturday, April 05, 2014}}\\[0.30cm]
{\Large \textbf{Total Time: 60 minutes}}\\[0.15cm]
{\Large \textbf{Total Marks: 50}}\\
\textbf{Course Instructor:}\\
Attique Dawood\\
\end{minipage}
\begin{minipage}{0.4\textwidth}
\textbf{Serial} \hrulefill \\[0.25cm]
\textbf{Name} \hrulefill\\[0.25cm]
\textbf{Section} \rule{1cm}{0.2mm} \textbf{Roll No:} \hrulefill\\[0.25cm]
\textbf{Signature:} \hrulefill\\[0.25cm]
\rule{6.6cm}{0.2mm}\\
\textbf{Signature of Invigilator}\\[0.25cm]
\end{minipage}
\begin{table}[H]
\begin{center}
\vspace{0.3cm}
	{\large \begin{tabular}{|l|c|c|c|c|c|c|}
	\hline
		\rule{0pt}{2.6ex} Question & \textbf{1} & \textbf{2} & \textbf{3} & \textbf{4} & \textbf{5} & \textbf{Total}\\
		\hline
		Marks Obtained \rule{0pt}{2.6ex} & & & & & &\\
		\hline
		Total Marks \rule{0pt}{2.6ex} & \QOne & \Qtwo & \Qthree & \Qfour & \Qfive & \TotalMarks\\		
	\hline
	\end{tabular}}
\end{center}
\end{table}
\noindent \textbf{You are advised to READ these notes:}
\begin{enumerate}
\item \textbf{Attempt on the Question Paper. \underline{NO EXTRA SHEET} will be provided/accepted. No
additional sheet will be provided for rough work. Use the back of the page where
provided space is not sufficient.}
\item After asked to commence the exam, please verify that you have \textbf{\pageref{LastPage} different
printed pages} including this title page.
\item There are 5 questions. Attempt all of them. It is advisable to go through the paper once
before starting with the first question.
\item Exam is closed books, closed notes. Please see that the area in your threshold is clean.
You will be charged for any material which can be classified as \textbf{`helping in the paper'}
found near you.
\item \textbf{Calculator sharing is strictly prohibited.}
\item Students who attempt the paper with lead pencils lose the right to get them rechecked.
\item \textbf{The invigilator present is not supposed to answer any questions. No one may come
to your room for corrections and you are not supposed to request to call anyone.
Make assumptions wherever required and clearly mark them.}
\end{enumerate}
\newpage
\noindent\textbf{Question 1: Continuous Charge Distributions\hfill \QOne~marks}\\
Volume charge density in the region $0<x<1$, $-1<y<0$ and $0<z<1$ is $\rho_V=(1-yz)$ C/m$^3$. Find the total charge.
\newpage

\noindent \textbf{Question 2: Gauss's Law\hfill \Qtwo~marks}\\
A uniform volume charge density $\rho_V=8\pi\epsilon_0$ C/m$^3$ exists in the region $1<r<2$. Find electric field everywhere.
\newpage

\noindent\textbf{Question 3: Conducters and Electric Field \hfill \Qthree~marks}\\
A spherical conductor of radius $1$ m has net charge $1$ C. This sphere is placed inside a conducting shell of inner radius $2$ m and outer radius $3$ m. The shell is also given a net charge $1$ C. Find electric field everywhere.
\newpage

\noindent\textbf{Question 4: Work Done in an Electric Field \hfill \Qfour~marks}\\
Electric field in a region is: $\textbf{E}=\dfrac{1}{x^2}\hat x$ V/m. Find the work done in moving a $5$ C charge from $x=2$ to $x=1$. Use $W_{AB}=-q\int\limits_{A}^{B} \textbf{E}\cdot d\textbf{\textit{l}}$.
\newpage

\noindent\textbf{Question 5: Energy of System of Charges \hfill \Qfive~marks}\\
$1$ nC, $-2$ nC and $3$ nC charges are located are (1, 0, 1), (0, 1, 1) and (0, 0, 1) respectively. Find the total work done in assembling this system of charges.

\end{document}