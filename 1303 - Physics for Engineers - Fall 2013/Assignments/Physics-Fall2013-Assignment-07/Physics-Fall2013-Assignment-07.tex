\documentclass[12pt,a4paper]{article}

% Margins.
\setlength{\oddsidemargin}{0in}
\setlength{\evensidemargin}{0in}
\setlength{\headheight}{12pt}
\setlength{\headsep}{42pt}
\setlength{\topmargin}{-54pt}
\setlength{\textwidth}{6.5in}
\setlength{\textheight}{10in}

\usepackage{amsmath}
\usepackage{float}
\usepackage{graphicx}
\usepackage[hyphens]{url}
\usepackage{hyperref}	% Clickable links to figures, references and urls.
\usepackage{datetime}
\usepackage{subfigure}

% Links direct to top of figures.
\usepackage[all]{hypcap}

% Drawing.
\usepackage{pgf}
\usepackage{tikz}

% Listings for formatting code.
\usepackage{listings}
\usepackage{textcomp}
% General options.
\lstset{breaklines=true, basicstyle=\small\ttfamily, tabsize=4, numbers=left, stepnumber=1, frame=single, showstringspaces=false, upquote=true}
% C++ specific high-lighting. Comments are 50/50 shades of green/black and strings coloured with 60/40 red/black mixture.
\lstset{language=[ISO]C++, commentstyle=\color{green!50!black}, keywordstyle=\color{blue}, stringstyle=\color{red!60!black}}

%opening
\title{\vspace{-3cm}Physics for Engineers\\Assignment 07\\Electric Potential}
\author{Arshad Hassan\and Attique Dawood}
\date{October 29, 2013\\Due: November 08, 2013\\[0.2cm] Last Modified: \today, \currenttime}
\begin{document}
\maketitle
\noindent\textbf{Question 1:} A 1 C charge is located at origin.
\begin{itemize}
\item[a.] Find electric potential, $V_A$ at point A(-1, 3, 7).
\item[b.] Find electric potential, $V_B$ at point B(2, 1, -5).
\item[c.] Calculate the potential difference $V_{AB}$ between A and B.
\item[d.] Calculate the work done in moving a 2 C charge from A to B.
\end{itemize}
\noindent\textbf{Question 2:} A 2 C charge is located at (-3, 0, 2) and a -4 C charge is located at (2, -1, 9).
\begin{itemize}
\item[a.] Find electric potential at point A(1, 1, 0).
\item[b.] Find electric potential at point B(3, 1, 2).
\end{itemize}
\noindent\textbf{Question 3:} Point charges $q_1=2$ C, $q_2=-1$ C and $q_3=3$ C are located at (-2, 4, 1), (1, -5, -2) and (-2, 0, 7), respectively. Find the total work done in assembling these charges, assuming they are brought from infinity. Also find electric potential at origin.\\[0.2cm]
\noindent\textbf{Question 4:} Solve Sessional - II.
%\bibliographystyle{plain}
%\bibliography{PhysicsRef}
\end{document}
