\documentclass[12pt,a4paper]{article}

% Margins.
\setlength{\oddsidemargin}{0in}
\setlength{\evensidemargin}{0in}
\setlength{\headheight}{12pt}
\setlength{\headsep}{42pt}
\setlength{\topmargin}{-54pt}
\setlength{\textwidth}{6.5in}
\setlength{\textheight}{10in}

\usepackage{amsmath}
\usepackage{float}
\usepackage{graphicx}
\usepackage[hyphens]{url}
\usepackage{hyperref}	% Clickable links to figures, references and urls.
\usepackage{datetime}
\usepackage{longtable}

% Links direct to top of figures.
\usepackage[all]{hypcap}

% Drawing.
\usepackage{pgf}
\usepackage{tikz}

% Listings for formatting code.
\usepackage{listings}
\usepackage{textcomp}
% General options.
\lstset{breaklines=true, basicstyle=\small\ttfamily, tabsize=4, numbers=left, stepnumber=1, frame=single, showstringspaces=false, upquote=true}
% C++ specific high-lighting. Comments are 50/50 shades of green/black and strings coloured with 60/40 red/black mixture.
\lstset{language=[ISO]C++, commentstyle=\color{green!50!black}, keywordstyle=\color{blue}, stringstyle=\color{red!60!black}}

%opening
\title{\vspace{-2cm}Physics for Engineers\\Class 09\\Line, Surface and Volume Integrals -- I}
\author{Attique Dawood}
\date{September 06, 2013\\[0.2cm] Last Modified: \today, \currenttime}
\begin{document}
\maketitle
\section{Announcements}
\begin{itemize}
\item None.
\end{itemize}
\section{Revision}
\begin{itemize}
\item Differential length, surface and volume elements.
\end{itemize}
\section{Surface and Volume Integrals}
In electromagnetics we need to solve surface and volume integrals. Some common integrals are given below.
\begin{itemize}
\item $\int_{S} d\mathrm{A}$~: Gives the total area of the surface `S'. This is generalised expression actually solved as a double integral.
\item $\int_{v} dv$~: Calculate the total volume defined by limits of `v'. This is also a generalised expression actually solved as a triple integral.
\item $\int_{S}\textbf{A}\cdot d\textbf{A}$~: Gives the total flux of the vector field \textbf{A} through the surface `S'.
\item $\int_{v}\rho_v dv$~: If $\rho_v$ is volumetric charge density in $C/m^3$ then this integrals gives the total charge in volume `v'.
\end{itemize}
%\section{Introduction to the Line Integral}
%Line integral is the generalisation of single variable integral. You may have only seen integrals of the kind $\int_{x=a}^{x=b} f(x)dx$ till now. The path of this integration is along the $x$--axis from $x=a$ to $x=b$. In electromagnetics we come across integrals where the path of integration may be any arbitrary (curved or zig zag) path in 3D space. Two important line integrals are
%\begin{itemize}
%\item $\int_{a}^{b}\hat n\cdot d\textbf{\textit{l}}$~: Gives the length of the path from a to b if $\hat n$ is parallel to the path.
%\item $\int_{a}^{b}\textbf{A}\cdot d\textbf{\textit{l}}$~: If \textbf{A} is force, then this integral gives the total work done from a to b. If \textbf{A} is electric field then this integral gives potential difference (or voltage) between a and b.
%\end{itemize}
%Notice, these integrals give a scalar. There are other forms of line integrals but we will restrict ourselves to these two for now.
%\section{Using Calculator to Solve Integrals}
%A calculator can be used to solve definite integrals. To find $\int_{0}^{3}x^2dx$, the key sequence is: $\boxed{\int dx}$ $\boxed{ALPHA}$ $\boxed{X}$ \fbox{\textasciicircum} $\boxed{2}$ $\boxed{,}$ $\boxed{0}$ $\boxed{,}$ $\boxed{3}$ $\boxed{)}$ $\boxed{=}$. The answer should come out to be 9. The sequence of instructions is: integration notation followed by expression to integrate and then limits of integration. On newer models limits may be entered in boxes.
%
%It is important to note that the calculator only treats `X' as the variable of integration. So, if you need to evaluate $\int_{0}^{3}t^2dt$, the expression would be the same as above replacing t with `X' during input.
\section{Exercises}
\noindent\textbf{Question 1:} Use an appropriate $dS$ to find the surface area of given structures.
\begin{itemize}
\item[(1)] The area of curved surface of a cylinder of radius $\rho=3$ and height $0<z<2$.
\item[(2)] The surface area of a sphere of radius $r=3$.
\item[(3)] The area of curved surface of a slice of cake described by $\rho=3$, height $0<z<2$ and $0<\phi<30^0$.
\item[(4)] The area of an icecream cone described by $\theta=30^0$ and $0<r<3$.
\end{itemize}
\noindent\textbf{Question 2:} Use a suitable $dv$ to find the volumes of structures given in question 1.\\[0.2cm]
%\noindent\textbf{Question 3:} Using $\int_{a}^{b}\hat n\cdot d\textbf{\textit{l}}$
%\begin{itemize}
%\item[(1)] Find the length of line segment from (0, 0) to (1, 2). Solve this in Cartesian as well as Cylindrical coordinates.
%\item[(2)] Find the length of body/space diagonal of a unit cube.
%\item[(3)] Find the arc length of a quarter circle of radius $\rho=3$ m in first quadrant.
%\item[(4)] Find the length of the curve $y=x^2$ from (0, 0) to (1, 1). For this problem a calculator would be handy in solving the integrals.
%\end{itemize}
%\noindent\textbf{Note:} For part (4) you need to find a unit vector along the curve $y=x^2$. A vector parallel (or tangent) to $y=x^2$ can be obtained from the slope. Numerator and denominator of slope ($\dfrac{dy}{dx}$) are the $y$-- and $x$--components, respectively, of the vector. Here, $\dfrac{dy}{dx}=\dfrac{2x}{1}$ so the parallel/tangent vector is $\textbf{n}=\hat x+2x\hat y$. The unit vector is then $\hat n=\dfrac{\hat x+2x\hat y}{\sqrt{1+4x^2}}$.\\[0.2cm]
%\noindent\textbf{Question 4:} A vector field is given by $\textbf{A}=xyz\hat y$ in a region. Evaluate the flux, $\int_{S}\textbf{A}\cdot d\textbf{A}$, through the surface given by $y=2$, $0<x<2$ and $2<z<5$.\\[0.2cm]
%\noindent\textbf{Question 5:} Electric field of a point charge is $\textbf{E}=\dfrac{1}{r^2}\hat r$ in a region. Evaluate the flux, $\int_{S}\textbf{A}\cdot d\textbf{A}$, through the spherical surface $r=2$.\\[0.2cm]
%\newpage
%\noindent\textbf{Question 6 (Example 2--4C \cite[Example 2--4, page 23]{Cheng}):} Given a force field $\textbf{F}=xy\hat x+(3x-y^2)\hat y$ in a region, evaluate the integral $\int_{P1}^{P2}\textbf{F}\cdot d\textbf{\textit{l}}$ to find the total work done in moving from P1 to P2 along path 1 and path 2.\\[0.2cm]
%\noindent\textbf{Question 7:} Electric field in a region is $\textbf{E}=x\hat x+y\hat y$. Refer to figure \ref{Cheng-integral}, evaluate the integral $\int_{P1}^{P2}\textbf{E}\cdot d\textbf{\textit{l}}$ to find the potential difference between P1 and P2. Is the potential difference same for path 1 and path 2?
%\begin{figure}[H]
%\centering
%\includegraphics[scale=0.6]{Figure2-10Cheng.png}
%\caption{Paths of integration for Question 2 \cite[Figure 2--10, page 23]{Cheng}}
%\label{Cheng-integral}
%\end{figure}
%\nocite{*}
%\bibliographystyle{plain}
%\bibliography{PhysicsRef}
\end{document}
