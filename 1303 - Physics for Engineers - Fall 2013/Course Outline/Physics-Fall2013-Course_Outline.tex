\documentclass[12pt,a4paper]{article}

% Margins.
\setlength{\oddsidemargin}{0in}
\setlength{\evensidemargin}{0in}
\setlength{\headheight}{12pt}
\setlength{\headsep}{42pt}
\setlength{\topmargin}{-54pt}
\setlength{\textwidth}{6.5in}
\setlength{\textheight}{10in}

\usepackage{amsmath}
\usepackage{float}
\usepackage{graphicx}
\usepackage[hyphens]{url}
\usepackage{hyperref}	% Clickable links to figures, references and urls.
\usepackage{datetime}
\usepackage{longtable}

% Drawing.
\usepackage{pgf}
\usepackage{tikz}

% Listings for formatting code.
\usepackage{listings}
\usepackage{textcomp}
% General options.
\lstset{breaklines=true, basicstyle=\small\ttfamily, tabsize=4, numbers=left, stepnumber=1, frame=single, showstringspaces=false, upquote=true}
% C++ specific high-lighting. Comments are 50/50 shades of green/black and strings coloured with 60/40 red/black mixture.
\lstset{language=[ISO]C++, commentstyle=\color{green!50!black}, keywordstyle=\color{blue}, stringstyle=\color{red!60!black}}

%opening
\title{\vspace{-3.5cm}Physics for Engineers\\Fall - 2013\\Course Outline}
\vspace{0.5cm}
\author{Attique Dawood}
\date{August 07, 2013\\[0.2cm] Last Modified: \today, \currenttime}
\begin{document}
\maketitle
\section{Course Information}
\noindent\textbf{Course Title:} Physics for Engineers\\
\textbf{Course Code:} NS110\\
\textbf{Pre-requisite(s):} None\\
\textbf{Credit Hrs:} 3\\
\section{Text Book}
\textbf{Title:} Physics for Scientists \& Engineers  (6th Edition)\\
\textbf{Author:} Serway, Jewett\\
\textbf{Publisher:} Thomson; Brooks Cole - Custom\\
\section{Reference Book}
\textbf{Title:} Elements of Electromagnetics (3rd Edition)\\
\textbf{Author:} M. N. O. Sadiku\\
\section{Objective}
The objective of this course is to introduce the basics of electromagnetics. Emphasis is on mathematically solving problems involving electric and magnetic fields. After taking this course students will be familiar with Maxwell’s equations and will be able to solve problems related to electrostatics and magnetostatics.
\section{Course Contents}
%\begin{table}[H]
\begin{center}
\vspace{-0.5cm}
	\begin{longtable}{|c|p{13cm}|}
	\hline \hline
		\rule{0pt}{2.6ex} \textbf{Week} & \textbf{Topics}\\
		\hline
		1 \rule{0pt}{2.6ex} & Motivation and introduction to electromagnetics; physical quantities, engineering and scientific notations (1.1); Introduction to the 2D and 3D coordinate system, scalars and vectors (1.3S); unit vector (1.4S); vector addition and subtraction (1.5S); Position and distance vectors (1.6S).\\
		\hline
		2 \rule{0pt}{2.6ex} & Function, concept of field, scalar and vector fields, Scalar and vector products (1.7S); Components of a vector, resolution of vector into rectangular components, vector algebra (1.8S); solving problems involving vectors; Introduction to 3D cartesian, cylindrical and spherical coordinate systems (2.2S, 2.3S, 2.4S).\\
		\hline
		3 \rule{0pt}{2.6ex} & Constant--coordinate surfaces and orientation of unit vectors in cartesian, cylindrical and spherical coordinate systems (2.5S); differential length, surface and volume (3.2S); concept of line, surface and volume integrals (3.3S).\\
		\hline
		4 \rule{0pt}{2.6ex} & Electrostatics, charge (23.1); Coulomb's law (23.3).\\
		\hline
		5 \rule{0pt}{2.6ex} &  The electric field, electric field of a point charge (23.4).\\
		\hline
		6 \rule{0pt}{2.6ex} & Continuous charge distributions, line, surface and volume charges,  field due to a line charge, ring of charge, disk of charge and infinite sheet of charge (23.5).\\
		\hline
		7 \rule{0pt}{2.6ex} & Electric field lines (23.6); motion of charged particle in uniform electric field (23.7).\\
		\hline
		8 \rule{0pt}{2.6ex} & Electric flux (24.1); Gauss's law (24.2); applications of Gauss's law (24.3); conductors in electrostatic equilibrium (24.4).\\
		\hline
		9 \rule{0pt}{2.6ex} & Conservative fields, potential difference and electric potential (25.1); potential difference in uniform electric field (25.2); potential energy due to point charges (25.3); obtaining electric field from electric potential (25.4); applications of electrostatics (25.8).\\
		\hline
		10 \rule{0pt}{2.6ex} & Magnetostatics, magnetic fields and forces (29.1); motion of a charged particle in magnetic field (29.4).\\
		\hline
		11 \rule{0pt}{2.6ex} & Source of magnetic field, Biot--Savart law (30.1); magnetic force between two parallel conductors (30.2).\\
		\hline
		12 \rule{0pt}{2.6ex} & Ampere's law, applications of Ampere's law, magnetic field of an infinite current carrying conductor, magnetic field of a coaxial conductor (30.3).\\
		\hline
		13 \rule{0pt}{2.6ex} & Magnetic flux (30.5); Gauss's law for magnetism (30.6); displacement current and general form of Gauss's law (30.7).\\
		\hline
		14 \rule{0pt}{2.6ex} & Faraday's law of induction (31.1); motional emf (31.2); Lenz's law (31.3); induced emf and electric fields (31.4); Final form of Maxwell's equations (31.7).\\
		\hline
		15 \rule{0pt}{2.6ex} & Time--varying fields, electromagnetic waves, source of electromagnetic radiation, electromagnetic wave propagation in free space, wave propagation in lossy and lossless media.\\
		\hline
		16 \rule{0pt}{2.6ex} & Wave propagation in lossy and lossless media, applications: antennas, metamaterials, wireless communication.\\
	\hline \hline
	\end{longtable}
\end{center}
\noindent \textbf{Note:} Topic references with an `S' are from reference book.
%\label{Physics-Course-Outline}
%\caption{Physics for Engineers Course Outline}
%\end{table}
\end{document}
