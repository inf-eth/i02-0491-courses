\documentclass[12pt,a4paper]{article}

% Margins.
\setlength{\oddsidemargin}{0in}
\setlength{\evensidemargin}{0in}
\setlength{\headheight}{12pt}
\setlength{\headsep}{42pt}
\setlength{\topmargin}{-54pt}
\setlength{\textwidth}{6.5in}
\setlength{\textheight}{10in}
\pagestyle{plain}

\usepackage{amsmath}
\usepackage{float}
\usepackage{graphicx}
\usepackage[hyphens]{url}
\usepackage[hidelinks]{hyperref}	% Clickable links to figures, references and urls.
\usepackage{lastpage}

% Drawing.
\usepackage{pgf}
\usepackage{tikz}

% Listings for formatting code.
\usepackage{listings}
\usepackage{textcomp}
% General options.
\lstset{breaklines=true, basicstyle=\footnotesize\ttfamily, tabsize=4, numbers=none, stepnumber=1, frame=single, showstringspaces=false, upquote=true}
% C++ specific high-lighting. Comments are 50/50 shades of green/black and strings coloured with 60/40 red/black mixture.
\lstset{language=[ISO]C++, commentstyle=\color{green!50!black}, keywordstyle=\color{blue}, stringstyle=\color{red!60!black}}

% Marks of each question.
\def\QOne{10}
\def\Qtwo{10}
\def\Qthree{10}
\def\Qfour{10}
\def\Qfive{10}
\def\TotalMarks{50}

\begin{document}
\begin{minipage}{0.55\textwidth}
{\LARGE \textbf{Physics for Engineers}}\\[0.15cm]
{\normalsize \textbf{Fall 2013 Semester}}\\
{\Large \textbf{$2^{nd}$ Sessional Exam}}\\
{\normalsize \textbf{Saturday, November 02, 2013}}\\[0.30cm]
{\Large \textbf{Total Time: 60 minutes}}\\[0.15cm]
{\Large \textbf{Total Marks: 50}}\\
\textbf{Course Instructors:}\\
Arshad Hassan\\
Attique Dawood\\
\end{minipage}
\begin{minipage}{0.4\textwidth}
\textbf{Serial} \hrulefill \\[0.25cm]
\textbf{Name} \hrulefill\\[0.25cm]
\textbf{Section} \rule{1cm}{0.2mm} \textbf{Roll No:} \hrulefill\\[0.25cm]
\textbf{Signature:} \hrulefill\\[0.25cm]
\rule{6.6cm}{0.2mm}\\
\textbf{Signature of Invigilator}\\[0.25cm]
\end{minipage}
\begin{table}[H]
\begin{center}
\vspace{0.3cm}
	{\large \begin{tabular}{|l|c|c|c|c|c|c|}
	\hline
		\rule{0pt}{2.6ex} Question & \textbf{1} & \textbf{2} & \textbf{3} & \textbf{4} & \textbf{5} & \textbf{Total}\\
		\hline
		Total Marks \rule{0pt}{2.6ex} & \QOne & \Qtwo & \Qthree & \Qfour & \Qfive & \TotalMarks\\
		\hline
		Marks Obtained \rule{0pt}{2.6ex} & & & & & &\\
	\hline
	\end{tabular}}
\end{center}
\end{table}
\noindent \textbf{You are advised to READ these notes:}
\begin{enumerate}
\item \textbf{Attempt on the Question Paper. \underline{NO EXTRA SHEET} will be provided/accepted. No
additional sheet will be provided for rough work. Use the back of the page where
provided space is not sufficient.}
\item After asked to commence the exam, please verify that you have \textbf{\pageref{LastPage} different
printed pages} including this title page.
\item There are 5 questions. Attempt all of them. It is advisable to go through the paper once
before starting with the first question.
\item Exam is closed books, closed notes. Please see that the area in your threshold is clean.
You will be charged for any material which can be classified as \textbf{`helping in the paper'}
found near you.
\item \textbf{Calculator sharing is strictly prohibited.}
\item Students who attempt the paper with lead pencils lose the right to get them rechecked.
\item \textbf{The invigilator present is not supposed to answer any questions. No one may come
to your room for corrections and you are not supposed to request to call anyone.
Make assumptions wherever required and clearly mark them.}
\end{enumerate}
\newpage
\noindent\textbf{Question 1: Electric Field of Continuous Charge Distributions\hfill \QOne~marks}\\
A line charge is placed on $x$--axis at $0<x<5$ with uniform line charge density $\rho_L=4\pi\epsilon_0$ C/m. Calculate electric field at a point (0, 1, 0) on $y$--axis. All distances and coordinates are in meters.\\
\textbf{Note:} $\int\dfrac{dx}{(x^2+a^2)^{(3/2)}}=\dfrac{x/a^2}{\sqrt{x^2+a^2}}$.
\begin{figure}[H]
\begin{tikzpicture}
	\draw[thick] (0cm,0cm) rectangle (\textwidth, 21.5cm);
\end{tikzpicture}
\end{figure}

\noindent \textbf{Question 2: Gauss's Law\hfill \Qtwo~marks}\\
A non--uniform charge density with radial symmetry $\rho_v=\dfrac{1}{r^2}$ C/m$^3$ exists in the region $1<r<2$. Sketch geometry of the problem and find electric field \textbf{everywhere}. All distances and coordinates are in meters.
\begin{figure}[H]
\begin{tikzpicture}
	\draw[thick] (0cm,0cm) rectangle (\textwidth, 22.5cm);
\end{tikzpicture}
\end{figure}

\noindent\textbf{Question 3: Conductors and Electric Field \hfill \Qthree~marks}\\
A uniform charge distribution with charge density $\rho_v=-\dfrac{3}{4\pi}$ C/m$^3$ exists in the region $r<1$. This spherical charge distribution is enclosed in a conducting shell from $r=2$ to $r=3$. Surface charge density on the outer surface of conductor is $\rho_S=-\dfrac{1}{36\pi}$ C/m$^2$. Find electric field \textbf{everywhere}. All distances and coordinates are in meters.
\begin{figure}[H]
\begin{tikzpicture}
	\draw[thick] (0cm,0cm) rectangle (\textwidth, 21.5cm);
\end{tikzpicture}
\end{figure}

\noindent\textbf{Question 4: Electric Potential \hfill \Qfour~marks}\\
A 1 C charge is located at (-3, 0, 2) and a -2 C charge is placed at (2, -1, 9). Calculate work done in moving a 1 nC charge from A(1, 1, 0) to B(2, 0, 0). All coordinates are in meters.
\begin{figure}[H]
\begin{tikzpicture}
	\draw[thick] (0cm,0cm) rectangle (\textwidth, 23.25cm);
\end{tikzpicture}
\end{figure}
\noindent\textbf{Question 5: Energy of System of Charges \hfill \Qfive~marks}\\
Point charges $q_1=2$ nC, $q_2=-1$ nC and $q_3=3$ nC are located at (-2, 1, 1), (1, -1, -2) and (-1, 0, 1), respectively. Find the total work done in assembling these charges, assuming they are brought from infinity. All coordinates are in meters.
\begin{figure}[H]
\begin{tikzpicture}
	\draw[thick] (0cm,0cm) rectangle (\textwidth, 22.5cm);
\end{tikzpicture}
\end{figure}

\end{document}