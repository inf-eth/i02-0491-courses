\documentclass[12pt,a4paper]{article}

% Margins.
\setlength{\oddsidemargin}{0in}
\setlength{\evensidemargin}{0in}
\setlength{\headheight}{12pt}
\setlength{\headsep}{0pt}
\setlength{\topmargin}{-60pt}
\setlength{\textwidth}{6.5in}
\setlength{\textheight}{10.75in}

\usepackage{amsmath}
\usepackage{float}
\usepackage{graphicx}
\usepackage[hyphens]{url}
\usepackage{hyperref}	% Clickable links to figures, references and urls.
\usepackage{datetime}
\usepackage{longtable}

% Drawing.
\usepackage{pgf}
\usepackage{tikz}

% Listings for formatting code.
\usepackage{listings}
\usepackage{textcomp}
% General options.
\lstset{breaklines=true, basicstyle=\small\ttfamily, tabsize=4, numbers=left, stepnumber=1, frame=single, showstringspaces=false, upquote=true}
% C++ specific high-lighting. Comments are 50/50 shades of green/black and strings coloured with 60/40 red/black mixture.
\lstset{language=[ISO]C++, commentstyle=\color{green!50!black}, keywordstyle=\color{blue}, stringstyle=\color{red!60!black}}

%opening
\title{Introduction to Computing\\Lab 02\\Handling Input}
\author{Moomal Bukhari\and Attique Dawood}
\date{February 24, 2015\\[0.2cm] Last Modified: \today, \currenttime}
\begin{document}
\maketitle
\section{Taking Input from User}
Following program uses the function \verb|cin| from the \verb|iostream| library to take two numbers from user, calculates the sum and displays it on--screen.
\begin{lstlisting}[caption={Input}]
#include <iostream> 
using namespace std; 

int main() 
{
	int x;
	int y;
	cin >> x;
	cin >> y;
	int sum = x+y;
	cout << "Sum of numbers is " << sum << endl; 
	return 0; 
}
\end{lstlisting}
\section{The Remainder Operator}
The remainder operator \verb|%| gives the remainder of a division when two numbers are involved in a division operation. \verb|a%b| gives the remainder left over when \verb|a| is divided by \verb|b|. For example, \verb|13%5| gives 3 because 12 cannot be completely divided by 5. In this case the quotient or result of divsion is 2.
\section{Exercises}
\noindent\textbf{Question No. 1:} Make a simple calculator. Ask the user to input two numbers. Calculate their sum, difference, product and result of division.\\[0.2cm]
\noindent\textbf{Question No. 2:} Write a program that takes days as input from user. Convert days into years, months and leftover days. For example, 45 days are 1 month and 15 days; 400 days are 1 year, 1 month and 5 days.\\[0.2cm]
\noindent\textbf{Question No. 3:} Write a program to divide two numbers and give the answer in fractional form. You can integrate this into calculator.
\end{document}