\documentclass[12pt,a4paper]{article}

% Margins.
\setlength{\oddsidemargin}{0in}
\setlength{\evensidemargin}{0in}
\setlength{\headheight}{12pt}
\setlength{\headsep}{0pt}
\setlength{\topmargin}{-60pt}
\setlength{\textwidth}{6.5in}
\setlength{\textheight}{10.75in}

\usepackage{amsmath}
\usepackage{float}
\usepackage{graphicx}
\usepackage[hyphens]{url}
\usepackage{hyperref}	% Clickable links to figures, references and urls.
\usepackage{datetime}

% Drawing.
\usepackage{pgf}
\usepackage{tikz}

% Listings for formatting code.
\usepackage{listings}
\usepackage{textcomp}
% General options.
\lstset{breaklines=true, basicstyle=\small\ttfamily, tabsize=4, numbers=left, stepnumber=1, frame=single, showstringspaces=false, upquote=true}
% C++ specific high-lighting. Comments are 50/50 shades of green/black and strings coloured with 60/40 red/black mixture.
\lstset{language=[ISO]C++, commentstyle=\color{green!50!black}, keywordstyle=\color{blue}, stringstyle=\color{red!60!black}}

%opening
\title{Programming for Engineers I\\Class 09\\Introduction to \texttt{struct}}
\author{Attique Dawood}
\date{July 13, 2014\\[0.2cm] Last Modified: \today, \currenttime}
\begin{document}
\maketitle
\section{Announcements}
\begin{itemize}
\item None.
\end{itemize}
\section{Revision}
\begin{itemize}
\item Pointers arithmetic.
\item Dynamic memory allocation in C++. \verb|new|, \verb|new []|, \verb|delete| and \verb|delete[]|.
\end{itemize}
\section{\texttt{struct}}
\begin{itemize}
\item Can create new data types from basic types.
\item Types can be anything int, float, double. Different types can be mixed in a \texttt{struct}.
\item Group similar data together.
\item Can also include other structs (matrix of complex numbers?).
\end{itemize}
\begin{lstlisting}[caption={\texttt{struct}}]
struct complex
{
	float real;
	float img;
};

int main()
{
	complex c1;
	complex c2;
	c1.real = 1.f;
	c2.img = 0.f;
	c2.real = c1.real + 34.f;
	c2.img = 3.3f * c1.img - 2.2f;
	
	return 0;
}
\end{lstlisting}
\section{Structured Approach}
\begin{lstlisting}[caption={Using \texttt{struct} to define complex number and add them using functions}]
#include <iostream>
using std::cin;
using std::cout;
using std::endl;

struct complex
{
	float real;
	float img;
};

// Input function prototype.
void InputComplex(complex&);
// Input function definition.
void InputComplex(complex& c)
{
	cout << "Enter complex number: ";
	cin >> c.real >> c.img;
}

// Addition function prototype.
complex AddComplex(complex, complex);
// Addition function definition.
complex AddComplex(complex c1, complex c2)
{
	complex sum;
	sum.real = c1.real + c2.real;
	sum.img = c1.img + c2.img;
	return sum;
}

// Display function prototype.
void DisplayComplex(complex&);
// Display function definition.
void DisplayComplex(complex& c)
{
	if (c.img < 0.f)
		cout << c.real << " - j" << -1.f*c.img << endl;
	else
		cout << c.real << " + j" << c.img << endl;
}

int main()
{
	// First complex number.
	complex c1;
	// Second complex number.
	complex c2;
	// Resultant complex number.
	complex c3;
	
	// First complex input.
	InputComplex(c1);
	// Second complex input.
	InputComplex(c2);
	
	// Addition.
	c3 = AddComplex(c1, c2);
	
	// Result.
	cout << "Result is: ";
	DisplayComplex(c3);
	
	return 0;
}
\end{lstlisting}

\end{document}