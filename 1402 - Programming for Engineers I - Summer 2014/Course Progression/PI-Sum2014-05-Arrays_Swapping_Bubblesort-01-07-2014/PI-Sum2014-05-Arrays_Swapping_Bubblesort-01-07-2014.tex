\documentclass[12pt,a4paper]{article}

% Margins.
\setlength{\oddsidemargin}{0in}
\setlength{\evensidemargin}{0in}
\setlength{\headheight}{12pt}
\setlength{\headsep}{0pt}
\setlength{\topmargin}{-60pt}
\setlength{\textwidth}{6.5in}
\setlength{\textheight}{10.75in}

\usepackage{amsmath}
\usepackage{float}
\usepackage{graphicx}
\usepackage[hyphens]{url}
\usepackage{hyperref}	% Clickable links to figures, references and urls.
\usepackage{datetime}
% Drawing.
\usepackage{pgf}
\usepackage{tikz}

% Listings for formatting code.
\usepackage{listings}
\usepackage{textcomp}
% General options.
\lstset{breaklines=true, basicstyle=\small\ttfamily, tabsize=4, numbers=left, stepnumber=1, frame=single, showstringspaces=false, upquote=true}
% C++ specific high-lighting. Comments are 50/50 shades of green/black and strings coloured with 60/40 red/black mixture.
\lstset{language=[ISO]C++, commentstyle=\color{green!50!black}, keywordstyle=\color{blue}, stringstyle=\color{red!60!black}}

%opening
\title{Programming for Engineers I\\Class 05\\Arrays\\Swapping\\Bubblesort}
\author{Attique Dawood}
\date{July 01, 2014\\[0.2cm] Last Modified: \today, \currenttime}
\begin{document}
\maketitle
\section{Announcements}
\begin{itemize}
\item None.
\end{itemize}
\section{Revision}
\begin{itemize}
\item \verb|switch| statement.
\end{itemize}
\section{Arrays}
\begin{itemize}
\item Arrays can be used to create multiple variables of the same type.
\item To store marks of 50 students it would be tedious to declare 50 integers. However, an array makes it easier.
\item Arrays are normally used to group and store similar data.
\item Elements of an array can be accessed with subscript operator \verb|[]|.
\item A \verb|for| loop is naturally suited for array manipulation.
\end{itemize}
\begin{lstlisting}[caption={A Simple Array}]
#include <iostream>
using namespace std;

int main()
{
	int Grades[5];
	// Input all grades in a loop.
	for (int i=0; i<5; i++)
		cin >> Grades[i];
	// Output grades on console.
	cout << "Grades entered: ";
	for (int i=0; i<5; i++)
		cout << Grades[i] << " ";

	cout << endl;

	return 0;
}
\end{lstlisting}
\section{Initialization}
Array can either be initialised at declaration or at a later stage using loop.
\subsection{Initialisation at Declaration}
\begin{lstlisting}[caption={Array initialisation at declaration}]
int myArray1[10] = {0}; // Initialises whole array with zeroes.
int myArray2[10] = {-1,11,-12,10,15,2,4,-7,4,21}; // Initialise individual elements.
\end{lstlisting}
\subsection{Using \texttt{for} loop}
\begin{lstlisting}[caption={Array initialisation using for loop}]
int myArray1[10]; // Declaration.
for(int i=0; i<10; i++)
{
	myArray[i] = 0; // Initialise all elements to zero.
}
\end{lstlisting}
\section{Sample Program}
\begin{lstlisting}[caption={Sample program}]
#include <iostream>
using namespace std;

int main()
{
	int Age[4]; // Declaration.
	Age[0] = 20;
	Age[1] = 19;
	Age[2] = 22;
	Age[3] = 21;

	cout << "First element  = Age[0]" << endl;
	cout << "Second element  = Age[1]" << endl;
	cout << "Third element  = Age[2]" << endl;
	cout << "Fourth element  = Age[3]" << endl;

	return 0;
}
\end{lstlisting}
\section{Copying Arrays}
Arrays cannot be copied using assignment operator, so individual elements must be copied using a loop. Similarly, other operations like addition, subtraction etc. must performed on individual elements rather than whole arrays.
\begin{lstlisting}[caption={Copying an array}]
#include <iostream>
using namespace std;

int main()
{
	float Array1[5] = {-0.33f,-0.45f,0.22f,0.532f,0.101f};
	float Array2[5];

	// Copying...
	for (int i=0; i<5; i++)
	{
		Array2[i] = Array1[i];
	}
	// Display.
	for (int i=0; i<5; i++)
	{
		cout << "Element " << i+1 << " = " << Array2[i] << endl;
	}

	return 0;
}
\end{lstlisting}
\section{\texttt{const} Variables and \texttt{\#define}}
\begin{itemize}
\item A variable can be declared as a constant with \verb|const| directive before declaration.
\item Value of a constant variable cannot be changed.
\item Constant variable can be assigned (initialised) a value at declaration.
\item If constant variable is not assigned a value at declaration then it is useless because it will contain garbage.
\item Size of array must be constant. If a non-const variable is used as array size this will generate an error.
\item Another way of passing array size is to use \verb|#define| directive.
\end{itemize}
\begin{lstlisting}[caption={Using const as Array Size}]
int main()
{
	const int Size1 = 10;
	int Size2 = 10;
	
	int Array1[Size1]; // This will work.
	int Array2[Size2]; // ERROR: Size must be constant.

	return 0;
}
\end{lstlisting}
\begin{lstlisting}[caption={Using \#define to Declare Array Size}]
#define Size 10
int main()
{
	int Array[Size];

	return 0;
}
\end{lstlisting}
\section{Variable Swapping}
To swap two variables a third temporary variable is required.
\begin{lstlisting}[caption={Swapping Variables}]
int main()
{
	int A = 3;
	int B = 4;
	int temp;
	
	// Swapping.
	temp = A;
	A = B;
	B = temp;
	
	return 0;
}
\end{lstlisting}
\section{Using \texttt{rand()} to Generate Random Integers}
\begin{itemize}
\item \verb|rand()| is located in \verb|<cstdlib>|.
\item Every time \verb|rand()| is called it returns a different random integer in the range 0-MAX.
\item \verb|rand()| can be used with modulus operator to generate variables in certain range.
\end{itemize}
\begin{lstlisting}[caption={Using rand() to Fill an Array}]
#include <iostream>
#include <cstdlib>

#define size 10

int main()
{
	int Random[size];
	
	for (int i=0; i<size; i++)
		Random[i] = rand() % 6; // Fill Random[] array with random numbers in range 0-5.
		
	cout << "Displaying array: ";
	for (int i=0; i<size; i++)
		cout << Random[i] << " ";
		
	cout << endl;
	
	return 0;
}
\end{lstlisting}
\section{2D Array}
\begin{lstlisting}[caption={Using rand() to Fill a 2D Array}]
#include <iostream>
#include <cstdlib>

#define Rows 4
#define Cols 5

int main()
{
	int Random[Rows][Cols];
	
	for (int i=0; i<Rows; i++)
		for (int j=0; j<Cols; j++)
			Random[i][j] = rand() % 5; // Fill Random[] array with random numbers in range 0-5.
		
	cout << "Displaying random matrix of size " << Rows << "x" << Cols << ": " << endl;
	for (int i=0; i<Rows; i++)
	{
		cout << "Row " << i+1 << ": ";
		for (int j=0; j<Cols; j++)
		{
			cout << Random[i][j] << " ";
		}		
		cout << endl;
	}
		
	return 0;
}
\end{lstlisting}
\section{How to Find Smallest and Greatest Integer From an Array?}
\begin{enumerate}
\item Take first element as smallest.
\item Test next element.
\item If next element is less than smallest then new smallest is the variable being tested.
\end{enumerate}
\begin{lstlisting}[caption={Smallest Number in an Array}]
#include <cstdlib>
#include <iostream>
using namespace std;

int main()
{
	const int Size = 10;
	int data[Size];
	int min;
	
	// Storing random numbers in range 0-99 in data[]
	for (int i=0; i<Size; i++)
		data[i] = rand() % 100;
		
	// Displaying data.
	cout << "Data: ";
	for (int i=0; i<Size; i++)
	{
		cout << data[i] << " ";
	}
	cout << endl;

	// Finding smallest value.
	min = data[0];
	for (int i=0; i<Size; i++)
	{
		if (data[i] < min)
			min = data[i];
	}
	cout << "Smallest value is " << min << endl;

	return 0;
}
\end{lstlisting}
\begin{enumerate}
\item Take first element as largest.
\item Test next element.
\item If next element is greater than largest then new largest is the variable being tested.
\end{enumerate}
\begin{lstlisting}[caption={Largest Number in an Array}]
#include <cstdlib>
#include <iostream>
using namespace std;

int main()
{
	const int Size = 10;
	int data[Size];
	int max;
	
	// Storing random numbers in range 0-99 in data[]
	for (int i=0; i<Size; i++)
		data[i] = rand() % 100;
		
	// Displaying data.
	cout << "Data: ";
	for (int i=0; i<Size; i++)
	{
		cout << data[i] << " ";
	}
	cout << endl;

	// Finding largest value.
	max = data[0];
	for (int i=0; i<Size; i++)
	{
		if (data[i] > max)
			max = data[i];
	}
	cout << "Largest value is " << max << endl;

	return 0;
}
\end{lstlisting}
\section{Bubblesort}
\begin{itemize}
\item Bubblesort is a sorting algorithm.
\item Sorting an array in ascending order.
\item Larger values are `bubbled' to the top.
\end{itemize}
\begin{lstlisting}[caption={Bubble Sort}]
#include <cstdlib>
#include <iostream>
using namespace std;

int main()
{
	const int Size = 5;
	int data[Size];

	// Storing random numbers in range 0-99 in data[]
	for (int i=0; i<Size; i++)
		data[i] = rand() % 100;
		
	// Displaying data before sorting.
	cout << "Data before sorting: ";
	for (int i=0; i<Size; i++)
		cout << data[i] << " ";
	cout << endl;

	// Bubblesort
	for (int c=0; c<Size; c++)
	{
		int currentsize = Size-c;
		cout << "===== Iteration no: " << c+1 << " ====" << endl;
		for (int i=0; i<currentsize-1; i++)
		{
			// Sorting condition.
			if (data[i] > data[i+1])
			{
				// Swap.
				int temp = data[i];
				data[i] = data[i+1];
				data[i+1] = temp;
			}
			cout << "Array after swapping no: " << i << endl;
			for (int z=0; z<Size; z++)
				cout << data[z] << " ";
			cout << endl;
		}
		cout << "===================================" << endl;
	}

	// Displaying data after sorting.
	cout << "Data after sorting: ";
	for (int i=0; i<Size; i++)
		cout << data[i] << " ";
	cout << endl;
	
	return 0;
}
\end{lstlisting}
\end{document}