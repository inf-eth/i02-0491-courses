\documentclass[12pt,a4paper]{article}

% Margins.
\setlength{\oddsidemargin}{0in}
\setlength{\evensidemargin}{0in}
\setlength{\headheight}{12pt}
\setlength{\headsep}{0pt}
\setlength{\topmargin}{-60pt}
\setlength{\textwidth}{6.5in}
\setlength{\textheight}{10.75in}

\usepackage{amsmath}
\usepackage{float}
\usepackage{graphicx}
\usepackage[hyphens]{url}
\usepackage{hyperref}	% Clickable links to figures, references and urls.
\usepackage{datetime}
% Drawing.
\usepackage{pgf}
\usepackage{tikz}

% Listings for formatting code.
\usepackage{listings}
\usepackage{textcomp}
% General options.
\lstset{breaklines=true, basicstyle=\small\ttfamily, tabsize=4, numbers=left, stepnumber=1, frame=single, showstringspaces=false, upquote=true}
% C++ specific high-lighting. Comments are 50/50 shades of green/black and strings coloured with 60/40 red/black mixture.
\lstset{language=[ISO]C++, commentstyle=\color{green!50!black}, keywordstyle=\color{blue}, stringstyle=\color{red!60!black}}

%opening
\title{Programming for Engineers I\\Class 03\\Conditional Statements and Basic Variable Types\\ \texttt{char} and ASCII Table\\Float Representation}
\author{Attique Dawood}
\date{June 24, 2014\\[0.2cm] Last Modified: \today, \currenttime}
\begin{document}
\maketitle
\section{Revision}
\begin{itemize}
\item There are mathematical operators and logic operators.
\item Math or arithmetic operators are +, -, *, /, \%. Operator precedence?
\item Basic logic operations are NOT, AND, OR, XOR.
\item In C/C++ there are two types of logic operations: bitwise and logical statements.
\item The bitwise logic operators are \verb|~| for NOT, \verb|&| for AND, \verb@|@ for OR and \verb|^| for XOR. These operate on individual bits of variable(s).
\item The logic operators for evaluating statements are \verb|!| for NOT, \verb|&&| for AND and \verb@||@ for OR. Logical operators can operate on logical statements.
\item The operators \verb|<<| and \verb|>>| have different uses depending on context.
\item As stream operator used with \verb|cin| and \verb|cout| for input/output.
\item As shift operators, \verb|<<| and \verb|>>| operate on individual bits of a variable. For example, \verb|x<<2| which means bits of variable \verb|x| left--shifted by 2.
\end{itemize}
\section{Comparison Operators}
The comparison operators can be used to compare two values. The values can also be variables.  For example, \verb|x == y|. This should be read as \textit{Is x equal to y?} Most commonly used operators are:
\begin{itemize}
\item \verb|==| (equals?)
\item \verb|!=| (not equals?)
\item \verb|>| (greater than?)
\item \verb|<| (less than?)
\item \verb|>=| (equals or greater than?)
\item \verb|<=| (equals or less than?)
\end{itemize}

\section{Conditional Statements}
The syntax for conditional statement is,
\begin{lstlisting}
if (condition...)
{
    do this if condition is true...
}
else
{
    do this if condition is false...
}
\end{lstlisting}
If the condition is evaluated to be 0 then it is false. If condition is non--zero it is true.
Operators in previous section are useful for testing a certain condition. For example,
\begin{lstlisting}
int x = 2;
int y = 4;
if (x == y)
{
    cout << "x is equal to y" << endl;
}
else
{
   cout << "x is not equal to y" endl;
}
\end{lstlisting}
In this case the output will be \verb|x is not equal to y|. The \verb|else| part is not mandatory and an \verb|if| statement can be used standalone. For example,
\begin{lstlisting}
int x;
cin >> x;
if (x > 0)
{
    cout << "You have entered a positive number" << endl;
}
\end{lstlisting}
\section{Characters}
\begin{itemize}
\item ASCII table.
\item char input/output.
\item Usage of single quotes.
\item char and int conversions.
\end{itemize}
\begin{lstlisting}
// Display char.
char x = 'A';
cout << x;

// Char to int conversion. Display ASCII code.
cout << (int)x;

// Int to char conversion. Display ASCII symbol of code.
int z = 65;
cout << (char)z;
\end{lstlisting}
\section{Floating Point Number Representation}
To review floating point numbers and their conversion please take a look at \url{http://www.tfinley.net/notes/cps104/floating.html} or your class lectures/notes etc. Following example is taken from the above link.
\subsection{Example: Convert 329.390625 to 32 bit Floating Point Notation}
\subsubsection{Compute binary of integral part}
\verb|329 = 101001001|
\subsubsection{Compute binary of fractional part}
This can be done by repeatedly multiplying the fractional part by 2 and examining the bit left of decimal point. Repeat until fractional part becomes zero.
\begin{verbatim}
0.390625* 2 = 0.78125   0
0.78125 * 2 = 1.5625    1
0.5625  * 2 = 1.125     1
0.125   * 2 = 0.25      0
0.25    * 2 = 0.5       0
0.5     * 2 = 1         1
0

329 = 101001001
.390625 = 0110011

329.390625 = 101001001.0110011
\end{verbatim}
\subsubsection{Write the resulting number in scientific notation}
$1.010010010110011 \times 2^8$
\subsubsection{Compute Exponent}
Adding 127 to 8 we get 135. Binary of 135 is \verb|10000111| which is our 8 bit exponent field.
\subsubsection{Write in 32 Floating Point Notation}
Number is positive so sign bit is 0. Exponent was calculated above and mantissa is the fractional part in scientific notation with additional zeros added to make it 23 bits long.
\begin{verbatim}
Sign   Exponent         23 bit Mantissa
0      10000111      01001001011001000000000

Byte Arrangement
MSB                                                  LSB
    0100 0011    1010 0100    1011 0010    0000 0000

Memory Representation
Address    Data(bin)    Data(hex)
base       0000 0000       00
base+1     1011 0010       B2
base+2     1010 0100       A4
base+3     0100 0011       43
\end{verbatim}
\end{document}