\documentclass[12pt,a4paper]{article}

% Margins.
\setlength{\oddsidemargin}{0in}
\setlength{\evensidemargin}{0in}
\setlength{\headheight}{12pt}
\setlength{\headsep}{0pt}
\setlength{\topmargin}{-60pt}
\setlength{\textwidth}{6.5in}
\setlength{\textheight}{10.75in}

\usepackage{amsmath}
\usepackage{float}
\usepackage{graphicx}
\usepackage[hyphens]{url}
\usepackage{hyperref}	% Clickable links to figures, references and urls.
\usepackage{datetime}

% Drawing.
\usepackage{pgf}
\usepackage{tikz}

% Listings for formatting code.
\usepackage{listings}
\usepackage{textcomp}
% General options.
\lstset{breaklines=true, basicstyle=\small\ttfamily, tabsize=4, numbers=left, stepnumber=1, frame=single, showstringspaces=false, upquote=true}
% C++ specific high-lighting. Comments are 50/50 shades of green/black and strings coloured with 60/40 red/black mixture.
\lstset{language=[ISO]C++, commentstyle=\color{green!50!black}, keywordstyle=\color{blue}, stringstyle=\color{red!60!black}}

%opening
\title{Programming for Engineers I\\Assignment 04\\Electronic Dictionary}
\author{Attique Dawood}
\date{July 13, 2014\\Due: July 24, 2014 11:30 PM\\[0.2cm] Last Modified: \today, \currenttime}
\begin{document}
\maketitle
\noindent\textbf{Objectives}
\begin{itemize}
\item[1.] Learn how C--style strings work.
\item[2.] Use \verb|<cstring>| library functions for string manipulation.
\item[3.] Usage of \verb|struct|
\item[4.] Text file handling.
\end{itemize}
\noindent\textbf{Question} Create an electronic dictionary to store words along with their meanings. The dictionary should be sorted based on alphabetical order of words. First create a struct called dictionary entry with pre--allocated \verb|char| arrays of fixed size to store word and its meaning. Create an array of dictionary entries of some fixed maximum size. Provide user with options to add a new word with its meaning, search for an existing word and display its meaning and display whole dictionary at once.
\begin{verbatim}
struct DictionaryEntry
{
    char Word[50];
    char Meaning[100];
};
const int MAXSIZE = 2000;
DictionaryEntry Dictionary[MAXSIZE];
\end{verbatim}
Dictionary menu is given below:
\begin{verbatim}
1. Add word
   - Input word and meaning
2. Search word
   - Input word or part of word
3. Display whole dictionary
4. Read in dictionary from file (Dictionary.txt)
5. Store dictionary to file
6. Exit program.
\end{verbatim}
\noindent\textbf{Error Checks:}
\begin{itemize}
\item[-] Track actual dictionary size.
\item[-] When adding a word input must be taken in a temporary \verb|DictionaryEntry| variable. The input word must be matched with all existing words in the dictionary. If a match is found replace meaning without incrementing size and break loop. If the loop reaches \verb|size-1| and no match is found then the word must be added at index location \verb|size| and \verb|size| incremented by one.
\item[-] If a word already exists then replace the existing word. Exact match can be found using \verb|strcmp()|.
\item[-] Sort the dictionary after a new word is added so that dictionary remains in alphabetical order.
\item[-] Words can also be searched with a substring. For example, if dictionary contains \verb|bat|, \verb|batter| and \verb|battery| then searching for \verb|bat| should display all three words containing \verb|bat|. Substring search can be done using \verb|strstr()| function.
\item[-] The \verb|Dictionary.txt| file contains words and meanings on alternate lines.
\end{itemize}
\end{document}
