\documentclass[12pt,a4paper]{article}

% Margins.
\setlength{\oddsidemargin}{0in}
\setlength{\evensidemargin}{0in}
\setlength{\headheight}{12pt}
\setlength{\headsep}{0pt}
\setlength{\topmargin}{-60pt}
\setlength{\textwidth}{6.5in}
\setlength{\textheight}{10.75in}

\usepackage{amsmath}
\usepackage{float}
\usepackage{graphicx}
\usepackage[hyphens]{url}
\usepackage{hyperref}	% Clickable links to figures, references and urls.
\usepackage{datetime}

% Drawing.
\usepackage{pgf}
\usepackage{tikz}

% Listings for formatting code.
\usepackage{listings}
\usepackage{textcomp}
% General options.
\lstset{breaklines=true, basicstyle=\small\ttfamily, tabsize=4, numbers=left, stepnumber=1, frame=single, showstringspaces=false, upquote=true}
% C++ specific high-lighting. Comments are 50/50 shades of green/black and strings coloured with 60/40 red/black mixture.
\lstset{language=[ISO]C++, commentstyle=\color{green!50!black}, keywordstyle=\color{blue}, stringstyle=\color{red!60!black}}

%opening
\title{Programming for Engineers I\\Assignment 01\\Programming Basics}
\author{Attique Dawood}
\date{June 19, 2014\\Due: June 24, 2014\\[0.2cm] Last Modified: \today, \currenttime}
\begin{document}
\maketitle
\noindent\textbf{Question 1:} Using \verb|cout| display following patterns on screen:
\begin{lstlisting}
*      *****      *     *****
**      ****     ***    *****
***      ***    *****   *****
****      **   *******  *****
*****      *  ********* *****
\end{lstlisting}
\noindent\textbf{Question 2:} Create a program to calculate SGPA. Inputs should be GPA of five courses and their credit hours.\\
\noindent\textbf{Question 3:} Convert the following binary values into decimal and hexadecimal.
\begin{enumerate}
\item[a.] 1001
\item[b.] 1011
\item[c.] 10100010
\item[d.] 111000001101
\item[e.] 1010011000001110
\end{enumerate}
\noindent\textbf{Question 4:} Convert the following decimal values into binary and hexadecimal.
\begin{enumerate}
\item[a.] 4
\item[b.] 8
\item[c.] 39
\item[d.] 95
\item[e.] 127
\end{enumerate}
\noindent\textbf{Question 5:} Convert the following hexadecimal values into binary and decimal.
\begin{enumerate}
\item[a.] 0xA
\item[b.] 0xB3
\item[c.] 0xCE05
\item[d.] 0xFF78
\item[e.] 0x1C2FEA
\end{enumerate}
\noindent\textbf{Question 6:} A particular RAM has a total of $2^{32}$ addresses. Considering each address holds exactly one byte, find the total amount of RAM in mega bytes.\\
\noindent\textbf{Hint:} 1MB = 1024kB and 1kB = 1024B. It is easier to break down $2^{32}$ into powers of 10 and work with them. $2^{32}=2^{10}\times 2^{10}\times 2^{10}\times 2^2$. Also, how much is $2^{10}$?\\
\noindent\textbf{Question 7:} For the following variable declarations, find the exact binary and hexadecimal representation and draw a memory map of RAM showing how they are exactly stored in RAM. How much space in RAM is required to store all these variables?
\begin{enumerate}
\item[a.] \verb|int x1 = -1;|
\item[b.] \verb|unsigned int x2 = 99;|
\item[c.] \verb|unsigned short int x3 = 65001;|
\item[d.] \verb|short int x4 = -0x128;|
\item[e.] \verb|unsigned int x5 = 0xFFFFFFFF;|
\end{enumerate}
\end{document}
