\documentclass[12pt,a4paper]{article}

% Margins.
\setlength{\oddsidemargin}{0in}
\setlength{\evensidemargin}{0in}
\setlength{\headheight}{12pt}
\setlength{\headsep}{0pt}
\setlength{\topmargin}{-60pt}
\setlength{\textwidth}{6.5in}
\setlength{\textheight}{10.75in}

\usepackage{amsmath}
\usepackage{float}
\usepackage{graphicx}
\usepackage[hyphens]{url}
\usepackage{hyperref}	% Clickable links to figures, references and urls.
\usepackage{datetime}

% Drawing.
\usepackage{pgf}
\usepackage{tikz}

% Listings for formatting code.
\usepackage{listings}
\usepackage{textcomp}
% General options.
\lstset{breaklines=true, basicstyle=\small\ttfamily, tabsize=4, numbers=left, stepnumber=1, frame=single, showstringspaces=false, upquote=true}
% C++ specific high-lighting. Comments are 50/50 shades of green/black and strings coloured with 60/40 red/black mixture.
\lstset{language=[ISO]C++, commentstyle=\color{green!50!black}, keywordstyle=\color{blue}, stringstyle=\color{red!60!black}}

%opening
\title{Programming for Engineers I\\Assignment 05\\Network Programming}
\author{Attique Dawood}
\date{July 25, 2014\\Due: August 07, 2014 11:30 PM\\[0.2cm] Last Modified: \today, \currenttime}
\begin{document}
\maketitle
\noindent\textbf{Objectives}
\begin{itemize}
\item[1.] Learn the mechanics of network communication.
\item[2.] TCP and UDP models of network communication.
\item[3.] Use provided libraries to develop TCP or UDP network applications.
\end{itemize}
\noindent\textbf{Question} This assignment requires you to develop a TCP network application. The Server maintains a database of student records in a file. A Client with proper authorisation can connect to Server and perform database operations. Database operations are: add record, display all records, search for a record and update a record. The Client is presented a menu to navigate through these options. Student record structure and menu are given below.
\begin{verbatim}
// Each StudentRecord takes 64 bytes.
struct StudentRecord
{
    char Name[40];
    int Roll;
    int Age;
    float CGPA;
    int CourseMarks[3];
};
\end{verbatim}
\begin{verbatim}
============== Server Menu ==============
1. Add record
   - Input record
2. Display all records
3. Search a record
   - Input student name to search for
4. Update record
   - Input student name to search for
5. Download course outline
6. Disconnect from Server.
=========================================
\end{verbatim}
\newpage
\noindent\textbf{Algorithm:}
\begin{lstlisting}
        Server                              Client
StudentRecord temp                 StudentRecord temp
int choice = -1                    int choice = -1
while(choice != 6)                 while(choice != 6)
{                                  {
 - Send menu                        - Receive menu
 - Receive choice                   - cin and Send choice
    
 switch(choice)                     switch(choice)
 {                                  {
  case 1:                            case 1:
  {                                  {
   - Receive record                   - Input and Send record
   - Open file in append mode        }
   - Write record to file
  }
  case 2:                            case 2:
  {                                  {
   - int NoOfRecords 
   - Open file                        - int NoOfRecords
   - Find NoOfRecords
   - Send NoOfRecords to Client       - Receive NoOfRecords
   - Reset seekg pointer
   for (NoOfRecords times)            for (NoOfRecords times)
   {                                  {
    - Read a record from file          - Receive record
    - Send record to Client            - Display record
   }                                  }
  }                                  }
  case 3:                            case 3:
  {                                  {
   - Make an int found = 0            - int found
     and index = -1                   - Input name to search
   - Receive name from Client         - Send name to Server
   - Open file and read             
     individual records
     in a temp struct
   - Set found = 1 if found,
     index = i and break loop         - Receive found
   - Send found to Client             - if (found == 1)
   - if (found == 1)                     Receive record
      Send Client temp                   and display it
  }                                  }
  case 4:                            case 4:
  {                                  {
   - Same steps as case 3             - ...
   - ...
   - if (found == 1)                  - if (found ==1)
      Receive temp from Client           Input and send
   - Write to file at                    new record
     appropriate position            
     index*sizeof(StudentRecord)
  }                                  }
  case 5:                            case 5:
  {                                  { 
   - Refer to TCP file                - ...
     transfer lab
  }                                  }
  case 6:                            case 6:
  {                                  {
   - break                            - break
  }                                  }
 }                                  }
}                                  }
\end{lstlisting}
\noindent\textbf{Considerations:}
\begin{itemize}
\item[-] Server and Client can only use one \verb|StudentRecord| variable each at their ends to perform all operations.
\item[-] All the data is stored in a file \verb|records.rec| at Server.
\item[-] Declare a string to store menu at Server side and send it to Client in a loop.
\item[-] After Server sends the menu to Client it will receive a choice (an integer) and perform requested operation.
\item[-] Keep course outline in a pdf file at Server side. Programming -- I course outline is available at SLATE.
\item[-] Server and Client will both run in infinite loops. Selecting `disconnect' will properly break the loops for both Server and Client so that they can exit.
\end{itemize}
\end{document}
