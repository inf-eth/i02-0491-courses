\documentclass[12pt,a4paper]{article}

% Margins.
\setlength{\oddsidemargin}{0in}
\setlength{\evensidemargin}{0in}
\setlength{\headheight}{12pt}
\setlength{\headsep}{0pt}
\setlength{\topmargin}{-60pt}
\setlength{\textwidth}{6.5in}
\setlength{\textheight}{10.75in}

\usepackage{amsmath}
\usepackage{float}
\usepackage{graphicx}
\usepackage[hyphens]{url}
\usepackage{hyperref}	% Clickable links to figures, references and urls.
\usepackage{datetime}

% Drawing.
\usepackage{pgf}
\usepackage{tikz}

% Listings for formatting code.
\usepackage{listings}
\usepackage{textcomp}
% General options.
\lstset{breaklines=true, basicstyle=\small\ttfamily, tabsize=4, numbers=left, stepnumber=1, frame=single, showstringspaces=false, upquote=true}
% C++ specific high-lighting. Comments are 50/50 shades of green/black and strings coloured with 60/40 red/black mixture.
\lstset{language=[ISO]C++, commentstyle=\color{green!50!black}, keywordstyle=\color{blue}, stringstyle=\color{red!60!black}}

%opening
\title{\vspace{-1.5cm}Programming for Engineers I - Summer 2014\\Quiz \#02}
\date{\vspace{-1.5cm}Date: 01--07--2014}
\begin{document}
\maketitle
\vspace{-0.5cm}
\noindent\textbf{Time: 10 minutes\hfill Total Marks: 10}\\[0.3cm]
\noindent\textbf{Name:\rule{8cm}{1pt}\hfill Roll Number:\rule{3cm}{1pt}}\\[0.5cm]
\noindent\textbf{Question 1:} How is \verb|float x4 = 19.125f;| stored in RAM?
\begin{verbatim}
19 = 10011

0.125*2 = 0.25
0.25*2  = 0.5
0.5*2   = 1.0

0.125 = 0.001
\end{verbatim}
\begin{math}
19.125 = 10011.001
       = 1.0011001\times2^{4}
\end{math}
\begin{verbatim}
Exponent = 4+127 = 131
         = 10000011 (binary)
         
Sign  Exponent  Mantissa
0     10000011  00110010000000000000000

0100 0001  1001 1001  0000 0000  0000 0000 (32 bit binary)
41 99 00 00 (32 bit hex)

Memory Storage:
Address     Data(hex)
0x00000000    00
0x00000001    00
0x00000002    99
0x00000003    41
\end{verbatim}
\noindent\textbf{Question 2:} What is the decimal representation of the following 32 bit float: \verb|40 18 00 00|?
\begin{verbatim}
40 18 00 00 = 0100 0000  0001 1000  0000 0000  0000 0000 (32 bit binary)

Sign  Exponent  Mantissa
0     10000000  00110000000000000000000

10000000 = 128
128 - 127 = 1
\end{verbatim}
\begin{math}
\mathrm{Binary Number} = 1.0011\times2^1
              = 10.011
              = 2^1+2^{-2}+2^{-3}
              = 2+\dfrac{1}{4}+\dfrac{1}{8}
              = 2.375
\end{math}
\end{document}
