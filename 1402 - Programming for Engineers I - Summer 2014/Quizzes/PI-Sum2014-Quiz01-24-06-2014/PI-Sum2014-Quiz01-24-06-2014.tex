\documentclass[12pt,a4paper]{article}

% Margins.
\setlength{\oddsidemargin}{0in}
\setlength{\evensidemargin}{0in}
\setlength{\headheight}{12pt}
\setlength{\headsep}{0pt}
\setlength{\topmargin}{-60pt}
\setlength{\textwidth}{6.5in}
\setlength{\textheight}{10.75in}

\usepackage{amsmath}
\usepackage{float}
\usepackage{graphicx}
\usepackage[hyphens]{url}
\usepackage{hyperref}	% Clickable links to figures, references and urls.
\usepackage{datetime}

% Drawing.
\usepackage{pgf}
\usepackage{tikz}

% Listings for formatting code.
\usepackage{listings}
\usepackage{textcomp}
% General options.
\lstset{breaklines=true, basicstyle=\small\ttfamily, tabsize=4, numbers=left, stepnumber=1, frame=single, showstringspaces=false, upquote=true}
% C++ specific high-lighting. Comments are 50/50 shades of green/black and strings coloured with 60/40 red/black mixture.
\lstset{language=[ISO]C++, commentstyle=\color{green!50!black}, keywordstyle=\color{blue}, stringstyle=\color{red!60!black}}

%opening
\title{\vspace{-1.5cm}Programming for Engineers I - Summer 2014\\Quiz \#01}
\date{\vspace{-1.5cm}Date: 24--06--2014}
\begin{document}
\maketitle
\vspace{-0.5cm}
\noindent\textbf{Time: 10 minutes\hfill Total Marks: 10}\\[0.3cm]
\noindent\textbf{Name:\rule{8cm}{1pt}\hfill Roll Number:\rule{3cm}{1pt}}\\[0.5cm]
\noindent\textbf{Question 1:} If a memory contains a total of $2^{22}$ addresses what is the size of RAM in MBs.\\
\begin{math}
2^{10}=1024\\
2^{22}~\mathrm{B}=2^2\times2^{10}\times2^{10}~\mathrm{B}\\
2^{22}~\mathrm{B}=4\times1024\times1024~\mathrm{B}\\
2^{22}~\mathrm{B}=4\times1024~\mathrm{KB}\\
2^{22}~\mathrm{B}=4~\mathrm{MB}\\
\end{math}
\noindent\textbf{Question 2:} How are following two variables stored in RAM? What is total space taken by these two variables?
\begin{itemize}
\item[a.] \verb|unsigned int x1 = 27;|
\item[b.] \verb|short int x2 = -0x120;|
\end{itemize}
\begin{verbatim}
27 = 0x1b (hex) = 1 1011 (binary)
   = 0000 0000  0000 0000  0000 0000  0001 1011 (32 bit binary)
   = 00 00 00 1b (32 bit hex)
  
0x120 = 01 20 (16 bit hex)
      = 0000 0001  0010 0000 (16 bit binary)
      
      0000 0001  0010 0000 (original bits)
      1111 1110  1101 1111 (bits inversion)
     +                   1
     ---------------------
      1111 1110  1110 0000 (2's complement)
     ---------------------
     
-0x120 = 1111 1110  1110 0000 (16 bit binary)
       = fe e0 (16 bit hex)
       
Memory Storage:
Address      Data (bin)    Data (hex)
0x00000000   0001 1011       1b
0x00000001   0000 0000       00
0x00000002   0000 0000       00
0x00000003   0000 0000       00
0x00000004   1110 0000       e0
0x00000005   1111 1110       fe
\end{verbatim}
Total size taken by both variables is $2 + 4=6$ bytes. \verb|x1| occupies address 0--3 and \verb|x2| occupies address 4--5.
\end{document}
