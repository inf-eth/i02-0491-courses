\documentclass[12pt,a4paper]{article}

% Margins.
\setlength{\oddsidemargin}{0in}
\setlength{\evensidemargin}{0in}
\setlength{\headheight}{12pt}
\setlength{\headsep}{0pt}
\setlength{\topmargin}{-60pt}
\setlength{\textwidth}{6.5in}
\setlength{\textheight}{10.75in}

\usepackage{float}
\usepackage{amsmath}
\usepackage{graphicx}
\usepackage{datetime}

%opening
\title{\vspace{-1cm}Programming for Engineers I\\Lab 04\\Loops - II}
% '\and' creates even spacing.
\author{Attique Dawood}
\date{February 28, 2013\\[0.2cm] Last Modified: \today, \currenttime}
\begin{document}

\maketitle

\section{Exercise}
\textbf{Question No. 1:} Calculate the factorial of an integer. Take integer input from user.\\
\noindent\textbf{Question No. 2:} Calculate the power of a float. Power can only be a positive integer. Take both the base (in float) and power (integer) as input from user.\\
\noindent\textbf{Question No. 3:} Write a program to compute $e^x$. $n$ and $x$ should be input by user. The number of terms to compute will be $n+1$. Also, keep in mind $0! = 1$, same as $1! = 1$.
\begin{equation}
e^x = 1+\dfrac{x^1}{1!}+\dfrac{x^2}{2!}+\dfrac{x^3}{3!}+\dfrac{x^4}{4!}+\dfrac{x^5}{5!}+\dfrac{x^6}{6!}+...+\dfrac{x^n}{n!}.
\label{eq:ex-Taylor-Series}
\end{equation}
\end{document}
