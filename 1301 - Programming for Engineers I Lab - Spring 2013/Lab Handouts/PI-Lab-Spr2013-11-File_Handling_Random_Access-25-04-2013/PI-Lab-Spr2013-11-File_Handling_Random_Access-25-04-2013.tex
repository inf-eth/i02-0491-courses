\documentclass[12pt,a4paper]{article}

% Margins.
\setlength{\oddsidemargin}{0in}
\setlength{\evensidemargin}{0in}
\setlength{\headheight}{12pt}
\setlength{\headsep}{0pt}
\setlength{\topmargin}{-60pt}
\setlength{\textwidth}{6.5in}
\setlength{\textheight}{10.75in}

\usepackage{amsmath}
\usepackage{float}
\usepackage{graphicx}
\usepackage[hyphens]{url}
\usepackage{hyperref}	% Clickable links to figures, references and urls.
\usepackage{datetime}
% Drawing.
\usepackage{pgf}
\usepackage{tikz}
\usepackage{amssymb}  % Tick mark
\usepackage{textcomp} % Cross
\usepackage{datetime}

% Listings for formatting code.
\usepackage{listings}
\usepackage{textcomp}
% General options.
\lstset{breaklines=true, basicstyle=\small\ttfamily, tabsize=4, numbers=left, stepnumber=1, frame=single, showstringspaces=false, upquote=true}
% C++ specific high-lighting. Comments are 50/50 shades of green/black and strings coloured with 60/40 red/black mixture.
\lstset{language=[ISO]C++, commentstyle=\color{green!50!black}, keywordstyle=\color{blue}, stringstyle=\color{red!60!black}}

%opening
\title{Programming for Engineers I\\Lab 11\\File Handling\\Random Access}
\author{Attique Dawood}
\date{April 25, 2013\\[0.2cm] Last Modified: \today, \currenttime}
\begin{document}
\maketitle
\section{Put and Get Pointers}
\begin{itemize}
\item When you open a file a special read pointer is set to point at the start of file.
\item The read pointer (called a `\textbf{get}' pointer in C++) is moved forward as you read from file.
\item If you read 1 byte, get pointer is moved 1 byte forward.
\item Essentially the get pointer points to the location in the file where next read is expected.
\item Similar to the get pointer, a `\textbf{put}' is used when writing to a file.
\item The put pointer keeps track where last write was performed. Normally it points to the end of file if data is being appended.
\item The get and put pointers can be manipulated to write/read data from any random location in file.
\end{itemize}
\begin{table}[H]
\begin{center}
\vspace{0.3cm}
	\begin{tabular}{lcc}
	\hline \hline
		\textbf{Type} \rule{0pt}{2.6ex} & \textbf{Get Pointer} & \textbf{Set Pointer}\\
		\hline
		Read \rule{0pt}{2.6ex} &  \verb|tellg()| & \verb|seekg()|\\
		Write \rule{0pt}{2.6ex} & \verb|tellp()| & \verb|seekp()|\\
	\hline \hline
	\end{tabular}
\end{center}
\label{GetandPutPointers}
\caption{Get and Put Pointers}
\end{table}
\begin{table}[H]
\begin{center}
\vspace{0.3cm}
	\begin{tabular}{lp{9cm}}
	\hline \hline
		\textbf{Syntax} \rule{0pt}{2.6ex} & \textbf{Description}\\
		\hline
		\verb|cout << (int) tellp();| \rule{0pt}{2.6ex} & Display the current location of put pointer\\
		\verb|seekg(ios::beg);| \rule{0pt}{2.6ex} & Set the location of get pointer at start of file\\
		\verb|seekp(0, ios::end);| \rule{0pt}{2.6ex} & Set the location of put pointer at end of file\\
		\verb|seekp(10);| \rule{0pt}{2.6ex} & Set the location of put pointer to 10$^{th}$ byte in file\\
	\hline \hline
	\end{tabular}
\end{center}
\label{ManipulatingGetandPutPointers}
\caption{Manipulating Get and Put Pointers}
\end{table}

\section{Exercise}
\noindent\textbf{Question No. 1 (Determine file size):}\\
\vspace{-0.5cm}\begin{enumerate}
\item Open your code file (.cpp) in `read' mode.
\item Move the \verb|seekg()| pointer to end of file.
\item Use \verb|tellg()| to display size of file.
\item \underline{Remember to reset the seek pointer (\texttt{seekg(0)}) if you want to read from file!}
\end{enumerate}
\noindent\textbf{Question No. 2 (Records in a text file):}\\
\vspace{-0.5cm}\begin{enumerate}
\item Given text file, \verb|txtrecords.txt| contains records of different items.
\item Each attribute of individual item is stored in a different line.
\item Count the number of lines in text file and determine number of items.
\item Display all the records.
\end{enumerate}
\noindent\textbf{Hint:} You can either count the number of lines with \verb|getline()| function or check for total number of \verb|'\n'| characters in file by reading char--by--char.
\begin{lstlisting}
struct Item
{
	char ItemName[20];
	int Quantity;
	float Price;
};
\end{lstlisting}
\noindent\textbf{Question No. 3 (Records in a binary file):}\\
\vspace{-0.5cm}\begin{enumerate}
\item Given binary file, \verb|binrecords.txt| contains records of different items.
\item Determine the total number of items in file.
\item Display all the records.
\item Ask user to change any record (for example, $3^{rd}$ record). Display the file after changing a record.
\end{enumerate}
\noindent\textbf{Hint:} Total number of records = filesize/sizeof(Item).
\end{document}
