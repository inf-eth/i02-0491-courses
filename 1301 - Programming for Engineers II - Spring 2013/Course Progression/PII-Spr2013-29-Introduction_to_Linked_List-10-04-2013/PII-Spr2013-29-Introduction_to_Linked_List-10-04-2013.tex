\documentclass[12pt,a4paper]{article}

% Margins.
\setlength{\oddsidemargin}{0in}
\setlength{\evensidemargin}{0in}
\setlength{\headheight}{12pt}
\setlength{\headsep}{42pt}
\setlength{\topmargin}{-54pt}
\setlength{\textwidth}{6.5in}
\setlength{\textheight}{10in}

\usepackage{amsmath}
\usepackage{float}
\usepackage{graphicx}
\usepackage[hyphens]{url}
\usepackage{hyperref}	% Clickable links to figures, references and urls.
\usepackage{enumerate}
\usepackage{datetime}

% Drawing.
\usepackage{pgf}
\usepackage{tikz}

% Listings for formatting code.
\usepackage{listings}
\usepackage{textcomp}
% General options.
\lstset{breaklines=true, basicstyle=\small\ttfamily, tabsize=4, numbers=left, stepnumber=1, frame=single, showstringspaces=false, upquote=true}
% C++ specific high-lighting. Comments are 50/50 shades of green/black and strings coloured with 60/40 red/black mixture.
\lstset{language=[ISO]C++, commentstyle=\color{green!50!black}, keywordstyle=\color{blue}, stringstyle=\color{red!60!black}}

%opening
\title{\vspace{-2cm}Programming for Engineers II\\Class 29\\Introduction to Linked List}
\author{Attique Dawood}
\date{April 10, 2013\\[0.2cm] Last Modified: \today, \currenttime}
\begin{document}
\maketitle
\section{Announcements}
\begin{itemize}
\item No Quiz this week due to sessional exams.
\end{itemize}
\section{Linked List}
\begin{itemize}
\item Storage problems with fragmented memory or hard disk.
\item Array requires contiguous space in memory. Array cannot dynamically grow incrementally. Must de--allocate and re--allocate whole array to change size.
\item Linked list solves the above problems. Doesn't require contiguous memory. Linked list is a chain of pointer--connected nodes. Nodes store information just like array indices.
\item Linked list access is slower than array. Must traverse whole list to reach a particular node. But, storage and size is flexible.
\item Array is preferable when speed is desired and storage size is known. Static data. Example: Number of students in a class cannot exceed 50. Create an array of 50 size for students.
\item Linked list is preferable when speed is not a major concern and storage size is highly dynamic. Example: Daily record of cars using motorway. On weekends traffic is high but there aren't many cars on weekdays. Numbers of cars can be highly varied.
\end{itemize}
\begin{lstlisting}[caption={Creating a Linked List}]
#include <iostream>
using namespace std;

class Node
{
	public:
	int Data;
	Node* Next;
};

class LinkedList
{
	private:
	Node* First;
	public:
	LinkedList(): First(NULL)
	{
	}
	void AddNode(int pData)
	{
		// If creating first node we need to change first pointer.
		if (First == NULL)
		{
			Node* NewNode;
			NewNode = new Node;
			NewNode->Data = pData;
			NewNode->Next = NULL;
			First = NewNode;
		}
		// We need to traverse the whole list to get to last node.
		else
		{
			// Find last node.
			Node* temp;
			temp = First;
			while (temp->Next != NULL)
				temp = temp->Next;
			// temp now points to last node.
			Node* NewNode;
			NewNode = new Node;
			temp->Next = NewNode;
			NewNode->Data = pData;
			NewNode->Next = NULL;
		}
	}
	void Display()
	{
		Node* temp;
		temp = First;
		while(temp != NULL)
		{
			cout << "Data = " << temp->Data << endl;
			temp = temp->Next;
		}
	}			
};

int main()
{
	LinkedList TestList;

	TestList.AddNode(-1);
	TestList.AddNode(2);
	TestList.AddNode(0);
	TestList.AddNode(7);
	
	TestList.Display();
	
	return 0;
}
\end{lstlisting}
\textbf{Question} You must traverse the whole list to add a new node at the end. By keeping a \verb|Last| pointer, that always points to last node in the list, you can add a node at the end without traversal. Add a \verb|Last| pointer to linked list class and modify the \verb|AddNode()| function to provide this functionality.\\
\textbf{Note: } This question is worth 1 absolute point. Only first correct submission will be accepted.
%\nocite{*}
%\bibliographystyle{plain}
%\bibliography{OOPref}
\end{document}
