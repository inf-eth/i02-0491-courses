\documentclass[12pt,a4paper]{article}

% Margins.
\setlength{\oddsidemargin}{0in}
\setlength{\evensidemargin}{0in}
\setlength{\headheight}{12pt}
\setlength{\headsep}{42pt}
\setlength{\topmargin}{-54pt}
\setlength{\textwidth}{6.5in}
\setlength{\textheight}{10in}

\usepackage{amsmath}
\usepackage{float}
\usepackage{graphicx}
\usepackage[hyphens]{url}
\usepackage{hyperref}	% Clickable links to figures, references and urls.
\usepackage{datetime}

% Drawing.
\usepackage{pgf}
\usepackage{tikz}

% Listings for formatting code.
\usepackage{listings}
\usepackage{textcomp}
% General options.
\lstset{breaklines=true, basicstyle=\small\ttfamily, tabsize=4, numbers=left, stepnumber=1, frame=single, showstringspaces=false, upquote=true}
% C++ specific high-lighting. Comments are 50/50 shades of green/black and strings coloured with 60/40 red/black mixture.
\lstset{language=[ISO]C++, commentstyle=\color{green!50!black}, keywordstyle=\color{blue}, stringstyle=\color{red!60!black}}

%opening
\title{\vspace{-2cm}Programming for Engineers II\\Class 15\\Sessional Exam Discussion - I}
\author{Attique Dawood}
\date{February 27, 2013\\[0.2cm] Last Modified: \today, \currenttime}
\begin{document}
\maketitle
\section{Announcements}
\begin{itemize}
\item None.
\end{itemize}
\section{A few Misconceptions}
Following points have been shamelessly copied from Sir Aamer Munir's lecture \# 17:\\
\noindent\textbf{Misconception \# 1:} Data hiding is achieved through making data members private!\\
\noindent\textbf{Reality:} By virtue of grouping data \& funcs into a class and avoiding global variables is how data hiding is achieved.\\
\noindent\textbf{Misconception \# 2:} OOP can handle real world ‘problems’!\\
\noindent\textbf{Reality:} OOP provides ease in ‘modeling’ real-world ‘object’ (not problems).\\
\noindent\textbf{Misconception \# 3:} Many features of programming taught in this semester are considered specific to OOP, like inline functions, const etc.\\
\noindent\textbf{Reality:} These are relevant to both SP and OOP. Many things were not taught in Prog-1. Not everything being taught now relates to OOP.\\
%\nocite{*}
\bibliographystyle{plain}
\bibliography{OOPref}
\end{document}
