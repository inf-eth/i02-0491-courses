\documentclass[12pt,a4paper]{article}

% Margins.
\setlength{\oddsidemargin}{0in}
\setlength{\evensidemargin}{0in}
\setlength{\headheight}{12pt}
\setlength{\headsep}{0pt}
\setlength{\topmargin}{-60pt}
\setlength{\textwidth}{6.5in}
\setlength{\textheight}{10.75in}

\usepackage{amsmath}
\usepackage{float}
\usepackage{graphicx}
\usepackage[hyphens]{url}
\usepackage{hyperref}	% Clickable links to figures, references and urls.
\usepackage{datetime}
\usepackage{longtable}

% Drawing.
\usepackage{pgf}
\usepackage{tikz}

% Listings for formatting code.
\usepackage{listings}
\usepackage{textcomp}
% General options.
\lstset{breaklines=true, basicstyle=\small\ttfamily, tabsize=4, numbers=left, stepnumber=1, frame=single, showstringspaces=false, upquote=true}
% C++ specific high-lighting. Comments are 50/50 shades of green/black and strings coloured with 60/40 red/black mixture.
\lstset{language=[ISO]C++, commentstyle=\color{green!50!black}, keywordstyle=\color{blue}, stringstyle=\color{red!60!black}}

%opening
\title{Data Structures and Algorithms\\Class 05\\ADT: List of Integers}
\author{Attique Dawood}
\date{February 17, 2015\\[0.2cm] Last Modified: \today, \currenttime}
\begin{document}
\maketitle
\section{Revision}
\begin{itemize}
\item Basic data type, abstract data type and data structure.
\end{itemize}
\section{ADT: List of Integers}
Following operations can be performed on a list of integers
\begin{itemize}
\item \verb|Insert(x,p)|: Insert $x$ at position/index $p$ in list. Elements starting at $p$ are shifted one position.
\item \verb|Locate(x)|: Searches $x$ in list; if found position $p$ of $x$ is returned.
\item \verb|Delete(p)|: Deletes the element at position $p$.
\item \verb|Retrieve(p)|: Returns the value at position $p$.
\item \verb|Next(p)|: Returns the value at position $p+1$.
\item \verb|Previous(p)|: Returns the value at position $p-1$.
\item \verb|MakeNull()|: Empties the list.
\item \verb|First()|: Returns the first element.
\item \verb|PrintList()|: Display the list.
\item \verb|Purge(x)|: Delete duplicate instances of value $x$.
\end{itemize}
\section{Exercises}
\noindent\textbf{Question 1:} If list of integers is implemented as an array then what is the growth order of operations?\\[0.2cm]
\noindent\textbf{Question 2:} Repeat Q1 for a linked--list implementation?
%\nocite{*}
\bibliographystyle{plain}
\bibliography{DSRef}
\end{document}
