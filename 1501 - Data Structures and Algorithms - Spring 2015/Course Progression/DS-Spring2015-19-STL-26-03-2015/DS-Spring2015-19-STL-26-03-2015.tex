\documentclass[12pt,a4paper]{article}

% Margins.
\setlength{\oddsidemargin}{0in}
\setlength{\evensidemargin}{0in}
\setlength{\headheight}{12pt}
\setlength{\headsep}{0pt}
\setlength{\topmargin}{-60pt}
\setlength{\textwidth}{6.5in}
\setlength{\textheight}{10.75in}

\usepackage{amsmath}
\usepackage{float}
\usepackage{graphicx}
\usepackage[hyphens]{url}
\usepackage{hyperref}	% Clickable links to figures, references and urls.
\usepackage{datetime}
\usepackage{longtable}

% Drawing.
\usepackage{pgf}
\usepackage{tikz}

% Listings for formatting code.
\usepackage{listings}
\usepackage{textcomp}
% General options.
\lstset{breaklines=true, basicstyle=\small\ttfamily, tabsize=4, numbers=left, stepnumber=1, frame=single, showstringspaces=false, upquote=true}
% C++ specific high-lighting. Comments are 50/50 shades of green/black and strings coloured with 60/40 red/black mixture.
\lstset{language=[ISO]C++, commentstyle=\color{green!50!black}, keywordstyle=\color{blue}, stringstyle=\color{red!60!black}}

%opening
\title{Data Structures and Algorithms\\Class 19\\Standard Template Library}
\author{Attique Dawood}
\date{March 30, 2015\\[0.2cm] Last Modified: \today, \currenttime}
\begin{document}
\maketitle
\section{Announcements}
\begin{itemize}
\item None.
\end{itemize}
\section{Revision}
\begin{itemize}
\item Templates.
\end{itemize}
\section{Standard Template Library (STL Containers)}
A container is a holder object that stores a collection of other objects (its elements). 
They are implemented as class templates, which allows a great flexibility in the types supported as elements.

The container manages the storage space for its elements and provides member functions to access them,
either directly or through iterators (reference objects with similar properties to pointers).

Containers replicate structures very commonly used in programming:
\begin{itemize}
\item[-] dynamic arrays (vector)
\item[-] linked lists (list)
\item[-] queues (queue)
\item[-] stacks (stack)
\item[-] heaps (priority\_queue)
\item[-] trees (set)
\item[-] associative arrays (map)
\end{itemize}
Many containers have several member functions in common, and share functionalities.
The decision of which type of container to use for a specific need does not generally depend only on the
functionality offered by the container, but also on the efficiency of some of its members (complexity).
This is especially true for sequence containers, which offer different trade-offs in complexity between
inserting/removing elements and accessing them.

stack, queue and priority\_queue are implemented as container adaptors.
Container adaptors are not full container classes, but classes that provide a specific interface relying
on an object of one of the container classes (such as deque or list) to handle the elements.
The underlying container is encapsulated in such a way that its elements are accessed by the members of the
container class independently of the underlying container class used.

This excerpt was taken from \cite{Ref:CPP-STL}
\section{Vector Usage}
- Header:
\begin{verbatim}
  #include <vector>
  using std::vector;
\end{verbatim}
- Create vector object:
\begin{verbatim}
  vector<int> intvector;
  vector<myclassname> myclassvector;
\end{verbatim}
- Initialization:
\begin{verbatim}
  intvector.resize (20, 0); // int vector of size 20, all values intialized to 0.
\end{verbatim}
- Get size: Total number of elements in vector.
\begin{verbatim}
  int mysize = intvector.size();
  cout << intvector.size(); 
\end{verbatim}
- Insertion:
\begin{lstlisting}
  Single insertion at back: intvector.push_back (3); // Creates a new integer, NOT the same as assignment. Size increases by one.
  Single or multiple insertions at a specific location:
  - intvector.insert (intvector.begin()+1, 1, 33); // Insert 33 at 2nd location.
  - intvector.insert (intvector.begin(), 3, 1); // Insert 3 1's at the beginning.
\end{lstlisting}
- Access:
\begin{lstlisting}
  Operator [] can be used to access entries:
  - int x = intvector[2]; // NOTE: This is NOT the same as insertion. Size remains same.
  - intvector[5] = 22;
\end{lstlisting}
- Deletion:
\begin{lstlisting}
  Delete a single entry: 
  - myvector.erase (myvector.begin()+i); // Delete ith entry. Size is reduced by one.
  Delete a range of entries: 
  - myvector.erase (myvector.begin()+1, myvector.begin()+5); // Delete elements in the range 2-6
  Delete vector:
  - myvector.clear ();
\end{lstlisting}

\section{Tasks}
\textbf{1. Create a dynamic array of integers using vector. Display contents after each step.}
\begin{itemize}
\item[-] Insert 10 random numbers.
\item[-] Display the contents.
\item[-] Delete first element.
\item[-] Delete 3rd , 4th and 5th elements.
\item[-] Insert 2 4's at beginning (index 0).
\item[-] Insert 4 -1's starting at second last location.
\item[-] Display final size.
\item[-] Delete vector.
\item[-] Resize the vector to 20 size initialized to -1's.
\item[-] Set values randomly. Size remains 20.
\end{itemize}
\textbf{2. Create dynamic 2D and 3D arrays using vectors.}
\begin{verbatim}
int rows = 3;
int cols = 5;
- vector<vector<int>> my2DArray (rows, vector<int>(cols)); // Declaration.
- Elements can be accessed using double subscript notation [][] in nested for loops.
\end{verbatim}
\subsection{Helpful Resources}
- If string input is being skipped, use the following before getline() statement.
\begin{verbatim}
  if (cin.peek() == '\n')
      cin.ignore();
  getline (cin, stringname, '\n');
\end{verbatim}
- getline() can be used to read from a txt file line by line.
\begin{verbatim}
  fstream file("abc.txt", ios::in);
  while (true)
  {
      getline(file, tempstring, '\n');
      if (file.eof())
          break;
      cout << "Read: " << tempstring << endl;
  }
  file.close();
\end{verbatim}
- To access the char* in string use stringname.c\_str();
\begin{verbatim}
  fstream file("abc.txt", ios::out);
  string message = "this is a mesage. Writing this to file.";
  file << message.c_str() << endl;
  file.close();
\end{verbatim}
%\section{Exercises}
%\noindent\textbf{Question:} ?\\[0.2cm]
%\nocite{*}
\bibliographystyle{plain}
\bibliography{DSRef}
\end{document}
