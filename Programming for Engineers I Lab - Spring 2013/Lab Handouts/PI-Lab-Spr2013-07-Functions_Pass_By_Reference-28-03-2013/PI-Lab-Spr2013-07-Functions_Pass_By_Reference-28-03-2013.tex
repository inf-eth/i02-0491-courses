\documentclass[12pt,a4paper]{article}

% Margins.
\setlength{\oddsidemargin}{0in}
\setlength{\evensidemargin}{0in}
\setlength{\headheight}{12pt}
\setlength{\headsep}{42pt}
\setlength{\topmargin}{-54pt}
\setlength{\textwidth}{6.5in}
\setlength{\textheight}{10in}

\usepackage{amsmath}
\usepackage{float}
\usepackage{graphicx}
\usepackage[hyphens]{url}
\usepackage{hyperref}	% Clickable links to figures, references and urls.

% Drawing.
\usepackage{pgf}
\usepackage{tikz}
\usepackage{amssymb}  % Tick mark
\usepackage{textcomp} % Cross

% Listings for formatting code.
\usepackage{listings}
\usepackage{textcomp}
% General options.
\lstset{breaklines=true, basicstyle=\small\ttfamily, tabsize=4, numbers=left, stepnumber=1, frame=single, showstringspaces=false, upquote=true}
% C++ specific high-lighting. Comments are 50/50 shades of green/black and strings coloured with 60/40 red/black mixture.
\lstset{language=[ISO]C++, commentstyle=\color{green!50!black}, keywordstyle=\color{blue}, stringstyle=\color{red!60!black}}

%opening
\title{Programming for Engineers I\\Lab 07\\Functions\\Pointers and Pass By Reference\\Dynamic Memory Allocation}
\author{Attique Dawood}

\begin{document}
\maketitle
\section{Exercise}
\textbf{Question No. 1:} Write a function with void return type that squares an input number by reference.
\begin{lstlisting}[caption={Square by reference}]
void Square(int&); // Prototype.
\end{lstlisting}
\textbf{Question No. 2:} Write a function with void return type that squares an input number using pointer.
\begin{lstlisting}[caption={Square using pointer}]
void Square(int*); // Prototype.
\end{lstlisting}
\textbf{Question No. 3:} Write a function with void return type that will square an input array. Assume size is fixed.
\begin{lstlisting}[caption={Square using array pointer}]
void SquareArray(int*, int); // Prototype. Array pointer and array size are arguments
int main()
{
	int Data[5] = {2, 3, 7, -1, 1};
	SquareArray(Data, 5);
	// Display squared array here.

	return 0;
}
void SquareArray(int* array, int size)
{
	// Write code to square array
}
\end{lstlisting}
\textbf{Question No. 4:} Repeat question no. 3 but array should be dynamically allocated. Take array size from user and initialise it with random numbers.
\end{document}
