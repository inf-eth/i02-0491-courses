\documentclass[12pt,a4paper]{article}

% Margins.
\setlength{\oddsidemargin}{0in}
\setlength{\evensidemargin}{0in}
\setlength{\headheight}{12pt}
\setlength{\headsep}{42pt}
\setlength{\topmargin}{-54pt}
\setlength{\textwidth}{6.5in}
\setlength{\textheight}{10in}

\usepackage{amsmath}
\usepackage{float}
\usepackage{graphicx}
\usepackage[hyphens]{url}
\usepackage{hyperref}	% Clickable links to figures, references and urls.

% Drawing.
\usepackage{pgf}
\usepackage{tikz}

% Listings for formatting code.
\usepackage{listings}
\usepackage{textcomp}
% General options.
\lstset{breaklines=true, basicstyle=\small\ttfamily, tabsize=4, numbers=left, stepnumber=1, frame=single, showstringspaces=false, upquote=true}
% C++ specific high-lighting. Comments are 50/50 shades of green/black and strings coloured with 60/40 red/black mixture.
\lstset{language=[ISO]C++, commentstyle=\color{green!50!black}, keywordstyle=\color{blue}, stringstyle=\color{red!60!black}}

%opening
\title{\vspace{-2cm}Programming for Engineers I\\Class 06\\Character Conversion and ASCII Table}
\author{Attique Dawood}

\begin{document}
\maketitle
\section{Revision}
\begin{itemize}
\item Conditional statements.
\item Variable types.
\item Signed and unsigned variables.
\item 2's complement notation.
\item Storing signed and unsigned integers in memory.
\end{itemize}
\section{Characters}
\begin{itemize}
\item ASCII table.
\item char input/output.
\item Usage of single quotes.
\item char and int conversions.
\end{itemize}
\section{Floating Point Number Representation}
To review floating point numbers and their conversion please take a look at \url{http://www.tfinley.net/notes/cps104/floating.html} or your class lectures/notes etc. Following example is taken from the above link.
\begin{lstlisting}
// Display char.
char x = 'A';
cout << x;

// Char to int conversion. Display ASCII code.
cout << (int)x;

// Int to char conversion. Display ASCII symbol of code.
int z = 65;
cout << (char)z;
\end{lstlisting}
\end{document}