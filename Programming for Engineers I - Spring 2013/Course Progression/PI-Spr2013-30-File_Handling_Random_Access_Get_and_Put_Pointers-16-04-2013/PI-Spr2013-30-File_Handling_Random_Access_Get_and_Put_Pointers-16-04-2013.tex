\documentclass[12pt,a4paper]{article}

% Margins.
\setlength{\oddsidemargin}{0in}
\setlength{\evensidemargin}{0in}
\setlength{\headheight}{12pt}
\setlength{\headsep}{42pt}
\setlength{\topmargin}{-54pt}
\setlength{\textwidth}{6.5in}
\setlength{\textheight}{10in}

\usepackage{amsmath}
\usepackage{float}
\usepackage{graphicx}
\usepackage[hyphens]{url}
\usepackage{hyperref}	% Clickable links to figures, references and urls.
\usepackage{datetime}

% Drawing.
\usepackage{pgf}
\usepackage{tikz}

% Listings for formatting code.
\usepackage{listings}
\usepackage{textcomp}
% General options.
\lstset{breaklines=true, basicstyle=\small\ttfamily, tabsize=4, numbers=left, stepnumber=1, frame=single, showstringspaces=false, upquote=true}
% C++ specific high-lighting. Comments are 50/50 shades of green/black and strings coloured with 60/40 red/black mixture.
\lstset{language=[ISO]C++, commentstyle=\color{green!50!black}, keywordstyle=\color{blue}, stringstyle=\color{red!60!black}}

%opening
\title{\vspace{-2cm}Programming for Engineers I\\Class 30\\File Handling\\Random Access: Get and Put Pointers}
\author{Attique Dawood}
\date{April 16, 2013\\[0.2cm] Last Modified: \today, \currenttime}
\begin{document}
\maketitle
\section{Announcements}
\begin{itemize}
\item Quiz on Friday 19-04-2013.
\end{itemize}
\section{Revision}
\begin{itemize}
\item Binary files: writing and reading.
\end{itemize}
\section{Writing to Binary File}
\begin{lstlisting}[caption={Writing a \texttt{struct} to binary file}]
#include <iostream>
#include <fstream>		// File handling library.
using namespace std;

struct StudentInfo
{
	char Name[20];
	int Age;
	float CGPA;
	char Gender;
};

int main()
{
	StudentInfo S;
	cout << "Enter student information: " << endl;
	cout << "Name: ";
	cin.getline(S.Name, 20);
	cout << "Age: ";
	cin >> S.Age;
	cout << "CGPA: ";
	cin >> S.CGPA;
	cout << "Gender (M/F): ";
	cin >> S.Gender;

	fstream OutFile;	// File handling object.

	// Open "abc.txt" for writing in binary mode (ios::out|ios::binary). This will overwrite existing file.
	OutFile.open("abc.txt", ios::out|ios::binary);
	OutFile.write((char*)&x, sizeof(int));
	OutFile.close();	// Close file.

	return 0;
}
\end{lstlisting}
\section{Reading from Binary File}
\begin{lstlisting}[caption={Reading from Binary file}]
#include <iostream>
#include <fstream>		// File handling library.
using namespace std;

int main()
{
	int x;
	double d;
	char c;

	fstream InFile;	// File handling object.

	// Open "abc.txt" for reading in binary mode (ios::in|ios::binary).
	InFile.open("abc.txt", ios::in|ios::binary);
	InFile.read((char*)&x, sizeof(int));
	InFile.read((char*)&d, sizeof(double));
	InFile.read((char*)&c, sizeof(char));
	InFile.close();	// Close file.

	cout << "x = " << x << endl;
	cout << "d = " << d << endl;
	cout << "c = " << c << endl;

	return 0;
}
\end{lstlisting}
\end{document}
