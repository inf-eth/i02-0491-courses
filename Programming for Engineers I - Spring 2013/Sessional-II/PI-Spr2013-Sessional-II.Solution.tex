\documentclass[12pt,a4paper]{article}

% Margins.
\setlength{\oddsidemargin}{0in}
\setlength{\evensidemargin}{0in}
\setlength{\headheight}{12pt}
\setlength{\headsep}{42pt}
\setlength{\topmargin}{-54pt}
\setlength{\textwidth}{6.5in}
\setlength{\textheight}{10in}
\pagestyle{plain}

\usepackage{amsmath}
\usepackage{float}
\usepackage{graphicx}
\usepackage[hyphens]{url}
\usepackage[hidelinks]{hyperref}	% Clickable links to figures, references and urls.
\usepackage{lastpage}
\usepackage{ulem}

% Drawing.
\usepackage{pgf}
\usepackage{tikz}

% Listings for formatting code.
\usepackage{listings}
\usepackage{textcomp}
% General options.
\lstset{breaklines=true, basicstyle=\footnotesize\ttfamily, tabsize=4, numbers=none, stepnumber=1, frame=single, showstringspaces=false, upquote=true}
% C++ specific high-lighting. Comments are 50/50 shades of green/black and strings coloured with 60/40 red/black mixture.
\lstset{language=[ISO]C++, commentstyle=\color{green!50!black}, keywordstyle=\color{blue}, stringstyle=\color{red!60!black}}

% Marks of each question.
\def\QOne{10}
\def\Qtwo{10}
\def\Qthree{20}
\def\Qfour{20}
\def\Qfive{20}
\def\Qsix{20}
\def\TotalMarks{100}

\begin{document}
\begin{minipage}{0.55\textwidth}
{\LARGE \textbf{Programming for\\ Engineers I}}\\[0.15cm]
{\normalsize \textbf{Spring 2013 Semester}}\\
{\Large \textbf{$2^{nd}$ Sessional Exam}}\\
{\normalsize \textbf{Saturday, April 06, 2013}}\\[0.30cm]
{\Large \textbf{Total Time: 60 minutes}}\\[0.15cm]
{\Large \textbf{Total Marks: 100}}\\
\textbf{Course Instructor:}\\
Atif Mughees\\
Attique Dawood\\
\end{minipage}
\begin{minipage}{0.4\textwidth}
\textbf{Serial} \hrulefill \\[0.25cm]
\textbf{Name} \hrulefill\\[0.25cm]
\textbf{Section} \rule{1cm}{0.2mm} \textbf{Roll No:} \hrulefill\\[0.25cm]
\textbf{Signature:} \hrulefill\\[0.25cm]
\rule{6.6cm}{0.2mm}\\
\textbf{Signature of Invigilator}\\[0.25cm]
\end{minipage}
\begin{table}[H]
\begin{center}
\vspace{0.3cm}
	{\Large \begin{tabular}{|l|c|c|c|c|c|c|c|}
	\hline
		\rule{0pt}{2.6ex} Question & \textbf{1} & \textbf{2} & \textbf{3} & \textbf{4} & \textbf{5} & \textbf{6} & \textbf{Total}\\
		\hline
		Total Marks \rule{0pt}{2.6ex} & \QOne & \Qtwo & \Qthree & \Qfour & \Qfive & \Qsix & \TotalMarks\\
		\hline
		Marks Obtained & & & & & & &\\
	\hline
	\end{tabular}}
\end{center}
\end{table}
\noindent \textbf{You are advised to READ these notes:}
\begin{enumerate}
\item \textbf{Attempt on the Question Paper. \underline{NO EXTRA SHEET} will be provided/accepted. No
additional sheet will be provided for rough work. Use the back of the page where
provided space is not sufficient.}
\item After asked to commence the exam, please verify that you have \textbf{\pageref{LastPage} different
printed pages} including this title page.
\item There are 6 questions. Attempt all of them. It is advisable to go through the paper once
before starting with the first question.
\item Exam is closed books, closed notes. Please see that the area in your threshold is clean.
You will be charged for any material which can be classified as \textbf{`helping in the paper'}
found near you.
\item \textbf{Calculator sharing is strictly prohibited.}
\item Students who attempt the paper with lead pencils lose the right to get them rechecked.
\item \textbf{The invigilator present is not supposed to answer any questions. No one may come
to your room for corrections and you are not supposed to request to call anyone.
Make assumptions wherever required and clearly mark them.}
\end{enumerate}
\newpage
\noindent\textbf{Question 1: Code correction\hfill \QOne~marks}\\
Find errors in the given code and suggest corrections. Errors may be logical or in syntax.
\begin{lstlisting}
#include <iostream>
using namespace std;

void Sum(int a, int b)
{
	return (a+b);
}

int main()
{
	const int z;
	z = 5;

	int size;
	cin >> size;
	int arr[size];

	char name[3] = "abc";
	cout << name << endl;

	int x = 3;
	int y = 5;
	cout << "Sum of x and y = " << Sum(int x, int y) << endl;

	return 0;
}
\end{lstlisting}
\begin{lstlisting}[escapechar=!]
#include <iostream>
using namespace std;

!\textcolor{red}{int}! Sum(int a, int b)
{
	return (a+b);
}

int main()
{
	const int z !\textcolor{red}{= 5}!;
	!\textcolor{red}{\sout{z = 5;}}!

	int size;
	cin >> size;
	!\textcolor{red}{int* arr = new int[size];}!

	char name[!\textcolor{red}{4}!] = "abc";
	cout << name << endl;

	int x = 3;
	int y = 5;
	cout << "Sum of x and y = " << Sum(!\textcolor{red}{\sout{int}}! x, !\textcolor{red}{\sout{int}}! y) << endl;

	return 0;
}
\end{lstlisting}
\newpage
\noindent\textbf{Question 2: Integer Arrays \hfill \Qtwo~marks}\\
Write code to create an integer array of size 10. Fill this array with random numbers in the range 0 to 5. Display the array.
\begin{lstlisting}[escapechar=!]
#include <iostream>
#include <cstdlib>
using namespace std;

int main()
{
	const int size = 10;
	int Array[size];

	// Filling array with random numbers.
	for (int i=0; i<size; i++)
	{
		Array[i] = rand() % 6;
	}
	
	// Display.
	for (int i=0; i<size; i++)
	{
		cout << Array[i] << " ";
	}
	cout << endl;
	
	return 0;
}
\end{lstlisting}

\noindent\textbf{Question 3: Sorting \hfill \Qthree~marks}\\
Given an integer array of size 10 filled with random numbers, write code to sort this array in descending order. You can use bubble sort OR insertion sort.\\
\verb|int Data[10] = {-1, 8, 2, 11, 7, 3, 5, 17, 0, 9};|
\begin{lstlisting}[escapechar=!]
int main()
{
	int Data[10] = {-1, 8, 2, 11, 7, 3, 5, 17, 0, 9};

	for (int c=0; c<10; c++)
	{
		int currentsize = 10-c;
		for (int i=0; i<curentsize-1; i++)
		{
			if (Data[i+1] > Data[i])
			{
				int temp = Data[i];
				Data[i] = Data[i+1];
				Data[i+1] = temp;
			}
		}
	}

	return 0;
}
\end{lstlisting}
\newpage
\noindent\textbf{Question 4: C--Strings \hfill \Qfour~marks}\\
Take a 2D char array to store 3 names. Names can be a maximum of 20 characters long including NULL. Names should be input by user. After this, user will input a name to search for. If the name is found, display an appropriate message. You can use any library functions.
\begin{lstlisting}[escapechar=!]
#include <iostream>
#include <cstring>
using namespace std;

int main()
{
	char Names[3][20];
	for (int i=0; i<3; i++)
	{
		cout << "Enter name: ";
		cin.getline(Names[i], 20);
	}
	char Temp[20;
	cout << "Enter a name to search: ";
	cin.getline(Temp, 20)
	
	for (int i=0; i<3; i++)
	{
		if (strcmp(Temp, Names[i]) == 0)
		{
			cout << "Matched..." << endl;
			return 0;
		}
	}
	cout << "No match found." <<endl;
	
	return 0;
}
\end{lstlisting}
\newpage
\noindent\textbf{Question 5: Dynamic Memory Allocation \hfill $4\times 5=$\Qfive~marks}\\
\begin{enumerate}
\item Write code for a function to square an integer array of given size.\\
\verb|void SquareArray(int* Data, int size)|
\item In \verb|main()|, create a dynamic integer array. Take size of array as input from user.
\item Using a loop, fill this array with increasing real numbers starting from 0. For example, an array of size 6 will contain \verb|{0, 1, 2, 3, 4, 5}|.
\item Square the array using \verb|SquareArray| function and display it in \verb|main()|.
\item De--allocate array memory before exiting \verb|main()|.
\end{enumerate}
%\begin{figure}[H]
%\begin{tikzpicture}
%	\draw[thick] (0cm,0cm) rectangle (\textwidth, 18cm);
%\end{tikzpicture}
%\end{figure}
\begin{lstlisting}
#include <iostream>
using namespace std;
// 1. Function to square array.
void SquareArray(int* Data, int Size)
{
	for (int i=0; i<Size; i++)
	{
		Data[i] = Data[i] * Data[i];
	}
}
int main()
{
	// 2. Dynamically create array of given size.
	int* Array;
	int s;

	cout << "Enter size: ";
	cin >> s;
	Array = new int[s];
	
	// 3. Fill array.
	for (int i=0; i<s; i++)
	{
		Array[i] = i;
	}

	// 4a. Function call.
	SquareArray(Array, s);

	// 4b. Loop to display squared array.
	for (int i=0; i<s; i++)
	{
		cout << Array[i] << " ";
	}
	cout << endl;
	
	// 5. De-allocation.
	delete[] Array;

	return 0;
}
\end{lstlisting}
\newpage
\noindent\textbf{Question 6: \texttt{struct} \hfill \Qsix~marks}\\
\begin{enumerate}
\item Create a \texttt{struct} \verb|StudentInfo| to store student record. Student attributes are name, age, CGPA and gender.
\item Create two functions \verb|void Input(StudentInfo& S)| and\\ \verb|void Display(const StudentInfo& S)| to input and display student information.
\item In \verb|main()| create two \verb|StudentInfo| variables and then input/display their records.
\end{enumerate}
\begin{lstlisting}
#include <iostream>
using namespace std;

// StudentInfo definition
struct StudentInfo
{
	char name[20];
	int Age;
	float CGPA;
	char Gender;
};
void Input(StudentInfo& S)
{
	cin.getline(S.name, 20);
	cin >> S.Age;
	cin >> S.CPGPA;
	cin >> S.Gender;
}
void Display(const StudentInfo& S)
{
	switch (S.Gender)
	{
		case 'M':
			cout << "Name: Mr. " << S.name << endl;
			break;
		case 'F':
			cout << "Name: Ms. " << S.name << endl;
			break;
	}
	cout << "Age: " << S.Age << endl;
	cout << "CGPA: " << S.CGPA << endl;
}
int main()
{
	// Create StudentInfo variables.
	StudentInfo S1;
	StudentInfo s2;

	// Input.
	Input(S1);
	Input(S2);
	
	// Display
	Display(S1);
	Display(S2);

	return 0;
}
\end{lstlisting}
%\begin{figure}[H]
%\begin{tikzpicture}
%	\draw[thick] (0cm,0cm) rectangle (\textwidth, 22.5cm);
%\end{tikzpicture}
%\end{figure}
\end{document}