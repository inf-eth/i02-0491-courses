\documentclass[12pt,a4paper]{article}

% Margins.
\setlength{\oddsidemargin}{0in}
\setlength{\evensidemargin}{0in}
\setlength{\headheight}{12pt}
\setlength{\headsep}{42pt}
\setlength{\topmargin}{-40pt}
\setlength{\textwidth}{6.5in}
\setlength{\textheight}{10in}
\pagestyle{empty}

\usepackage{amsmath}
\usepackage{float}
\usepackage{graphicx}
\usepackage[hyphens]{url}
\usepackage[hidelinks]{hyperref}	% Clickable links to figures, references and urls.
\usepackage{lastpage}

% Drawing.
\usepackage{pgf}
\usepackage{tikz}

% Listings for formatting code.
\usepackage{listings}
\usepackage{textcomp}
% General options.
\lstset{breaklines=true, basicstyle=\footnotesize\ttfamily, tabsize=4, numbers=none, stepnumber=1, frame=single, showstringspaces=false, upquote=true}
% C++ specific high-lighting. Comments are 50/50 shades of green/black and strings coloured with 60/40 red/black mixture.
\lstset{language=[ISO]C++, commentstyle=\color{green!50!black}, keywordstyle=\color{blue}, stringstyle=\color{red!60!black}}

% Marks of each question.
\def\QOne{10}
\def\Qtwo{10}
\def\Qthree{10}
\def\Qfour{10}
\def\Qfive{10}
\def\TotalMarks{50}

\begin{document}
\begin{minipage}{0.55\textwidth}
{\LARGE \textbf{Programming for Engineers}}\\[0.15cm]
{\normalsize \textbf{Spring 2014 Semester}}\\
{\Large \textbf{$1^{st}$ Sessional Exam}}\\
{\normalsize \textbf{Saturday, February 22, 2014}}\\[0.30cm]
{\Large \textbf{Total Time: 60 minutes}}\\[0.15cm]
{\Large \textbf{Total Marks: 50}}\\
\textbf{Course Instructors:}\\
Attique Dawood\\
\end{minipage}
\begin{minipage}{0.4\textwidth}
\textbf{Serial} \hrulefill \\[0.25cm]
\textbf{Name} \hrulefill\\[0.25cm]
\textbf{Section} \rule{1cm}{0.2mm} \textbf{Roll No:} \hrulefill\\[0.25cm]
\textbf{Signature:} \hrulefill\\[0.25cm]
\rule{6.6cm}{0.2mm}\\
\textbf{Signature of Invigilator}\\[0.25cm]
\end{minipage}
\begin{table}[H]
\begin{center}
\vspace{0.3cm}
	{\large \begin{tabular}{|l|c|c|c|c|c|c|}
	\hline
		\rule{0pt}{2.6ex} Question & \textbf{1} & \textbf{2} & \textbf{3} & \textbf{4} & \textbf{5} & \textbf{Total}\\
		\hline
		Marks Obtained \rule{0pt}{2.6ex} & & & & & &\\
		\hline
		Total Marks \rule{0pt}{2.6ex} & \QOne & \Qtwo & \Qthree & \Qfour & \Qfive & \TotalMarks\\
	\hline
	\end{tabular}}
\end{center}
\end{table}
\noindent \textbf{You are advised to READ these notes:}
\begin{enumerate}
\item \textbf{Attempt on the Question Paper. \underline{NO EXTRA SHEET} will be provided/accepted. No
additional sheet will be provided for rough work. Use the back of the page where
provided space is not sufficient.}
\item After asked to commence the exam, please verify that you have \textbf{\pageref{LastPage} different
printed pages} including this title page.
\item There are 5 questions. Attempt all of them. It is advisable to go through the paper once
before starting with the first question.
\item Exam is closed books, closed notes. Please see that the area in your threshold is clean.
You will be charged for any material which can be classified as \textbf{`helping in the paper'}
found near you.
\item \textbf{Calculator sharing is strictly prohibited.}
\item Students who attempt the paper with lead pencils lose the right to get them rechecked.
\item \textbf{The invigilator present is not supposed to answer any questions. No one may come
to your room for corrections and you are not supposed to request to call anyone.
Make assumptions wherever required and clearly mark them.}
\end{enumerate}
\newpage

\noindent\textbf{Question 1: Short Answers \hfill $5\times 4=$\QOne~marks}\\
Answer the following in 1--4 lines.
\begin{enumerate}
\item[a.] What is new in object--oriented programming?
\begin{figure}[H]
\begin{tikzpicture}
	\draw[thick] (0cm,0cm) rectangle (\textwidth, 3.5cm);
\end{tikzpicture}
\end{figure}
\item[b.] What is the difference between class and object?
\begin{figure}[H]
\begin{tikzpicture}
	\draw[thick] (0cm,0cm) rectangle (\textwidth, 3.5cm);
\end{tikzpicture}
\end{figure}
\item[c.] What is member initialiser list?
\begin{figure}[H]
\begin{tikzpicture}
	\draw[thick] (0cm,0cm) rectangle (\textwidth, 3.5cm);
\end{tikzpicture}
\end{figure}
\item[d.] What is scope resolution operator.
\begin{figure}[H]
\begin{tikzpicture}
	\draw[thick] (0cm,0cm) rectangle (\textwidth, 3.5cm);
\end{tikzpicture}
\end{figure}
\end{enumerate}
\newpage

\noindent\textbf{Question 2: Code correction\hfill \Qtwo~marks}\\
Find errors in the given code and suggest corrections. Errors may be logical or in syntax.
\begin{lstlisting}
#include <iostream>
using namespace std;
class Circle
{
	private:
	float radius;
	float area = 0.f;
	const float pi;
	public:
	// Default constructor.
	void Circle(): radius(0.f), pi(3.14f)
	{
	}
	// User-defined constructor.
	Circle(float r)
	{
		radius = r;
		CaclulateArea();
	}
	void CalculateArea()
	{
		area = 2*pi*r;
	}
	void Display()
	{
		cout << "Radius = " << radius << endl;
		cout << "Area = " << area << endl;
	}
};
\end{lstlisting}
\newpage

\noindent\textbf{Question 3:Makefile\hfill \Qthree~marks}\\
You have a project with following files: \verb|complex.hpp|, \verb|complex.cpp|, \verb|matrix.hpp|, \verb|matrix.cpp| and \verb|main.cpp|. Write \verb|Makefile| for this project. Ensure that individual object files are created first and then linked together so that build process is efficient.
\newpage

\noindent\textbf{Question 4: Constructors/Destructors \hfill \Qfour~marks}\\
Write the output of following program. Mention object name with constructor/destructor calls.
\begin{lstlisting}
#include <iostream>
using namespace std;
class test
{
	public:
	test() { cout << "DC" << endl; }
	test(int x) { cout << "UDC" << endl; }
	test(const test& t) { cout << "CC" << endl; }
	~test() { cout << "D" << endl; }
};
void func1(test t)
{
}
void func2(test& t)
{
}
int main()
{
	test A, B(3);
	test C = B;
	func1(C);
	func2(A);
	test* D = new test;
	test* E = new test;
	delete D;
	test* F;
	return 0;
}
\end{lstlisting}
\newpage

\noindent\textbf{Question 5: Point Class \hfill \Qfive~marks}\\
You are required to make a Point class to store x and y coordinates. Define the Point class so that code given in main() runs without any errors.
\begin{lstlisting}[escapechar=!]
#include <iostream>
using namespace std;
class Point
{
	private:
	int x;
	int y;
	public:




























};

int main()
{
	Point A(-2, 3); // x=-2, y=3
	Point B(4, 7); // x=4, y=7
	Point C; // x=0, y=0
	C = A + B;
	C.display(); // OUTPUT: x = 2, y = 10;

	return 0;
}
\end{lstlisting}
\end{document}