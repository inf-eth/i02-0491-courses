\documentclass[12pt,a4paper]{article}

% Margins.
\setlength{\oddsidemargin}{0in}
\setlength{\evensidemargin}{0in}
\setlength{\headheight}{12pt}
\setlength{\headsep}{42pt}
\setlength{\topmargin}{-54pt}
\setlength{\textwidth}{6.5in}
\setlength{\textheight}{10in}

\usepackage{amsmath}
\usepackage{float}
\usepackage{graphicx}
\usepackage[hyphens]{url}
\usepackage{hyperref}	% Clickable links to figures, references and urls.
\usepackage{datetime}

% Drawing.
\usepackage{pgf}
\usepackage{tikz}

% Listings for formatting code.
\usepackage{listings}
\usepackage{textcomp}
% General options.
\lstset{breaklines=true, basicstyle=\small\ttfamily, tabsize=4, numbers=left, stepnumber=1, frame=single, showstringspaces=false, upquote=true}
% C++ specific high-lighting. Comments are 50/50 shades of green/black and strings coloured with 60/40 red/black mixture.
\lstset{language=[ISO]C++, commentstyle=\color{green!50!black}, keywordstyle=\color{blue}, stringstyle=\color{red!60!black}}

%opening
\title{\vspace{-3cm}Programming for Engineers II\\Assignment 02\\Matrix Class}
\author{Attique Dawood}
\date{February 06, 2014\\Due: February 13, 2014\\[0.2cm] Last Modified: \today, \currenttime}
\begin{document}
\maketitle
\begin{enumerate}
\item Create a \verb|Matrix| class to add, subtract and multiply matrices of complex numbers. Additionally, create \verb|Input()| and \verb|Display()| functions for the \verb|Matrix| class.
\item \verb|Matrix| data should be a fixed 2D array. There are additional marks for variable \verb|Matrix| size (rows and columns) that can be input by user. However, you \textbf{cannot} use dynamic allocation (\verb|new| or \verb|malloc()|). One way of achieving this is to keep a 2D array of some maximum dimensions like 100x100. If user wants to work with 2x2 matrix then only use the 2x2 elements from actual 2D array.
\item It is recommended that you first work with fixed matrix sizes of \verb|float| type and then convert this into matrix of complex numbers.
\item You can use the \verb|Complex| class from previous assignment.
\item Submit a hand--written UML diagram clearly showing the relationship between both classes as well as their attributes and functions.
\item Your program should contain separate files for class definition, class implementation and main function. This assignment should contain six appropriately named files.
\item \verb|main()| should create three matrices and demonstrate matrix operations.
\item Provide a Makefile with your program.
\end{enumerate}
\end{document}
