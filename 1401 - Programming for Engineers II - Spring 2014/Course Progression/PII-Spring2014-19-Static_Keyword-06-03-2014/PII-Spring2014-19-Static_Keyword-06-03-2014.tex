\documentclass[12pt,a4paper]{article}

% Margins.
\setlength{\oddsidemargin}{0in}
\setlength{\evensidemargin}{0in}
\setlength{\headheight}{12pt}
\setlength{\headsep}{42pt}
\setlength{\topmargin}{-54pt}
\setlength{\textwidth}{6.5in}
\setlength{\textheight}{10in}

\usepackage{amsmath}
\usepackage{float}
\usepackage{graphicx}
\usepackage[hyphens]{url}
\usepackage{hyperref}	% Clickable links to figures, references and urls.
\usepackage{enumerate}
\usepackage{datetime}

% Drawing.
\usepackage{pgf}
\usepackage{tikz}

% Listings for formatting code.
\usepackage{listings}
\usepackage{textcomp}
% General options.
\lstset{breaklines=true, basicstyle=\small\ttfamily, tabsize=4, numbers=left, stepnumber=1, frame=single, showstringspaces=false, upquote=true}
% C++ specific high-lighting. Comments are 50/50 shades of green/black and strings coloured with 60/40 red/black mixture.
\lstset{language=[ISO]C++, commentstyle=\color{green!50!black}, keywordstyle=\color{blue}, stringstyle=\color{red!60!black}}

%opening
\title{\vspace{-2cm}Programming for Engineers II\\Class 19\\Static Data Members}
\author{Attique Dawood}
\date{March 06, 2014\\[0.2cm] Last Modified: \today, \currenttime}
\begin{document}
\maketitle
\section{Announcements}
\begin{itemize}
\item Quiz next week (Monday, 10--03--2014).
\end{itemize}
\section{Revision}
\begin{itemize}
\item Data conversion.
\item To do \verb|A=B| we would ideally want to overload assignment operator for class A that takes object of class B as argument, \verb|void operator=(B)|.
\item To do \verb|B=A| we would ideally want to overload assignment operator for class B that takes object of class A as argument, \verb|void operator=(A)|.
\item If we cannot change class B code and still want to do \verb|B=A| we'll define \verb|operator B()| in class A.
\end{itemize}
\section{Variable Scope: Global and Local Variables}
\begin{itemize}
\item \underline{Variable classification on the basis of `scope'.}
\item Global variables are accessible everywhere in a single file, in class, any function and main. Their scope is `global'.
\item Local variables are limited to the structure in which they are declared.
\item Global variables exist for the entirety of the program and will only be destroyed when program exits.
\item Local variables will only exist for the duration of the function in which they are declared. They will be destroyed when function ends.
\end{itemize}
\begin{lstlisting}[caption={Variable Scope}]
#include <iostream>
using namespace std;

int x; // Global variable

class Simple
{
	int y; // y is local variable for Simple class.
	public:
	// Default constructor.
	Simple()
	{
		cout << "Value of x in constructor: " << x << endl;
		x = 100;
		y = 0;
		cout << "Value of x after constructor call: " << x << endl;
	}
	~Simple()
	{
		x = -1;
		cout << "Value of x after destructor call: " << x << endl;
	}
};
void SimpleFunction()
{
	Simple functionobj; // functionobj is a local object of SimpleFunction().
	x = 1;
}
int main()
{
	x = 33;
	Simple* obj = new Simple;

	SimpleFunction();

	delete obj;

	for (int i=0; i<3; i++)
	{
		int j;
	}
	cout << i << j << endl; // Compilation error: i and j are local to for loop.

	cout << "main() is exiting now..." << endl;
	return 0;
}
\end{lstlisting}
\section{\texttt{static} Variables}
\begin{itemize}
\item \textbf{WARNING!} We have loosely used the term `static' for non-dynamic variables. In the context of C/C++ as a language \verb|static| actually means something else.
\item Variables declared \verb|static| are only destroyed when the program ends.
\item Even if declared in a function, \verb|static| variables will continue to exist and take up memory until the end of program.
\item \verb|static| variable declaration/creation will only happen once.
\item There is no point making global variables \verb|static| because they already function as if they were \verb|static|... but nobody is stopping you from doing so.
\item One possible use of using static variables in a function is when you're passing a pointer or reference to a local variable. If the local variable isn't static it's pointer or reference is no longer useful outside the function.
\end{itemize}
\begin{minipage}{7.5cm}
\begin{lstlisting}[caption={Local Variable in a Function}]
#include <iostream>
using namespace std;

int Simple()
{
	int x = 0;
	x++;
	return x;
}
int main()
{
	cout << x << endl;

	cout << Simple() << endl;
	cout << Simple() << endl;
	cout << Simple() << endl;
	
	return 0;
}
\end{lstlisting}
\end{minipage}
\hspace*{1cm}
\begin{minipage}{7.5cm}
\begin{lstlisting}[caption={\texttt{static} Local Variable in a Function}]
#include <iostream>
using namespace std;

int Simple()
{
	static int x = 0;
	x++;
	return x;
}
int main()
{
	cout << x << endl;

	cout << Simple() << endl;
	cout << Simple() << endl;
	cout << Simple() << endl;
	
	return 0;
}
\end{lstlisting}
\end{minipage}
\section{\texttt{static} Data Member}
\begin{itemize}
\item A \verb|static| data member of class has only one instance. It belongs to the class rather than any object.
\item \verb|static| members are not associated with object but with the class.
\item Even there aren't any objects, \verb|static| data member will continue to exist.
\item One use of \verb|static| is to provide a common variable that can be accessed by objects. Communication?
\end{itemize}
\begin{lstlisting}[caption={\texttt{static} Data Member}]
#include <iostream>
using namespace std;

class Complex
{
	private:
	static int count; // Declaration.
	public:
	Complex()
	{
		count++;
	}
	int GetCount()
	{
		return count;
	}
	~Complex()
	{
		count--;
	}
};
int Complex::count = 0; // Initialisation (think of definition).

int main()
{
	Complex A, B, C;
	cout << A.GetCount() << endl;
	Complex* Ap;
	Ap = new Complex;
	cout << A.GetCount() << endl;
	delete Ap;
	cout << A.GetCount() << endl;
	
	return 0;
}
\end{lstlisting}
\section{\texttt{static} Functions}
\begin{itemize}
\item \verb|static| functions can only access \verb|static| data members. Why?
\item \verb|static| functions can be called even if objects don't exist.
\item If objects don't exist, \verb|static| variables still exist!
\end{itemize}
\begin{lstlisting}[caption={\texttt{static} Member Function}]
#include <iostream>
using namespace std;

class Complex
{
	private:
	static int count; // Declaration.
	public:
	Complex()
	{
		count++;
	}
	static int GetCount()
	{
		return count;
	}
	~Complex()
	{
		count--;
	}
};
int Complex::count = 0; // Initialisation (think of definition).

int main()
{
	Complex A, B, C;
	cout << Complex::GetCount() << endl;
	Complex* Ap;
	Ap = new Complex;
	cout << Complex::GetCount() << endl;
	delete Ap;
	cout << Complex::GetCount() << endl;
	
	return 0;
}
\end{lstlisting}
%\nocite{*}
\bibliographystyle{plain}
\bibliography{OOPref}
\end{document}
