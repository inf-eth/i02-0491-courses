\documentclass[12pt,a4paper]{article}

% Margins.
\setlength{\oddsidemargin}{0in}
\setlength{\evensidemargin}{0in}
\setlength{\headheight}{12pt}
\setlength{\headsep}{42pt}
\setlength{\topmargin}{-54pt}
\setlength{\textwidth}{6.5in}
\setlength{\textheight}{10in}

\usepackage{amsmath}
\usepackage{float}
\usepackage{graphicx}
\usepackage[hyphens]{url}
\usepackage{hyperref}	% Clickable links to figures, references and urls.
\usepackage{enumerate}
\usepackage{datetime}

% Drawing.
\usepackage{pgf}
\usepackage{tikz}

% Listings for formatting code.
\usepackage{listings}
\usepackage{textcomp}
% General options.
\lstset{breaklines=true, basicstyle=\small\ttfamily, tabsize=4, numbers=left, stepnumber=1, frame=single, showstringspaces=false, upquote=true}
% C++ specific high-lighting. Comments are 50/50 shades of green/black and strings coloured with 60/40 red/black mixture.
\lstset{language=[ISO]C++, commentstyle=\color{green!50!black}, keywordstyle=\color{blue}, stringstyle=\color{red!60!black}}

%opening
\title{\vspace{-2cm}Programming for Engineers II\\Class 36\\File Handling\\Text and Binary Formats\\Reading and Writing}
\author{Attique Dawood}
\date{April 17, 2014\\[0.2cm] Last Modified: \today, \currenttime}
\begin{document}
\maketitle
\section{Announcements}
\begin{itemize}
\item None.
\end{itemize}
\section{File Handling in C++}
\begin{itemize}
\item In C++ file handling library is \verb|<fstream>|.
\item You create `objects' of type (or class) \verb|fstream| and use them for reading to and writing from files.
\item Text files can be manipulated using \verb|<<| and \verb|>>| operators exactly like \verb|cout/cin|.
\end{itemize}
\section{Text and Binary File Formats}
\begin{itemize}
\item The number -3.7182193 can be either a double or float depending on data type being used.
\item For example \verb|float f = -3.7182193;| is float and takes 4 bytes for storage in memory. These 4 bytes contain the binary representation of float.
\item If stored as a string, it is 10 characters long including the minus sign and decimal point: \verb|char c[] = "-3.7182193";|. Note: \verb|NULL| is implicit and doesn't count towards length of string.
\item Take another example of an integer \verb|int x = 1;|. If stored as a string of characters, it will take only 1 byte. If stored as an integer it will take 4 bytes, regardless of the value.
\item In a text file, data is stored as a string of characters. It's a copy of what you see on your console. As can be expected, a text file is easily readable if opened in a text editor.
\item In a binary file, data is stored as it would in a memory. It's a copy of what is in the memory. A binary file, just like memory, isn't so obviously readable in raw form.
\end{itemize}
\section{File Modes}
Modes are options that can be specified when opening a file. Options are specified as the second argument of \verb|open()| function. Options can be combined with the `OR' operator \verb\'|'\. A few examples are given below:
\begin{table}[H]
\begin{center}
\vspace{0.3cm}
	\begin{tabular}{lp{9cm}}
	\hline \hline
		\textbf{Mode} \rule{0pt}{2.6ex} & \textbf{Description}\\
		\hline
		\verb|ios::in| \rule{0pt}{2.6ex} & Read mode\\
		\verb|ios::out| \rule{0pt}{2.6ex} & Write mode\\
		\verb|ios::binary| \rule{0pt}{2.6ex} & Binary mode operation\\
		\verb|ios::app| \rule{0pt}{2.6ex} & Append mode\\
		\verb|File.open("abc.txt", ios::in);| \rule{0pt}{2.6ex} & Open file for reading. File must exist.\\
		\verb|File.open("abc.txt", ios::out);| \rule{0pt}{2.6ex} & Open file for writing. File will be created if it doesn't exist. Existing file will be overwritten.\\
		\verb\..., ios::out|ios::app);\ \rule{0pt}{2.6ex} & Open file for writing. File will be created if it doesn't exist. Existing data will be preserved. Data will be written at the end of file.\\
		\verb\..., ios::out|ios::binary);\ \rule{0pt}{2.6ex} & Open file for writing in binary mode. File will be created if it doesn't exist. Existing file will be overwritten.\\
		\verb\ios::out|ios::binary|ios::app\ \rule{0pt}{2.6ex} & Open file for writing in binary mode. File will be created if it doesn't exist. Existing data will be preserved. Data will be written at the end of file.\\
	\hline \hline
	\end{tabular}
\end{center}
\label{FileModes}
\caption{File modes}
\end{table}
\section{Writing to Text File}
\begin{lstlisting}[caption={Writing to text file}]
#include <iostream>
#include <fstream>		// File handling library.
using namespace std;

int main()
{
	fstream OutFile;	// File handling object.

	// Open "abc.txt" for writing (ios::out). This will overwrite existing file.
	OutFile.open("abc.txt", ios::out);
	OutFile << "Hello world?" << endl;
	OutFile << "Some numbers: 2 3 7 11 13.33 -2.448" << endl;
	OutFile << "Grades: A A- B+ D C- F A+" << endl;
	OutFile.close();	// Close file.

	return 0;
}
\end{lstlisting}
\section{Reading from Text File}
\begin{lstlisting}[caption={Reading from text file (Creating file)}]
#include <iostream>
#include <fstream>		// File handling library.
using namespace std;

int main()
{
	int x = 3;
	float f = -2.348193f;
	char c = '^';

	fstream OutFile;	// File handling object.

	// Open "abc.txt" for writing (ios::out). This will overwrite existing file.
	OutFile.open("abc.txt", ios::out);
	OutFile << x << " " << f << " " << c << " " << endl;
	OutFile.close();	// Close file.
	
	return 0;
}
\end{lstlisting}
\begin{lstlisting}[caption={Reading from text file}]
#include <iostream>
#include <fstream>		// File handling library.
using namespace std;
int main()
{
	int x;
	float f;
	char c;

	fstream InFile;	// File handling object.

	// Open "abc.txt" for reading (ios::in).
	InFile.open("abc.txt", ios::in);
	InFile >> x >> f >> c;
	InFile.close();	// Close file.
	
	cout << "Data read is..." << endl;
	cout << "x = " << x << ", f = " << f << ", c = " << c << endl;

	return 0;
}
\end{lstlisting}
\begin{lstlisting}[caption={Reading a line from text file}]
#include <iostream>
#include <fstream>		// File handling library.
using namespace std;
int main()
{
	char Line[50];

	fstream InFile;	// File handling object.

	// Open "abc.txt" for reading (ios::in).
	InFile.open("abc.txt", ios::in);
	InFile.getline(Line, 50);
	InFile.close();	// Close file.
	
	cout << "Data read is..." << endl;
	cout << Line << endl;

	return 0;
}
\end{lstlisting}
\begin{lstlisting}[caption={Reading a whole text file line--by--line}]
#include <iostream>
#include <fstream>		// File handling library.
using namespace std;
int main()
{
	char Line[50];

	fstream InFile;	// File handling object.

	// Open "abc.txt" for reading (ios::in).
	InFile.open("abc.txt", ios::in);
	while(InFile)
	{
		InFile.getline(Line, 50);
		cout << "Read: " << Line << endl;
	}
	InFile.close();	// Close file.

	return 0;
}
\end{lstlisting}
\section{Writing to Binary File}
\begin{lstlisting}[caption={Writing to binary file}]
#include <fstream>		// File handling library.
using namespace std;

int main()
{
	fstream OutFile;	// File handling object.

	int x = 7;
	double d = -1.32489;
	char c = '$';

	// Open "abc.txt" for writing in binary mode (ios::out|ios::binary). This will overwrite existing file.
	OutFile.open("abc.txt", ios::out|ios::binary);
	OutFile.write((char*)&x, sizeof(int));
	OutFile.write((char*)&d, sizeof(double));
	OutFile.write((char*)&c, sizeof(char));
	OutFile.close();	// Close file.

	return 0;
}
\end{lstlisting}
\section{Reading from Binary File}
\begin{lstlisting}[caption={Reading from Binary file}]
#include <iostream>
#include <fstream>		// File handling library.
using namespace std;

int main()
{
	int x;
	double d;
	char c;

	fstream InFile;	// File handling object.

	// Open "abc.txt" for reading in binary mode (ios::in|ios::binary).
	InFile.open("abc.txt", ios::in|ios::binary);
	InFile.read((char*)&x, sizeof(int));
	InFile.read((char*)&d, sizeof(double));
	InFile.read((char*)&c, sizeof(char));
	InFile.close();	// Close file.

	cout << "x = " << x << endl;
	cout << "d = " << d << endl;
	cout << "c = " << c << endl;

	return 0;
}
\end{lstlisting}
\section{Writing Object to Binary File}
\begin{lstlisting}[caption={Writing an object to binary file}]
#include <iostream>
#include <fstream>		// File handling library.
using namespace std;

class StudentInfo
{
	public:
	char Name[20];
	int Age;
	float CGPA;
	char Gender;
};

int main()
{
	StudentInfo S;
	cout << "Enter student information: " << endl;
	cout << "Name: ";
	cin.getline(S.Name, 20);
	cout << "Age: ";
	cin >> S.Age;
	cout << "CGPA: ";
	cin >> S.CGPA;
	cout << "Gender (M/F): ";
	cin >> S.Gender;

	fstream OutFile;	// File handling object.

	// Open "abc.txt" for writing in binary mode (ios::out|ios::binary). This will overwrite existing file.
	OutFile.open("abc.txt", ios::out|ios::binary);
	OutFile.write((char*)&S, sizeof(StudentInfo));
	OutFile.close();	// Close file.

	return 0;
}
\end{lstlisting}
\newpage
\section{Reading Object from Binary File}
\begin{lstlisting}[caption={Reading an object from binary file}]
#include <iostream>
#include <fstream>		// File handling library.
using namespace std;
class StudentInfo
{
	public:
	char Name[20];
	int Age;
	float CGPA;
	char Gender;
};

int main()
{
	StudentInfo S;

	fstream InFile;	// File handling object.

	// Open "abc.txt" for reading in binary mode (ios::in|ios::binary).
	InFile.open("abc.txt", ios::in|ios::binary);
	InFile.read((char*)&S, sizeof(StudentInfo));
	InFile.close();	// Close file.

	switch (S.Gender)
	{
		case 'M':
			cout << "Name: Mr. " << S.Name << endl;
			break;
		case 'F':
			cout << "Name: Ms. " << S.Name << endl;
			break;
	}
	cout << "Age: " << S.Age << endl;
	cout << "CGPA: " << S.CGPA << endl;

	return 0;
}
\end{lstlisting}
%\nocite{*}
%\bibliographystyle{plain}
%\bibliography{OOPref}
\end{document}
