\documentclass[12pt,a4paper]{article}

% Margins.
\setlength{\oddsidemargin}{0in}
\setlength{\evensidemargin}{0in}
\setlength{\headheight}{12pt}
\setlength{\headsep}{42pt}
\setlength{\topmargin}{-54pt}
\setlength{\textwidth}{6.5in}
\setlength{\textheight}{10in}

\usepackage{amsmath}
\usepackage{float}
\usepackage{graphicx}
\usepackage[hyphens]{url}
\usepackage{hyperref}	% Clickable links to figures, references and urls.
\usepackage{enumerate}
\usepackage{datetime}

% Drawing.
\usepackage{pgf}
\usepackage{tikz}

% Listings for formatting code.
\usepackage{listings}
\usepackage{textcomp}
% General options.
\lstset{breaklines=true, basicstyle=\small\ttfamily, tabsize=4, numbers=left, stepnumber=1, frame=single, showstringspaces=false, upquote=true}
% C++ specific high-lighting. Comments are 50/50 shades of green/black and strings coloured with 60/40 red/black mixture.
\lstset{language=[ISO]C++, commentstyle=\color{green!50!black}, keywordstyle=\color{blue}, stringstyle=\color{red!60!black}}

%opening
\title{\vspace{-2cm}Programming for Engineers II\\Class 16\\Friend Functions\\Overloading Stream Operators\\Array Subscript Operator}
\author{Attique Dawood}
\date{February 27, 2014\\[0.2cm] Last Modified: \today, \currenttime}
\begin{document}
\maketitle
\section{Announcements}
\begin{itemize}
\item Assignment \# 03: Add copy constructor and assignment operator functions to matrix class. Make stream operator functions for both complex and matrix classes.
\end{itemize}
\section{Revision}
\begin{itemize}
\item Copy constructor and assignment operator.
\item When an appropriate constructor is already provided we can return an object by writing class name and passing constructor arguments.
\end{itemize}
\section{\texttt{friend} Functions}
\begin{itemize}
\item Sometimes we need to access private data of a class in certain trusted functions.
\item These functions can be declared \verb|friend| of the class.
\item \verb|friend| is specified with the prototype of the function inside the class declaration.
\item \verb|friend| function definition/implementation is outside the class.
\item \verb|friend| \underline{functions are not members of the class}. Remember, member functions can already access private members.
\end{itemize}
\begin{lstlisting}[caption={\texttt{friend} function to display Complex number}]
#include <iostream>
using namespace std;

class Complex
{
	private:
	float real;
	float img;
	public:
	Complex(): real(0.f), img(0.f) {}
	Complex(float a, float b): real(a), img(b) {}
	friend void Display(Complex&); // Friend function specified.
};

// Display function definition. This is an ordinary function.
void Display(Complex& c)
{
	cout << c.real << "+j" << c.img << endl;
}

int main()
{
	Complex A(2.f,3.3f);
	Display(A);

	return 0;
}
\end{lstlisting}
\section{Overloading the Insertion and Extraction Operators}
\begin{itemize}
\item Stream operators \verb|>>| and \verb|<<| operators are used for input and output.
\item \verb|cin| and \verb|cout| are objects of \verb|istream| and \verb|ostream| classes.
\item Stream operators must be overloaded as friends. Convenient way to display() and input().
\item Warning! Following code snippet will not work with VC6.
\end{itemize}
\begin{lstlisting}[caption={Overloading Stream Operators for Complex class}]
#include <iostream>
using namespace std;

class Complex
{
	private:
	float real;
	float img;
	public:
	Complex(): real(0.f), img(0.f) {}
	Complex(float a, float b): real(a), img(b) {}
	friend ostream& operator <<(ostream&, Complex&);
	friend istream& operator >>(istream&, Complex&);
};

ostream& operator <<(ostream& out, Complex& c)
{
	out << c.real << "+j" << c.img;
	return out;
}
istream& operator >>(istream& in, Complex& c)
{
	in >> c.real >> c.img;
	return in;
}

int main()
{
	Complex A;
	cout << "Enter a complex number: ";
	cin >> A;
	cout << "You entered: " << A << endl;

	return 0;
}
\end{lstlisting}
\section{Variable Scope: Global and Local Variables}
\begin{itemize}
\item \underline{Variable classification on the basis of `scope'.}
\item Global variables are accessible everywhere in a single file, in class, any function and main. Their scope is `global'.
\item Local variables are limited to the structure in which they are declared.
\item Global variables exist for the entirety of the program and will only be destroyed when program exits.
\item Local variables will only exist for the duration of the function in which they are declared. They will be destroyed when function ends.
\end{itemize}
\begin{lstlisting}[caption={Variable Scope}]
#include <iostream>
using namespace std;

int x; // Global variable

class Simple
{
	int y; // y is local variable for Simple class.
	public:
	// Default constructor.
	Simple()
	{
		cout << "Value of x in constructor: " << x << endl;
		x = 100;
		y = 0;
		cout << "Value of x after constructor call: " << x << endl;
	}
	~Simple()
	{
		x = -1;
		cout << "Value of x after destructor call: " << x << endl;
	}
};
void SimpleFunction()
{
	Simple functionobj; // functionobj is a local object of SimpleFunction().
	x = 1;
}
int main()
{
	x = 33;
	Simple* obj = new Simple;

	SimpleFunction();

	delete obj;

	for (int i=0; i<3; i++)
	{
		int j;
	}
	cout << i << j << endl; // Compilation error: i and j are local to for loop.

	cout << "main() is exiting now..." << endl;
	return 0;
}
\end{lstlisting}
\section{Function Return By Reference}
\begin{itemize}
\item Returning a local variable reference doesn't make sense. The local variable is destroyed after function ends.
\item Returning a global variable reference can make sense but global variables are part of the problem why we're using OOP anyway.
\item There is a cute trick when returning by reference. A function that returns a variable by reference can be used either left OR right of the equality sign.
\end{itemize}
\begin{lstlisting}[caption={Function Returning Reference}]
#include <iostream>
using namespace std;
int x = 0;
int& SimpleFunction()
{
	int y=0;
	return x; // return y won't make sense but return x would, but wait! return x also doesn't make any sense because x is accessible everywhere! Why return a global variable?
}
int main()
{
	SimpleFunction() = 100; // SimpleFunction() returns reference to x and that is assigned a value.

	cout << "Value of x = " << x << endl;
	
	int z;
	z = SimpleFunction(); // Assigning z the value of x.
	cout << "Value of z = " << z << endl;

	return 0;
}
\end{lstlisting}
\section{The Safe Array Class}
\begin{itemize}
\item Array bounds check? What happens if you access an invalid index like \verb|array[-2]|?
\item We can make a class that checks for bounds.
\item We'll do it first with simple \verb|set()| and \verb|get()| functions and then overload the subscript operator `\verb|[]|'.
\item \textbf{Remember: Operator overloading is a means to make the \underline{syntax easier}.}
\item What's the use of bounds checking when giving an invalid index merely exits the program instead of crashing it? Really, not much use but exiting is better than unexpected crash. You know exactly what happened and may even know who did it. Additionally, you always have the option of simply returning from function and continue with the program if user supplies invalid index.
\end{itemize}
\begin{lstlisting}[caption={Safe Array With Set() and Get()}]
#include <iostream>
#include <cstdlib> // for exit()
using namespace std;
const int SIZE = 10;
class SafeArray
{
	private:
	int element[SIZE];
	public:
	void Set(int index, int value)
	{
		if (index < 0 || index >= SIZE)
		{
			cout << "Array out of bounds." << endl;
			exit(-1);
		}
		element[index] = value;
	}
	 // Notice, a get() function doesn't change any variable, hence, const.
	int Get(int index) const
	{
		if (index < 0 || index >= SIZE)
		{
			cout << "Array out of bounds." << endl;
			exit(-1);
		}
		return element[index];
	}
};
int main()
{
	SafeArray obj;
	obj.Set(0, 10);
	cout << "Array[0] = " << obj.Get(0);

	obj.Set(10, 33); // ERROR!
	cout << "Array[-1] = " << obj.Get(-1); // ERROR!

	return 0;
}
\end{lstlisting}
\begin{lstlisting}[caption={Safe Array With Subscript Operator[]}]
#include <iostream>
#include <cstdlib> // for exit()
using namespace std;
const int SIZE = 10;
class SafeArray
{
	private:
	int element[SIZE];
	public:
	int& operator[](int index)
	{
		if (index < 0 || index >= SIZE)
		{
			cout << "Array out of bounds." << endl;
			exit(-1);
		}
		return element[index];
	}
};
int main()
{
	SafeArray obj;
	obj[0] = 10; // obj.Set(0, 10);
	cout << "Array[0] = " << obj[0];

	obj[10] = 33; //obj.Set(10, 33); // ERROR!
	cout << "Array[-1] = " << obj[-1]; // ERROR!

	return 0;
}
\end{lstlisting}
%\nocite{*}
\bibliographystyle{plain}
\bibliography{OOPref}
\end{document}
