\documentclass[12pt,a4paper]{article}

% Margins.
\setlength{\oddsidemargin}{0in}
\setlength{\evensidemargin}{0in}
\setlength{\headheight}{12pt}
\setlength{\headsep}{42pt}
\setlength{\topmargin}{-54pt}
\setlength{\textwidth}{6.5in}
\setlength{\textheight}{10in}

\usepackage{amsmath}
\usepackage{float}
\usepackage{graphicx}
\usepackage[hyphens]{url}
\usepackage{hyperref}	% Clickable links to figures, references and urls.
\usepackage{enumerate}
\usepackage{datetime}

% Drawing.
\usepackage{pgf}
\usepackage{tikz}

% Listings for formatting code.
\usepackage{listings}
\usepackage{textcomp}
% General options.
\lstset{breaklines=true, basicstyle=\small\ttfamily, tabsize=4, numbers=left, stepnumber=1, frame=single, showstringspaces=false, upquote=true}
% C++ specific high-lighting. Comments are 50/50 shades of green/black and strings coloured with 60/40 red/black mixture.
\lstset{language=[ISO]C++, commentstyle=\color{green!50!black}, keywordstyle=\color{blue}, stringstyle=\color{red!60!black}}

%opening
\title{\vspace{-2cm}Programming for Engineers II\\Class 39\\Function and Class Templates -- I}
\author{Attique Dawood}
\date{April 24, 2014\\[0.2cm] Last Modified: \today, \currenttime}
\begin{document}
\maketitle
\section{Announcements}
\begin{itemize}
\item None
\end{itemize}
\section{\texttt{iomanip} Library}
\begin{itemize}
\item \verb|<iomanip>| libaray is used for manipulating out put.
\item \verb|setw(x)| will fix the output to x chars.
\item \verb|setprecision(x)| will set the desired floating point precision.
\item \verb|setbase(x)| and \verb|dec|, \verb|oct|, \verb|hex| will format the output for indicated base. Valid bases are 10, 8 and 16.
\end{itemize}
\section{Templates in C++}
\begin{itemize}
\item Templates offer a mechanism to write generic code.
\item If we want to make a sum function to add two data types then data types can be \verb|int|, \verb|float|, \verb|double| or even \verb|complex|.
\item We would have to make sum function for all these types.
\item By making a template, we let compiler decide which instance of sum function to call.
\item Template calling can be implicit or explicit.
\item If no type is mentioned with function call, compiler will decide what is best instantiation of template.
\item If a type is mentioned with function call (e.g., \verb|sum<int> (3, 7)|) then that particular instance of template is instantiated by compiler.
\item Templates can also be defined for multiple types.
\item Templates can be defined for functions as well as classes.
\end{itemize}
\begin{lstlisting}[caption={Traditional way of writing sum function}]
int sum(int a, int b)
{
	return (a+b);
}
float sum(float a, float b)
{
	return (a+b);
}
...
complex sum(complex a, complex b)
{
	return (a+b);
}
...
\end{lstlisting}
\begin{lstlisting}[caption={Sum function template}]
// Sum function template.
// Depending on type of argument passed in main, that will be substituted in place of 'MyType'.
// For example, if called as sum(2.3, 7.7), MyType would be double.
template <class MyType> MyType sum(MyType a, MyType b)
{
	return (a+b);
}
int main()
{
	sum(3.2, 7.8); // implicit instantiation with double type.
	sum(3, 8); // implicit instantiation with int type.
	sum<double> (3.2, 7.8); // explicit double instantiation.
	
	return 0;
}
\end{lstlisting}
\begin{lstlisting}[caption={Sum function template for mixed types}]
template <class MyType1, class MyType2> MyType1 sum(MyType1 a, MyType2 b)
{
	return (a+b);
}
int main()
{
	sum(3.2, 7.8); // implicit instantiation with both double types.
	sum(3, 8.9); // implicit instantiation for int and double types.
	sum<int, double> (3, 7.8); // explicit instantiation.
	
	return 0;
}
\end{lstlisting}
\begin{lstlisting}[caption={Class template}]
#include <iostream>
using namespace std;
template <class MyType> class GenericArray
{
	private:
	MyType Data[5];
	public:
	void Input()
	{
		for (int i=0; i<5; i++)
			cin >> Data[i];
	}
	void Display()
	{
		for (int i=0; i<5; i++)
			cout << Data[i] << " ";
		cout << endl;
	}
};
int main()
{
	GenericArray<int> IntArray;
	IntArray.Input();
	IntArray.Display();
	
	return 0;
}
\end{lstlisting}
\section{Linked List Template}
\begin{itemize}
\item We can define a linked list template to hold any type of data.
\item Linked list code from lecture 31 is modified here.
\end{itemize}
\begin{lstlisting}[caption={Linked List Template},escapechar=$]
#include <iostream>
using namespace std;

// Node
template <class T> class Node
{
public:
	T Data;
	Node<T>* Next;
};

// Linked list class.
template <class Y> class LinkedList
{
private:
	Node$\textcolor{red}{<Y>}$* First;
	int Count;
public:
	// Constructor.
	LinkedList(): First(NULL), Count(0)
	{
	}
	// Adds a node at the end of linked list.
	void AddNode(Y pData)
	{
		// If creating first node we need to change first pointer.
		if (First == NULL)
		{
			Node$\textcolor{red}{<Y>}$* NewNode;
			NewNode = new Node$\textcolor{red}{<Y>}$;
			NewNode->Data = pData;
			NewNode->Next = NULL;
			First = NewNode;
			Count++;
		}
		// We need to traverse the whole list to get to last node.
		else
		{
			// Find last node.
			Node$\textcolor{red}{<Y>}$* temp;
			temp = First;
			while (temp->Next != NULL)
				temp = temp->Next;
			// temp now points to last node.
			Node$\textcolor{red}{<Y>}$* NewNode;
			NewNode = new Node$\textcolor{red}{<Y>}$;
			temp->Next = NewNode;
			NewNode->Data = pData;
			NewNode->Next = NULL;
			Count++;
		}
	}
	// Displays all entries in linked list.
	void Display()
	{
		Node$\textcolor{red}{<Y>}$* temp;
		temp = First;
		while(temp != NULL)
		{
			cout << "Data = " << temp->Data << endl;
			temp = temp->Next;
		}
	}
	// Search a data value.
	int Search(Y Key)
	{
		Node$\textcolor{red}{<Y>}$* temp;
		temp = First;
		while(temp != NULL)
		{
			if (temp->Data == Key)
				return 0;

			temp = temp->Next;
		}
		return -1; // If we ever get here then search was unsuccessful.
	}
	// Delete a Node by data value.
	int Delete(Y Key)
	{
		// If List is empty then nothing to do.
		if (First == NULL)
			return -1;

		Node$\textcolor{red}{<Y>}$* previous;
		Node$\textcolor{red}{<Y>}$* current;
		previous = current = First;
		// Checking first node.
		if (current->Data == Key)
		{
			// Delete First node. First will now point to next node.
			First = First->Next;
			delete current;
			Count--;
			return 0;
		}
		else
		{
			current = current->Next; // Move current pointer to next node.
			while(current != NULL)
			{
				// If entry is matched then link previous node to next node and delete current node.
				if (current->Data == Key)
				{
					previous->Next = current->Next;
					delete current;
					Count--;
					return 0;
				}
				// Move pointers forward.
				previous = current;
				current = current->Next;
			}
		}
		return -1; // If we ever get here then search was unsuccessful.
	}
	// Get size of linked list.
	int GetCount()
	{
		return Count;
	}
	// Subscript operator.
	Y& operator[](int Index)
	{
		// Checking index bounds.
		if (Index < 0 || Index >= Count)
		{
			cout << "Index out of bounds. Exiting..." << endl;
			exit(-1);
		}
		else
		{
			// Move temp pointer to point to node at desired index.
			Node$\textcolor{red}{<Y>}$* temp = First;
			for (int i=0; i<Index; i++)
				temp = temp->Next;

			return temp->Data;
		}
	}
};

int main()
{
	LinkedList<int> TestList;

	TestList.AddNode(-1);
	TestList.AddNode(2);
	TestList.AddNode(0);
	TestList.AddNode(7);

	cout << "Displaying Linked List: " << endl;
	TestList.Display();

	int SearchKey = 2;
	if (TestList.Search(SearchKey) == 0)
		cout << SearchKey << " is in the list." << endl;

	if (TestList.Delete(SearchKey) == 0)
		cout << SearchKey << " was deleted." << endl;

	cout << "Displaying Linked List: " << endl;
	TestList.Display();
	
	TestList[0] = -122;
	cout << "Displaying Linked List: " << endl;
	for (int i=0; i<TestList.GetCount(); i++)
		cout << TestList[i] << endl;

	return 0;
}
\end{lstlisting}
\section{Standard Template Library (STL Containers)}

A container is a holder object that stores a collection of other objects (its elements). 
They are implemented as class templates, which allows a great flexibility in the types supported as elements.

The container manages the storage space for its elements and provides member functions to access them,
either directly or through iterators (reference objects with similar properties to pointers).

Containers replicate structures very commonly used in programming:
\begin{itemize}
\item[-] dynamic arrays (vector)
\item[-] linked lists (list)
\item[-] queues (queue)
\item[-] stacks (stack)
\item[-] heaps (priority\_queue)
\item[-] trees (set)
\item[-] associative arrays (map)
\end{itemize}
Many containers have several member functions in common, and share functionalities.
The decision of which type of container to use for a specific need does not generally depend only on the
functionality offered by the container, but also on the efficiency of some of its members (complexity).
This is especially true for sequence containers, which offer different trade-offs in complexity between
inserting/removing elements and accessing them.

stack, queue and priority\_queue are implemented as container adaptors.
Container adaptors are not full container classes, but classes that provide a specific interface relying
on an object of one of the container classes (such as deque or list) to handle the elements.
The underlying container is encapsulated in such a way that its elements are accessed by the members of the
container class independently of the underlying container class used.

This excerpt was taken from \cite{Ref:CPP-STL}

\section{Vector Usage}
- Header:
\begin{verbatim}
  #include <vector>
  using std::vector;
\end{verbatim}
- Create vector object:
\begin{verbatim}
  vector<int> intvector;
  vector<myclassname> myclassvector;
\end{verbatim}
- Initialization:
\begin{verbatim}
  intvector.resize (20, 0); // int vector of size 20, all values intialized to 0.
\end{verbatim}
- Get size: Total number of elements in vector.
\begin{verbatim}
  int mysize = intvector.size();
  cout << intvector.size(); 
\end{verbatim}
- Insertion:
\begin{verbatim}
  Single insertion at back: intvector.push_back (3); // Creates a new integer, NOT the same as assignment. Size increases by one.
  Single or multiple insertions at a specific location:
  - intvector.insert (intvector.begin()+1, 1, 33); // Insert 33 at 2nd location.
  - intvector.insert (intvector.begin(), 3, 1); // Insert 3 1's at the beginning.
\end{verbatim}
- Access:
\begin{verbatim}
  Operator [] can be used to access entries:
  - int x = intvector[2]; // NOTE: This is NOT the same as insertion. Size remains same.
  - intvector[5] = 22;
\end{verbatim}
- Deletion:
\begin{verbatim}
  Delete a single entry: 
  - myvector.erase (myvector.begin()+i); // Delete ith entry. Size is reduced by one.
  Delete a range of entries: 
  - myvector.erase (myvector.begin()+1, myvector.begin()+5); // Delete elements in the range 2-6
  Delete vector:
  - myvector.clear ();
\end{verbatim}

\section{Tasks}
\textbf{1. Create a dynamic array of integers using vector. Display contents after each step.}
\begin{itemize}
\item[-] Insert 10 random numbers.
\item[-] Display the contents.
\item[-] Delete first element.
\item[-] Delete 3rd , 4th and 5th elements.
\item[-] Insert 2 4's at beginning (index 0).
\item[-] Insert 4 -1's starting at second last location.
\item[-] Display final size.
\item[-] Delete vector.
\item[-] Resize the vector to 20 size initialized to -1's.
\item[-] Set values randomly. Size remains 20.
\end{itemize}
\textbf{2. Create dynamic 2D and 3D arrays using vectors.}
\begin{verbatim}
int rows = 3;
int cols = 5;
- vector<vector<int>> my2DArray (rows, vector<int>(cols)); // Declaration.
- Elements can be accessed using double subscript notation [][] in nested for loops.
\end{verbatim}
\subsection{Helpful Resources}
- If string input is being skipped, use the following before getline() statement.
\begin{verbatim}
  if (cin.peek() == '\n')
      cin.ignore();
  getline (cin, stringname, '\n');
\end{verbatim}
- getline() can be used to read from a txt file line by line.
\begin{verbatim}
  fstream file("abc.txt", ios::in);
  while (true)
  {
      getline(file, tempstring, '\n');
      if (file.eof())
          break;
      cout << "Read: " << tempstring << endl;
  }
  file.close();
\end{verbatim}
- To access the char* in string use stringname.c\_str();
\begin{verbatim}
  fstream file("abc.txt", ios::out);
  string message = "this is a mesage. Writing this to file.";
  file << message.c_str() << endl;
  file.close();
\end{verbatim}
%\nocite{*}
\bibliographystyle{plain}
\bibliography{OOPref}
\end{document}
