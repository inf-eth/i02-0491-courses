\documentclass[12pt,a4paper]{article}

% Margins.
\setlength{\oddsidemargin}{0in}
\setlength{\evensidemargin}{0in}
\setlength{\headheight}{12pt}
\setlength{\headsep}{42pt}
\setlength{\topmargin}{-54pt}
\setlength{\textwidth}{6.5in}
\setlength{\textheight}{10in}

\usepackage{amsmath}
\usepackage{float}
\usepackage{graphicx}
\usepackage[hyphens]{url}
\usepackage{hyperref}	% Clickable links to figures, references and urls.

% Drawing.
\usepackage{pgf}
\usepackage{tikz}

% Listings for formatting code.
\usepackage{listings}
\usepackage{textcomp}
% General options.
\lstset{breaklines=true, basicstyle=\small\ttfamily, tabsize=4, numbers=left, stepnumber=1, frame=single, showstringspaces=false, upquote=true}
% C++ specific high-lighting. Comments are 50/50 shades of green/black and strings coloured with 60/40 red/black mixture.
\lstset{language=[ISO]C++, commentstyle=\color{green!50!black}, keywordstyle=\color{blue}, stringstyle=\color{red!60!black}}

%opening
\title{\vspace{-2cm}Programming for Engineers I\\Class 09\\Loops}
\author{Attique Dawood}

\begin{document}
\maketitle
\section{Revision}
\begin{itemize}
\item Float representation.
\item Float conversion.
\end{itemize}
\section{While Loop}
\begin{itemize}
\item \verb|while| loop runs until a condition is met.
\item The loop should have a terminating condition or it will run indefinitely.
\item \verb|while| loop checks condition first and then executes loop structure.
\end{itemize}
\begin{lstlisting}[caption={while Loop}]
while (condition is true)
{
	// Do this
}
\end{lstlisting}
\begin{lstlisting}[caption={Using while Loop to Sum}]
#include <iostream>
using namespace std;

int main()
{
	int Sum=0;
	while (Sum<20)
	{
		cout << "Sum is now: " << i << endl;
		Sum = Sum + 6;		
	}
	
	return 0;
}
\end{lstlisting}
\begin{lstlisting}[caption={Using Counter With while Loop}]
#include <iostream>
using namespace std;

int main()
{
	int i=0;
	while (i<10)
	{
		cout << "Counter is now: " << i << endl;
		i=i+1;		
	}
	
	return 0;
}
\end{lstlisting}
\section{do--while Loop}
\begin{itemize}
\item Executes loop structure first and then checks for condition.
\item Guaranteed to execute at least once.
\item Generally used when at least one execution of loop is required.
\end{itemize}
\begin{lstlisting}[caption={Using Counter With do-while Loop}]
#include <iostream>
using namespace std;

int main()
{
	int i=0;
	do
	{
		cout << "Counter is now: " << i << endl;
		i=i+1;		
	}
	while (i<10); // Notice ';' after while.
	
	return 0;
}
\end{lstlisting}
\section{for Loop}
\begin{itemize}
\item A somewhat better while loop.
\item Provides counter in addition to condition check. Don't need to explicitly make our own counter.
\item A \verb|for| is generally used instead of \verb|while| when it is known beforehand how many times the loop will run. Although not a strict rule but \verb|for| and \verb|while| can be used interchangeably.
\item A \verb|for| loop effectively becomes \verb|while| loop if counter isn't used.
\end{itemize}
\begin{lstlisting}[caption={for Loop}]
for (initialise counter; condition; increment counter)
{
	// Do this
}
\end{lstlisting}
\begin{lstlisting}[caption={for Loop With Counter}]
#include <iostream>
using namespace std;

int main()
{
	for (int i=0; i<10; i++)
	{
		cout << "Counter is now: " << i << endl;
	}
	
	return 0;
}
\end{lstlisting}
\section{Applications of Loops}
\begin{itemize}
\item Series calculation: Sum up $1/2 + 1/4 + 1/8 + 1/16 + ...$ ?
\item Print a box or fixed width.
\item Take student grades as input in a loop and calculate CGPA.
\end{itemize}
\end{document}