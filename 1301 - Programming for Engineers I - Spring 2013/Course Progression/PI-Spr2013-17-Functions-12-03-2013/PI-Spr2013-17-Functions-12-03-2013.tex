\documentclass[12pt,a4paper]{article}

% Margins.
\setlength{\oddsidemargin}{0in}
\setlength{\evensidemargin}{0in}
\setlength{\headheight}{12pt}
\setlength{\headsep}{42pt}
\setlength{\topmargin}{-54pt}
\setlength{\textwidth}{6.5in}
\setlength{\textheight}{10in}

\usepackage{amsmath}
\usepackage{float}
\usepackage{graphicx}
\usepackage[hyphens]{url}
\usepackage{hyperref}	% Clickable links to figures, references and urls.

% Drawing.
\usepackage{pgf}
\usepackage{tikz}

% Listings for formatting code.
\usepackage{listings}
\usepackage{textcomp}
% General options.
\lstset{breaklines=true, basicstyle=\small\ttfamily, tabsize=4, numbers=left, stepnumber=1, frame=single, showstringspaces=false, upquote=true}
% C++ specific high-lighting. Comments are 50/50 shades of green/black and strings coloured with 60/40 red/black mixture.
\lstset{language=[ISO]C++, commentstyle=\color{green!50!black}, keywordstyle=\color{blue}, stringstyle=\color{red!60!black}}

%opening
\title{\vspace{-2cm}Programming for Engineers I\\Class 17\\Functions}
\author{Attique Dawood}

\begin{document}
\maketitle
\section{Announcements}
\begin{itemize}
\item None.
\end{itemize}
\section{Revision}
\begin{itemize}
\item Bubblesort.
\item Insertion sort.
\end{itemize}
\section{Functions}
The functions are like subtasks. They receive some information, do some process and provide a result. Functions are invoked through a calling program. Calling program does not need to know what the function is doing and how it is performing its task. There is a specific function-calling methodology. The calling program calls a function by giving it some information and receives the result.

We have a main ( ) in every C program. ‘main ( )’ is also a function. When we write a function, it must start with a name, parentheses, and surrounding braces just like with main ( ). Functions are very important in code reusing.

There are two categories of functions:

\begin{description}
\item [1]Functions that return a value
\item [2]Functions that do not return a value
\end{description}

Suppose, we have a function that calculates the square of an integer such that function will return the square of the integer. Similarly we may have a function which displays some information on the screen so this function is not supposed to return any value to the calling program.
\subsection{Structure of a Function}
The declaration syntax of a function is as follows:\\[0.3cm]
\textit{return-value-type    function-name( argument-list )\\
$\left\lbrace\right.$\\
\hspace{1cm}declarations and statements\\
$\left.\right\rbrace$}
\begin{lstlisting}
#include <iostream>
using namespace std;

// Function definition/implementation.
void Display()
{
	cout << "Hello" << endl;
}

int main()
{
	Display(); // Function call	
	
	return 0;
}
\end{lstlisting}
\begin{lstlisting}
#include <iostream>
using namespace std;

// Function definition/implementation.
void Display()
{
	cout << "Hello" << endl;
}

int main()
{
	Display(); // Function call	
	
	return 0;
}
\end{lstlisting}
\begin{lstlisting}
//This function calculates the square of a number and returns it.
#include <iostream>
using namespace std;

int square(int number)
{
    int result = 0;
    result = number * number;
    return result;
}

void main()
{
    int number, result;
    result = 0;
    number = 0;
    // Getting the  input from the user
    cout << "Please enter the number to calculate the square: ";
    cin >> number;

    // Calling the function square(int number)
    result = square(number);
    cout << "The square of " << number << " is " << result << endl;
}
\end{lstlisting}
\subsection{Advantages of using Functions:}
\begin{itemize}
\item Code Reusability
\item Modularity
\item Readability
\end{itemize}
\section{Exiting a Function}
\begin{itemize}
\item To terminate a function that returns a value use the statement \verb|return somevalue;|.
\item To exit a function that doesn't return a value simply use \verb|return;|
\item This is similar to \verb|break|. A function will immediately exit on \verb|return| statement.
\end{itemize}
\section{Recursive Functions}
\begin{itemize}
\item Recursive functions call themselves.
\item There must be a terminating condition for recursive functions.
\item \textbf{Interesting Fact:} You can call \verb|main()| in \verb|main()| function! This will put your program in a loop. Although not really a good programming practice but it does give a convenient way to test your program when you don't want to re-run the exe.
\end{itemize}
\begin{lstlisting}[caption={Recursive Factorial Function}]
#include <iostream>
using namespace std;

// Function definition/implementation.
int Factorial(int n)
{
	if (n == 0 || n == 1)
	{
		return 1;
	}
	return (Factorial(n-1) * n);
}

int main()
{
	int k;
	cout << "Enter a number to calculate factorial: ";
	cin >> k;
	cout << "Factorial of " << k << " = " << Factorial(k) << endl;
	
	return 0;
}
\end{lstlisting}
\end{document}