\documentclass[12pt,a4paper]{article}

% To use this template make changes to following:
% 1. Fill-ables section.
% 2. Instructions.
% 3. Marks table.
% 4. Actual questions.

% ================================ 1. Fill-ables ================================
\newcommand\University{National University of Computer and Emerging Sciences}
\newcommand\Department{School of Engineering}
\newcommand\Campus{Islamabad Campus}
\newcommand\Semester{Fall 2014}
\newcommand\Exam{Final Exam}
\newcommand\Subject{EE305--Electromagnetic Theory}
\newcommand\ExamDate{Tuesday, January 20, 2015}
\newcommand\InstructorOne{Attique Dawood}
\newcommand\InstructorTwo{\null~}
\newcommand\InstructorThree{\null}
\newcommand\TotalTime{03 Hours}
\newcommand\TotalMarks{100}
\newcommand\TotalQuestions{10}
\newcommand\TotalPages{\pageref{LastPage}} % Automatic: No need to change this.
% Marks of each question
\def\Qone{10}
\def\Qtwo{10}
\def\Qthree{10}
\def\Qfour{10}
\def\Qfive{10}
\def\Qsix{10}
\def\Qseven{10}
\def\Qeight{10}
\def\Qnine{10}
\def\Qten{10}
% ============================================================================

% ============== 2. Packages ==============
\usepackage{amsmath}
\usepackage{float}
\usepackage{graphicx}
\usepackage[hyphens]{url}
\usepackage[hidelinks]{hyperref}	% Clickable links to figures, references and urls.
\usepackage{lastpage}
\usepackage{array}
\usepackage{fancyhdr}
% Drawing packages.
\usepackage{pgf}
\usepackage{tikz}
% Listings for formatting code.
\usepackage{listings}
\usepackage{textcomp}

% General listings options.
\lstset{breaklines=true, basicstyle=\footnotesize\ttfamily, tabsize=4, numbers=left, stepnumber=1, frame=none, showstringspaces=false, upquote=true}
% C++ specific high-lighting. Comments are 50/50 shades of green/black and strings coloured with 60/40 red/black mixture.
\lstset{language=[ISO]C++, commentstyle=\color{green!50!black}, keywordstyle=\color{blue}, stringstyle=\color{red!60!black}}

% Table cell alignment directives.
\newcolumntype{L}[1]{>{\raggedright\let\newline\\\arraybackslash\hspace{0pt}}m{#1}}
\newcolumntype{C}[1]{>{\centering\let\newline\\\arraybackslash\hspace{0pt}}m{#1}}
\newcolumntype{R}[1]{>{\raggedleft\let\newline\\\arraybackslash\hspace{0pt}}m{#1}}

% Line spacing.
\def\SingleSpacing{\def\baselinestretch{1}\large\normalsize}
\def\DoubleSpacing{\def\baselinestretch{1.5}\large\normalsize}

% Margins.
\setlength{\oddsidemargin}{0in}
\setlength{\evensidemargin}{0in}
\setlength{\headheight}{28pt}
\setlength{\headsep}{2.5pt}
\setlength{\topmargin}{-55pt}
\setlength{\textwidth}{6.5in}
\setlength{\textheight}{10.75in} % Actual: 10.75in

% ============================= 3. Header and Footer ============================
\pagestyle{empty}
% Header
\chead
{
	{\large\textbf{\University}}\\
	\begin{minipage}{0.45\textwidth}
	\begin{center}
	{\small\textbf{\Department}}
	\end{center}
	\end{minipage}
	\begin{minipage}{0.45\textwidth}
	\begin{center}
	{\small\textbf{\Campus}}
	\end{center}
	\end{minipage}
}
% Footer
\lfoot{{\small\Exam}}
\cfoot{{\small\Semester}}
\rfoot{{\small Page \textbf{\thepage}~of \textbf{\TotalPages}}}
\renewcommand{\headrulewidth}{0.4pt}
\renewcommand{\footrulewidth}{0.4pt}
% ================================= 4. Front Page ===============================
\begin{document}
% A cute macro to add up marks of all individual questions. Uncomment if you want to use this.
\pgfmathtruncatemacro\TotalMarks{\Qone+\Qtwo+\Qthree+\Qfour+\Qfive+\Qsix+\Qseven+\Qeight+\Qnine+\Qten}
% Use this macro if marks are in decimal points
%\newcommand\TotalMarks{\pgfmathsetmacro\TotalMarks{\Qone+\Qtwo+\Qthree+\Qfour+\Qfive+\Qsix+\Qseven+\Qeight+\Qnine+\Qten}}
\begin{minipage}[t]{0.6\textwidth}
\begin{flushleft}
\DoubleSpacing
{\Large\textbf{\Subject}}\\
{\normalsize\ExamDate}\\
{\large\textbf{Course Instructor}}\\
{\normalsize\InstructorOne}\\
{\normalsize\InstructorTwo}\\
{\normalsize\InstructorThree}
\end{flushleft}
\end{minipage}
\begin{minipage}[t]{0.01\textwidth}
~
\end{minipage}
\begin{minipage}[t]{0.325\textwidth}
\DoubleSpacing
{\normalsize Serial No:}\\
{\Large\textbf{\Exam}}\\
{\large\textbf{Total Time: \TotalTime}}\\
{\large\textbf{Total Marks: \TotalMarks}}\\[1cm]
\rule{5cm}{0.2mm}\\[-0.25cm]
{\small Signature of Invigilator}
\end{minipage}
\SingleSpacing
~\\[1.5cm] % Extra space.
\rule{7cm}{0.2mm}~\rule{2.5cm}{0.2mm}~\rule{2cm}{0.2mm}~\rule{4.5cm}{0.2mm}\\
{\small Student Name\hspace{4.75cm}Roll No\hspace{1.35cm}Section\hspace{0.95cm}Signature}\\[1cm]
% ============================ 5. Instructions ==================================
\textbf{DO NOT OPEN THE QUESTION BOOK OR START UNTIL INSTRUCTED.}\\
\textbf{Instructions:}
\begin{enumerate}
\itemsep0em
\item Verify at the start of the exam that you have a total of \TotalQuestions~questions printed on \TotalPages~pages including this title page.
\item Attempt all questions on the question-book and in the given order.
\item The exam is closed books, closed notes. Please see that the area in your threshold is free of any material classified as `useful in the paper' or else there may be a charge of cheating.
\item Read the questions carefully for clarity of context and understanding of meaning and make assumptions wherever required, for neither the invigilator will address your queries, nor the teacher/examiner will come to the examination hall for any assistance.
\item Fit in all your answers in the provided space. You may use extra space on the last page if required. If you do so, clearly mark question/part number on that page to avoid confusion. 
\item Use only your own stationery and calculator. If you do not have your own calculator, use manual calculations. 
\item Use only permanent ink-pens. Only the questions attempted with permanent ink-pens will be considered. Any part of paper done in lead pencil cannot be claimed for checking/rechecking.
\item \textbf{All distances and dimensions are in meters}.
\end{enumerate}
% =============================== 6. Marks Table ================================
\begin{table}[H]
\begin{center}
\vspace{0.3cm}
	{\footnotesize \begin{tabular}{|C{1.8cm}|C{0.75cm}|C{0.75cm}|C{0.75cm}|C{0.75cm}|C{0.75cm}|C{0.75cm}|C{0.75cm}|C{0.75cm}|C{0.75cm}|C{0.75cm}|c|}
	\hline
		\rule{0pt}{4.6ex} & Q-1 & Q-2 & Q-3 & Q-4 & Q-5 & Q-6 & Q-7 & Q-8 & Q-9 & Q-10 &\textbf{Total}\\[-0.5ex]
		\hline
		\rule{0pt}{2.5ex}\textbf{Total Marks}& \Qone & \Qtwo & \Qthree & \Qfour & \Qfive & \Qsix & \Qseven & \Qeight & \Qnine & \Qten & \TotalMarks\\
		\hline
		\rule{0pt}{2.5ex}\textbf{Marks Obtained}& & & & & & & & & & &\\
	\hline
	\end{tabular}}
\end{center}
\end{table}
{\small \textbf{Vetted By: \rule{6cm}{0.2mm} Vetter Signature: \rule{4.5cm}{0.2mm}}}
\setlength{\textheight}{10.4in}
\newpage
\pagestyle{fancy}
% ================================== 7. Questions ===============================
\noindent\textbf{Question 1: Coulomb's Law \hfill \Qone~marks}\\
Point charges $q_1=1$ C and $q_2=2$ C are located at (0, 0, 0) and (1, 0, 1). Find electric field at (1, -1, 0).
\newpage
\noindent\textbf{Question 2: Gauss' Law \hfill \Qtwo~marks}\\
A volume charge with density $\rho_v=\dfrac{1}{2\pi r^2}$ C/m$^3$ exists in the region $r<1$. This charge is enclosed in a conducting shell at $2<r<3$. Surface charge density on the outer surface of conductor was found to be $\rho_S=\dfrac{1}{36\pi}$ C/m$^2$. Find electric field everywhere.
\newpage
\noindent\textbf{Question 3: Potential and Electrostatic Energy\hfill \Qthree~marks}\\
Electric potential in a region of free space is $V=2yz+3x^2$ V.\\[0.2cm]
a. Find the work done in moving a 1 C charge from origin to (1, 1, 0). Is the field conservative?\\
b. Calculate electrostatic energy stored in the unit cube defined by $0<x<1$, $0<y<1$ and $0<z<1$.
\newpage
\noindent\textbf{Question 4: Electric Field in Materials\hfill \Qfour~marks}\\
An infinite dielectric slab ($\epsilon=3\epsilon_0$) is placed in the region $0<z<2$. If $\mathrm{\textbf E}=6x\hat x$ V/m in slab find\\[0.2cm]
a. \textbf{D} and \textbf{P} in slab.\\
b. $\rho_{pv}$ and $\rho_{ps}$.
\newpage
\noindent\textbf{Question 5: Electrostatic Boundary--Value Problems\hfill \Qfive~marks}\\
A circular disk of inner radius $\rho=a$ and outer radius $\rho=b$ has thickness $t$ and conductivity $\sigma$. Find the resistance between the top and bottom, $z=0$ and $z=t$. Laplace equation in cylindrical coordinates is $\nabla^2V=\dfrac{1}{\rho}\dfrac{\partial}{\partial\rho}\left(\rho\dfrac{\partial V}{\partial\rho}\right)+\dfrac{1}{\rho^2}\dfrac{\partial^2V}{\partial\phi^2}+\dfrac{\partial^2V}{\partial z^2}=0$.
\newpage
\noindent\textbf{Question 6: Biot--Savart Law\hfill \Qsix~marks}\\
A loop of radius 1 cm carries 2 A current. Find magnetic field at the centre of the ring.
\newpage
\noindent\textbf{Question 7: Magnetic Flux\hfill \Qseven~marks}\\
Vector magnetic potential in a region is $\mathrm{\textbf A}=2xy\hat x+x\hat y+yz\hat z$ Wb/m. Find the flux of magnetic field through the surface defined by $z=0$, $1<x<2$, $0<y<1$.
\newpage
\noindent\textbf{Question 8: Boundary Conditions\hfill \Qeight~marks}\\
Region 1 ($z<0$) is occupied by a material of permeability $\mu_1=3\mu_0$ while region 2 ($z>0$) is free space. If $\textbf{H}_2=2\hat y-4\hat z$ A/m and surface current density at $z=0$ is $\textbf{K}=2\hat x$ A/m find \textbf{H}$_1$ and $\theta_1$.
\newpage
\noindent\textbf{Question 9: Induced EMF\hfill \Qnine~marks}\\
A car is travelling at 100 km/hr on a straight road. Earth's magnetic field is $4\times10^{-5}$ Wb/m$^2$ and makes and angle of 30$^0$ with surface of the earth. Find induced EMF in the bumper of the car if bumper length is 1.6 m.
\newpage
\noindent\textbf{Question 10: Maxwell's Equations\hfill \Qten~marks}\\
Write final Maxwell's equations in differential form.
\end{document}