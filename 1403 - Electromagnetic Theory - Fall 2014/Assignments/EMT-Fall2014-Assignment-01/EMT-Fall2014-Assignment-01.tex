\documentclass[12pt,a4paper]{article}

% Margins.
\setlength{\oddsidemargin}{0in}
\setlength{\evensidemargin}{0in}
\setlength{\headheight}{12pt}
\setlength{\headsep}{0pt}
\setlength{\topmargin}{-60pt}
\setlength{\textwidth}{6.5in}
\setlength{\textheight}{10.75in}

\usepackage{amsmath}
\usepackage{float}
\usepackage{graphicx}
\usepackage[hyphens]{url}
\usepackage{hyperref}	% Clickable links to figures, references and urls.
\usepackage{datetime}

% Drawing.
\usepackage{pgf}
\usepackage{tikz}

% Listings for formatting code.
\usepackage{listings}
\usepackage{textcomp}
% General options.
\lstset{breaklines=true, basicstyle=\small\ttfamily, tabsize=4, numbers=left, stepnumber=1, frame=single, showstringspaces=false, upquote=true}
% C++ specific high-lighting. Comments are 50/50 shades of green/black and strings coloured with 60/40 red/black mixture.
\lstset{language=[ISO]C++, commentstyle=\color{green!50!black}, keywordstyle=\color{blue}, stringstyle=\color{red!60!black}}

%opening
\title{Electromagnetic Theory\\Assignment 01\\Vectors}
\author{Attique Dawood}
\date{August 29, 2014\\Due: September 01, 2014\\[0.2cm] Last Modified: \today, \currenttime}
\begin{document}
\maketitle
\begin{enumerate}
\item Find the angle between face diagonals of a cube.
\item Find the angle between space diagonals of a cube.
\item Do the following problems given at the end of chapter 01 from Sadiku: 1.1, 1.2, 1.5, 1.11, 1.21, 1.22.
\item A vector field is given by $\textbf{A}=-y\hat x+x\hat y$.
\begin{enumerate}
\item[a.] Draw the field vector at points (1, 0, 0), (0, 1, 0), (-1, 0, 0) and (0, -1, 0).
\item[b.] Draw the field vector at points (1, 1, 0), (-1, 1, 0), (-1, -1, 0) and (1, -1, 0).
\item[c.] What is the magnitude of the field at (2, 3, 0).
\item[d.] Find at unit vector along \textbf{A} at (1, 0, 0), (0, 1, 0), (1, 1, 0) and (1, -1, 0).
\item[e.] The given field \textbf{A} has a counter--clockwise rotation around the z-axis. Can you suggest a vector field with a clockwise rotation?
\end{enumerate}
\end{enumerate}
\end{document}
