\documentclass[12pt,a4paper]{article}

% Margins.
\setlength{\oddsidemargin}{0in}
\setlength{\evensidemargin}{0in}
\setlength{\headheight}{12pt}
\setlength{\headsep}{0pt}
\setlength{\topmargin}{-60pt}
\setlength{\textwidth}{6.5in}
\setlength{\textheight}{10.75in}

\usepackage{amsmath}
\usepackage{float}
\usepackage{graphicx}
\usepackage[hyphens]{url}
\usepackage{hyperref}	% Clickable links to figures, references and urls.
\usepackage{datetime}
\usepackage{longtable}
\usepackage{subfigure}

% Links direct to top of figures.
\usepackage[all]{hypcap}

% Drawing.
\usepackage{pgf}
\usepackage{tikz}

% Listings for formatting code.
\usepackage{listings}
\usepackage{textcomp}
% General options.
\lstset{breaklines=true, basicstyle=\small\ttfamily, tabsize=4, numbers=left, stepnumber=1, frame=single, showstringspaces=false, upquote=true}
% C++ specific high-lighting. Comments are 50/50 shades of green/black and strings coloured with 60/40 red/black mixture.
\lstset{language=[ISO]C++, commentstyle=\color{green!50!black}, keywordstyle=\color{blue}, stringstyle=\color{red!60!black}}

%opening
\title{Electromagnetic Theory\\Class 14\\Divergence and Curl}
\author{Attique Dawood}
\date{September 26, 2014\\[0.2cm] Last Modified: \today, \currenttime}
\begin{document}
\maketitle
\section{Announcements}
\begin{itemize}
\item None.
\end{itemize}
\section{Divergence of a Vector Field}
Divergence is defined as the flux per unit volume of a vector field as the volume shrinks. Mathematically
\begin{equation}
\mathrm{div. \textbf{A}}=\nabla\cdot\mathrm{\textbf{A}}=\lim_{\Delta v \to 0}\dfrac{\oint\limits_S\mathrm{\textbf{A}}\cdot d\mathrm{\textbf{S}}}{\Delta v}.
\end{equation}
Divergence of a vector field is a scalar quantity.
\section{Curl of a Vector Field}
Curl is defined as circulation of a vector field as the enclosed area shrinks. Mathematically
\begin{equation}
\mathrm{curl~of~\textbf{A}}=\nabla\times\mathrm{\textbf{A}}=\lim_{\Delta S \to 0}\dfrac{\oint\limits_L\mathrm{\textbf{A}}\cdot d\mathrm{\textbf{l}}}{\Delta S}\hat n.
\end{equation}
Curl of a vector field is a vector quantity and its direction is normal to the enclosed surface.
\section{Exercises}
\noindent\textbf{Question 4 \cite{Sadiku}[PE 3.6]:} Determine the divergence of following vector fields and determine them at the specified points.
\begin{itemize}
\item[(1)] 
\end{itemize}
%\nocite{*}
\bibliographystyle{plain}
\bibliography{EMTRef}
\end{document}
