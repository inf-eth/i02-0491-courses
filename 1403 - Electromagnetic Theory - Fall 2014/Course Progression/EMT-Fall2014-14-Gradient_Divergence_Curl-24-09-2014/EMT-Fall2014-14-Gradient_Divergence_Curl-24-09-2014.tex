\documentclass[12pt,a4paper]{article}

% Margins.
\setlength{\oddsidemargin}{0in}
\setlength{\evensidemargin}{0in}
\setlength{\headheight}{12pt}
\setlength{\headsep}{0pt}
\setlength{\topmargin}{-60pt}
\setlength{\textwidth}{6.5in}
\setlength{\textheight}{10.75in}

\usepackage{amsmath}
\usepackage{float}
\usepackage{graphicx}
\usepackage[hyphens]{url}
\usepackage{hyperref}	% Clickable links to figures, references and urls.
\usepackage{datetime}
\usepackage{longtable}
\usepackage{subfigure}

% Links direct to top of figures.
\usepackage[all]{hypcap}

% Drawing.
\usepackage{pgf}
\usepackage{tikz}

% Listings for formatting code.
\usepackage{listings}
\usepackage{textcomp}
% General options.
\lstset{breaklines=true, basicstyle=\small\ttfamily, tabsize=4, numbers=left, stepnumber=1, frame=single, showstringspaces=false, upquote=true}
% C++ specific high-lighting. Comments are 50/50 shades of green/black and strings coloured with 60/40 red/black mixture.
\lstset{language=[ISO]C++, commentstyle=\color{green!50!black}, keywordstyle=\color{blue}, stringstyle=\color{red!60!black}}

%opening
\title{Electromagnetic Theory\\Class 14\\Gradient, Divergence and Curl}
\author{Attique Dawood}
\date{September 24, 2014\\[0.2cm] Last Modified: \today, \currenttime}
\begin{document}
\maketitle
\section{Announcements}
\begin{itemize}
\item Assignment 03 to be submitted after sessional exam..
\end{itemize}
\section{Del Operator}
The Del operator is defined as
\begin{equation}
\nabla=\dfrac{\partial}{\partial x}\hat x+\dfrac{\partial}{\partial y}\hat y+\dfrac{\partial}{\partial z}\hat z.
\end{equation}
It must be kept in mind that Del operator is a vector and must be written with a vector bar.
\section{Gradient}
Given a scalar field $f(x,y,z)$ the gradient of $f$ is defined as
\begin{equation}
\nabla f=\dfrac{\partial f}{\partial x}\hat x+\dfrac{\partial f}{\partial y}\hat y+\dfrac{\partial f}{\partial z}\hat z.
\end{equation}
In cylindrical coordinates
\begin{equation}
\nabla f=\dfrac{\partial f}{\partial\rho}\hat\rho+\dfrac{1}{\rho}\dfrac{\partial f}{\partial\phi}\hat\phi+\dfrac{\partial f}{\partial z}\hat z.
\end{equation}
And in spherical coordinates
\begin{equation}
\nabla f=\dfrac{\partial f}{\partial r}\hat r+\dfrac{1}{r}\dfrac{\partial f}{\partial\theta}\hat\theta+\dfrac{1}{r\sin\theta}\dfrac{\partial f}{\partial\phi}\hat\phi.
\end{equation}
The gradient of $f$ is a vector. $|\nabla f|$ gives the maximum rate of change of $f$. The direction of $\nabla f$ is the direction of maximum rate of change. To find the rate of change in a particular direction $\hat n$ we can use the relation $\hat n\cdot\nabla f$ which is known as \textit{directional derivative} of $f$ in $\hat n$ direction. If $\hat n$ is in the same direction as that of $\nabla f$ then $\hat n\cdot\nabla f=|\nabla f|$.
\section{Divergence of a Vector Field}
Divergence is defined as the flux per unit volume of a vector field as the volume shrinks. Mathematically
\begin{equation}
\mathrm{div. \textbf{A}}=\nabla\cdot\mathrm{\textbf{A}}=\lim_{\Delta v \to 0}\dfrac{\oint\limits_S\mathrm{\textbf{A}}\cdot d\mathrm{\textbf{S}}}{\Delta v}.
\end{equation}
Divergence of a vector field is a scalar quantity.
\section{Curl of a Vector Field}
Curl is defined as circulation of a vector field as the enclosed area shrinks. Mathematically
\begin{equation}
\mathrm{curl~of~\textbf{A}}=\nabla\times\mathrm{\textbf{A}}=\lim_{\Delta S \to 0}\dfrac{\oint\limits_L\mathrm{\textbf{A}}\cdot d\mathrm{\textbf{l}}}{\Delta S}\hat n.
\end{equation}
Curl of a vector field is a vector quantity and its direction is normal to the enclosed surface.
\section{Exercises}
\noindent\textbf{Question 1 \cite{Sadiku}[Example 3.3]:} Find the gradient of following scalar fields
\begin{itemize}
\item[(1)] $U=\rho^2z\cos\phi$
\item[(2)] $W=10r\sin^2\theta\cos\phi$
\end{itemize}
\noindent\textbf{Question 2 \cite{Sadiku}[Example 3.4]:} Given $W=x^2y^2+xyz$, compute $\nabla W$ and directional derivative of $W$ in the direction $3\hat x+4\hat y+12\hat z$ at (2, -1, 0).\\[0.2cm]
\noindent\textbf{Question 3 \cite{Sadiku}[PE 3.6]:} Determine the divergence of following vector fields and determine them at the specified points.
\begin{itemize}
\item[(1)] 
\end{itemize}
%\nocite{*}
\bibliographystyle{plain}
\bibliography{EMTRef}
\end{document}
