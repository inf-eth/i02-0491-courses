\documentclass[12pt,a4paper]{article}

% Margins.
\setlength{\oddsidemargin}{0in}
\setlength{\evensidemargin}{0in}
\setlength{\headheight}{12pt}
\setlength{\headsep}{0pt}
\setlength{\topmargin}{-60pt}
\setlength{\textwidth}{6.5in}
\setlength{\textheight}{10.75in}

\usepackage{amsmath}
\usepackage{float}
\usepackage{graphicx}
\usepackage[hyphens]{url}
\usepackage{hyperref}	% Clickable links to figures, references and urls.
\usepackage{datetime}
\usepackage{longtable}
\usepackage{subfigure}

% Links direct to top of figures.
\usepackage[all]{hypcap}

% Drawing.
\usepackage{pgf}
\usepackage{tikz}

% Listings for formatting code.
\usepackage{listings}
\usepackage{textcomp}
% General options.
\lstset{breaklines=true, basicstyle=\small\ttfamily, tabsize=4, numbers=left, stepnumber=1, frame=single, showstringspaces=false, upquote=true}
% C++ specific high-lighting. Comments are 50/50 shades of green/black and strings coloured with 60/40 red/black mixture.
\lstset{language=[ISO]C++, commentstyle=\color{green!50!black}, keywordstyle=\color{blue}, stringstyle=\color{red!60!black}}

%opening
\title{Electromagnetic Theory\\Class 21\\Electric Field Lines}
\author{Attique Dawood}
\date{October 15, 2014\\[0.2cm] Last Modified: \today, \currenttime}
\begin{document}
\maketitle
\section{Announcements}
\begin{itemize}
\item Quiz \#03 today.
\end{itemize}
\section{Revision}
\begin{itemize}
\item Electric field of continuous charge distributions.
\end{itemize}
\section{Electric Field Lines}
\begin{itemize}
\item Electric field lines is a way of depicting electric field. This is an old concept and less frequently used in recent times.
\item Another way of representing electric field is to plot electric field vectors at each point in space. This is mostly used nowadays due to computing softwares that can easily plot electric fields.
\item Electric field line representation is different from vector representation because in case of vectors, length of vector gives field strength or magnitude, whereas, density (or number) of field lines give field strength.
\item If there are charges present then electric field lines will originate from positive charges and terminate at negative charges.
\end{itemize}
%\nocite{*}
%\bibliographystyle{plain}
%\bibliography{EMTRef}
\end{document}
