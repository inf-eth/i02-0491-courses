\documentclass[12pt,a4paper]{article}

% Margins.
\setlength{\oddsidemargin}{0in}
\setlength{\evensidemargin}{0in}
\setlength{\headheight}{12pt}
\setlength{\headsep}{0pt}
\setlength{\topmargin}{-60pt}
\setlength{\textwidth}{6.5in}
\setlength{\textheight}{10.75in}

\usepackage{amsmath}
\usepackage{float}
\usepackage{graphicx}
\usepackage[hyphens]{url}
\usepackage{hyperref}	% Clickable links to figures, references and urls.
\usepackage{datetime}
\usepackage{longtable}

% Links direct to top of figures.
\usepackage[all]{hypcap}

% Drawing.
\usepackage{pgf}
\usepackage{tikz}

% Listings for formatting code.
\usepackage{listings}
\usepackage{textcomp}
% General options.
\lstset{breaklines=true, basicstyle=\small\ttfamily, tabsize=4, numbers=left, stepnumber=1, frame=single, showstringspaces=false, upquote=true}
% C++ specific high-lighting. Comments are 50/50 shades of green/black and strings coloured with 60/40 red/black mixture.
\lstset{language=[ISO]C++, commentstyle=\color{green!50!black}, keywordstyle=\color{blue}, stringstyle=\color{red!60!black}}

%opening
\title{Electromagnetic Theory\\Class 09\\Line, Surface and Volume Integrals -- I}
\author{Attique Dawood}
\date{September 12, 2014\\[0.2cm] Last Modified: \today, \currenttime}
\begin{document}
\maketitle
\section{Revision}
\begin{itemize}
\item Differential length, surface and volume.
\end{itemize}
\section{Surface and Volume Integrals}
In electromagnetics we need to solve surface and volume integrals. Some common integrals are given below.
\begin{itemize}
\item $\int\limits_{S} d\mathrm{S}$~: Gives the total area of the surface `S'. This is generalised expression actually solved as a double integral.
\item $\int\limits_{v} dv$~: Calculate the total volume defined by limits of `v'. This is also a generalised expression actually solved as a triple integral.
\item $\int\limits_{S}\textbf{A}\cdot d\textbf{S}$~: Gives the total flux of the vector field \textbf{A} through the surface `S'.
\item $\int\limits_{v}\rho_v dv$~: If $\rho_v$ is volumetric charge density in $C/m^3$ then this integrals gives the total charge in volume `v'.
\end{itemize}
\section{Exercises}
\noindent\textbf{Question 1:} Use an appropriate $dS$ to find the surface area of given structures.
\begin{itemize}
\item[(1)] The area of curved surface of a cylinder of radius $\rho=3$ and height $0<z<2$.
\item[(2)] The surface area of a sphere of radius $r=3$.
\item[(3)] The area of curved surface of a slice of cake described by $\rho=3$, height $0<z<2$ and $0<\phi<30^0$.
\item[(4)] The area of an icecream cone described by $\theta=60^0$ and $0<r<3$.
\end{itemize}
\noindent\textbf{Question 2:} Use a suitable $dv$ to find the volumes of structures given in question 1.\\[0.2cm]
%\nocite{*}
\bibliographystyle{plain}
\bibliography{EMTRef}
\end{document}
