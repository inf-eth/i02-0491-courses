\documentclass[12pt,a4paper]{article}

% Margins.
\setlength{\oddsidemargin}{0in}
\setlength{\evensidemargin}{0in}
\setlength{\headheight}{12pt}
\setlength{\headsep}{0pt}
\setlength{\topmargin}{-60pt}
\setlength{\textwidth}{6.5in}
\setlength{\textheight}{10.75in}

\usepackage{amsmath}
\usepackage{float}
\usepackage{graphicx}
\usepackage[hyphens]{url}
\usepackage{hyperref}	% Clickable links to figures, references and urls.
\usepackage{datetime}
\usepackage{longtable}

% Links direct to top of figures.
\usepackage[all]{hypcap}

% Drawing.
\usepackage{pgf}
\usepackage{tikz}

% Listings for formatting code.
\usepackage{listings}
\usepackage{textcomp}
% General options.
\lstset{breaklines=true, basicstyle=\small\ttfamily, tabsize=4, numbers=left, stepnumber=1, frame=single, showstringspaces=false, upquote=true}
% C++ specific high-lighting. Comments are 50/50 shades of green/black and strings coloured with 60/40 red/black mixture.
\lstset{language=[ISO]C++, commentstyle=\color{green!50!black}, keywordstyle=\color{blue}, stringstyle=\color{red!60!black}}

%opening
\title{Electromagnetic Theory\\Class 07\\Vector Transformations\\Problems}
\author{Attique Dawood}
\date{September 08, 2014\\[0.2cm] Last Modified: \today, \currenttime}
\begin{document}
\maketitle
\section{Announcements}
\begin{itemize}
\item None.
\end{itemize}
\section{Revision}
\begin{itemize}
\item Vector conversions and transformation matrices.
\end{itemize}
\section{Point Conversions Using Calculator}
Following relations are useful when using calculator,
\begin{equation*}
\begin{split}
&(\rho, \phi): pol(x,y)\\
&(r, \theta): pol(z, \rho) = pol(z, pol(x, y))\\
&(x, y): rec(\rho, \phi)\\
&(z, \rho): rec(r, \theta)
\end{split}
\end{equation*}
\textbf{Warning: \underline{These are NOT vector conversions.} These are only point conversions between coordinate variables. Refer to transformation matrices for vector conversions.}
\section{Exercises}
\noindent\textbf{Question 1:}
\begin{itemize}
\item[(1)] Convert points P(1, 3, 5) T(0, -4, 3) and S(-3, -4, -10) to cylindrical and spherical coordinates.
\item[(2)] Transform \textbf{A}$=\hat x$ to cylindrical and spherical coordinates.
\item[(3)] Calculate \textbf{A} at T in cylindrical and spherical coordinates.
\end{itemize}
%\nocite{*}
\bibliographystyle{plain}
\bibliography{EMTRef}
\end{document}
