\documentclass[12pt,a4paper]{article}

% Margins.
\setlength{\oddsidemargin}{0in}
\setlength{\evensidemargin}{0in}
\setlength{\headheight}{12pt}
\setlength{\headsep}{0pt}
\setlength{\topmargin}{-60pt}
\setlength{\textwidth}{6.5in}
\setlength{\textheight}{10.75in}

\usepackage{amsmath}
\usepackage{float}
\usepackage{graphicx}
\usepackage[hyphens]{url}
\usepackage{hyperref}	% Clickable links to figures, references and urls.
\usepackage{datetime}
\usepackage{longtable}
\usepackage{subfigure}

% Links direct to top of figures.
\usepackage[all]{hypcap}

% Drawing.
\usepackage{pgf}
\usepackage{tikz}

% Listings for formatting code.
\usepackage{listings}
\usepackage{textcomp}
% General options.
\lstset{breaklines=true, basicstyle=\small\ttfamily, tabsize=4, numbers=left, stepnumber=1, frame=single, showstringspaces=false, upquote=true}
% C++ specific high-lighting. Comments are 50/50 shades of green/black and strings coloured with 60/40 red/black mixture.
\lstset{language=[ISO]C++, commentstyle=\color{green!50!black}, keywordstyle=\color{blue}, stringstyle=\color{red!60!black}}

%opening
\title{Electromagnetic Theory\\Class 13\\Line, Surface and Volume Integrals -- IV\\Solving Line Integrals\\Length of a Curve}
\author{Attique Dawood}
\date{September 22, 2014\\[0.2cm] Last Modified: \today, \currenttime}
\begin{document}
\maketitle
\section{Announcements}
\begin{itemize}
\item None.
\end{itemize}
\section{Revision}
\begin{itemize}
\item Line integrals.
\item Open and close paths and surfaces.
\end{itemize}
\section{Exercises}
\noindent\textbf{Question 1:} Using $\int\limits_{a}^{b}\hat n\cdot d\textbf{\textit{l}}$
\begin{itemize}
\item[(1)] Find the length of line segment from (0, 0) to (1, 2). Solve this in Cartesian as well as Cylindrical coordinates.
\item[(2)] Find the length of body/space diagonal of a unit cube.
\item[(3)] Find the arc length of a quarter circle of radius $\rho=3$ m in first quadrant.
\item[(4)] Find the length of the curve $y=x^2$ from (0, 0) to (1, 1). For this problem a calculator would be handy in solving the integrals.
\end{itemize}
\noindent\textbf{Note:} For part (4) you need to find a unit vector along the curve $y=x^2$. A vector parallel (or tangent) to $y=x^2$ can be obtained from the slope. Numerator and denominator of slope ($\dfrac{dy}{dx}$) are the $y$-- and $x$--components, respectively, of the vector. Here, $\dfrac{dy}{dx}=\dfrac{2x}{1}$ so the parallel/tangent vector is $\textbf{n}=\hat x+2x\hat y$. The unit vector is then $\hat n=\dfrac{\hat x+2x\hat y}{\sqrt{1+4x^2}}$.\\[0.2cm]
%\nocite{*}
\bibliographystyle{plain}
\bibliography{EMTRef}
\end{document}
