\documentclass[12pt,a4paper]{article}

% Margins.
\setlength{\oddsidemargin}{0in}
\setlength{\evensidemargin}{0in}
\setlength{\headheight}{12pt}
\setlength{\headsep}{0pt}
\setlength{\topmargin}{-60pt}
\setlength{\textwidth}{6.5in}
\setlength{\textheight}{10.75in}

\usepackage{amsmath}
\usepackage{float}
\usepackage{graphicx}
\usepackage[hyphens]{url}
\usepackage{hyperref}	% Clickable links to figures, references and urls.
\usepackage{datetime}

% Drawing.
\usepackage{pgf}
\usepackage{tikz}

% Listings for formatting code.
\usepackage{listings}
\usepackage{textcomp}
% General options.
\lstset{breaklines=true, basicstyle=\small\ttfamily, tabsize=4, numbers=left, stepnumber=1, frame=single, showstringspaces=false, upquote=true}
% C++ specific high-lighting. Comments are 50/50 shades of green/black and strings coloured with 60/40 red/black mixture.
\lstset{language=[ISO]C++, commentstyle=\color{green!50!black}, keywordstyle=\color{blue}, stringstyle=\color{red!60!black}}

%opening
\title{\vspace{-1.5cm}Electromagnetic Theory - Fall 2014\\Quiz \#04B}
\date{\vspace{-1.5cm}Date: 05--11--2014}
\begin{document}
\maketitle
\vspace{-0.5cm}
\noindent \textbf{Time: 15 minutes\hfill Total Marks: 10}\\[0.3cm]
\noindent \textbf{Name:\rule{8cm}{1pt}\hfill Roll Number:\rule{3cm}{1pt}}\\[0.5cm]
\noindent \textbf{Question 1:} $1$ nC, $-2$ nC and $3$ nC charges are located are (1, 0, 1), (0, 1, 1) and (0, 0, 1) respectively. Find the total work done in assembling this system of charges.\\[0.2cm]
\noindent \textbf{Question 2:} A line charge is placed on x--axis with uniform charge density 1 C/m at $1<x<5$. Find electric potential at origin.\\[0.2cm]
\end{document}
