\documentclass{article}

% Margins.
\setlength{\oddsidemargin}{0in}
\setlength{\evensidemargin}{0in}
\setlength{\headheight}{12pt}
\setlength{\headsep}{42pt}
\setlength{\topmargin}{-54pt}
\setlength{\textwidth}{6.5in}
\setlength{\textheight}{9in}

\usepackage{float}
\usepackage{graphicx}

%opening
\title{Programming for Engineers I\\Lab 10\\Functions}
\author{Hina Ashraf, Attique Dawood}

\begin{document}

\maketitle


\section{Functions}
The functions are like subtasks. They receive some information, do some process and provide a result. Functions are invoked through a calling program. Calling program does not need to know what the function is doing and how it is performing its task. There is a specific function-calling methodology. The calling program calls a function by giving it some information and receives the result.

We have a main ( ) in every C program. ‘main ( )’ is also a function. When we write a function, it must start with a name, parentheses, and surrounding braces just like with main ( ). Functions are very important in code reusing.

There are two categories of functions:

\begin{description}
\item [1]Functions that return a value
\item [2]Functions that do not return a value
\end{description}

Suppose, we have a function that calculates the square of an integer such that function will return the square of the integer. Similarly we may have a function which displays some information on the screen so this function is not supposed to return any value to the calling program.
\subsection{Structure of a Function}
The declaration syntax of a function is as follows:\\[0.3cm]
\textit{return-value-type    function-name( argument-list )\\
$\left\lbrace\right.$\\
\hspace{1cm}declarations and statements\\
$\left.\right\rbrace$}

\begin{verbatim}
//This function calculates the square of a number and returns it.
#include <stdio.h>
int square(int number)
{
    int result = 0;
    result = number * number;
    return result;
}

void main()
{
    int number, result;
    result = 0;
    number = 0;
    // Getting the  input from the user
    printf("Please enter the number to calculate the square");
    scanf("%d", &number);

    // Calling the function square(int number)
    result = square(number);
    printf("The square of %d is %d.\n", number, result);
}
\end{verbatim}

\subsection{Advantages of using Functions:}

\begin{itemize}
\item Code Reusability
\item Modularity
\item Readability
\end{itemize}

\section{Exercise}
\textbf{Question No. 1:} Write a function that calculates the area of a rectangle. Function takes two integers (length and width) as input and returns an integer that is the area of rectangle.\\
\textbf{Question No. 2:} Write a program that takes an integer as input from the user. Pass that integer as an argument to a function named as evenodd (int ) that prints “number is even ” or “number is odd” after evaluating the number.\\
\textbf{Question No. 3:} Write a function that calculates the power of an integer number. For example you want to calculate $(2)^5$. Your function takes 2 and 5 as arguments and return the value of $(2)^5$.\\

\end{document}
